%\documentclass[12pt,a4paper]{article}
\documentclass[20pt,a4paper]{article}
\usepackage[margin=0.5in]{geometry}
\usepackage{polyglossia}
\usepackage[dvipsnames]{xcolor}
\pagenumbering{gobble}
% This beautiful one line disable the initial spacing at the beginning of a line
\usepackage[parfill]{parskip} 
\usepackage{setspace}
\setstretch{2}

\setdefaultlanguage[numerals=maghrib]{arabic}
\newfontfamily\arabicfont[Script=Arabic]{Amiri}

\title{}
\author{}
\date{}
\definecolor{cl_page}{gray}{0.98}
\definecolor{cl_aya}{HTML}{DEEEFF}

\begin{document}
\pagecolor{cl_page}

% Start %


{\tiny\colorbox{cl_aya}{1}} الْحَمْدُ لِلَّهِ الَّذِى لَهُ مَا فِى السَّمَوَتِ وَمَا فِى الْأَرْضِ وَلَهُ الْحَمْدُ فِى الْءَاخِرَةِ وَهُوَ الْحَكِيمُ الْخَبِيرُ
{\tiny\colorbox{cl_aya}{2}} يَعْلَمُ مَا يَلِجُ فِى الْأَرْضِ وَمَا يَخْرُجُ مِنْهَا وَمَا يَنزِلُ مِنَ السَّمَاءِ وَمَا يَعْرُجُ فِيهَا وَهُوَ الرَّحِيمُ الْغَفُورُ
{\tiny\colorbox{cl_aya}{3}} وَقَالَ الَّذِينَ كَفَرُوا لَا تَأْتِينَا السَّاعَةُ قُلْ بَلَى وَرَبِّى لَتَأْتِيَنَّكُمْ عَلِمِ الْغَيْبِ لَا يَعْزُبُ عَنْهُ مِثْقَالُ ذَرَّةٍ فِى السَّمَوَتِ وَلَا فِى الْأَرْضِ وَلَا أَصْغَرُ مِن ذَلِكَ وَلَا أَكْبَرُ إِلَّا فِى كِتَبٍ مُّبِينٍ
{\tiny\colorbox{cl_aya}{4}} لِّيَجْزِىَ الَّذِينَ ءَامَنُوا وَعَمِلُوا الصَّلِحَتِ أُولَئِكَ لَهُم مَّغْفِرَةٌ وَرِزْقٌ كَرِيمٌ
{\tiny\colorbox{cl_aya}{5}} وَالَّذِينَ سَعَوْ فِى ءَايَتِنَا مُعَجِزِينَ أُولَئِكَ لَهُمْ عَذَابٌ مِّن رِّجْزٍ أَلِيمٌ
{\tiny\colorbox{cl_aya}{6}} وَيَرَى الَّذِينَ أُوتُوا الْعِلْمَ الَّذِى أُنزِلَ إِلَيْكَ مِن رَّبِّكَ هُوَ الْحَقَّ وَيَهْدِى إِلَى صِرَطِ الْعَزِيزِ الْحَمِيدِ
{\tiny\colorbox{cl_aya}{7}} وَقَالَ الَّذِينَ كَفَرُوا هَلْ نَدُلُّكُمْ عَلَى رَجُلٍ يُنَبِّئُكُمْ إِذَا مُزِّقْتُمْ كُلَّ مُمَزَّقٍ إِنَّكُمْ لَفِى خَلْقٍ جَدِيدٍ
{\tiny\colorbox{cl_aya}{8}} أَفْتَرَى عَلَى اللَّهِ كَذِبًا أَم بِهِ جِنَّةٌ بَلِ الَّذِينَ لَا يُؤْمِنُونَ بِالْءَاخِرَةِ فِى الْعَذَابِ وَالضَّلَلِ الْبَعِيدِ
{\tiny\colorbox{cl_aya}{9}} أَفَلَمْ يَرَوْا إِلَى مَا بَيْنَ أَيْدِيهِمْ وَمَا خَلْفَهُم مِّنَ السَّمَاءِ وَالْأَرْضِ إِن نَّشَأْ نَخْسِفْ بِهِمُ الْأَرْضَ أَوْ نُسْقِطْ عَلَيْهِمْ كِسَفًا مِّنَ السَّمَاءِ إِنَّ فِى ذَلِكَ لَءَايَةً لِّكُلِّ عَبْدٍ مُّنِيبٍ
{\tiny\colorbox{cl_aya}{10}} وَلَقَدْ ءَاتَيْنَا دَاوُدَ مِنَّا فَضْلًا يَجِبَالُ أَوِّبِى مَعَهُ وَالطَّيْرَ وَأَلَنَّا لَهُ الْحَدِيدَ
{\tiny\colorbox{cl_aya}{11}} أَنِ اعْمَلْ سَبِغَتٍ وَقَدِّرْ فِى السَّرْدِ وَاعْمَلُوا صَلِحًا إِنِّى بِمَا تَعْمَلُونَ بَصِيرٌ
{\tiny\colorbox{cl_aya}{12}} وَلِسُلَيْمَنَ الرِّيحَ غُدُوُّهَا شَهْرٌ وَرَوَاحُهَا شَهْرٌ وَأَسَلْنَا لَهُ عَيْنَ الْقِطْرِ وَمِنَ الْجِنِّ مَن يَعْمَلُ بَيْنَ يَدَيْهِ بِإِذْنِ رَبِّهِ وَمَن يَزِغْ مِنْهُمْ عَنْ أَمْرِنَا نُذِقْهُ مِنْ عَذَابِ السَّعِيرِ
{\tiny\colorbox{cl_aya}{13}} يَعْمَلُونَ لَهُ مَا يَشَاءُ مِن مَّحَرِيبَ وَتَمَثِيلَ وَجِفَانٍ كَالْجَوَابِ وَقُدُورٍ رَّاسِيَتٍ اعْمَلُوا ءَالَ دَاوُدَ شُكْرًا وَقَلِيلٌ مِّنْ عِبَادِىَ الشَّكُورُ
{\tiny\colorbox{cl_aya}{14}} فَلَمَّا قَضَيْنَا عَلَيْهِ الْمَوْتَ مَا دَلَّهُمْ عَلَى مَوْتِهِ إِلَّا دَابَّةُ الْأَرْضِ تَأْكُلُ مِنسَأَتَهُ فَلَمَّا خَرَّ تَبَيَّنَتِ الْجِنُّ أَن لَّوْ كَانُوا يَعْلَمُونَ الْغَيْبَ مَا لَبِثُوا فِى الْعَذَابِ الْمُهِينِ
{\tiny\colorbox{cl_aya}{15}} لَقَدْ كَانَ لِسَبَإٍ فِى مَسْكَنِهِمْ ءَايَةٌ جَنَّتَانِ عَن يَمِينٍ وَشِمَالٍ كُلُوا مِن رِّزْقِ رَبِّكُمْ وَاشْكُرُوا لَهُ بَلْدَةٌ طَيِّبَةٌ وَرَبٌّ غَفُورٌ
{\tiny\colorbox{cl_aya}{16}} فَأَعْرَضُوا فَأَرْسَلْنَا عَلَيْهِمْ سَيْلَ الْعَرِمِ وَبَدَّلْنَهُم بِجَنَّتَيْهِمْ جَنَّتَيْنِ ذَوَاتَىْ أُكُلٍ خَمْطٍ وَأَثْلٍ وَشَىْءٍ مِّن سِدْرٍ قَلِيلٍ
{\tiny\colorbox{cl_aya}{17}} ذَلِكَ جَزَيْنَهُم بِمَا كَفَرُوا وَهَلْ نُجَزِى إِلَّا الْكَفُورَ
{\tiny\colorbox{cl_aya}{18}} وَجَعَلْنَا بَيْنَهُمْ وَبَيْنَ الْقُرَى الَّتِى بَرَكْنَا فِيهَا قُرًى ظَهِرَةً وَقَدَّرْنَا فِيهَا السَّيْرَ سِيرُوا فِيهَا لَيَالِىَ وَأَيَّامًا ءَامِنِينَ
{\tiny\colorbox{cl_aya}{19}} فَقَالُوا رَبَّنَا بَعِدْ بَيْنَ أَسْفَارِنَا وَظَلَمُوا أَنفُسَهُمْ فَجَعَلْنَهُمْ أَحَادِيثَ وَمَزَّقْنَهُمْ كُلَّ مُمَزَّقٍ إِنَّ فِى ذَلِكَ لَءَايَتٍ لِّكُلِّ صَبَّارٍ شَكُورٍ
{\tiny\colorbox{cl_aya}{20}} وَلَقَدْ صَدَّقَ عَلَيْهِمْ إِبْلِيسُ ظَنَّهُ فَاتَّبَعُوهُ إِلَّا فَرِيقًا مِّنَ الْمُؤْمِنِينَ
{\tiny\colorbox{cl_aya}{21}} وَمَا كَانَ لَهُ عَلَيْهِم مِّن سُلْطَنٍ إِلَّا لِنَعْلَمَ مَن يُؤْمِنُ بِالْءَاخِرَةِ مِمَّنْ هُوَ مِنْهَا فِى شَكٍّ وَرَبُّكَ عَلَى كُلِّ شَىْءٍ حَفِيظٌ
{\tiny\colorbox{cl_aya}{22}} قُلِ ادْعُوا الَّذِينَ زَعَمْتُم مِّن دُونِ اللَّهِ لَا يَمْلِكُونَ مِثْقَالَ ذَرَّةٍ فِى السَّمَوَتِ وَلَا فِى الْأَرْضِ وَمَا لَهُمْ فِيهِمَا مِن شِرْكٍ وَمَا لَهُ مِنْهُم مِّن ظَهِيرٍ
{\tiny\colorbox{cl_aya}{23}} وَلَا تَنفَعُ الشَّفَعَةُ عِندَهُ إِلَّا لِمَنْ أَذِنَ لَهُ حَتَّى إِذَا فُزِّعَ عَن قُلُوبِهِمْ قَالُوا مَاذَا قَالَ رَبُّكُمْ قَالُوا الْحَقَّ وَهُوَ الْعَلِىُّ الْكَبِيرُ
{\tiny\colorbox{cl_aya}{24}} قُلْ مَن يَرْزُقُكُم مِّنَ السَّمَوَتِ وَالْأَرْضِ قُلِ اللَّهُ وَإِنَّا أَوْ إِيَّاكُمْ لَعَلَى هُدًى أَوْ فِى ضَلَلٍ مُّبِينٍ
{\tiny\colorbox{cl_aya}{25}} قُل لَّا تُسَْٔلُونَ عَمَّا أَجْرَمْنَا وَلَا نُسَْٔلُ عَمَّا تَعْمَلُونَ
{\tiny\colorbox{cl_aya}{26}} قُلْ يَجْمَعُ بَيْنَنَا رَبُّنَا ثُمَّ يَفْتَحُ بَيْنَنَا بِالْحَقِّ وَهُوَ الْفَتَّاحُ الْعَلِيمُ
{\tiny\colorbox{cl_aya}{27}} قُلْ أَرُونِىَ الَّذِينَ أَلْحَقْتُم بِهِ شُرَكَاءَ كَلَّا بَلْ هُوَ اللَّهُ الْعَزِيزُ الْحَكِيمُ
{\tiny\colorbox{cl_aya}{28}} وَمَا أَرْسَلْنَكَ إِلَّا كَافَّةً لِّلنَّاسِ بَشِيرًا وَنَذِيرًا وَلَكِنَّ أَكْثَرَ النَّاسِ لَا يَعْلَمُونَ
{\tiny\colorbox{cl_aya}{29}} وَيَقُولُونَ مَتَى هَذَا الْوَعْدُ إِن كُنتُمْ صَدِقِينَ
{\tiny\colorbox{cl_aya}{30}} قُل لَّكُم مِّيعَادُ يَوْمٍ لَّا تَسْتَْٔخِرُونَ عَنْهُ سَاعَةً وَلَا تَسْتَقْدِمُونَ
{\tiny\colorbox{cl_aya}{31}} وَقَالَ الَّذِينَ كَفَرُوا لَن نُّؤْمِنَ بِهَذَا الْقُرْءَانِ وَلَا بِالَّذِى بَيْنَ يَدَيْهِ وَلَوْ تَرَى إِذِ الظَّلِمُونَ مَوْقُوفُونَ عِندَ رَبِّهِمْ يَرْجِعُ بَعْضُهُمْ إِلَى بَعْضٍ الْقَوْلَ يَقُولُ الَّذِينَ اسْتُضْعِفُوا لِلَّذِينَ اسْتَكْبَرُوا لَوْلَا أَنتُمْ لَكُنَّا مُؤْمِنِينَ
{\tiny\colorbox{cl_aya}{32}} قَالَ الَّذِينَ اسْتَكْبَرُوا لِلَّذِينَ اسْتُضْعِفُوا أَنَحْنُ صَدَدْنَكُمْ عَنِ الْهُدَى بَعْدَ إِذْ جَاءَكُم بَلْ كُنتُم مُّجْرِمِينَ
{\tiny\colorbox{cl_aya}{33}} وَقَالَ الَّذِينَ اسْتُضْعِفُوا لِلَّذِينَ اسْتَكْبَرُوا بَلْ مَكْرُ الَّيْلِ وَالنَّهَارِ إِذْ تَأْمُرُونَنَا أَن نَّكْفُرَ بِاللَّهِ وَنَجْعَلَ لَهُ أَندَادًا وَأَسَرُّوا النَّدَامَةَ لَمَّا رَأَوُا الْعَذَابَ وَجَعَلْنَا الْأَغْلَلَ فِى أَعْنَاقِ الَّذِينَ كَفَرُوا هَلْ يُجْزَوْنَ إِلَّا مَا كَانُوا يَعْمَلُونَ
{\tiny\colorbox{cl_aya}{34}} وَمَا أَرْسَلْنَا فِى قَرْيَةٍ مِّن نَّذِيرٍ إِلَّا قَالَ مُتْرَفُوهَا إِنَّا بِمَا أُرْسِلْتُم بِهِ كَفِرُونَ
{\tiny\colorbox{cl_aya}{35}} وَقَالُوا نَحْنُ أَكْثَرُ أَمْوَلًا وَأَوْلَدًا وَمَا نَحْنُ بِمُعَذَّبِينَ
{\tiny\colorbox{cl_aya}{36}} قُلْ إِنَّ رَبِّى يَبْسُطُ الرِّزْقَ لِمَن يَشَاءُ وَيَقْدِرُ وَلَكِنَّ أَكْثَرَ النَّاسِ لَا يَعْلَمُونَ
{\tiny\colorbox{cl_aya}{37}} وَمَا أَمْوَلُكُمْ وَلَا أَوْلَدُكُم بِالَّتِى تُقَرِّبُكُمْ عِندَنَا زُلْفَى إِلَّا مَنْ ءَامَنَ وَعَمِلَ صَلِحًا فَأُولَئِكَ لَهُمْ جَزَاءُ الضِّعْفِ بِمَا عَمِلُوا وَهُمْ فِى الْغُرُفَتِ ءَامِنُونَ
{\tiny\colorbox{cl_aya}{38}} وَالَّذِينَ يَسْعَوْنَ فِى ءَايَتِنَا مُعَجِزِينَ أُولَئِكَ فِى الْعَذَابِ مُحْضَرُونَ
{\tiny\colorbox{cl_aya}{39}} قُلْ إِنَّ رَبِّى يَبْسُطُ الرِّزْقَ لِمَن يَشَاءُ مِنْ عِبَادِهِ وَيَقْدِرُ لَهُ وَمَا أَنفَقْتُم مِّن شَىْءٍ فَهُوَ يُخْلِفُهُ وَهُوَ خَيْرُ الرَّزِقِينَ
{\tiny\colorbox{cl_aya}{40}} وَيَوْمَ يَحْشُرُهُمْ جَمِيعًا ثُمَّ يَقُولُ لِلْمَلَئِكَةِ أَهَؤُلَاءِ إِيَّاكُمْ كَانُوا يَعْبُدُونَ
{\tiny\colorbox{cl_aya}{41}} قَالُوا سُبْحَنَكَ أَنتَ وَلِيُّنَا مِن دُونِهِم بَلْ كَانُوا يَعْبُدُونَ الْجِنَّ أَكْثَرُهُم بِهِم مُّؤْمِنُونَ
{\tiny\colorbox{cl_aya}{42}} فَالْيَوْمَ لَا يَمْلِكُ بَعْضُكُمْ لِبَعْضٍ نَّفْعًا وَلَا ضَرًّا وَنَقُولُ لِلَّذِينَ ظَلَمُوا ذُوقُوا عَذَابَ النَّارِ الَّتِى كُنتُم بِهَا تُكَذِّبُونَ
{\tiny\colorbox{cl_aya}{43}} وَإِذَا تُتْلَى عَلَيْهِمْ ءَايَتُنَا بَيِّنَتٍ قَالُوا مَا هَذَا إِلَّا رَجُلٌ يُرِيدُ أَن يَصُدَّكُمْ عَمَّا كَانَ يَعْبُدُ ءَابَاؤُكُمْ وَقَالُوا مَا هَذَا إِلَّا إِفْكٌ مُّفْتَرًى وَقَالَ الَّذِينَ كَفَرُوا لِلْحَقِّ لَمَّا جَاءَهُمْ إِنْ هَذَا إِلَّا سِحْرٌ مُّبِينٌ
{\tiny\colorbox{cl_aya}{44}} وَمَا ءَاتَيْنَهُم مِّن كُتُبٍ يَدْرُسُونَهَا وَمَا أَرْسَلْنَا إِلَيْهِمْ قَبْلَكَ مِن نَّذِيرٍ
{\tiny\colorbox{cl_aya}{45}} وَكَذَّبَ الَّذِينَ مِن قَبْلِهِمْ وَمَا بَلَغُوا مِعْشَارَ مَا ءَاتَيْنَهُمْ فَكَذَّبُوا رُسُلِى فَكَيْفَ كَانَ نَكِيرِ
{\tiny\colorbox{cl_aya}{46}} قُلْ إِنَّمَا أَعِظُكُم بِوَحِدَةٍ أَن تَقُومُوا لِلَّهِ مَثْنَى وَفُرَدَى ثُمَّ تَتَفَكَّرُوا مَا بِصَاحِبِكُم مِّن جِنَّةٍ إِنْ هُوَ إِلَّا نَذِيرٌ لَّكُم بَيْنَ يَدَىْ عَذَابٍ شَدِيدٍ
{\tiny\colorbox{cl_aya}{47}} قُلْ مَا سَأَلْتُكُم مِّنْ أَجْرٍ فَهُوَ لَكُمْ إِنْ أَجْرِىَ إِلَّا عَلَى اللَّهِ وَهُوَ عَلَى كُلِّ شَىْءٍ شَهِيدٌ
{\tiny\colorbox{cl_aya}{48}} قُلْ إِنَّ رَبِّى يَقْذِفُ بِالْحَقِّ عَلَّمُ الْغُيُوبِ
{\tiny\colorbox{cl_aya}{49}} قُلْ جَاءَ الْحَقُّ وَمَا يُبْدِئُ الْبَطِلُ وَمَا يُعِيدُ
{\tiny\colorbox{cl_aya}{50}} قُلْ إِن ضَلَلْتُ فَإِنَّمَا أَضِلُّ عَلَى نَفْسِى وَإِنِ اهْتَدَيْتُ فَبِمَا يُوحِى إِلَىَّ رَبِّى إِنَّهُ سَمِيعٌ قَرِيبٌ
{\tiny\colorbox{cl_aya}{51}} وَلَوْ تَرَى إِذْ فَزِعُوا فَلَا فَوْتَ وَأُخِذُوا مِن مَّكَانٍ قَرِيبٍ
{\tiny\colorbox{cl_aya}{52}} وَقَالُوا ءَامَنَّا بِهِ وَأَنَّى لَهُمُ التَّنَاوُشُ مِن مَّكَانٍ بَعِيدٍ
{\tiny\colorbox{cl_aya}{53}} وَقَدْ كَفَرُوا بِهِ مِن قَبْلُ وَيَقْذِفُونَ بِالْغَيْبِ مِن مَّكَانٍ بَعِيدٍ
{\tiny\colorbox{cl_aya}{54}} وَحِيلَ بَيْنَهُمْ وَبَيْنَ مَا يَشْتَهُونَ كَمَا فُعِلَ بِأَشْيَاعِهِم مِّن قَبْلُ إِنَّهُمْ كَانُوا فِى شَكٍّ مُّرِيبٍ
\end{document}