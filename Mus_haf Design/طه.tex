%\documentclass[12pt,a4paper]{article}
\documentclass[20pt,a4paper]{article}
\usepackage[margin=0.5in]{geometry}
\usepackage{polyglossia}
\usepackage[dvipsnames]{xcolor}
\pagenumbering{gobble}
% This beautiful one line disable the initial spacing at the beginning of a line
\usepackage[parfill]{parskip} 
\usepackage{setspace}
\setstretch{2}

\setdefaultlanguage[numerals=maghrib]{arabic}
\newfontfamily\arabicfont[Script=Arabic]{Amiri}

\title{}
\author{}
\date{}
\definecolor{cl_page}{gray}{0.98}
\definecolor{cl_aya}{HTML}{DEEEFF}

\begin{document}
\pagecolor{cl_page}

% Start %


{\tiny\colorbox{cl_aya}{1}} طه
{\tiny\colorbox{cl_aya}{2}} مَا أَنزَلْنَا عَلَيْكَ الْقُرْءَانَ لِتَشْقَى
{\tiny\colorbox{cl_aya}{3}} إِلَّا تَذْكِرَةً لِّمَن يَخْشَى
{\tiny\colorbox{cl_aya}{4}} تَنزِيلًا مِّمَّنْ خَلَقَ الْأَرْضَ وَالسَّمَوَتِ الْعُلَى
{\tiny\colorbox{cl_aya}{5}} الرَّحْمَنُ عَلَى الْعَرْشِ اسْتَوَى
{\tiny\colorbox{cl_aya}{6}} لَهُ مَا فِى السَّمَوَتِ وَمَا فِى الْأَرْضِ وَمَا بَيْنَهُمَا وَمَا تَحْتَ الثَّرَى
{\tiny\colorbox{cl_aya}{7}} وَإِن تَجْهَرْ بِالْقَوْلِ فَإِنَّهُ يَعْلَمُ السِّرَّ وَأَخْفَى
{\tiny\colorbox{cl_aya}{8}} اللَّهُ لَا إِلَهَ إِلَّا هُوَ لَهُ الْأَسْمَاءُ الْحُسْنَى
{\tiny\colorbox{cl_aya}{9}} وَهَلْ أَتَىكَ حَدِيثُ مُوسَى
{\tiny\colorbox{cl_aya}{10}} إِذْ رَءَا نَارًا فَقَالَ لِأَهْلِهِ امْكُثُوا إِنِّى ءَانَسْتُ نَارًا لَّعَلِّى ءَاتِيكُم مِّنْهَا بِقَبَسٍ أَوْ أَجِدُ عَلَى النَّارِ هُدًى
{\tiny\colorbox{cl_aya}{11}} فَلَمَّا أَتَىهَا نُودِىَ يَمُوسَى
{\tiny\colorbox{cl_aya}{12}} إِنِّى أَنَا رَبُّكَ فَاخْلَعْ نَعْلَيْكَ إِنَّكَ بِالْوَادِ الْمُقَدَّسِ طُوًى
{\tiny\colorbox{cl_aya}{13}} وَأَنَا اخْتَرْتُكَ فَاسْتَمِعْ لِمَا يُوحَى
{\tiny\colorbox{cl_aya}{14}} إِنَّنِى أَنَا اللَّهُ لَا إِلَهَ إِلَّا أَنَا فَاعْبُدْنِى وَأَقِمِ الصَّلَوةَ لِذِكْرِى
{\tiny\colorbox{cl_aya}{15}} إِنَّ السَّاعَةَ ءَاتِيَةٌ أَكَادُ أُخْفِيهَا لِتُجْزَى كُلُّ نَفْسٍ بِمَا تَسْعَى
{\tiny\colorbox{cl_aya}{16}} فَلَا يَصُدَّنَّكَ عَنْهَا مَن لَّا يُؤْمِنُ بِهَا وَاتَّبَعَ هَوَىهُ فَتَرْدَى
{\tiny\colorbox{cl_aya}{17}} وَمَا تِلْكَ بِيَمِينِكَ يَمُوسَى
{\tiny\colorbox{cl_aya}{18}} قَالَ هِىَ عَصَاىَ أَتَوَكَّؤُا عَلَيْهَا وَأَهُشُّ بِهَا عَلَى غَنَمِى وَلِىَ فِيهَا مََٔارِبُ أُخْرَى
{\tiny\colorbox{cl_aya}{19}} قَالَ أَلْقِهَا يَمُوسَى
{\tiny\colorbox{cl_aya}{20}} فَأَلْقَىهَا فَإِذَا هِىَ حَيَّةٌ تَسْعَى
{\tiny\colorbox{cl_aya}{21}} قَالَ خُذْهَا وَلَا تَخَفْ سَنُعِيدُهَا سِيرَتَهَا الْأُولَى
{\tiny\colorbox{cl_aya}{22}} وَاضْمُمْ يَدَكَ إِلَى جَنَاحِكَ تَخْرُجْ بَيْضَاءَ مِنْ غَيْرِ سُوءٍ ءَايَةً أُخْرَى
{\tiny\colorbox{cl_aya}{23}} لِنُرِيَكَ مِنْ ءَايَتِنَا الْكُبْرَى
{\tiny\colorbox{cl_aya}{24}} اذْهَبْ إِلَى فِرْعَوْنَ إِنَّهُ طَغَى
{\tiny\colorbox{cl_aya}{25}} قَالَ رَبِّ اشْرَحْ لِى صَدْرِى
{\tiny\colorbox{cl_aya}{26}} وَيَسِّرْ لِى أَمْرِى
{\tiny\colorbox{cl_aya}{27}} وَاحْلُلْ عُقْدَةً مِّن لِّسَانِى
{\tiny\colorbox{cl_aya}{28}} يَفْقَهُوا قَوْلِى
{\tiny\colorbox{cl_aya}{29}} وَاجْعَل لِّى وَزِيرًا مِّنْ أَهْلِى
{\tiny\colorbox{cl_aya}{30}} هَرُونَ أَخِى
{\tiny\colorbox{cl_aya}{31}} اشْدُدْ بِهِ أَزْرِى
{\tiny\colorbox{cl_aya}{32}} وَأَشْرِكْهُ فِى أَمْرِى
{\tiny\colorbox{cl_aya}{33}} كَىْ نُسَبِّحَكَ كَثِيرًا
{\tiny\colorbox{cl_aya}{34}} وَنَذْكُرَكَ كَثِيرًا
{\tiny\colorbox{cl_aya}{35}} إِنَّكَ كُنتَ بِنَا بَصِيرًا
{\tiny\colorbox{cl_aya}{36}} قَالَ قَدْ أُوتِيتَ سُؤْلَكَ يَمُوسَى
{\tiny\colorbox{cl_aya}{37}} وَلَقَدْ مَنَنَّا عَلَيْكَ مَرَّةً أُخْرَى
{\tiny\colorbox{cl_aya}{38}} إِذْ أَوْحَيْنَا إِلَى أُمِّكَ مَا يُوحَى
{\tiny\colorbox{cl_aya}{39}} أَنِ اقْذِفِيهِ فِى التَّابُوتِ فَاقْذِفِيهِ فِى الْيَمِّ فَلْيُلْقِهِ الْيَمُّ بِالسَّاحِلِ يَأْخُذْهُ عَدُوٌّ لِّى وَعَدُوٌّ لَّهُ وَأَلْقَيْتُ عَلَيْكَ مَحَبَّةً مِّنِّى وَلِتُصْنَعَ عَلَى عَيْنِى
{\tiny\colorbox{cl_aya}{40}} إِذْ تَمْشِى أُخْتُكَ فَتَقُولُ هَلْ أَدُلُّكُمْ عَلَى مَن يَكْفُلُهُ فَرَجَعْنَكَ إِلَى أُمِّكَ كَىْ تَقَرَّ عَيْنُهَا وَلَا تَحْزَنَ وَقَتَلْتَ نَفْسًا فَنَجَّيْنَكَ مِنَ الْغَمِّ وَفَتَنَّكَ فُتُونًا فَلَبِثْتَ سِنِينَ فِى أَهْلِ مَدْيَنَ ثُمَّ جِئْتَ عَلَى قَدَرٍ يَمُوسَى
{\tiny\colorbox{cl_aya}{41}} وَاصْطَنَعْتُكَ لِنَفْسِى
{\tiny\colorbox{cl_aya}{42}} اذْهَبْ أَنتَ وَأَخُوكَ بَِٔايَتِى وَلَا تَنِيَا فِى ذِكْرِى
{\tiny\colorbox{cl_aya}{43}} اذْهَبَا إِلَى فِرْعَوْنَ إِنَّهُ طَغَى
{\tiny\colorbox{cl_aya}{44}} فَقُولَا لَهُ قَوْلًا لَّيِّنًا لَّعَلَّهُ يَتَذَكَّرُ أَوْ يَخْشَى
{\tiny\colorbox{cl_aya}{45}} قَالَا رَبَّنَا إِنَّنَا نَخَافُ أَن يَفْرُطَ عَلَيْنَا أَوْ أَن يَطْغَى
{\tiny\colorbox{cl_aya}{46}} قَالَ لَا تَخَافَا إِنَّنِى مَعَكُمَا أَسْمَعُ وَأَرَى
{\tiny\colorbox{cl_aya}{47}} فَأْتِيَاهُ فَقُولَا إِنَّا رَسُولَا رَبِّكَ فَأَرْسِلْ مَعَنَا بَنِى إِسْرَءِيلَ وَلَا تُعَذِّبْهُمْ قَدْ جِئْنَكَ بَِٔايَةٍ مِّن رَّبِّكَ وَالسَّلَمُ عَلَى مَنِ اتَّبَعَ الْهُدَى
{\tiny\colorbox{cl_aya}{48}} إِنَّا قَدْ أُوحِىَ إِلَيْنَا أَنَّ الْعَذَابَ عَلَى مَن كَذَّبَ وَتَوَلَّى
{\tiny\colorbox{cl_aya}{49}} قَالَ فَمَن رَّبُّكُمَا يَمُوسَى
{\tiny\colorbox{cl_aya}{50}} قَالَ رَبُّنَا الَّذِى أَعْطَى كُلَّ شَىْءٍ خَلْقَهُ ثُمَّ هَدَى
{\tiny\colorbox{cl_aya}{51}} قَالَ فَمَا بَالُ الْقُرُونِ الْأُولَى
{\tiny\colorbox{cl_aya}{52}} قَالَ عِلْمُهَا عِندَ رَبِّى فِى كِتَبٍ لَّا يَضِلُّ رَبِّى وَلَا يَنسَى
{\tiny\colorbox{cl_aya}{53}} الَّذِى جَعَلَ لَكُمُ الْأَرْضَ مَهْدًا وَسَلَكَ لَكُمْ فِيهَا سُبُلًا وَأَنزَلَ مِنَ السَّمَاءِ مَاءً فَأَخْرَجْنَا بِهِ أَزْوَجًا مِّن نَّبَاتٍ شَتَّى
{\tiny\colorbox{cl_aya}{54}} كُلُوا وَارْعَوْا أَنْعَمَكُمْ إِنَّ فِى ذَلِكَ لَءَايَتٍ لِّأُولِى النُّهَى
{\tiny\colorbox{cl_aya}{55}} مِنْهَا خَلَقْنَكُمْ وَفِيهَا نُعِيدُكُمْ وَمِنْهَا نُخْرِجُكُمْ تَارَةً أُخْرَى
{\tiny\colorbox{cl_aya}{56}} وَلَقَدْ أَرَيْنَهُ ءَايَتِنَا كُلَّهَا فَكَذَّبَ وَأَبَى
{\tiny\colorbox{cl_aya}{57}} قَالَ أَجِئْتَنَا لِتُخْرِجَنَا مِنْ أَرْضِنَا بِسِحْرِكَ يَمُوسَى
{\tiny\colorbox{cl_aya}{58}} فَلَنَأْتِيَنَّكَ بِسِحْرٍ مِّثْلِهِ فَاجْعَلْ بَيْنَنَا وَبَيْنَكَ مَوْعِدًا لَّا نُخْلِفُهُ نَحْنُ وَلَا أَنتَ مَكَانًا سُوًى
{\tiny\colorbox{cl_aya}{59}} قَالَ مَوْعِدُكُمْ يَوْمُ الزِّينَةِ وَأَن يُحْشَرَ النَّاسُ ضُحًى
{\tiny\colorbox{cl_aya}{60}} فَتَوَلَّى فِرْعَوْنُ فَجَمَعَ كَيْدَهُ ثُمَّ أَتَى
{\tiny\colorbox{cl_aya}{61}} قَالَ لَهُم مُّوسَى وَيْلَكُمْ لَا تَفْتَرُوا عَلَى اللَّهِ كَذِبًا فَيُسْحِتَكُم بِعَذَابٍ وَقَدْ خَابَ مَنِ افْتَرَى
{\tiny\colorbox{cl_aya}{62}} فَتَنَزَعُوا أَمْرَهُم بَيْنَهُمْ وَأَسَرُّوا النَّجْوَى
{\tiny\colorbox{cl_aya}{63}} قَالُوا إِنْ هَذَنِ لَسَحِرَنِ يُرِيدَانِ أَن يُخْرِجَاكُم مِّنْ أَرْضِكُم بِسِحْرِهِمَا وَيَذْهَبَا بِطَرِيقَتِكُمُ الْمُثْلَى
{\tiny\colorbox{cl_aya}{64}} فَأَجْمِعُوا كَيْدَكُمْ ثُمَّ ائْتُوا صَفًّا وَقَدْ أَفْلَحَ الْيَوْمَ مَنِ اسْتَعْلَى
{\tiny\colorbox{cl_aya}{65}} قَالُوا يَمُوسَى إِمَّا أَن تُلْقِىَ وَإِمَّا أَن نَّكُونَ أَوَّلَ مَنْ أَلْقَى
{\tiny\colorbox{cl_aya}{66}} قَالَ بَلْ أَلْقُوا فَإِذَا حِبَالُهُمْ وَعِصِيُّهُمْ يُخَيَّلُ إِلَيْهِ مِن سِحْرِهِمْ أَنَّهَا تَسْعَى
{\tiny\colorbox{cl_aya}{67}} فَأَوْجَسَ فِى نَفْسِهِ خِيفَةً مُّوسَى
{\tiny\colorbox{cl_aya}{68}} قُلْنَا لَا تَخَفْ إِنَّكَ أَنتَ الْأَعْلَى
{\tiny\colorbox{cl_aya}{69}} وَأَلْقِ مَا فِى يَمِينِكَ تَلْقَفْ مَا صَنَعُوا إِنَّمَا صَنَعُوا كَيْدُ سَحِرٍ وَلَا يُفْلِحُ السَّاحِرُ حَيْثُ أَتَى
{\tiny\colorbox{cl_aya}{70}} فَأُلْقِىَ السَّحَرَةُ سُجَّدًا قَالُوا ءَامَنَّا بِرَبِّ هَرُونَ وَمُوسَى
{\tiny\colorbox{cl_aya}{71}} قَالَ ءَامَنتُمْ لَهُ قَبْلَ أَنْ ءَاذَنَ لَكُمْ إِنَّهُ لَكَبِيرُكُمُ الَّذِى عَلَّمَكُمُ السِّحْرَ فَلَأُقَطِّعَنَّ أَيْدِيَكُمْ وَأَرْجُلَكُم مِّنْ خِلَفٍ وَلَأُصَلِّبَنَّكُمْ فِى جُذُوعِ النَّخْلِ وَلَتَعْلَمُنَّ أَيُّنَا أَشَدُّ عَذَابًا وَأَبْقَى
{\tiny\colorbox{cl_aya}{72}} قَالُوا لَن نُّؤْثِرَكَ عَلَى مَا جَاءَنَا مِنَ الْبَيِّنَتِ وَالَّذِى فَطَرَنَا فَاقْضِ مَا أَنتَ قَاضٍ إِنَّمَا تَقْضِى هَذِهِ الْحَيَوةَ الدُّنْيَا
{\tiny\colorbox{cl_aya}{73}} إِنَّا ءَامَنَّا بِرَبِّنَا لِيَغْفِرَ لَنَا خَطَيَنَا وَمَا أَكْرَهْتَنَا عَلَيْهِ مِنَ السِّحْرِ وَاللَّهُ خَيْرٌ وَأَبْقَى
{\tiny\colorbox{cl_aya}{74}} إِنَّهُ مَن يَأْتِ رَبَّهُ مُجْرِمًا فَإِنَّ لَهُ جَهَنَّمَ لَا يَمُوتُ فِيهَا وَلَا يَحْيَى
{\tiny\colorbox{cl_aya}{75}} وَمَن يَأْتِهِ مُؤْمِنًا قَدْ عَمِلَ الصَّلِحَتِ فَأُولَئِكَ لَهُمُ الدَّرَجَتُ الْعُلَى
{\tiny\colorbox{cl_aya}{76}} جَنَّتُ عَدْنٍ تَجْرِى مِن تَحْتِهَا الْأَنْهَرُ خَلِدِينَ فِيهَا وَذَلِكَ جَزَاءُ مَن تَزَكَّى
{\tiny\colorbox{cl_aya}{77}} وَلَقَدْ أَوْحَيْنَا إِلَى مُوسَى أَنْ أَسْرِ بِعِبَادِى فَاضْرِبْ لَهُمْ طَرِيقًا فِى الْبَحْرِ يَبَسًا لَّا تَخَفُ دَرَكًا وَلَا تَخْشَى
{\tiny\colorbox{cl_aya}{78}} فَأَتْبَعَهُمْ فِرْعَوْنُ بِجُنُودِهِ فَغَشِيَهُم مِّنَ الْيَمِّ مَا غَشِيَهُمْ
{\tiny\colorbox{cl_aya}{79}} وَأَضَلَّ فِرْعَوْنُ قَوْمَهُ وَمَا هَدَى
{\tiny\colorbox{cl_aya}{80}} يَبَنِى إِسْرَءِيلَ قَدْ أَنجَيْنَكُم مِّنْ عَدُوِّكُمْ وَوَعَدْنَكُمْ جَانِبَ الطُّورِ الْأَيْمَنَ وَنَزَّلْنَا عَلَيْكُمُ الْمَنَّ وَالسَّلْوَى
{\tiny\colorbox{cl_aya}{81}} كُلُوا مِن طَيِّبَتِ مَا رَزَقْنَكُمْ وَلَا تَطْغَوْا فِيهِ فَيَحِلَّ عَلَيْكُمْ غَضَبِى وَمَن يَحْلِلْ عَلَيْهِ غَضَبِى فَقَدْ هَوَى
{\tiny\colorbox{cl_aya}{82}} وَإِنِّى لَغَفَّارٌ لِّمَن تَابَ وَءَامَنَ وَعَمِلَ صَلِحًا ثُمَّ اهْتَدَى
{\tiny\colorbox{cl_aya}{83}} وَمَا أَعْجَلَكَ عَن قَوْمِكَ يَمُوسَى
{\tiny\colorbox{cl_aya}{84}} قَالَ هُمْ أُولَاءِ عَلَى أَثَرِى وَعَجِلْتُ إِلَيْكَ رَبِّ لِتَرْضَى
{\tiny\colorbox{cl_aya}{85}} قَالَ فَإِنَّا قَدْ فَتَنَّا قَوْمَكَ مِن بَعْدِكَ وَأَضَلَّهُمُ السَّامِرِىُّ
{\tiny\colorbox{cl_aya}{86}} فَرَجَعَ مُوسَى إِلَى قَوْمِهِ غَضْبَنَ أَسِفًا قَالَ يَقَوْمِ أَلَمْ يَعِدْكُمْ رَبُّكُمْ وَعْدًا حَسَنًا أَفَطَالَ عَلَيْكُمُ الْعَهْدُ أَمْ أَرَدتُّمْ أَن يَحِلَّ عَلَيْكُمْ غَضَبٌ مِّن رَّبِّكُمْ فَأَخْلَفْتُم مَّوْعِدِى
{\tiny\colorbox{cl_aya}{87}} قَالُوا مَا أَخْلَفْنَا مَوْعِدَكَ بِمَلْكِنَا وَلَكِنَّا حُمِّلْنَا أَوْزَارًا مِّن زِينَةِ الْقَوْمِ فَقَذَفْنَهَا فَكَذَلِكَ أَلْقَى السَّامِرِىُّ
{\tiny\colorbox{cl_aya}{88}} فَأَخْرَجَ لَهُمْ عِجْلًا جَسَدًا لَّهُ خُوَارٌ فَقَالُوا هَذَا إِلَهُكُمْ وَإِلَهُ مُوسَى فَنَسِىَ
{\tiny\colorbox{cl_aya}{89}} أَفَلَا يَرَوْنَ أَلَّا يَرْجِعُ إِلَيْهِمْ قَوْلًا وَلَا يَمْلِكُ لَهُمْ ضَرًّا وَلَا نَفْعًا
{\tiny\colorbox{cl_aya}{90}} وَلَقَدْ قَالَ لَهُمْ هَرُونُ مِن قَبْلُ يَقَوْمِ إِنَّمَا فُتِنتُم بِهِ وَإِنَّ رَبَّكُمُ الرَّحْمَنُ فَاتَّبِعُونِى وَأَطِيعُوا أَمْرِى
{\tiny\colorbox{cl_aya}{91}} قَالُوا لَن نَّبْرَحَ عَلَيْهِ عَكِفِينَ حَتَّى يَرْجِعَ إِلَيْنَا مُوسَى
{\tiny\colorbox{cl_aya}{92}} قَالَ يَهَرُونُ مَا مَنَعَكَ إِذْ رَأَيْتَهُمْ ضَلُّوا
{\tiny\colorbox{cl_aya}{93}} أَلَّا تَتَّبِعَنِ أَفَعَصَيْتَ أَمْرِى
{\tiny\colorbox{cl_aya}{94}} قَالَ يَبْنَؤُمَّ لَا تَأْخُذْ بِلِحْيَتِى وَلَا بِرَأْسِى إِنِّى خَشِيتُ أَن تَقُولَ فَرَّقْتَ بَيْنَ بَنِى إِسْرَءِيلَ وَلَمْ تَرْقُبْ قَوْلِى
{\tiny\colorbox{cl_aya}{95}} قَالَ فَمَا خَطْبُكَ يَسَمِرِىُّ
{\tiny\colorbox{cl_aya}{96}} قَالَ بَصُرْتُ بِمَا لَمْ يَبْصُرُوا بِهِ فَقَبَضْتُ قَبْضَةً مِّنْ أَثَرِ الرَّسُولِ فَنَبَذْتُهَا وَكَذَلِكَ سَوَّلَتْ لِى نَفْسِى
{\tiny\colorbox{cl_aya}{97}} قَالَ فَاذْهَبْ فَإِنَّ لَكَ فِى الْحَيَوةِ أَن تَقُولَ لَا مِسَاسَ وَإِنَّ لَكَ مَوْعِدًا لَّن تُخْلَفَهُ وَانظُرْ إِلَى إِلَهِكَ الَّذِى ظَلْتَ عَلَيْهِ عَاكِفًا لَّنُحَرِّقَنَّهُ ثُمَّ لَنَنسِفَنَّهُ فِى الْيَمِّ نَسْفًا
{\tiny\colorbox{cl_aya}{98}} إِنَّمَا إِلَهُكُمُ اللَّهُ الَّذِى لَا إِلَهَ إِلَّا هُوَ وَسِعَ كُلَّ شَىْءٍ عِلْمًا
{\tiny\colorbox{cl_aya}{99}} كَذَلِكَ نَقُصُّ عَلَيْكَ مِنْ أَنبَاءِ مَا قَدْ سَبَقَ وَقَدْ ءَاتَيْنَكَ مِن لَّدُنَّا ذِكْرًا
{\tiny\colorbox{cl_aya}{100}} مَّنْ أَعْرَضَ عَنْهُ فَإِنَّهُ يَحْمِلُ يَوْمَ الْقِيَمَةِ وِزْرًا
{\tiny\colorbox{cl_aya}{101}} خَلِدِينَ فِيهِ وَسَاءَ لَهُمْ يَوْمَ الْقِيَمَةِ حِمْلًا
{\tiny\colorbox{cl_aya}{102}} يَوْمَ يُنفَخُ فِى الصُّورِ وَنَحْشُرُ الْمُجْرِمِينَ يَوْمَئِذٍ زُرْقًا
{\tiny\colorbox{cl_aya}{103}} يَتَخَفَتُونَ بَيْنَهُمْ إِن لَّبِثْتُمْ إِلَّا عَشْرًا
{\tiny\colorbox{cl_aya}{104}} نَّحْنُ أَعْلَمُ بِمَا يَقُولُونَ إِذْ يَقُولُ أَمْثَلُهُمْ طَرِيقَةً إِن لَّبِثْتُمْ إِلَّا يَوْمًا
{\tiny\colorbox{cl_aya}{105}} وَيَسَْٔلُونَكَ عَنِ الْجِبَالِ فَقُلْ يَنسِفُهَا رَبِّى نَسْفًا
{\tiny\colorbox{cl_aya}{106}} فَيَذَرُهَا قَاعًا صَفْصَفًا
{\tiny\colorbox{cl_aya}{107}} لَّا تَرَى فِيهَا عِوَجًا وَلَا أَمْتًا
{\tiny\colorbox{cl_aya}{108}} يَوْمَئِذٍ يَتَّبِعُونَ الدَّاعِىَ لَا عِوَجَ لَهُ وَخَشَعَتِ الْأَصْوَاتُ لِلرَّحْمَنِ فَلَا تَسْمَعُ إِلَّا هَمْسًا
{\tiny\colorbox{cl_aya}{109}} يَوْمَئِذٍ لَّا تَنفَعُ الشَّفَعَةُ إِلَّا مَنْ أَذِنَ لَهُ الرَّحْمَنُ وَرَضِىَ لَهُ قَوْلًا
{\tiny\colorbox{cl_aya}{110}} يَعْلَمُ مَا بَيْنَ أَيْدِيهِمْ وَمَا خَلْفَهُمْ وَلَا يُحِيطُونَ بِهِ عِلْمًا
{\tiny\colorbox{cl_aya}{111}} وَعَنَتِ الْوُجُوهُ لِلْحَىِّ الْقَيُّومِ وَقَدْ خَابَ مَنْ حَمَلَ ظُلْمًا
{\tiny\colorbox{cl_aya}{112}} وَمَن يَعْمَلْ مِنَ الصَّلِحَتِ وَهُوَ مُؤْمِنٌ فَلَا يَخَافُ ظُلْمًا وَلَا هَضْمًا
{\tiny\colorbox{cl_aya}{113}} وَكَذَلِكَ أَنزَلْنَهُ قُرْءَانًا عَرَبِيًّا وَصَرَّفْنَا فِيهِ مِنَ الْوَعِيدِ لَعَلَّهُمْ يَتَّقُونَ أَوْ يُحْدِثُ لَهُمْ ذِكْرًا
{\tiny\colorbox{cl_aya}{114}} فَتَعَلَى اللَّهُ الْمَلِكُ الْحَقُّ وَلَا تَعْجَلْ بِالْقُرْءَانِ مِن قَبْلِ أَن يُقْضَى إِلَيْكَ وَحْيُهُ وَقُل رَّبِّ زِدْنِى عِلْمًا
{\tiny\colorbox{cl_aya}{115}} وَلَقَدْ عَهِدْنَا إِلَى ءَادَمَ مِن قَبْلُ فَنَسِىَ وَلَمْ نَجِدْ لَهُ عَزْمًا
{\tiny\colorbox{cl_aya}{116}} وَإِذْ قُلْنَا لِلْمَلَئِكَةِ اسْجُدُوا لِءَادَمَ فَسَجَدُوا إِلَّا إِبْلِيسَ أَبَى
{\tiny\colorbox{cl_aya}{117}} فَقُلْنَا ئََادَمُ إِنَّ هَذَا عَدُوٌّ لَّكَ وَلِزَوْجِكَ فَلَا يُخْرِجَنَّكُمَا مِنَ الْجَنَّةِ فَتَشْقَى
{\tiny\colorbox{cl_aya}{118}} إِنَّ لَكَ أَلَّا تَجُوعَ فِيهَا وَلَا تَعْرَى
{\tiny\colorbox{cl_aya}{119}} وَأَنَّكَ لَا تَظْمَؤُا فِيهَا وَلَا تَضْحَى
{\tiny\colorbox{cl_aya}{120}} فَوَسْوَسَ إِلَيْهِ الشَّيْطَنُ قَالَ ئََادَمُ هَلْ أَدُلُّكَ عَلَى شَجَرَةِ الْخُلْدِ وَمُلْكٍ لَّا يَبْلَى
{\tiny\colorbox{cl_aya}{121}} فَأَكَلَا مِنْهَا فَبَدَتْ لَهُمَا سَوْءَتُهُمَا وَطَفِقَا يَخْصِفَانِ عَلَيْهِمَا مِن وَرَقِ الْجَنَّةِ وَعَصَى ءَادَمُ رَبَّهُ فَغَوَى
{\tiny\colorbox{cl_aya}{122}} ثُمَّ اجْتَبَهُ رَبُّهُ فَتَابَ عَلَيْهِ وَهَدَى
{\tiny\colorbox{cl_aya}{123}} قَالَ اهْبِطَا مِنْهَا جَمِيعًا بَعْضُكُمْ لِبَعْضٍ عَدُوٌّ فَإِمَّا يَأْتِيَنَّكُم مِّنِّى هُدًى فَمَنِ اتَّبَعَ هُدَاىَ فَلَا يَضِلُّ وَلَا يَشْقَى
{\tiny\colorbox{cl_aya}{124}} وَمَنْ أَعْرَضَ عَن ذِكْرِى فَإِنَّ لَهُ مَعِيشَةً ضَنكًا وَنَحْشُرُهُ يَوْمَ الْقِيَمَةِ أَعْمَى
{\tiny\colorbox{cl_aya}{125}} قَالَ رَبِّ لِمَ حَشَرْتَنِى أَعْمَى وَقَدْ كُنتُ بَصِيرًا
{\tiny\colorbox{cl_aya}{126}} قَالَ كَذَلِكَ أَتَتْكَ ءَايَتُنَا فَنَسِيتَهَا وَكَذَلِكَ الْيَوْمَ تُنسَى
{\tiny\colorbox{cl_aya}{127}} وَكَذَلِكَ نَجْزِى مَنْ أَسْرَفَ وَلَمْ يُؤْمِن بَِٔايَتِ رَبِّهِ وَلَعَذَابُ الْءَاخِرَةِ أَشَدُّ وَأَبْقَى
{\tiny\colorbox{cl_aya}{128}} أَفَلَمْ يَهْدِ لَهُمْ كَمْ أَهْلَكْنَا قَبْلَهُم مِّنَ الْقُرُونِ يَمْشُونَ فِى مَسَكِنِهِمْ إِنَّ فِى ذَلِكَ لَءَايَتٍ لِّأُولِى النُّهَى
{\tiny\colorbox{cl_aya}{129}} وَلَوْلَا كَلِمَةٌ سَبَقَتْ مِن رَّبِّكَ لَكَانَ لِزَامًا وَأَجَلٌ مُّسَمًّى
{\tiny\colorbox{cl_aya}{130}} فَاصْبِرْ عَلَى مَا يَقُولُونَ وَسَبِّحْ بِحَمْدِ رَبِّكَ قَبْلَ طُلُوعِ الشَّمْسِ وَقَبْلَ غُرُوبِهَا وَمِنْ ءَانَائِ الَّيْلِ فَسَبِّحْ وَأَطْرَافَ النَّهَارِ لَعَلَّكَ تَرْضَى
{\tiny\colorbox{cl_aya}{131}} وَلَا تَمُدَّنَّ عَيْنَيْكَ إِلَى مَا مَتَّعْنَا بِهِ أَزْوَجًا مِّنْهُمْ زَهْرَةَ الْحَيَوةِ الدُّنْيَا لِنَفْتِنَهُمْ فِيهِ وَرِزْقُ رَبِّكَ خَيْرٌ وَأَبْقَى
{\tiny\colorbox{cl_aya}{132}} وَأْمُرْ أَهْلَكَ بِالصَّلَوةِ وَاصْطَبِرْ عَلَيْهَا لَا نَسَْٔلُكَ رِزْقًا نَّحْنُ نَرْزُقُكَ وَالْعَقِبَةُ لِلتَّقْوَى
{\tiny\colorbox{cl_aya}{133}} وَقَالُوا لَوْلَا يَأْتِينَا بَِٔايَةٍ مِّن رَّبِّهِ أَوَلَمْ تَأْتِهِم بَيِّنَةُ مَا فِى الصُّحُفِ الْأُولَى
{\tiny\colorbox{cl_aya}{134}} وَلَوْ أَنَّا أَهْلَكْنَهُم بِعَذَابٍ مِّن قَبْلِهِ لَقَالُوا رَبَّنَا لَوْلَا أَرْسَلْتَ إِلَيْنَا رَسُولًا فَنَتَّبِعَ ءَايَتِكَ مِن قَبْلِ أَن نَّذِلَّ وَنَخْزَى
{\tiny\colorbox{cl_aya}{135}} قُلْ كُلٌّ مُّتَرَبِّصٌ فَتَرَبَّصُوا فَسَتَعْلَمُونَ مَنْ أَصْحَبُ الصِّرَطِ السَّوِىِّ وَمَنِ اهْتَدَى
\end{document}