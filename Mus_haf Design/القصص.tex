%\documentclass[12pt,a4paper]{article}
\documentclass[20pt,a4paper]{article}
\usepackage[margin=0.5in]{geometry}
\usepackage{polyglossia}
\usepackage[dvipsnames]{xcolor}
\pagenumbering{gobble}
% This beautiful one line disable the initial spacing at the beginning of a line
\usepackage[parfill]{parskip} 
\usepackage{setspace}
\setstretch{2}

\setdefaultlanguage[numerals=maghrib]{arabic}
\newfontfamily\arabicfont[Script=Arabic]{Amiri}

\title{}
\author{}
\date{}
\definecolor{cl_page}{gray}{0.98}
\definecolor{cl_aya}{HTML}{DEEEFF}

\begin{document}
\pagecolor{cl_page}

% Start %


{\tiny\colorbox{cl_aya}{1}} طسم
{\tiny\colorbox{cl_aya}{2}} تِلْكَ ءَايَتُ الْكِتَبِ الْمُبِينِ
{\tiny\colorbox{cl_aya}{3}} نَتْلُوا عَلَيْكَ مِن نَّبَإِ مُوسَى وَفِرْعَوْنَ بِالْحَقِّ لِقَوْمٍ يُؤْمِنُونَ
{\tiny\colorbox{cl_aya}{4}} إِنَّ فِرْعَوْنَ عَلَا فِى الْأَرْضِ وَجَعَلَ أَهْلَهَا شِيَعًا يَسْتَضْعِفُ طَائِفَةً مِّنْهُمْ يُذَبِّحُ أَبْنَاءَهُمْ وَيَسْتَحْىِ نِسَاءَهُمْ إِنَّهُ كَانَ مِنَ الْمُفْسِدِينَ
{\tiny\colorbox{cl_aya}{5}} وَنُرِيدُ أَن نَّمُنَّ عَلَى الَّذِينَ اسْتُضْعِفُوا فِى الْأَرْضِ وَنَجْعَلَهُمْ أَئِمَّةً وَنَجْعَلَهُمُ الْوَرِثِينَ
{\tiny\colorbox{cl_aya}{6}} وَنُمَكِّنَ لَهُمْ فِى الْأَرْضِ وَنُرِىَ فِرْعَوْنَ وَهَمَنَ وَجُنُودَهُمَا مِنْهُم مَّا كَانُوا يَحْذَرُونَ
{\tiny\colorbox{cl_aya}{7}} وَأَوْحَيْنَا إِلَى أُمِّ مُوسَى أَنْ أَرْضِعِيهِ فَإِذَا خِفْتِ عَلَيْهِ فَأَلْقِيهِ فِى الْيَمِّ وَلَا تَخَافِى وَلَا تَحْزَنِى إِنَّا رَادُّوهُ إِلَيْكِ وَجَاعِلُوهُ مِنَ الْمُرْسَلِينَ
{\tiny\colorbox{cl_aya}{8}} فَالْتَقَطَهُ ءَالُ فِرْعَوْنَ لِيَكُونَ لَهُمْ عَدُوًّا وَحَزَنًا إِنَّ فِرْعَوْنَ وَهَمَنَ وَجُنُودَهُمَا كَانُوا خَطِِٔينَ
{\tiny\colorbox{cl_aya}{9}} وَقَالَتِ امْرَأَتُ فِرْعَوْنَ قُرَّتُ عَيْنٍ لِّى وَلَكَ لَا تَقْتُلُوهُ عَسَى أَن يَنفَعَنَا أَوْ نَتَّخِذَهُ وَلَدًا وَهُمْ لَا يَشْعُرُونَ
{\tiny\colorbox{cl_aya}{10}} وَأَصْبَحَ فُؤَادُ أُمِّ مُوسَى فَرِغًا إِن كَادَتْ لَتُبْدِى بِهِ لَوْلَا أَن رَّبَطْنَا عَلَى قَلْبِهَا لِتَكُونَ مِنَ الْمُؤْمِنِينَ
{\tiny\colorbox{cl_aya}{11}} وَقَالَتْ لِأُخْتِهِ قُصِّيهِ فَبَصُرَتْ بِهِ عَن جُنُبٍ وَهُمْ لَا يَشْعُرُونَ
{\tiny\colorbox{cl_aya}{12}} وَحَرَّمْنَا عَلَيْهِ الْمَرَاضِعَ مِن قَبْلُ فَقَالَتْ هَلْ أَدُلُّكُمْ عَلَى أَهْلِ بَيْتٍ يَكْفُلُونَهُ لَكُمْ وَهُمْ لَهُ نَصِحُونَ
{\tiny\colorbox{cl_aya}{13}} فَرَدَدْنَهُ إِلَى أُمِّهِ كَىْ تَقَرَّ عَيْنُهَا وَلَا تَحْزَنَ وَلِتَعْلَمَ أَنَّ وَعْدَ اللَّهِ حَقٌّ وَلَكِنَّ أَكْثَرَهُمْ لَا يَعْلَمُونَ
{\tiny\colorbox{cl_aya}{14}} وَلَمَّا بَلَغَ أَشُدَّهُ وَاسْتَوَى ءَاتَيْنَهُ حُكْمًا وَعِلْمًا وَكَذَلِكَ نَجْزِى الْمُحْسِنِينَ
{\tiny\colorbox{cl_aya}{15}} وَدَخَلَ الْمَدِينَةَ عَلَى حِينِ غَفْلَةٍ مِّنْ أَهْلِهَا فَوَجَدَ فِيهَا رَجُلَيْنِ يَقْتَتِلَانِ هَذَا مِن شِيعَتِهِ وَهَذَا مِنْ عَدُوِّهِ فَاسْتَغَثَهُ الَّذِى مِن شِيعَتِهِ عَلَى الَّذِى مِنْ عَدُوِّهِ فَوَكَزَهُ مُوسَى فَقَضَى عَلَيْهِ قَالَ هَذَا مِنْ عَمَلِ الشَّيْطَنِ إِنَّهُ عَدُوٌّ مُّضِلٌّ مُّبِينٌ
{\tiny\colorbox{cl_aya}{16}} قَالَ رَبِّ إِنِّى ظَلَمْتُ نَفْسِى فَاغْفِرْ لِى فَغَفَرَ لَهُ إِنَّهُ هُوَ الْغَفُورُ الرَّحِيمُ
{\tiny\colorbox{cl_aya}{17}} قَالَ رَبِّ بِمَا أَنْعَمْتَ عَلَىَّ فَلَنْ أَكُونَ ظَهِيرًا لِّلْمُجْرِمِينَ
{\tiny\colorbox{cl_aya}{18}} فَأَصْبَحَ فِى الْمَدِينَةِ خَائِفًا يَتَرَقَّبُ فَإِذَا الَّذِى اسْتَنصَرَهُ بِالْأَمْسِ يَسْتَصْرِخُهُ قَالَ لَهُ مُوسَى إِنَّكَ لَغَوِىٌّ مُّبِينٌ
{\tiny\colorbox{cl_aya}{19}} فَلَمَّا أَنْ أَرَادَ أَن يَبْطِشَ بِالَّذِى هُوَ عَدُوٌّ لَّهُمَا قَالَ يَمُوسَى أَتُرِيدُ أَن تَقْتُلَنِى كَمَا قَتَلْتَ نَفْسًا بِالْأَمْسِ إِن تُرِيدُ إِلَّا أَن تَكُونَ جَبَّارًا فِى الْأَرْضِ وَمَا تُرِيدُ أَن تَكُونَ مِنَ الْمُصْلِحِينَ
{\tiny\colorbox{cl_aya}{20}} وَجَاءَ رَجُلٌ مِّنْ أَقْصَا الْمَدِينَةِ يَسْعَى قَالَ يَمُوسَى إِنَّ الْمَلَأَ يَأْتَمِرُونَ بِكَ لِيَقْتُلُوكَ فَاخْرُجْ إِنِّى لَكَ مِنَ النَّصِحِينَ
{\tiny\colorbox{cl_aya}{21}} فَخَرَجَ مِنْهَا خَائِفًا يَتَرَقَّبُ قَالَ رَبِّ نَجِّنِى مِنَ الْقَوْمِ الظَّلِمِينَ
{\tiny\colorbox{cl_aya}{22}} وَلَمَّا تَوَجَّهَ تِلْقَاءَ مَدْيَنَ قَالَ عَسَى رَبِّى أَن يَهْدِيَنِى سَوَاءَ السَّبِيلِ
{\tiny\colorbox{cl_aya}{23}} وَلَمَّا وَرَدَ مَاءَ مَدْيَنَ وَجَدَ عَلَيْهِ أُمَّةً مِّنَ النَّاسِ يَسْقُونَ وَوَجَدَ مِن دُونِهِمُ امْرَأَتَيْنِ تَذُودَانِ قَالَ مَا خَطْبُكُمَا قَالَتَا لَا نَسْقِى حَتَّى يُصْدِرَ الرِّعَاءُ وَأَبُونَا شَيْخٌ كَبِيرٌ
{\tiny\colorbox{cl_aya}{24}} فَسَقَى لَهُمَا ثُمَّ تَوَلَّى إِلَى الظِّلِّ فَقَالَ رَبِّ إِنِّى لِمَا أَنزَلْتَ إِلَىَّ مِنْ خَيْرٍ فَقِيرٌ
{\tiny\colorbox{cl_aya}{25}} فَجَاءَتْهُ إِحْدَىهُمَا تَمْشِى عَلَى اسْتِحْيَاءٍ قَالَتْ إِنَّ أَبِى يَدْعُوكَ لِيَجْزِيَكَ أَجْرَ مَا سَقَيْتَ لَنَا فَلَمَّا جَاءَهُ وَقَصَّ عَلَيْهِ الْقَصَصَ قَالَ لَا تَخَفْ نَجَوْتَ مِنَ الْقَوْمِ الظَّلِمِينَ
{\tiny\colorbox{cl_aya}{26}} قَالَتْ إِحْدَىهُمَا يَأَبَتِ اسْتَْٔجِرْهُ إِنَّ خَيْرَ مَنِ اسْتَْٔجَرْتَ الْقَوِىُّ الْأَمِينُ
{\tiny\colorbox{cl_aya}{27}} قَالَ إِنِّى أُرِيدُ أَنْ أُنكِحَكَ إِحْدَى ابْنَتَىَّ هَتَيْنِ عَلَى أَن تَأْجُرَنِى ثَمَنِىَ حِجَجٍ فَإِنْ أَتْمَمْتَ عَشْرًا فَمِنْ عِندِكَ وَمَا أُرِيدُ أَنْ أَشُقَّ عَلَيْكَ سَتَجِدُنِى إِن شَاءَ اللَّهُ مِنَ الصَّلِحِينَ
{\tiny\colorbox{cl_aya}{28}} قَالَ ذَلِكَ بَيْنِى وَبَيْنَكَ أَيَّمَا الْأَجَلَيْنِ قَضَيْتُ فَلَا عُدْوَنَ عَلَىَّ وَاللَّهُ عَلَى مَا نَقُولُ وَكِيلٌ
{\tiny\colorbox{cl_aya}{29}} فَلَمَّا قَضَى مُوسَى الْأَجَلَ وَسَارَ بِأَهْلِهِ ءَانَسَ مِن جَانِبِ الطُّورِ نَارًا قَالَ لِأَهْلِهِ امْكُثُوا إِنِّى ءَانَسْتُ نَارًا لَّعَلِّى ءَاتِيكُم مِّنْهَا بِخَبَرٍ أَوْ جَذْوَةٍ مِّنَ النَّارِ لَعَلَّكُمْ تَصْطَلُونَ
{\tiny\colorbox{cl_aya}{30}} فَلَمَّا أَتَىهَا نُودِىَ مِن شَطِئِ الْوَادِ الْأَيْمَنِ فِى الْبُقْعَةِ الْمُبَرَكَةِ مِنَ الشَّجَرَةِ أَن يَمُوسَى إِنِّى أَنَا اللَّهُ رَبُّ الْعَلَمِينَ
{\tiny\colorbox{cl_aya}{31}} وَأَنْ أَلْقِ عَصَاكَ فَلَمَّا رَءَاهَا تَهْتَزُّ كَأَنَّهَا جَانٌّ وَلَّى مُدْبِرًا وَلَمْ يُعَقِّبْ يَمُوسَى أَقْبِلْ وَلَا تَخَفْ إِنَّكَ مِنَ الْءَامِنِينَ
{\tiny\colorbox{cl_aya}{32}} اسْلُكْ يَدَكَ فِى جَيْبِكَ تَخْرُجْ بَيْضَاءَ مِنْ غَيْرِ سُوءٍ وَاضْمُمْ إِلَيْكَ جَنَاحَكَ مِنَ الرَّهْبِ فَذَنِكَ بُرْهَنَانِ مِن رَّبِّكَ إِلَى فِرْعَوْنَ وَمَلَإِيهِ إِنَّهُمْ كَانُوا قَوْمًا فَسِقِينَ
{\tiny\colorbox{cl_aya}{33}} قَالَ رَبِّ إِنِّى قَتَلْتُ مِنْهُمْ نَفْسًا فَأَخَافُ أَن يَقْتُلُونِ
{\tiny\colorbox{cl_aya}{34}} وَأَخِى هَرُونُ هُوَ أَفْصَحُ مِنِّى لِسَانًا فَأَرْسِلْهُ مَعِىَ رِدْءًا يُصَدِّقُنِى إِنِّى أَخَافُ أَن يُكَذِّبُونِ
{\tiny\colorbox{cl_aya}{35}} قَالَ سَنَشُدُّ عَضُدَكَ بِأَخِيكَ وَنَجْعَلُ لَكُمَا سُلْطَنًا فَلَا يَصِلُونَ إِلَيْكُمَا بَِٔايَتِنَا أَنتُمَا وَمَنِ اتَّبَعَكُمَا الْغَلِبُونَ
{\tiny\colorbox{cl_aya}{36}} فَلَمَّا جَاءَهُم مُّوسَى بَِٔايَتِنَا بَيِّنَتٍ قَالُوا مَا هَذَا إِلَّا سِحْرٌ مُّفْتَرًى وَمَا سَمِعْنَا بِهَذَا فِى ءَابَائِنَا الْأَوَّلِينَ
{\tiny\colorbox{cl_aya}{37}} وَقَالَ مُوسَى رَبِّى أَعْلَمُ بِمَن جَاءَ بِالْهُدَى مِنْ عِندِهِ وَمَن تَكُونُ لَهُ عَقِبَةُ الدَّارِ إِنَّهُ لَا يُفْلِحُ الظَّلِمُونَ
{\tiny\colorbox{cl_aya}{38}} وَقَالَ فِرْعَوْنُ يَأَيُّهَا الْمَلَأُ مَا عَلِمْتُ لَكُم مِّنْ إِلَهٍ غَيْرِى فَأَوْقِدْ لِى يَهَمَنُ عَلَى الطِّينِ فَاجْعَل لِّى صَرْحًا لَّعَلِّى أَطَّلِعُ إِلَى إِلَهِ مُوسَى وَإِنِّى لَأَظُنُّهُ مِنَ الْكَذِبِينَ
{\tiny\colorbox{cl_aya}{39}} وَاسْتَكْبَرَ هُوَ وَجُنُودُهُ فِى الْأَرْضِ بِغَيْرِ الْحَقِّ وَظَنُّوا أَنَّهُمْ إِلَيْنَا لَا يُرْجَعُونَ
{\tiny\colorbox{cl_aya}{40}} فَأَخَذْنَهُ وَجُنُودَهُ فَنَبَذْنَهُمْ فِى الْيَمِّ فَانظُرْ كَيْفَ كَانَ عَقِبَةُ الظَّلِمِينَ
{\tiny\colorbox{cl_aya}{41}} وَجَعَلْنَهُمْ أَئِمَّةً يَدْعُونَ إِلَى النَّارِ وَيَوْمَ الْقِيَمَةِ لَا يُنصَرُونَ
{\tiny\colorbox{cl_aya}{42}} وَأَتْبَعْنَهُمْ فِى هَذِهِ الدُّنْيَا لَعْنَةً وَيَوْمَ الْقِيَمَةِ هُم مِّنَ الْمَقْبُوحِينَ
{\tiny\colorbox{cl_aya}{43}} وَلَقَدْ ءَاتَيْنَا مُوسَى الْكِتَبَ مِن بَعْدِ مَا أَهْلَكْنَا الْقُرُونَ الْأُولَى بَصَائِرَ لِلنَّاسِ وَهُدًى وَرَحْمَةً لَّعَلَّهُمْ يَتَذَكَّرُونَ
{\tiny\colorbox{cl_aya}{44}} وَمَا كُنتَ بِجَانِبِ الْغَرْبِىِّ إِذْ قَضَيْنَا إِلَى مُوسَى الْأَمْرَ وَمَا كُنتَ مِنَ الشَّهِدِينَ
{\tiny\colorbox{cl_aya}{45}} وَلَكِنَّا أَنشَأْنَا قُرُونًا فَتَطَاوَلَ عَلَيْهِمُ الْعُمُرُ وَمَا كُنتَ ثَاوِيًا فِى أَهْلِ مَدْيَنَ تَتْلُوا عَلَيْهِمْ ءَايَتِنَا وَلَكِنَّا كُنَّا مُرْسِلِينَ
{\tiny\colorbox{cl_aya}{46}} وَمَا كُنتَ بِجَانِبِ الطُّورِ إِذْ نَادَيْنَا وَلَكِن رَّحْمَةً مِّن رَّبِّكَ لِتُنذِرَ قَوْمًا مَّا أَتَىهُم مِّن نَّذِيرٍ مِّن قَبْلِكَ لَعَلَّهُمْ يَتَذَكَّرُونَ
{\tiny\colorbox{cl_aya}{47}} وَلَوْلَا أَن تُصِيبَهُم مُّصِيبَةٌ بِمَا قَدَّمَتْ أَيْدِيهِمْ فَيَقُولُوا رَبَّنَا لَوْلَا أَرْسَلْتَ إِلَيْنَا رَسُولًا فَنَتَّبِعَ ءَايَتِكَ وَنَكُونَ مِنَ الْمُؤْمِنِينَ
{\tiny\colorbox{cl_aya}{48}} فَلَمَّا جَاءَهُمُ الْحَقُّ مِنْ عِندِنَا قَالُوا لَوْلَا أُوتِىَ مِثْلَ مَا أُوتِىَ مُوسَى أَوَلَمْ يَكْفُرُوا بِمَا أُوتِىَ مُوسَى مِن قَبْلُ قَالُوا سِحْرَانِ تَظَهَرَا وَقَالُوا إِنَّا بِكُلٍّ كَفِرُونَ
{\tiny\colorbox{cl_aya}{49}} قُلْ فَأْتُوا بِكِتَبٍ مِّنْ عِندِ اللَّهِ هُوَ أَهْدَى مِنْهُمَا أَتَّبِعْهُ إِن كُنتُمْ صَدِقِينَ
{\tiny\colorbox{cl_aya}{50}} فَإِن لَّمْ يَسْتَجِيبُوا لَكَ فَاعْلَمْ أَنَّمَا يَتَّبِعُونَ أَهْوَاءَهُمْ وَمَنْ أَضَلُّ مِمَّنِ اتَّبَعَ هَوَىهُ بِغَيْرِ هُدًى مِّنَ اللَّهِ إِنَّ اللَّهَ لَا يَهْدِى الْقَوْمَ الظَّلِمِينَ
{\tiny\colorbox{cl_aya}{51}} وَلَقَدْ وَصَّلْنَا لَهُمُ الْقَوْلَ لَعَلَّهُمْ يَتَذَكَّرُونَ
{\tiny\colorbox{cl_aya}{52}} الَّذِينَ ءَاتَيْنَهُمُ الْكِتَبَ مِن قَبْلِهِ هُم بِهِ يُؤْمِنُونَ
{\tiny\colorbox{cl_aya}{53}} وَإِذَا يُتْلَى عَلَيْهِمْ قَالُوا ءَامَنَّا بِهِ إِنَّهُ الْحَقُّ مِن رَّبِّنَا إِنَّا كُنَّا مِن قَبْلِهِ مُسْلِمِينَ
{\tiny\colorbox{cl_aya}{54}} أُولَئِكَ يُؤْتَوْنَ أَجْرَهُم مَّرَّتَيْنِ بِمَا صَبَرُوا وَيَدْرَءُونَ بِالْحَسَنَةِ السَّيِّئَةَ وَمِمَّا رَزَقْنَهُمْ يُنفِقُونَ
{\tiny\colorbox{cl_aya}{55}} وَإِذَا سَمِعُوا اللَّغْوَ أَعْرَضُوا عَنْهُ وَقَالُوا لَنَا أَعْمَلُنَا وَلَكُمْ أَعْمَلُكُمْ سَلَمٌ عَلَيْكُمْ لَا نَبْتَغِى الْجَهِلِينَ
{\tiny\colorbox{cl_aya}{56}} إِنَّكَ لَا تَهْدِى مَنْ أَحْبَبْتَ وَلَكِنَّ اللَّهَ يَهْدِى مَن يَشَاءُ وَهُوَ أَعْلَمُ بِالْمُهْتَدِينَ
{\tiny\colorbox{cl_aya}{57}} وَقَالُوا إِن نَّتَّبِعِ الْهُدَى مَعَكَ نُتَخَطَّفْ مِنْ أَرْضِنَا أَوَلَمْ نُمَكِّن لَّهُمْ حَرَمًا ءَامِنًا يُجْبَى إِلَيْهِ ثَمَرَتُ كُلِّ شَىْءٍ رِّزْقًا مِّن لَّدُنَّا وَلَكِنَّ أَكْثَرَهُمْ لَا يَعْلَمُونَ
{\tiny\colorbox{cl_aya}{58}} وَكَمْ أَهْلَكْنَا مِن قَرْيَةٍ بَطِرَتْ مَعِيشَتَهَا فَتِلْكَ مَسَكِنُهُمْ لَمْ تُسْكَن مِّن بَعْدِهِمْ إِلَّا قَلِيلًا وَكُنَّا نَحْنُ الْوَرِثِينَ
{\tiny\colorbox{cl_aya}{59}} وَمَا كَانَ رَبُّكَ مُهْلِكَ الْقُرَى حَتَّى يَبْعَثَ فِى أُمِّهَا رَسُولًا يَتْلُوا عَلَيْهِمْ ءَايَتِنَا وَمَا كُنَّا مُهْلِكِى الْقُرَى إِلَّا وَأَهْلُهَا ظَلِمُونَ
{\tiny\colorbox{cl_aya}{60}} وَمَا أُوتِيتُم مِّن شَىْءٍ فَمَتَعُ الْحَيَوةِ الدُّنْيَا وَزِينَتُهَا وَمَا عِندَ اللَّهِ خَيْرٌ وَأَبْقَى أَفَلَا تَعْقِلُونَ
{\tiny\colorbox{cl_aya}{61}} أَفَمَن وَعَدْنَهُ وَعْدًا حَسَنًا فَهُوَ لَقِيهِ كَمَن مَّتَّعْنَهُ مَتَعَ الْحَيَوةِ الدُّنْيَا ثُمَّ هُوَ يَوْمَ الْقِيَمَةِ مِنَ الْمُحْضَرِينَ
{\tiny\colorbox{cl_aya}{62}} وَيَوْمَ يُنَادِيهِمْ فَيَقُولُ أَيْنَ شُرَكَاءِىَ الَّذِينَ كُنتُمْ تَزْعُمُونَ
{\tiny\colorbox{cl_aya}{63}} قَالَ الَّذِينَ حَقَّ عَلَيْهِمُ الْقَوْلُ رَبَّنَا هَؤُلَاءِ الَّذِينَ أَغْوَيْنَا أَغْوَيْنَهُمْ كَمَا غَوَيْنَا تَبَرَّأْنَا إِلَيْكَ مَا كَانُوا إِيَّانَا يَعْبُدُونَ
{\tiny\colorbox{cl_aya}{64}} وَقِيلَ ادْعُوا شُرَكَاءَكُمْ فَدَعَوْهُمْ فَلَمْ يَسْتَجِيبُوا لَهُمْ وَرَأَوُا الْعَذَابَ لَوْ أَنَّهُمْ كَانُوا يَهْتَدُونَ
{\tiny\colorbox{cl_aya}{65}} وَيَوْمَ يُنَادِيهِمْ فَيَقُولُ مَاذَا أَجَبْتُمُ الْمُرْسَلِينَ
{\tiny\colorbox{cl_aya}{66}} فَعَمِيَتْ عَلَيْهِمُ الْأَنبَاءُ يَوْمَئِذٍ فَهُمْ لَا يَتَسَاءَلُونَ
{\tiny\colorbox{cl_aya}{67}} فَأَمَّا مَن تَابَ وَءَامَنَ وَعَمِلَ صَلِحًا فَعَسَى أَن يَكُونَ مِنَ الْمُفْلِحِينَ
{\tiny\colorbox{cl_aya}{68}} وَرَبُّكَ يَخْلُقُ مَا يَشَاءُ وَيَخْتَارُ مَا كَانَ لَهُمُ الْخِيَرَةُ سُبْحَنَ اللَّهِ وَتَعَلَى عَمَّا يُشْرِكُونَ
{\tiny\colorbox{cl_aya}{69}} وَرَبُّكَ يَعْلَمُ مَا تُكِنُّ صُدُورُهُمْ وَمَا يُعْلِنُونَ
{\tiny\colorbox{cl_aya}{70}} وَهُوَ اللَّهُ لَا إِلَهَ إِلَّا هُوَ لَهُ الْحَمْدُ فِى الْأُولَى وَالْءَاخِرَةِ وَلَهُ الْحُكْمُ وَإِلَيْهِ تُرْجَعُونَ
{\tiny\colorbox{cl_aya}{71}} قُلْ أَرَءَيْتُمْ إِن جَعَلَ اللَّهُ عَلَيْكُمُ الَّيْلَ سَرْمَدًا إِلَى يَوْمِ الْقِيَمَةِ مَنْ إِلَهٌ غَيْرُ اللَّهِ يَأْتِيكُم بِضِيَاءٍ أَفَلَا تَسْمَعُونَ
{\tiny\colorbox{cl_aya}{72}} قُلْ أَرَءَيْتُمْ إِن جَعَلَ اللَّهُ عَلَيْكُمُ النَّهَارَ سَرْمَدًا إِلَى يَوْمِ الْقِيَمَةِ مَنْ إِلَهٌ غَيْرُ اللَّهِ يَأْتِيكُم بِلَيْلٍ تَسْكُنُونَ فِيهِ أَفَلَا تُبْصِرُونَ
{\tiny\colorbox{cl_aya}{73}} وَمِن رَّحْمَتِهِ جَعَلَ لَكُمُ الَّيْلَ وَالنَّهَارَ لِتَسْكُنُوا فِيهِ وَلِتَبْتَغُوا مِن فَضْلِهِ وَلَعَلَّكُمْ تَشْكُرُونَ
{\tiny\colorbox{cl_aya}{74}} وَيَوْمَ يُنَادِيهِمْ فَيَقُولُ أَيْنَ شُرَكَاءِىَ الَّذِينَ كُنتُمْ تَزْعُمُونَ
{\tiny\colorbox{cl_aya}{75}} وَنَزَعْنَا مِن كُلِّ أُمَّةٍ شَهِيدًا فَقُلْنَا هَاتُوا بُرْهَنَكُمْ فَعَلِمُوا أَنَّ الْحَقَّ لِلَّهِ وَضَلَّ عَنْهُم مَّا كَانُوا يَفْتَرُونَ
{\tiny\colorbox{cl_aya}{76}} إِنَّ قَرُونَ كَانَ مِن قَوْمِ مُوسَى فَبَغَى عَلَيْهِمْ وَءَاتَيْنَهُ مِنَ الْكُنُوزِ مَا إِنَّ مَفَاتِحَهُ لَتَنُوأُ بِالْعُصْبَةِ أُولِى الْقُوَّةِ إِذْ قَالَ لَهُ قَوْمُهُ لَا تَفْرَحْ إِنَّ اللَّهَ لَا يُحِبُّ الْفَرِحِينَ
{\tiny\colorbox{cl_aya}{77}} وَابْتَغِ فِيمَا ءَاتَىكَ اللَّهُ الدَّارَ الْءَاخِرَةَ وَلَا تَنسَ نَصِيبَكَ مِنَ الدُّنْيَا وَأَحْسِن كَمَا أَحْسَنَ اللَّهُ إِلَيْكَ وَلَا تَبْغِ الْفَسَادَ فِى الْأَرْضِ إِنَّ اللَّهَ لَا يُحِبُّ الْمُفْسِدِينَ
{\tiny\colorbox{cl_aya}{78}} قَالَ إِنَّمَا أُوتِيتُهُ عَلَى عِلْمٍ عِندِى أَوَلَمْ يَعْلَمْ أَنَّ اللَّهَ قَدْ أَهْلَكَ مِن قَبْلِهِ مِنَ الْقُرُونِ مَنْ هُوَ أَشَدُّ مِنْهُ قُوَّةً وَأَكْثَرُ جَمْعًا وَلَا يُسَْٔلُ عَن ذُنُوبِهِمُ الْمُجْرِمُونَ
{\tiny\colorbox{cl_aya}{79}} فَخَرَجَ عَلَى قَوْمِهِ فِى زِينَتِهِ قَالَ الَّذِينَ يُرِيدُونَ الْحَيَوةَ الدُّنْيَا يَلَيْتَ لَنَا مِثْلَ مَا أُوتِىَ قَرُونُ إِنَّهُ لَذُو حَظٍّ عَظِيمٍ
{\tiny\colorbox{cl_aya}{80}} وَقَالَ الَّذِينَ أُوتُوا الْعِلْمَ وَيْلَكُمْ ثَوَابُ اللَّهِ خَيْرٌ لِّمَنْ ءَامَنَ وَعَمِلَ صَلِحًا وَلَا يُلَقَّىهَا إِلَّا الصَّبِرُونَ
{\tiny\colorbox{cl_aya}{81}} فَخَسَفْنَا بِهِ وَبِدَارِهِ الْأَرْضَ فَمَا كَانَ لَهُ مِن فِئَةٍ يَنصُرُونَهُ مِن دُونِ اللَّهِ وَمَا كَانَ مِنَ الْمُنتَصِرِينَ
{\tiny\colorbox{cl_aya}{82}} وَأَصْبَحَ الَّذِينَ تَمَنَّوْا مَكَانَهُ بِالْأَمْسِ يَقُولُونَ وَيْكَأَنَّ اللَّهَ يَبْسُطُ الرِّزْقَ لِمَن يَشَاءُ مِنْ عِبَادِهِ وَيَقْدِرُ لَوْلَا أَن مَّنَّ اللَّهُ عَلَيْنَا لَخَسَفَ بِنَا وَيْكَأَنَّهُ لَا يُفْلِحُ الْكَفِرُونَ
{\tiny\colorbox{cl_aya}{83}} تِلْكَ الدَّارُ الْءَاخِرَةُ نَجْعَلُهَا لِلَّذِينَ لَا يُرِيدُونَ عُلُوًّا فِى الْأَرْضِ وَلَا فَسَادًا وَالْعَقِبَةُ لِلْمُتَّقِينَ
{\tiny\colorbox{cl_aya}{84}} مَن جَاءَ بِالْحَسَنَةِ فَلَهُ خَيْرٌ مِّنْهَا وَمَن جَاءَ بِالسَّيِّئَةِ فَلَا يُجْزَى الَّذِينَ عَمِلُوا السَّئَِّاتِ إِلَّا مَا كَانُوا يَعْمَلُونَ
{\tiny\colorbox{cl_aya}{85}} إِنَّ الَّذِى فَرَضَ عَلَيْكَ الْقُرْءَانَ لَرَادُّكَ إِلَى مَعَادٍ قُل رَّبِّى أَعْلَمُ مَن جَاءَ بِالْهُدَى وَمَنْ هُوَ فِى ضَلَلٍ مُّبِينٍ
{\tiny\colorbox{cl_aya}{86}} وَمَا كُنتَ تَرْجُوا أَن يُلْقَى إِلَيْكَ الْكِتَبُ إِلَّا رَحْمَةً مِّن رَّبِّكَ فَلَا تَكُونَنَّ ظَهِيرًا لِّلْكَفِرِينَ
{\tiny\colorbox{cl_aya}{87}} وَلَا يَصُدُّنَّكَ عَنْ ءَايَتِ اللَّهِ بَعْدَ إِذْ أُنزِلَتْ إِلَيْكَ وَادْعُ إِلَى رَبِّكَ وَلَا تَكُونَنَّ مِنَ الْمُشْرِكِينَ
{\tiny\colorbox{cl_aya}{88}} وَلَا تَدْعُ مَعَ اللَّهِ إِلَهًا ءَاخَرَ لَا إِلَهَ إِلَّا هُوَ كُلُّ شَىْءٍ هَالِكٌ إِلَّا وَجْهَهُ لَهُ الْحُكْمُ وَإِلَيْهِ تُرْجَعُونَ
\end{document}