%\documentclass[12pt,a4paper]{article}
\documentclass[20pt,a4paper]{article}
\usepackage[margin=0.5in]{geometry}
\usepackage{polyglossia}
\usepackage[dvipsnames]{xcolor}
\pagenumbering{gobble}
% This beautiful one line disable the initial spacing at the beginning of a line
\usepackage[parfill]{parskip} 
\usepackage{setspace}
\setstretch{2}

\setdefaultlanguage[numerals=maghrib]{arabic}
\newfontfamily\arabicfont[Script=Arabic]{Amiri}

\title{}
\author{}
\date{}
\definecolor{cl_page}{gray}{0.98}
\definecolor{cl_aya}{HTML}{DEEEFF}

\begin{document}
\pagecolor{cl_page}

% Start %


{\tiny\colorbox{cl_aya}{1}} اقْتَرَبَ لِلنَّاسِ حِسَابُهُمْ وَهُمْ فِى غَفْلَةٍ مُّعْرِضُونَ
{\tiny\colorbox{cl_aya}{2}} مَا يَأْتِيهِم مِّن ذِكْرٍ مِّن رَّبِّهِم مُّحْدَثٍ إِلَّا اسْتَمَعُوهُ وَهُمْ يَلْعَبُونَ
{\tiny\colorbox{cl_aya}{3}} لَاهِيَةً قُلُوبُهُمْ وَأَسَرُّوا النَّجْوَى الَّذِينَ ظَلَمُوا هَلْ هَذَا إِلَّا بَشَرٌ مِّثْلُكُمْ أَفَتَأْتُونَ السِّحْرَ وَأَنتُمْ تُبْصِرُونَ
{\tiny\colorbox{cl_aya}{4}} قَالَ رَبِّى يَعْلَمُ الْقَوْلَ فِى السَّمَاءِ وَالْأَرْضِ وَهُوَ السَّمِيعُ الْعَلِيمُ
{\tiny\colorbox{cl_aya}{5}} بَلْ قَالُوا أَضْغَثُ أَحْلَمٍ بَلِ افْتَرَىهُ بَلْ هُوَ شَاعِرٌ فَلْيَأْتِنَا بَِٔايَةٍ كَمَا أُرْسِلَ الْأَوَّلُونَ
{\tiny\colorbox{cl_aya}{6}} مَا ءَامَنَتْ قَبْلَهُم مِّن قَرْيَةٍ أَهْلَكْنَهَا أَفَهُمْ يُؤْمِنُونَ
{\tiny\colorbox{cl_aya}{7}} وَمَا أَرْسَلْنَا قَبْلَكَ إِلَّا رِجَالًا نُّوحِى إِلَيْهِمْ فَسَْٔلُوا أَهْلَ الذِّكْرِ إِن كُنتُمْ لَا تَعْلَمُونَ
{\tiny\colorbox{cl_aya}{8}} وَمَا جَعَلْنَهُمْ جَسَدًا لَّا يَأْكُلُونَ الطَّعَامَ وَمَا كَانُوا خَلِدِينَ
{\tiny\colorbox{cl_aya}{9}} ثُمَّ صَدَقْنَهُمُ الْوَعْدَ فَأَنجَيْنَهُمْ وَمَن نَّشَاءُ وَأَهْلَكْنَا الْمُسْرِفِينَ
{\tiny\colorbox{cl_aya}{10}} لَقَدْ أَنزَلْنَا إِلَيْكُمْ كِتَبًا فِيهِ ذِكْرُكُمْ أَفَلَا تَعْقِلُونَ
{\tiny\colorbox{cl_aya}{11}} وَكَمْ قَصَمْنَا مِن قَرْيَةٍ كَانَتْ ظَالِمَةً وَأَنشَأْنَا بَعْدَهَا قَوْمًا ءَاخَرِينَ
{\tiny\colorbox{cl_aya}{12}} فَلَمَّا أَحَسُّوا بَأْسَنَا إِذَا هُم مِّنْهَا يَرْكُضُونَ
{\tiny\colorbox{cl_aya}{13}} لَا تَرْكُضُوا وَارْجِعُوا إِلَى مَا أُتْرِفْتُمْ فِيهِ وَمَسَكِنِكُمْ لَعَلَّكُمْ تُسَْٔلُونَ
{\tiny\colorbox{cl_aya}{14}} قَالُوا يَوَيْلَنَا إِنَّا كُنَّا ظَلِمِينَ
{\tiny\colorbox{cl_aya}{15}} فَمَا زَالَت تِّلْكَ دَعْوَىهُمْ حَتَّى جَعَلْنَهُمْ حَصِيدًا خَمِدِينَ
{\tiny\colorbox{cl_aya}{16}} وَمَا خَلَقْنَا السَّمَاءَ وَالْأَرْضَ وَمَا بَيْنَهُمَا لَعِبِينَ
{\tiny\colorbox{cl_aya}{17}} لَوْ أَرَدْنَا أَن نَّتَّخِذَ لَهْوًا لَّاتَّخَذْنَهُ مِن لَّدُنَّا إِن كُنَّا فَعِلِينَ
{\tiny\colorbox{cl_aya}{18}} بَلْ نَقْذِفُ بِالْحَقِّ عَلَى الْبَطِلِ فَيَدْمَغُهُ فَإِذَا هُوَ زَاهِقٌ وَلَكُمُ الْوَيْلُ مِمَّا تَصِفُونَ
{\tiny\colorbox{cl_aya}{19}} وَلَهُ مَن فِى السَّمَوَتِ وَالْأَرْضِ وَمَنْ عِندَهُ لَا يَسْتَكْبِرُونَ عَنْ عِبَادَتِهِ وَلَا يَسْتَحْسِرُونَ
{\tiny\colorbox{cl_aya}{20}} يُسَبِّحُونَ الَّيْلَ وَالنَّهَارَ لَا يَفْتُرُونَ
{\tiny\colorbox{cl_aya}{21}} أَمِ اتَّخَذُوا ءَالِهَةً مِّنَ الْأَرْضِ هُمْ يُنشِرُونَ
{\tiny\colorbox{cl_aya}{22}} لَوْ كَانَ فِيهِمَا ءَالِهَةٌ إِلَّا اللَّهُ لَفَسَدَتَا فَسُبْحَنَ اللَّهِ رَبِّ الْعَرْشِ عَمَّا يَصِفُونَ
{\tiny\colorbox{cl_aya}{23}} لَا يُسَْٔلُ عَمَّا يَفْعَلُ وَهُمْ يُسَْٔلُونَ
{\tiny\colorbox{cl_aya}{24}} أَمِ اتَّخَذُوا مِن دُونِهِ ءَالِهَةً قُلْ هَاتُوا بُرْهَنَكُمْ هَذَا ذِكْرُ مَن مَّعِىَ وَذِكْرُ مَن قَبْلِى بَلْ أَكْثَرُهُمْ لَا يَعْلَمُونَ الْحَقَّ فَهُم مُّعْرِضُونَ
{\tiny\colorbox{cl_aya}{25}} وَمَا أَرْسَلْنَا مِن قَبْلِكَ مِن رَّسُولٍ إِلَّا نُوحِى إِلَيْهِ أَنَّهُ لَا إِلَهَ إِلَّا أَنَا فَاعْبُدُونِ
{\tiny\colorbox{cl_aya}{26}} وَقَالُوا اتَّخَذَ الرَّحْمَنُ وَلَدًا سُبْحَنَهُ بَلْ عِبَادٌ مُّكْرَمُونَ
{\tiny\colorbox{cl_aya}{27}} لَا يَسْبِقُونَهُ بِالْقَوْلِ وَهُم بِأَمْرِهِ يَعْمَلُونَ
{\tiny\colorbox{cl_aya}{28}} يَعْلَمُ مَا بَيْنَ أَيْدِيهِمْ وَمَا خَلْفَهُمْ وَلَا يَشْفَعُونَ إِلَّا لِمَنِ ارْتَضَى وَهُم مِّنْ خَشْيَتِهِ مُشْفِقُونَ
{\tiny\colorbox{cl_aya}{29}} وَمَن يَقُلْ مِنْهُمْ إِنِّى إِلَهٌ مِّن دُونِهِ فَذَلِكَ نَجْزِيهِ جَهَنَّمَ كَذَلِكَ نَجْزِى الظَّلِمِينَ
{\tiny\colorbox{cl_aya}{30}} أَوَلَمْ يَرَ الَّذِينَ كَفَرُوا أَنَّ السَّمَوَتِ وَالْأَرْضَ كَانَتَا رَتْقًا فَفَتَقْنَهُمَا وَجَعَلْنَا مِنَ الْمَاءِ كُلَّ شَىْءٍ حَىٍّ أَفَلَا يُؤْمِنُونَ
{\tiny\colorbox{cl_aya}{31}} وَجَعَلْنَا فِى الْأَرْضِ رَوَسِىَ أَن تَمِيدَ بِهِمْ وَجَعَلْنَا فِيهَا فِجَاجًا سُبُلًا لَّعَلَّهُمْ يَهْتَدُونَ
{\tiny\colorbox{cl_aya}{32}} وَجَعَلْنَا السَّمَاءَ سَقْفًا مَّحْفُوظًا وَهُمْ عَنْ ءَايَتِهَا مُعْرِضُونَ
{\tiny\colorbox{cl_aya}{33}} وَهُوَ الَّذِى خَلَقَ الَّيْلَ وَالنَّهَارَ وَالشَّمْسَ وَالْقَمَرَ كُلٌّ فِى فَلَكٍ يَسْبَحُونَ
{\tiny\colorbox{cl_aya}{34}} وَمَا جَعَلْنَا لِبَشَرٍ مِّن قَبْلِكَ الْخُلْدَ أَفَإِين مِّتَّ فَهُمُ الْخَلِدُونَ
{\tiny\colorbox{cl_aya}{35}} كُلُّ نَفْسٍ ذَائِقَةُ الْمَوْتِ وَنَبْلُوكُم بِالشَّرِّ وَالْخَيْرِ فِتْنَةً وَإِلَيْنَا تُرْجَعُونَ
{\tiny\colorbox{cl_aya}{36}} وَإِذَا رَءَاكَ الَّذِينَ كَفَرُوا إِن يَتَّخِذُونَكَ إِلَّا هُزُوًا أَهَذَا الَّذِى يَذْكُرُ ءَالِهَتَكُمْ وَهُم بِذِكْرِ الرَّحْمَنِ هُمْ كَفِرُونَ
{\tiny\colorbox{cl_aya}{37}} خُلِقَ الْإِنسَنُ مِنْ عَجَلٍ سَأُورِيكُمْ ءَايَتِى فَلَا تَسْتَعْجِلُونِ
{\tiny\colorbox{cl_aya}{38}} وَيَقُولُونَ مَتَى هَذَا الْوَعْدُ إِن كُنتُمْ صَدِقِينَ
{\tiny\colorbox{cl_aya}{39}} لَوْ يَعْلَمُ الَّذِينَ كَفَرُوا حِينَ لَا يَكُفُّونَ عَن وُجُوهِهِمُ النَّارَ وَلَا عَن ظُهُورِهِمْ وَلَا هُمْ يُنصَرُونَ
{\tiny\colorbox{cl_aya}{40}} بَلْ تَأْتِيهِم بَغْتَةً فَتَبْهَتُهُمْ فَلَا يَسْتَطِيعُونَ رَدَّهَا وَلَا هُمْ يُنظَرُونَ
{\tiny\colorbox{cl_aya}{41}} وَلَقَدِ اسْتُهْزِئَ بِرُسُلٍ مِّن قَبْلِكَ فَحَاقَ بِالَّذِينَ سَخِرُوا مِنْهُم مَّا كَانُوا بِهِ يَسْتَهْزِءُونَ
{\tiny\colorbox{cl_aya}{42}} قُلْ مَن يَكْلَؤُكُم بِالَّيْلِ وَالنَّهَارِ مِنَ الرَّحْمَنِ بَلْ هُمْ عَن ذِكْرِ رَبِّهِم مُّعْرِضُونَ
{\tiny\colorbox{cl_aya}{43}} أَمْ لَهُمْ ءَالِهَةٌ تَمْنَعُهُم مِّن دُونِنَا لَا يَسْتَطِيعُونَ نَصْرَ أَنفُسِهِمْ وَلَا هُم مِّنَّا يُصْحَبُونَ
{\tiny\colorbox{cl_aya}{44}} بَلْ مَتَّعْنَا هَؤُلَاءِ وَءَابَاءَهُمْ حَتَّى طَالَ عَلَيْهِمُ الْعُمُرُ أَفَلَا يَرَوْنَ أَنَّا نَأْتِى الْأَرْضَ نَنقُصُهَا مِنْ أَطْرَافِهَا أَفَهُمُ الْغَلِبُونَ
{\tiny\colorbox{cl_aya}{45}} قُلْ إِنَّمَا أُنذِرُكُم بِالْوَحْىِ وَلَا يَسْمَعُ الصُّمُّ الدُّعَاءَ إِذَا مَا يُنذَرُونَ
{\tiny\colorbox{cl_aya}{46}} وَلَئِن مَّسَّتْهُمْ نَفْحَةٌ مِّنْ عَذَابِ رَبِّكَ لَيَقُولُنَّ يَوَيْلَنَا إِنَّا كُنَّا ظَلِمِينَ
{\tiny\colorbox{cl_aya}{47}} وَنَضَعُ الْمَوَزِينَ الْقِسْطَ لِيَوْمِ الْقِيَمَةِ فَلَا تُظْلَمُ نَفْسٌ شَئًْا وَإِن كَانَ مِثْقَالَ حَبَّةٍ مِّنْ خَرْدَلٍ أَتَيْنَا بِهَا وَكَفَى بِنَا حَسِبِينَ
{\tiny\colorbox{cl_aya}{48}} وَلَقَدْ ءَاتَيْنَا مُوسَى وَهَرُونَ الْفُرْقَانَ وَضِيَاءً وَذِكْرًا لِّلْمُتَّقِينَ
{\tiny\colorbox{cl_aya}{49}} الَّذِينَ يَخْشَوْنَ رَبَّهُم بِالْغَيْبِ وَهُم مِّنَ السَّاعَةِ مُشْفِقُونَ
{\tiny\colorbox{cl_aya}{50}} وَهَذَا ذِكْرٌ مُّبَارَكٌ أَنزَلْنَهُ أَفَأَنتُمْ لَهُ مُنكِرُونَ
{\tiny\colorbox{cl_aya}{51}} وَلَقَدْ ءَاتَيْنَا إِبْرَهِيمَ رُشْدَهُ مِن قَبْلُ وَكُنَّا بِهِ عَلِمِينَ
{\tiny\colorbox{cl_aya}{52}} إِذْ قَالَ لِأَبِيهِ وَقَوْمِهِ مَا هَذِهِ التَّمَاثِيلُ الَّتِى أَنتُمْ لَهَا عَكِفُونَ
{\tiny\colorbox{cl_aya}{53}} قَالُوا وَجَدْنَا ءَابَاءَنَا لَهَا عَبِدِينَ
{\tiny\colorbox{cl_aya}{54}} قَالَ لَقَدْ كُنتُمْ أَنتُمْ وَءَابَاؤُكُمْ فِى ضَلَلٍ مُّبِينٍ
{\tiny\colorbox{cl_aya}{55}} قَالُوا أَجِئْتَنَا بِالْحَقِّ أَمْ أَنتَ مِنَ اللَّعِبِينَ
{\tiny\colorbox{cl_aya}{56}} قَالَ بَل رَّبُّكُمْ رَبُّ السَّمَوَتِ وَالْأَرْضِ الَّذِى فَطَرَهُنَّ وَأَنَا عَلَى ذَلِكُم مِّنَ الشَّهِدِينَ
{\tiny\colorbox{cl_aya}{57}} وَتَاللَّهِ لَأَكِيدَنَّ أَصْنَمَكُم بَعْدَ أَن تُوَلُّوا مُدْبِرِينَ
{\tiny\colorbox{cl_aya}{58}} فَجَعَلَهُمْ جُذَذًا إِلَّا كَبِيرًا لَّهُمْ لَعَلَّهُمْ إِلَيْهِ يَرْجِعُونَ
{\tiny\colorbox{cl_aya}{59}} قَالُوا مَن فَعَلَ هَذَا بَِٔالِهَتِنَا إِنَّهُ لَمِنَ الظَّلِمِينَ
{\tiny\colorbox{cl_aya}{60}} قَالُوا سَمِعْنَا فَتًى يَذْكُرُهُمْ يُقَالُ لَهُ إِبْرَهِيمُ
{\tiny\colorbox{cl_aya}{61}} قَالُوا فَأْتُوا بِهِ عَلَى أَعْيُنِ النَّاسِ لَعَلَّهُمْ يَشْهَدُونَ
{\tiny\colorbox{cl_aya}{62}} قَالُوا ءَأَنتَ فَعَلْتَ هَذَا بَِٔالِهَتِنَا يَإِبْرَهِيمُ
{\tiny\colorbox{cl_aya}{63}} قَالَ بَلْ فَعَلَهُ كَبِيرُهُمْ هَذَا فَسَْٔلُوهُمْ إِن كَانُوا يَنطِقُونَ
{\tiny\colorbox{cl_aya}{64}} فَرَجَعُوا إِلَى أَنفُسِهِمْ فَقَالُوا إِنَّكُمْ أَنتُمُ الظَّلِمُونَ
{\tiny\colorbox{cl_aya}{65}} ثُمَّ نُكِسُوا عَلَى رُءُوسِهِمْ لَقَدْ عَلِمْتَ مَا هَؤُلَاءِ يَنطِقُونَ
{\tiny\colorbox{cl_aya}{66}} قَالَ أَفَتَعْبُدُونَ مِن دُونِ اللَّهِ مَا لَا يَنفَعُكُمْ شَئًْا وَلَا يَضُرُّكُمْ
{\tiny\colorbox{cl_aya}{67}} أُفٍّ لَّكُمْ وَلِمَا تَعْبُدُونَ مِن دُونِ اللَّهِ أَفَلَا تَعْقِلُونَ
{\tiny\colorbox{cl_aya}{68}} قَالُوا حَرِّقُوهُ وَانصُرُوا ءَالِهَتَكُمْ إِن كُنتُمْ فَعِلِينَ
{\tiny\colorbox{cl_aya}{69}} قُلْنَا يَنَارُ كُونِى بَرْدًا وَسَلَمًا عَلَى إِبْرَهِيمَ
{\tiny\colorbox{cl_aya}{70}} وَأَرَادُوا بِهِ كَيْدًا فَجَعَلْنَهُمُ الْأَخْسَرِينَ
{\tiny\colorbox{cl_aya}{71}} وَنَجَّيْنَهُ وَلُوطًا إِلَى الْأَرْضِ الَّتِى بَرَكْنَا فِيهَا لِلْعَلَمِينَ
{\tiny\colorbox{cl_aya}{72}} وَوَهَبْنَا لَهُ إِسْحَقَ وَيَعْقُوبَ نَافِلَةً وَكُلًّا جَعَلْنَا صَلِحِينَ
{\tiny\colorbox{cl_aya}{73}} وَجَعَلْنَهُمْ أَئِمَّةً يَهْدُونَ بِأَمْرِنَا وَأَوْحَيْنَا إِلَيْهِمْ فِعْلَ الْخَيْرَتِ وَإِقَامَ الصَّلَوةِ وَإِيتَاءَ الزَّكَوةِ وَكَانُوا لَنَا عَبِدِينَ
{\tiny\colorbox{cl_aya}{74}} وَلُوطًا ءَاتَيْنَهُ حُكْمًا وَعِلْمًا وَنَجَّيْنَهُ مِنَ الْقَرْيَةِ الَّتِى كَانَت تَّعْمَلُ الْخَبَئِثَ إِنَّهُمْ كَانُوا قَوْمَ سَوْءٍ فَسِقِينَ
{\tiny\colorbox{cl_aya}{75}} وَأَدْخَلْنَهُ فِى رَحْمَتِنَا إِنَّهُ مِنَ الصَّلِحِينَ
{\tiny\colorbox{cl_aya}{76}} وَنُوحًا إِذْ نَادَى مِن قَبْلُ فَاسْتَجَبْنَا لَهُ فَنَجَّيْنَهُ وَأَهْلَهُ مِنَ الْكَرْبِ الْعَظِيمِ
{\tiny\colorbox{cl_aya}{77}} وَنَصَرْنَهُ مِنَ الْقَوْمِ الَّذِينَ كَذَّبُوا بَِٔايَتِنَا إِنَّهُمْ كَانُوا قَوْمَ سَوْءٍ فَأَغْرَقْنَهُمْ أَجْمَعِينَ
{\tiny\colorbox{cl_aya}{78}} وَدَاوُدَ وَسُلَيْمَنَ إِذْ يَحْكُمَانِ فِى الْحَرْثِ إِذْ نَفَشَتْ فِيهِ غَنَمُ الْقَوْمِ وَكُنَّا لِحُكْمِهِمْ شَهِدِينَ
{\tiny\colorbox{cl_aya}{79}} فَفَهَّمْنَهَا سُلَيْمَنَ وَكُلًّا ءَاتَيْنَا حُكْمًا وَعِلْمًا وَسَخَّرْنَا مَعَ دَاوُدَ الْجِبَالَ يُسَبِّحْنَ وَالطَّيْرَ وَكُنَّا فَعِلِينَ
{\tiny\colorbox{cl_aya}{80}} وَعَلَّمْنَهُ صَنْعَةَ لَبُوسٍ لَّكُمْ لِتُحْصِنَكُم مِّن بَأْسِكُمْ فَهَلْ أَنتُمْ شَكِرُونَ
{\tiny\colorbox{cl_aya}{81}} وَلِسُلَيْمَنَ الرِّيحَ عَاصِفَةً تَجْرِى بِأَمْرِهِ إِلَى الْأَرْضِ الَّتِى بَرَكْنَا فِيهَا وَكُنَّا بِكُلِّ شَىْءٍ عَلِمِينَ
{\tiny\colorbox{cl_aya}{82}} وَمِنَ الشَّيَطِينِ مَن يَغُوصُونَ لَهُ وَيَعْمَلُونَ عَمَلًا دُونَ ذَلِكَ وَكُنَّا لَهُمْ حَفِظِينَ
{\tiny\colorbox{cl_aya}{83}} وَأَيُّوبَ إِذْ نَادَى رَبَّهُ أَنِّى مَسَّنِىَ الضُّرُّ وَأَنتَ أَرْحَمُ الرَّحِمِينَ
{\tiny\colorbox{cl_aya}{84}} فَاسْتَجَبْنَا لَهُ فَكَشَفْنَا مَا بِهِ مِن ضُرٍّ وَءَاتَيْنَهُ أَهْلَهُ وَمِثْلَهُم مَّعَهُمْ رَحْمَةً مِّنْ عِندِنَا وَذِكْرَى لِلْعَبِدِينَ
{\tiny\colorbox{cl_aya}{85}} وَإِسْمَعِيلَ وَإِدْرِيسَ وَذَا الْكِفْلِ كُلٌّ مِّنَ الصَّبِرِينَ
{\tiny\colorbox{cl_aya}{86}} وَأَدْخَلْنَهُمْ فِى رَحْمَتِنَا إِنَّهُم مِّنَ الصَّلِحِينَ
{\tiny\colorbox{cl_aya}{87}} وَذَا النُّونِ إِذ ذَّهَبَ مُغَضِبًا فَظَنَّ أَن لَّن نَّقْدِرَ عَلَيْهِ فَنَادَى فِى الظُّلُمَتِ أَن لَّا إِلَهَ إِلَّا أَنتَ سُبْحَنَكَ إِنِّى كُنتُ مِنَ الظَّلِمِينَ
{\tiny\colorbox{cl_aya}{88}} فَاسْتَجَبْنَا لَهُ وَنَجَّيْنَهُ مِنَ الْغَمِّ وَكَذَلِكَ نُجِى الْمُؤْمِنِينَ
{\tiny\colorbox{cl_aya}{89}} وَزَكَرِيَّا إِذْ نَادَى رَبَّهُ رَبِّ لَا تَذَرْنِى فَرْدًا وَأَنتَ خَيْرُ الْوَرِثِينَ
{\tiny\colorbox{cl_aya}{90}} فَاسْتَجَبْنَا لَهُ وَوَهَبْنَا لَهُ يَحْيَى وَأَصْلَحْنَا لَهُ زَوْجَهُ إِنَّهُمْ كَانُوا يُسَرِعُونَ فِى الْخَيْرَتِ وَيَدْعُونَنَا رَغَبًا وَرَهَبًا وَكَانُوا لَنَا خَشِعِينَ
{\tiny\colorbox{cl_aya}{91}} وَالَّتِى أَحْصَنَتْ فَرْجَهَا فَنَفَخْنَا فِيهَا مِن رُّوحِنَا وَجَعَلْنَهَا وَابْنَهَا ءَايَةً لِّلْعَلَمِينَ
{\tiny\colorbox{cl_aya}{92}} إِنَّ هَذِهِ أُمَّتُكُمْ أُمَّةً وَحِدَةً وَأَنَا رَبُّكُمْ فَاعْبُدُونِ
{\tiny\colorbox{cl_aya}{93}} وَتَقَطَّعُوا أَمْرَهُم بَيْنَهُمْ كُلٌّ إِلَيْنَا رَجِعُونَ
{\tiny\colorbox{cl_aya}{94}} فَمَن يَعْمَلْ مِنَ الصَّلِحَتِ وَهُوَ مُؤْمِنٌ فَلَا كُفْرَانَ لِسَعْيِهِ وَإِنَّا لَهُ كَتِبُونَ
{\tiny\colorbox{cl_aya}{95}} وَحَرَمٌ عَلَى قَرْيَةٍ أَهْلَكْنَهَا أَنَّهُمْ لَا يَرْجِعُونَ
{\tiny\colorbox{cl_aya}{96}} حَتَّى إِذَا فُتِحَتْ يَأْجُوجُ وَمَأْجُوجُ وَهُم مِّن كُلِّ حَدَبٍ يَنسِلُونَ
{\tiny\colorbox{cl_aya}{97}} وَاقْتَرَبَ الْوَعْدُ الْحَقُّ فَإِذَا هِىَ شَخِصَةٌ أَبْصَرُ الَّذِينَ كَفَرُوا يَوَيْلَنَا قَدْ كُنَّا فِى غَفْلَةٍ مِّنْ هَذَا بَلْ كُنَّا ظَلِمِينَ
{\tiny\colorbox{cl_aya}{98}} إِنَّكُمْ وَمَا تَعْبُدُونَ مِن دُونِ اللَّهِ حَصَبُ جَهَنَّمَ أَنتُمْ لَهَا وَرِدُونَ
{\tiny\colorbox{cl_aya}{99}} لَوْ كَانَ هَؤُلَاءِ ءَالِهَةً مَّا وَرَدُوهَا وَكُلٌّ فِيهَا خَلِدُونَ
{\tiny\colorbox{cl_aya}{100}} لَهُمْ فِيهَا زَفِيرٌ وَهُمْ فِيهَا لَا يَسْمَعُونَ
{\tiny\colorbox{cl_aya}{101}} إِنَّ الَّذِينَ سَبَقَتْ لَهُم مِّنَّا الْحُسْنَى أُولَئِكَ عَنْهَا مُبْعَدُونَ
{\tiny\colorbox{cl_aya}{102}} لَا يَسْمَعُونَ حَسِيسَهَا وَهُمْ فِى مَا اشْتَهَتْ أَنفُسُهُمْ خَلِدُونَ
{\tiny\colorbox{cl_aya}{103}} لَا يَحْزُنُهُمُ الْفَزَعُ الْأَكْبَرُ وَتَتَلَقَّىهُمُ الْمَلَئِكَةُ هَذَا يَوْمُكُمُ الَّذِى كُنتُمْ تُوعَدُونَ
{\tiny\colorbox{cl_aya}{104}} يَوْمَ نَطْوِى السَّمَاءَ كَطَىِّ السِّجِلِّ لِلْكُتُبِ كَمَا بَدَأْنَا أَوَّلَ خَلْقٍ نُّعِيدُهُ وَعْدًا عَلَيْنَا إِنَّا كُنَّا فَعِلِينَ
{\tiny\colorbox{cl_aya}{105}} وَلَقَدْ كَتَبْنَا فِى الزَّبُورِ مِن بَعْدِ الذِّكْرِ أَنَّ الْأَرْضَ يَرِثُهَا عِبَادِىَ الصَّلِحُونَ
{\tiny\colorbox{cl_aya}{106}} إِنَّ فِى هَذَا لَبَلَغًا لِّقَوْمٍ عَبِدِينَ
{\tiny\colorbox{cl_aya}{107}} وَمَا أَرْسَلْنَكَ إِلَّا رَحْمَةً لِّلْعَلَمِينَ
{\tiny\colorbox{cl_aya}{108}} قُلْ إِنَّمَا يُوحَى إِلَىَّ أَنَّمَا إِلَهُكُمْ إِلَهٌ وَحِدٌ فَهَلْ أَنتُم مُّسْلِمُونَ
{\tiny\colorbox{cl_aya}{109}} فَإِن تَوَلَّوْا فَقُلْ ءَاذَنتُكُمْ عَلَى سَوَاءٍ وَإِنْ أَدْرِى أَقَرِيبٌ أَم بَعِيدٌ مَّا تُوعَدُونَ
{\tiny\colorbox{cl_aya}{110}} إِنَّهُ يَعْلَمُ الْجَهْرَ مِنَ الْقَوْلِ وَيَعْلَمُ مَا تَكْتُمُونَ
{\tiny\colorbox{cl_aya}{111}} وَإِنْ أَدْرِى لَعَلَّهُ فِتْنَةٌ لَّكُمْ وَمَتَعٌ إِلَى حِينٍ
{\tiny\colorbox{cl_aya}{112}} قَلَ رَبِّ احْكُم بِالْحَقِّ وَرَبُّنَا الرَّحْمَنُ الْمُسْتَعَانُ عَلَى مَا تَصِفُونَ
\end{document}