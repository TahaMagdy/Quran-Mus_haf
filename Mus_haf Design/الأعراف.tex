%\documentclass[12pt,a4paper]{article}
\documentclass[20pt,a4paper]{article}
\usepackage[margin=0.5in]{geometry}
\usepackage{polyglossia}
\usepackage[dvipsnames]{xcolor}
\pagenumbering{gobble}
% This beautiful one line disable the initial spacing at the beginning of a line
\usepackage[parfill]{parskip} 
\usepackage{setspace}
\setstretch{2}

\setdefaultlanguage[numerals=maghrib]{arabic}
\newfontfamily\arabicfont[Script=Arabic]{Amiri}

\title{}
\author{}
\date{}
\definecolor{cl_page}{gray}{0.98}
\definecolor{cl_aya}{HTML}{DEEEFF}

\begin{document}
\pagecolor{cl_page}

% Start %


{\tiny\colorbox{cl_aya}{1}} المص
{\tiny\colorbox{cl_aya}{2}} كِتَبٌ أُنزِلَ إِلَيْكَ فَلَا يَكُن فِى صَدْرِكَ حَرَجٌ مِّنْهُ لِتُنذِرَ بِهِ وَذِكْرَى لِلْمُؤْمِنِينَ
{\tiny\colorbox{cl_aya}{3}} اتَّبِعُوا مَا أُنزِلَ إِلَيْكُم مِّن رَّبِّكُمْ وَلَا تَتَّبِعُوا مِن دُونِهِ أَوْلِيَاءَ قَلِيلًا مَّا تَذَكَّرُونَ
{\tiny\colorbox{cl_aya}{4}} وَكَم مِّن قَرْيَةٍ أَهْلَكْنَهَا فَجَاءَهَا بَأْسُنَا بَيَتًا أَوْ هُمْ قَائِلُونَ
{\tiny\colorbox{cl_aya}{5}} فَمَا كَانَ دَعْوَىهُمْ إِذْ جَاءَهُم بَأْسُنَا إِلَّا أَن قَالُوا إِنَّا كُنَّا ظَلِمِينَ
{\tiny\colorbox{cl_aya}{6}} فَلَنَسَْٔلَنَّ الَّذِينَ أُرْسِلَ إِلَيْهِمْ وَلَنَسَْٔلَنَّ الْمُرْسَلِينَ
{\tiny\colorbox{cl_aya}{7}} فَلَنَقُصَّنَّ عَلَيْهِم بِعِلْمٍ وَمَا كُنَّا غَائِبِينَ
{\tiny\colorbox{cl_aya}{8}} وَالْوَزْنُ يَوْمَئِذٍ الْحَقُّ فَمَن ثَقُلَتْ مَوَزِينُهُ فَأُولَئِكَ هُمُ الْمُفْلِحُونَ
{\tiny\colorbox{cl_aya}{9}} وَمَنْ خَفَّتْ مَوَزِينُهُ فَأُولَئِكَ الَّذِينَ خَسِرُوا أَنفُسَهُم بِمَا كَانُوا بَِٔايَتِنَا يَظْلِمُونَ
{\tiny\colorbox{cl_aya}{10}} وَلَقَدْ مَكَّنَّكُمْ فِى الْأَرْضِ وَجَعَلْنَا لَكُمْ فِيهَا مَعَيِشَ قَلِيلًا مَّا تَشْكُرُونَ
{\tiny\colorbox{cl_aya}{11}} وَلَقَدْ خَلَقْنَكُمْ ثُمَّ صَوَّرْنَكُمْ ثُمَّ قُلْنَا لِلْمَلَئِكَةِ اسْجُدُوا لِءَادَمَ فَسَجَدُوا إِلَّا إِبْلِيسَ لَمْ يَكُن مِّنَ السَّجِدِينَ
{\tiny\colorbox{cl_aya}{12}} قَالَ مَا مَنَعَكَ أَلَّا تَسْجُدَ إِذْ أَمَرْتُكَ قَالَ أَنَا خَيْرٌ مِّنْهُ خَلَقْتَنِى مِن نَّارٍ وَخَلَقْتَهُ مِن طِينٍ
{\tiny\colorbox{cl_aya}{13}} قَالَ فَاهْبِطْ مِنْهَا فَمَا يَكُونُ لَكَ أَن تَتَكَبَّرَ فِيهَا فَاخْرُجْ إِنَّكَ مِنَ الصَّغِرِينَ
{\tiny\colorbox{cl_aya}{14}} قَالَ أَنظِرْنِى إِلَى يَوْمِ يُبْعَثُونَ
{\tiny\colorbox{cl_aya}{15}} قَالَ إِنَّكَ مِنَ الْمُنظَرِينَ
{\tiny\colorbox{cl_aya}{16}} قَالَ فَبِمَا أَغْوَيْتَنِى لَأَقْعُدَنَّ لَهُمْ صِرَطَكَ الْمُسْتَقِيمَ
{\tiny\colorbox{cl_aya}{17}} ثُمَّ لَءَاتِيَنَّهُم مِّن بَيْنِ أَيْدِيهِمْ وَمِنْ خَلْفِهِمْ وَعَنْ أَيْمَنِهِمْ وَعَن شَمَائِلِهِمْ وَلَا تَجِدُ أَكْثَرَهُمْ شَكِرِينَ
{\tiny\colorbox{cl_aya}{18}} قَالَ اخْرُجْ مِنْهَا مَذْءُومًا مَّدْحُورًا لَّمَن تَبِعَكَ مِنْهُمْ لَأَمْلَأَنَّ جَهَنَّمَ مِنكُمْ أَجْمَعِينَ
{\tiny\colorbox{cl_aya}{19}} وَئََادَمُ اسْكُنْ أَنتَ وَزَوْجُكَ الْجَنَّةَ فَكُلَا مِنْ حَيْثُ شِئْتُمَا وَلَا تَقْرَبَا هَذِهِ الشَّجَرَةَ فَتَكُونَا مِنَ الظَّلِمِينَ
{\tiny\colorbox{cl_aya}{20}} فَوَسْوَسَ لَهُمَا الشَّيْطَنُ لِيُبْدِىَ لَهُمَا مَا وُرِىَ عَنْهُمَا مِن سَوْءَتِهِمَا وَقَالَ مَا نَهَىكُمَا رَبُّكُمَا عَنْ هَذِهِ الشَّجَرَةِ إِلَّا أَن تَكُونَا مَلَكَيْنِ أَوْ تَكُونَا مِنَ الْخَلِدِينَ
{\tiny\colorbox{cl_aya}{21}} وَقَاسَمَهُمَا إِنِّى لَكُمَا لَمِنَ النَّصِحِينَ
{\tiny\colorbox{cl_aya}{22}} فَدَلَّىهُمَا بِغُرُورٍ فَلَمَّا ذَاقَا الشَّجَرَةَ بَدَتْ لَهُمَا سَوْءَتُهُمَا وَطَفِقَا يَخْصِفَانِ عَلَيْهِمَا مِن وَرَقِ الْجَنَّةِ وَنَادَىهُمَا رَبُّهُمَا أَلَمْ أَنْهَكُمَا عَن تِلْكُمَا الشَّجَرَةِ وَأَقُل لَّكُمَا إِنَّ الشَّيْطَنَ لَكُمَا عَدُوٌّ مُّبِينٌ
{\tiny\colorbox{cl_aya}{23}} قَالَا رَبَّنَا ظَلَمْنَا أَنفُسَنَا وَإِن لَّمْ تَغْفِرْ لَنَا وَتَرْحَمْنَا لَنَكُونَنَّ مِنَ الْخَسِرِينَ
{\tiny\colorbox{cl_aya}{24}} قَالَ اهْبِطُوا بَعْضُكُمْ لِبَعْضٍ عَدُوٌّ وَلَكُمْ فِى الْأَرْضِ مُسْتَقَرٌّ وَمَتَعٌ إِلَى حِينٍ
{\tiny\colorbox{cl_aya}{25}} قَالَ فِيهَا تَحْيَوْنَ وَفِيهَا تَمُوتُونَ وَمِنْهَا تُخْرَجُونَ
{\tiny\colorbox{cl_aya}{26}} يَبَنِى ءَادَمَ قَدْ أَنزَلْنَا عَلَيْكُمْ لِبَاسًا يُوَرِى سَوْءَتِكُمْ وَرِيشًا وَلِبَاسُ التَّقْوَى ذَلِكَ خَيْرٌ ذَلِكَ مِنْ ءَايَتِ اللَّهِ لَعَلَّهُمْ يَذَّكَّرُونَ
{\tiny\colorbox{cl_aya}{27}} يَبَنِى ءَادَمَ لَا يَفْتِنَنَّكُمُ الشَّيْطَنُ كَمَا أَخْرَجَ أَبَوَيْكُم مِّنَ الْجَنَّةِ يَنزِعُ عَنْهُمَا لِبَاسَهُمَا لِيُرِيَهُمَا سَوْءَتِهِمَا إِنَّهُ يَرَىكُمْ هُوَ وَقَبِيلُهُ مِنْ حَيْثُ لَا تَرَوْنَهُمْ إِنَّا جَعَلْنَا الشَّيَطِينَ أَوْلِيَاءَ لِلَّذِينَ لَا يُؤْمِنُونَ
{\tiny\colorbox{cl_aya}{28}} وَإِذَا فَعَلُوا فَحِشَةً قَالُوا وَجَدْنَا عَلَيْهَا ءَابَاءَنَا وَاللَّهُ أَمَرَنَا بِهَا قُلْ إِنَّ اللَّهَ لَا يَأْمُرُ بِالْفَحْشَاءِ أَتَقُولُونَ عَلَى اللَّهِ مَا لَا تَعْلَمُونَ
{\tiny\colorbox{cl_aya}{29}} قُلْ أَمَرَ رَبِّى بِالْقِسْطِ وَأَقِيمُوا وُجُوهَكُمْ عِندَ كُلِّ مَسْجِدٍ وَادْعُوهُ مُخْلِصِينَ لَهُ الدِّينَ كَمَا بَدَأَكُمْ تَعُودُونَ
{\tiny\colorbox{cl_aya}{30}} فَرِيقًا هَدَى وَفَرِيقًا حَقَّ عَلَيْهِمُ الضَّلَلَةُ إِنَّهُمُ اتَّخَذُوا الشَّيَطِينَ أَوْلِيَاءَ مِن دُونِ اللَّهِ وَيَحْسَبُونَ أَنَّهُم مُّهْتَدُونَ
{\tiny\colorbox{cl_aya}{31}} يَبَنِى ءَادَمَ خُذُوا زِينَتَكُمْ عِندَ كُلِّ مَسْجِدٍ وَكُلُوا وَاشْرَبُوا وَلَا تُسْرِفُوا إِنَّهُ لَا يُحِبُّ الْمُسْرِفِينَ
{\tiny\colorbox{cl_aya}{32}} قُلْ مَنْ حَرَّمَ زِينَةَ اللَّهِ الَّتِى أَخْرَجَ لِعِبَادِهِ وَالطَّيِّبَتِ مِنَ الرِّزْقِ قُلْ هِىَ لِلَّذِينَ ءَامَنُوا فِى الْحَيَوةِ الدُّنْيَا خَالِصَةً يَوْمَ الْقِيَمَةِ كَذَلِكَ نُفَصِّلُ الْءَايَتِ لِقَوْمٍ يَعْلَمُونَ
{\tiny\colorbox{cl_aya}{33}} قُلْ إِنَّمَا حَرَّمَ رَبِّىَ الْفَوَحِشَ مَا ظَهَرَ مِنْهَا وَمَا بَطَنَ وَالْإِثْمَ وَالْبَغْىَ بِغَيْرِ الْحَقِّ وَأَن تُشْرِكُوا بِاللَّهِ مَا لَمْ يُنَزِّلْ بِهِ سُلْطَنًا وَأَن تَقُولُوا عَلَى اللَّهِ مَا لَا تَعْلَمُونَ
{\tiny\colorbox{cl_aya}{34}} وَلِكُلِّ أُمَّةٍ أَجَلٌ فَإِذَا جَاءَ أَجَلُهُمْ لَا يَسْتَأْخِرُونَ سَاعَةً وَلَا يَسْتَقْدِمُونَ
{\tiny\colorbox{cl_aya}{35}} يَبَنِى ءَادَمَ إِمَّا يَأْتِيَنَّكُمْ رُسُلٌ مِّنكُمْ يَقُصُّونَ عَلَيْكُمْ ءَايَتِى فَمَنِ اتَّقَى وَأَصْلَحَ فَلَا خَوْفٌ عَلَيْهِمْ وَلَا هُمْ يَحْزَنُونَ
{\tiny\colorbox{cl_aya}{36}} وَالَّذِينَ كَذَّبُوا بَِٔايَتِنَا وَاسْتَكْبَرُوا عَنْهَا أُولَئِكَ أَصْحَبُ النَّارِ هُمْ فِيهَا خَلِدُونَ
{\tiny\colorbox{cl_aya}{37}} فَمَنْ أَظْلَمُ مِمَّنِ افْتَرَى عَلَى اللَّهِ كَذِبًا أَوْ كَذَّبَ بَِٔايَتِهِ أُولَئِكَ يَنَالُهُمْ نَصِيبُهُم مِّنَ الْكِتَبِ حَتَّى إِذَا جَاءَتْهُمْ رُسُلُنَا يَتَوَفَّوْنَهُمْ قَالُوا أَيْنَ مَا كُنتُمْ تَدْعُونَ مِن دُونِ اللَّهِ قَالُوا ضَلُّوا عَنَّا وَشَهِدُوا عَلَى أَنفُسِهِمْ أَنَّهُمْ كَانُوا كَفِرِينَ
{\tiny\colorbox{cl_aya}{38}} قَالَ ادْخُلُوا فِى أُمَمٍ قَدْ خَلَتْ مِن قَبْلِكُم مِّنَ الْجِنِّ وَالْإِنسِ فِى النَّارِ كُلَّمَا دَخَلَتْ أُمَّةٌ لَّعَنَتْ أُخْتَهَا حَتَّى إِذَا ادَّارَكُوا فِيهَا جَمِيعًا قَالَتْ أُخْرَىهُمْ لِأُولَىهُمْ رَبَّنَا هَؤُلَاءِ أَضَلُّونَا فََٔاتِهِمْ عَذَابًا ضِعْفًا مِّنَ النَّارِ قَالَ لِكُلٍّ ضِعْفٌ وَلَكِن لَّا تَعْلَمُونَ
{\tiny\colorbox{cl_aya}{39}} وَقَالَتْ أُولَىهُمْ لِأُخْرَىهُمْ فَمَا كَانَ لَكُمْ عَلَيْنَا مِن فَضْلٍ فَذُوقُوا الْعَذَابَ بِمَا كُنتُمْ تَكْسِبُونَ
{\tiny\colorbox{cl_aya}{40}} إِنَّ الَّذِينَ كَذَّبُوا بَِٔايَتِنَا وَاسْتَكْبَرُوا عَنْهَا لَا تُفَتَّحُ لَهُمْ أَبْوَبُ السَّمَاءِ وَلَا يَدْخُلُونَ الْجَنَّةَ حَتَّى يَلِجَ الْجَمَلُ فِى سَمِّ الْخِيَاطِ وَكَذَلِكَ نَجْزِى الْمُجْرِمِينَ
{\tiny\colorbox{cl_aya}{41}} لَهُم مِّن جَهَنَّمَ مِهَادٌ وَمِن فَوْقِهِمْ غَوَاشٍ وَكَذَلِكَ نَجْزِى الظَّلِمِينَ
{\tiny\colorbox{cl_aya}{42}} وَالَّذِينَ ءَامَنُوا وَعَمِلُوا الصَّلِحَتِ لَا نُكَلِّفُ نَفْسًا إِلَّا وُسْعَهَا أُولَئِكَ أَصْحَبُ الْجَنَّةِ هُمْ فِيهَا خَلِدُونَ
{\tiny\colorbox{cl_aya}{43}} وَنَزَعْنَا مَا فِى صُدُورِهِم مِّنْ غِلٍّ تَجْرِى مِن تَحْتِهِمُ الْأَنْهَرُ وَقَالُوا الْحَمْدُ لِلَّهِ الَّذِى هَدَىنَا لِهَذَا وَمَا كُنَّا لِنَهْتَدِىَ لَوْلَا أَنْ هَدَىنَا اللَّهُ لَقَدْ جَاءَتْ رُسُلُ رَبِّنَا بِالْحَقِّ وَنُودُوا أَن تِلْكُمُ الْجَنَّةُ أُورِثْتُمُوهَا بِمَا كُنتُمْ تَعْمَلُونَ
{\tiny\colorbox{cl_aya}{44}} وَنَادَى أَصْحَبُ الْجَنَّةِ أَصْحَبَ النَّارِ أَن قَدْ وَجَدْنَا مَا وَعَدَنَا رَبُّنَا حَقًّا فَهَلْ وَجَدتُّم مَّا وَعَدَ رَبُّكُمْ حَقًّا قَالُوا نَعَمْ فَأَذَّنَ مُؤَذِّنٌ بَيْنَهُمْ أَن لَّعْنَةُ اللَّهِ عَلَى الظَّلِمِينَ
{\tiny\colorbox{cl_aya}{45}} الَّذِينَ يَصُدُّونَ عَن سَبِيلِ اللَّهِ وَيَبْغُونَهَا عِوَجًا وَهُم بِالْءَاخِرَةِ كَفِرُونَ
{\tiny\colorbox{cl_aya}{46}} وَبَيْنَهُمَا حِجَابٌ وَعَلَى الْأَعْرَافِ رِجَالٌ يَعْرِفُونَ كُلًّا بِسِيمَىهُمْ وَنَادَوْا أَصْحَبَ الْجَنَّةِ أَن سَلَمٌ عَلَيْكُمْ لَمْ يَدْخُلُوهَا وَهُمْ يَطْمَعُونَ
{\tiny\colorbox{cl_aya}{47}} وَإِذَا صُرِفَتْ أَبْصَرُهُمْ تِلْقَاءَ أَصْحَبِ النَّارِ قَالُوا رَبَّنَا لَا تَجْعَلْنَا مَعَ الْقَوْمِ الظَّلِمِينَ
{\tiny\colorbox{cl_aya}{48}} وَنَادَى أَصْحَبُ الْأَعْرَافِ رِجَالًا يَعْرِفُونَهُم بِسِيمَىهُمْ قَالُوا مَا أَغْنَى عَنكُمْ جَمْعُكُمْ وَمَا كُنتُمْ تَسْتَكْبِرُونَ
{\tiny\colorbox{cl_aya}{49}} أَهَؤُلَاءِ الَّذِينَ أَقْسَمْتُمْ لَا يَنَالُهُمُ اللَّهُ بِرَحْمَةٍ ادْخُلُوا الْجَنَّةَ لَا خَوْفٌ عَلَيْكُمْ وَلَا أَنتُمْ تَحْزَنُونَ
{\tiny\colorbox{cl_aya}{50}} وَنَادَى أَصْحَبُ النَّارِ أَصْحَبَ الْجَنَّةِ أَنْ أَفِيضُوا عَلَيْنَا مِنَ الْمَاءِ أَوْ مِمَّا رَزَقَكُمُ اللَّهُ قَالُوا إِنَّ اللَّهَ حَرَّمَهُمَا عَلَى الْكَفِرِينَ
{\tiny\colorbox{cl_aya}{51}} الَّذِينَ اتَّخَذُوا دِينَهُمْ لَهْوًا وَلَعِبًا وَغَرَّتْهُمُ الْحَيَوةُ الدُّنْيَا فَالْيَوْمَ نَنسَىهُمْ كَمَا نَسُوا لِقَاءَ يَوْمِهِمْ هَذَا وَمَا كَانُوا بَِٔايَتِنَا يَجْحَدُونَ
{\tiny\colorbox{cl_aya}{52}} وَلَقَدْ جِئْنَهُم بِكِتَبٍ فَصَّلْنَهُ عَلَى عِلْمٍ هُدًى وَرَحْمَةً لِّقَوْمٍ يُؤْمِنُونَ
{\tiny\colorbox{cl_aya}{53}} هَلْ يَنظُرُونَ إِلَّا تَأْوِيلَهُ يَوْمَ يَأْتِى تَأْوِيلُهُ يَقُولُ الَّذِينَ نَسُوهُ مِن قَبْلُ قَدْ جَاءَتْ رُسُلُ رَبِّنَا بِالْحَقِّ فَهَل لَّنَا مِن شُفَعَاءَ فَيَشْفَعُوا لَنَا أَوْ نُرَدُّ فَنَعْمَلَ غَيْرَ الَّذِى كُنَّا نَعْمَلُ قَدْ خَسِرُوا أَنفُسَهُمْ وَضَلَّ عَنْهُم مَّا كَانُوا يَفْتَرُونَ
{\tiny\colorbox{cl_aya}{54}} إِنَّ رَبَّكُمُ اللَّهُ الَّذِى خَلَقَ السَّمَوَتِ وَالْأَرْضَ فِى سِتَّةِ أَيَّامٍ ثُمَّ اسْتَوَى عَلَى الْعَرْشِ يُغْشِى الَّيْلَ النَّهَارَ يَطْلُبُهُ حَثِيثًا وَالشَّمْسَ وَالْقَمَرَ وَالنُّجُومَ مُسَخَّرَتٍ بِأَمْرِهِ أَلَا لَهُ الْخَلْقُ وَالْأَمْرُ تَبَارَكَ اللَّهُ رَبُّ الْعَلَمِينَ
{\tiny\colorbox{cl_aya}{55}} ادْعُوا رَبَّكُمْ تَضَرُّعًا وَخُفْيَةً إِنَّهُ لَا يُحِبُّ الْمُعْتَدِينَ
{\tiny\colorbox{cl_aya}{56}} وَلَا تُفْسِدُوا فِى الْأَرْضِ بَعْدَ إِصْلَحِهَا وَادْعُوهُ خَوْفًا وَطَمَعًا إِنَّ رَحْمَتَ اللَّهِ قَرِيبٌ مِّنَ الْمُحْسِنِينَ
{\tiny\colorbox{cl_aya}{57}} وَهُوَ الَّذِى يُرْسِلُ الرِّيَحَ بُشْرًا بَيْنَ يَدَىْ رَحْمَتِهِ حَتَّى إِذَا أَقَلَّتْ سَحَابًا ثِقَالًا سُقْنَهُ لِبَلَدٍ مَّيِّتٍ فَأَنزَلْنَا بِهِ الْمَاءَ فَأَخْرَجْنَا بِهِ مِن كُلِّ الثَّمَرَتِ كَذَلِكَ نُخْرِجُ الْمَوْتَى لَعَلَّكُمْ تَذَكَّرُونَ
{\tiny\colorbox{cl_aya}{58}} وَالْبَلَدُ الطَّيِّبُ يَخْرُجُ نَبَاتُهُ بِإِذْنِ رَبِّهِ وَالَّذِى خَبُثَ لَا يَخْرُجُ إِلَّا نَكِدًا كَذَلِكَ نُصَرِّفُ الْءَايَتِ لِقَوْمٍ يَشْكُرُونَ
{\tiny\colorbox{cl_aya}{59}} لَقَدْ أَرْسَلْنَا نُوحًا إِلَى قَوْمِهِ فَقَالَ يَقَوْمِ اعْبُدُوا اللَّهَ مَا لَكُم مِّنْ إِلَهٍ غَيْرُهُ إِنِّى أَخَافُ عَلَيْكُمْ عَذَابَ يَوْمٍ عَظِيمٍ
{\tiny\colorbox{cl_aya}{60}} قَالَ الْمَلَأُ مِن قَوْمِهِ إِنَّا لَنَرَىكَ فِى ضَلَلٍ مُّبِينٍ
{\tiny\colorbox{cl_aya}{61}} قَالَ يَقَوْمِ لَيْسَ بِى ضَلَلَةٌ وَلَكِنِّى رَسُولٌ مِّن رَّبِّ الْعَلَمِينَ
{\tiny\colorbox{cl_aya}{62}} أُبَلِّغُكُمْ رِسَلَتِ رَبِّى وَأَنصَحُ لَكُمْ وَأَعْلَمُ مِنَ اللَّهِ مَا لَا تَعْلَمُونَ
{\tiny\colorbox{cl_aya}{63}} أَوَعَجِبْتُمْ أَن جَاءَكُمْ ذِكْرٌ مِّن رَّبِّكُمْ عَلَى رَجُلٍ مِّنكُمْ لِيُنذِرَكُمْ وَلِتَتَّقُوا وَلَعَلَّكُمْ تُرْحَمُونَ
{\tiny\colorbox{cl_aya}{64}} فَكَذَّبُوهُ فَأَنجَيْنَهُ وَالَّذِينَ مَعَهُ فِى الْفُلْكِ وَأَغْرَقْنَا الَّذِينَ كَذَّبُوا بَِٔايَتِنَا إِنَّهُمْ كَانُوا قَوْمًا عَمِينَ
{\tiny\colorbox{cl_aya}{65}} وَإِلَى عَادٍ أَخَاهُمْ هُودًا قَالَ يَقَوْمِ اعْبُدُوا اللَّهَ مَا لَكُم مِّنْ إِلَهٍ غَيْرُهُ أَفَلَا تَتَّقُونَ
{\tiny\colorbox{cl_aya}{66}} قَالَ الْمَلَأُ الَّذِينَ كَفَرُوا مِن قَوْمِهِ إِنَّا لَنَرَىكَ فِى سَفَاهَةٍ وَإِنَّا لَنَظُنُّكَ مِنَ الْكَذِبِينَ
{\tiny\colorbox{cl_aya}{67}} قَالَ يَقَوْمِ لَيْسَ بِى سَفَاهَةٌ وَلَكِنِّى رَسُولٌ مِّن رَّبِّ الْعَلَمِينَ
{\tiny\colorbox{cl_aya}{68}} أُبَلِّغُكُمْ رِسَلَتِ رَبِّى وَأَنَا لَكُمْ نَاصِحٌ أَمِينٌ
{\tiny\colorbox{cl_aya}{69}} أَوَعَجِبْتُمْ أَن جَاءَكُمْ ذِكْرٌ مِّن رَّبِّكُمْ عَلَى رَجُلٍ مِّنكُمْ لِيُنذِرَكُمْ وَاذْكُرُوا إِذْ جَعَلَكُمْ خُلَفَاءَ مِن بَعْدِ قَوْمِ نُوحٍ وَزَادَكُمْ فِى الْخَلْقِ بَصْطَةً فَاذْكُرُوا ءَالَاءَ اللَّهِ لَعَلَّكُمْ تُفْلِحُونَ
{\tiny\colorbox{cl_aya}{70}} قَالُوا أَجِئْتَنَا لِنَعْبُدَ اللَّهَ وَحْدَهُ وَنَذَرَ مَا كَانَ يَعْبُدُ ءَابَاؤُنَا فَأْتِنَا بِمَا تَعِدُنَا إِن كُنتَ مِنَ الصَّدِقِينَ
{\tiny\colorbox{cl_aya}{71}} قَالَ قَدْ وَقَعَ عَلَيْكُم مِّن رَّبِّكُمْ رِجْسٌ وَغَضَبٌ أَتُجَدِلُونَنِى فِى أَسْمَاءٍ سَمَّيْتُمُوهَا أَنتُمْ وَءَابَاؤُكُم مَّا نَزَّلَ اللَّهُ بِهَا مِن سُلْطَنٍ فَانتَظِرُوا إِنِّى مَعَكُم مِّنَ الْمُنتَظِرِينَ
{\tiny\colorbox{cl_aya}{72}} فَأَنجَيْنَهُ وَالَّذِينَ مَعَهُ بِرَحْمَةٍ مِّنَّا وَقَطَعْنَا دَابِرَ الَّذِينَ كَذَّبُوا بَِٔايَتِنَا وَمَا كَانُوا مُؤْمِنِينَ
{\tiny\colorbox{cl_aya}{73}} وَإِلَى ثَمُودَ أَخَاهُمْ صَلِحًا قَالَ يَقَوْمِ اعْبُدُوا اللَّهَ مَا لَكُم مِّنْ إِلَهٍ غَيْرُهُ قَدْ جَاءَتْكُم بَيِّنَةٌ مِّن رَّبِّكُمْ هَذِهِ نَاقَةُ اللَّهِ لَكُمْ ءَايَةً فَذَرُوهَا تَأْكُلْ فِى أَرْضِ اللَّهِ وَلَا تَمَسُّوهَا بِسُوءٍ فَيَأْخُذَكُمْ عَذَابٌ أَلِيمٌ
{\tiny\colorbox{cl_aya}{74}} وَاذْكُرُوا إِذْ جَعَلَكُمْ خُلَفَاءَ مِن بَعْدِ عَادٍ وَبَوَّأَكُمْ فِى الْأَرْضِ تَتَّخِذُونَ مِن سُهُولِهَا قُصُورًا وَتَنْحِتُونَ الْجِبَالَ بُيُوتًا فَاذْكُرُوا ءَالَاءَ اللَّهِ وَلَا تَعْثَوْا فِى الْأَرْضِ مُفْسِدِينَ
{\tiny\colorbox{cl_aya}{75}} قَالَ الْمَلَأُ الَّذِينَ اسْتَكْبَرُوا مِن قَوْمِهِ لِلَّذِينَ اسْتُضْعِفُوا لِمَنْ ءَامَنَ مِنْهُمْ أَتَعْلَمُونَ أَنَّ صَلِحًا مُّرْسَلٌ مِّن رَّبِّهِ قَالُوا إِنَّا بِمَا أُرْسِلَ بِهِ مُؤْمِنُونَ
{\tiny\colorbox{cl_aya}{76}} قَالَ الَّذِينَ اسْتَكْبَرُوا إِنَّا بِالَّذِى ءَامَنتُم بِهِ كَفِرُونَ
{\tiny\colorbox{cl_aya}{77}} فَعَقَرُوا النَّاقَةَ وَعَتَوْا عَنْ أَمْرِ رَبِّهِمْ وَقَالُوا يَصَلِحُ ائْتِنَا بِمَا تَعِدُنَا إِن كُنتَ مِنَ الْمُرْسَلِينَ
{\tiny\colorbox{cl_aya}{78}} فَأَخَذَتْهُمُ الرَّجْفَةُ فَأَصْبَحُوا فِى دَارِهِمْ جَثِمِينَ
{\tiny\colorbox{cl_aya}{79}} فَتَوَلَّى عَنْهُمْ وَقَالَ يَقَوْمِ لَقَدْ أَبْلَغْتُكُمْ رِسَالَةَ رَبِّى وَنَصَحْتُ لَكُمْ وَلَكِن لَّا تُحِبُّونَ النَّصِحِينَ
{\tiny\colorbox{cl_aya}{80}} وَلُوطًا إِذْ قَالَ لِقَوْمِهِ أَتَأْتُونَ الْفَحِشَةَ مَا سَبَقَكُم بِهَا مِنْ أَحَدٍ مِّنَ الْعَلَمِينَ
{\tiny\colorbox{cl_aya}{81}} إِنَّكُمْ لَتَأْتُونَ الرِّجَالَ شَهْوَةً مِّن دُونِ النِّسَاءِ بَلْ أَنتُمْ قَوْمٌ مُّسْرِفُونَ
{\tiny\colorbox{cl_aya}{82}} وَمَا كَانَ جَوَابَ قَوْمِهِ إِلَّا أَن قَالُوا أَخْرِجُوهُم مِّن قَرْيَتِكُمْ إِنَّهُمْ أُنَاسٌ يَتَطَهَّرُونَ
{\tiny\colorbox{cl_aya}{83}} فَأَنجَيْنَهُ وَأَهْلَهُ إِلَّا امْرَأَتَهُ كَانَتْ مِنَ الْغَبِرِينَ
{\tiny\colorbox{cl_aya}{84}} وَأَمْطَرْنَا عَلَيْهِم مَّطَرًا فَانظُرْ كَيْفَ كَانَ عَقِبَةُ الْمُجْرِمِينَ
{\tiny\colorbox{cl_aya}{85}} وَإِلَى مَدْيَنَ أَخَاهُمْ شُعَيْبًا قَالَ يَقَوْمِ اعْبُدُوا اللَّهَ مَا لَكُم مِّنْ إِلَهٍ غَيْرُهُ قَدْ جَاءَتْكُم بَيِّنَةٌ مِّن رَّبِّكُمْ فَأَوْفُوا الْكَيْلَ وَالْمِيزَانَ وَلَا تَبْخَسُوا النَّاسَ أَشْيَاءَهُمْ وَلَا تُفْسِدُوا فِى الْأَرْضِ بَعْدَ إِصْلَحِهَا ذَلِكُمْ خَيْرٌ لَّكُمْ إِن كُنتُم مُّؤْمِنِينَ
{\tiny\colorbox{cl_aya}{86}} وَلَا تَقْعُدُوا بِكُلِّ صِرَطٍ تُوعِدُونَ وَتَصُدُّونَ عَن سَبِيلِ اللَّهِ مَنْ ءَامَنَ بِهِ وَتَبْغُونَهَا عِوَجًا وَاذْكُرُوا إِذْ كُنتُمْ قَلِيلًا فَكَثَّرَكُمْ وَانظُرُوا كَيْفَ كَانَ عَقِبَةُ الْمُفْسِدِينَ
{\tiny\colorbox{cl_aya}{87}} وَإِن كَانَ طَائِفَةٌ مِّنكُمْ ءَامَنُوا بِالَّذِى أُرْسِلْتُ بِهِ وَطَائِفَةٌ لَّمْ يُؤْمِنُوا فَاصْبِرُوا حَتَّى يَحْكُمَ اللَّهُ بَيْنَنَا وَهُوَ خَيْرُ الْحَكِمِينَ
{\tiny\colorbox{cl_aya}{88}} قَالَ الْمَلَأُ الَّذِينَ اسْتَكْبَرُوا مِن قَوْمِهِ لَنُخْرِجَنَّكَ يَشُعَيْبُ وَالَّذِينَ ءَامَنُوا مَعَكَ مِن قَرْيَتِنَا أَوْ لَتَعُودُنَّ فِى مِلَّتِنَا قَالَ أَوَلَوْ كُنَّا كَرِهِينَ
{\tiny\colorbox{cl_aya}{89}} قَدِ افْتَرَيْنَا عَلَى اللَّهِ كَذِبًا إِنْ عُدْنَا فِى مِلَّتِكُم بَعْدَ إِذْ نَجَّىنَا اللَّهُ مِنْهَا وَمَا يَكُونُ لَنَا أَن نَّعُودَ فِيهَا إِلَّا أَن يَشَاءَ اللَّهُ رَبُّنَا وَسِعَ رَبُّنَا كُلَّ شَىْءٍ عِلْمًا عَلَى اللَّهِ تَوَكَّلْنَا رَبَّنَا افْتَحْ بَيْنَنَا وَبَيْنَ قَوْمِنَا بِالْحَقِّ وَأَنتَ خَيْرُ الْفَتِحِينَ
{\tiny\colorbox{cl_aya}{90}} وَقَالَ الْمَلَأُ الَّذِينَ كَفَرُوا مِن قَوْمِهِ لَئِنِ اتَّبَعْتُمْ شُعَيْبًا إِنَّكُمْ إِذًا لَّخَسِرُونَ
{\tiny\colorbox{cl_aya}{91}} فَأَخَذَتْهُمُ الرَّجْفَةُ فَأَصْبَحُوا فِى دَارِهِمْ جَثِمِينَ
{\tiny\colorbox{cl_aya}{92}} الَّذِينَ كَذَّبُوا شُعَيْبًا كَأَن لَّمْ يَغْنَوْا فِيهَا الَّذِينَ كَذَّبُوا شُعَيْبًا كَانُوا هُمُ الْخَسِرِينَ
{\tiny\colorbox{cl_aya}{93}} فَتَوَلَّى عَنْهُمْ وَقَالَ يَقَوْمِ لَقَدْ أَبْلَغْتُكُمْ رِسَلَتِ رَبِّى وَنَصَحْتُ لَكُمْ فَكَيْفَ ءَاسَى عَلَى قَوْمٍ كَفِرِينَ
{\tiny\colorbox{cl_aya}{94}} وَمَا أَرْسَلْنَا فِى قَرْيَةٍ مِّن نَّبِىٍّ إِلَّا أَخَذْنَا أَهْلَهَا بِالْبَأْسَاءِ وَالضَّرَّاءِ لَعَلَّهُمْ يَضَّرَّعُونَ
{\tiny\colorbox{cl_aya}{95}} ثُمَّ بَدَّلْنَا مَكَانَ السَّيِّئَةِ الْحَسَنَةَ حَتَّى عَفَوا وَّقَالُوا قَدْ مَسَّ ءَابَاءَنَا الضَّرَّاءُ وَالسَّرَّاءُ فَأَخَذْنَهُم بَغْتَةً وَهُمْ لَا يَشْعُرُونَ
{\tiny\colorbox{cl_aya}{96}} وَلَوْ أَنَّ أَهْلَ الْقُرَى ءَامَنُوا وَاتَّقَوْا لَفَتَحْنَا عَلَيْهِم بَرَكَتٍ مِّنَ السَّمَاءِ وَالْأَرْضِ وَلَكِن كَذَّبُوا فَأَخَذْنَهُم بِمَا كَانُوا يَكْسِبُونَ
{\tiny\colorbox{cl_aya}{97}} أَفَأَمِنَ أَهْلُ الْقُرَى أَن يَأْتِيَهُم بَأْسُنَا بَيَتًا وَهُمْ نَائِمُونَ
{\tiny\colorbox{cl_aya}{98}} أَوَأَمِنَ أَهْلُ الْقُرَى أَن يَأْتِيَهُم بَأْسُنَا ضُحًى وَهُمْ يَلْعَبُونَ
{\tiny\colorbox{cl_aya}{99}} أَفَأَمِنُوا مَكْرَ اللَّهِ فَلَا يَأْمَنُ مَكْرَ اللَّهِ إِلَّا الْقَوْمُ الْخَسِرُونَ
{\tiny\colorbox{cl_aya}{100}} أَوَلَمْ يَهْدِ لِلَّذِينَ يَرِثُونَ الْأَرْضَ مِن بَعْدِ أَهْلِهَا أَن لَّوْ نَشَاءُ أَصَبْنَهُم بِذُنُوبِهِمْ وَنَطْبَعُ عَلَى قُلُوبِهِمْ فَهُمْ لَا يَسْمَعُونَ
{\tiny\colorbox{cl_aya}{101}} تِلْكَ الْقُرَى نَقُصُّ عَلَيْكَ مِنْ أَنبَائِهَا وَلَقَدْ جَاءَتْهُمْ رُسُلُهُم بِالْبَيِّنَتِ فَمَا كَانُوا لِيُؤْمِنُوا بِمَا كَذَّبُوا مِن قَبْلُ كَذَلِكَ يَطْبَعُ اللَّهُ عَلَى قُلُوبِ الْكَفِرِينَ
{\tiny\colorbox{cl_aya}{102}} وَمَا وَجَدْنَا لِأَكْثَرِهِم مِّنْ عَهْدٍ وَإِن وَجَدْنَا أَكْثَرَهُمْ لَفَسِقِينَ
{\tiny\colorbox{cl_aya}{103}} ثُمَّ بَعَثْنَا مِن بَعْدِهِم مُّوسَى بَِٔايَتِنَا إِلَى فِرْعَوْنَ وَمَلَإِيهِ فَظَلَمُوا بِهَا فَانظُرْ كَيْفَ كَانَ عَقِبَةُ الْمُفْسِدِينَ
{\tiny\colorbox{cl_aya}{104}} وَقَالَ مُوسَى يَفِرْعَوْنُ إِنِّى رَسُولٌ مِّن رَّبِّ الْعَلَمِينَ
{\tiny\colorbox{cl_aya}{105}} حَقِيقٌ عَلَى أَن لَّا أَقُولَ عَلَى اللَّهِ إِلَّا الْحَقَّ قَدْ جِئْتُكُم بِبَيِّنَةٍ مِّن رَّبِّكُمْ فَأَرْسِلْ مَعِىَ بَنِى إِسْرَءِيلَ
{\tiny\colorbox{cl_aya}{106}} قَالَ إِن كُنتَ جِئْتَ بَِٔايَةٍ فَأْتِ بِهَا إِن كُنتَ مِنَ الصَّدِقِينَ
{\tiny\colorbox{cl_aya}{107}} فَأَلْقَى عَصَاهُ فَإِذَا هِىَ ثُعْبَانٌ مُّبِينٌ
{\tiny\colorbox{cl_aya}{108}} وَنَزَعَ يَدَهُ فَإِذَا هِىَ بَيْضَاءُ لِلنَّظِرِينَ
{\tiny\colorbox{cl_aya}{109}} قَالَ الْمَلَأُ مِن قَوْمِ فِرْعَوْنَ إِنَّ هَذَا لَسَحِرٌ عَلِيمٌ
{\tiny\colorbox{cl_aya}{110}} يُرِيدُ أَن يُخْرِجَكُم مِّنْ أَرْضِكُمْ فَمَاذَا تَأْمُرُونَ
{\tiny\colorbox{cl_aya}{111}} قَالُوا أَرْجِهْ وَأَخَاهُ وَأَرْسِلْ فِى الْمَدَائِنِ حَشِرِينَ
{\tiny\colorbox{cl_aya}{112}} يَأْتُوكَ بِكُلِّ سَحِرٍ عَلِيمٍ
{\tiny\colorbox{cl_aya}{113}} وَجَاءَ السَّحَرَةُ فِرْعَوْنَ قَالُوا إِنَّ لَنَا لَأَجْرًا إِن كُنَّا نَحْنُ الْغَلِبِينَ
{\tiny\colorbox{cl_aya}{114}} قَالَ نَعَمْ وَإِنَّكُمْ لَمِنَ الْمُقَرَّبِينَ
{\tiny\colorbox{cl_aya}{115}} قَالُوا يَمُوسَى إِمَّا أَن تُلْقِىَ وَإِمَّا أَن نَّكُونَ نَحْنُ الْمُلْقِينَ
{\tiny\colorbox{cl_aya}{116}} قَالَ أَلْقُوا فَلَمَّا أَلْقَوْا سَحَرُوا أَعْيُنَ النَّاسِ وَاسْتَرْهَبُوهُمْ وَجَاءُو بِسِحْرٍ عَظِيمٍ
{\tiny\colorbox{cl_aya}{117}} وَأَوْحَيْنَا إِلَى مُوسَى أَنْ أَلْقِ عَصَاكَ فَإِذَا هِىَ تَلْقَفُ مَا يَأْفِكُونَ
{\tiny\colorbox{cl_aya}{118}} فَوَقَعَ الْحَقُّ وَبَطَلَ مَا كَانُوا يَعْمَلُونَ
{\tiny\colorbox{cl_aya}{119}} فَغُلِبُوا هُنَالِكَ وَانقَلَبُوا صَغِرِينَ
{\tiny\colorbox{cl_aya}{120}} وَأُلْقِىَ السَّحَرَةُ سَجِدِينَ
{\tiny\colorbox{cl_aya}{121}} قَالُوا ءَامَنَّا بِرَبِّ الْعَلَمِينَ
{\tiny\colorbox{cl_aya}{122}} رَبِّ مُوسَى وَهَرُونَ
{\tiny\colorbox{cl_aya}{123}} قَالَ فِرْعَوْنُ ءَامَنتُم بِهِ قَبْلَ أَنْ ءَاذَنَ لَكُمْ إِنَّ هَذَا لَمَكْرٌ مَّكَرْتُمُوهُ فِى الْمَدِينَةِ لِتُخْرِجُوا مِنْهَا أَهْلَهَا فَسَوْفَ تَعْلَمُونَ
{\tiny\colorbox{cl_aya}{124}} لَأُقَطِّعَنَّ أَيْدِيَكُمْ وَأَرْجُلَكُم مِّنْ خِلَفٍ ثُمَّ لَأُصَلِّبَنَّكُمْ أَجْمَعِينَ
{\tiny\colorbox{cl_aya}{125}} قَالُوا إِنَّا إِلَى رَبِّنَا مُنقَلِبُونَ
{\tiny\colorbox{cl_aya}{126}} وَمَا تَنقِمُ مِنَّا إِلَّا أَنْ ءَامَنَّا بَِٔايَتِ رَبِّنَا لَمَّا جَاءَتْنَا رَبَّنَا أَفْرِغْ عَلَيْنَا صَبْرًا وَتَوَفَّنَا مُسْلِمِينَ
{\tiny\colorbox{cl_aya}{127}} وَقَالَ الْمَلَأُ مِن قَوْمِ فِرْعَوْنَ أَتَذَرُ مُوسَى وَقَوْمَهُ لِيُفْسِدُوا فِى الْأَرْضِ وَيَذَرَكَ وَءَالِهَتَكَ قَالَ سَنُقَتِّلُ أَبْنَاءَهُمْ وَنَسْتَحْىِ نِسَاءَهُمْ وَإِنَّا فَوْقَهُمْ قَهِرُونَ
{\tiny\colorbox{cl_aya}{128}} قَالَ مُوسَى لِقَوْمِهِ اسْتَعِينُوا بِاللَّهِ وَاصْبِرُوا إِنَّ الْأَرْضَ لِلَّهِ يُورِثُهَا مَن يَشَاءُ مِنْ عِبَادِهِ وَالْعَقِبَةُ لِلْمُتَّقِينَ
{\tiny\colorbox{cl_aya}{129}} قَالُوا أُوذِينَا مِن قَبْلِ أَن تَأْتِيَنَا وَمِن بَعْدِ مَا جِئْتَنَا قَالَ عَسَى رَبُّكُمْ أَن يُهْلِكَ عَدُوَّكُمْ وَيَسْتَخْلِفَكُمْ فِى الْأَرْضِ فَيَنظُرَ كَيْفَ تَعْمَلُونَ
{\tiny\colorbox{cl_aya}{130}} وَلَقَدْ أَخَذْنَا ءَالَ فِرْعَوْنَ بِالسِّنِينَ وَنَقْصٍ مِّنَ الثَّمَرَتِ لَعَلَّهُمْ يَذَّكَّرُونَ
{\tiny\colorbox{cl_aya}{131}} فَإِذَا جَاءَتْهُمُ الْحَسَنَةُ قَالُوا لَنَا هَذِهِ وَإِن تُصِبْهُمْ سَيِّئَةٌ يَطَّيَّرُوا بِمُوسَى وَمَن مَّعَهُ أَلَا إِنَّمَا طَئِرُهُمْ عِندَ اللَّهِ وَلَكِنَّ أَكْثَرَهُمْ لَا يَعْلَمُونَ
{\tiny\colorbox{cl_aya}{132}} وَقَالُوا مَهْمَا تَأْتِنَا بِهِ مِنْ ءَايَةٍ لِّتَسْحَرَنَا بِهَا فَمَا نَحْنُ لَكَ بِمُؤْمِنِينَ
{\tiny\colorbox{cl_aya}{133}} فَأَرْسَلْنَا عَلَيْهِمُ الطُّوفَانَ وَالْجَرَادَ وَالْقُمَّلَ وَالضَّفَادِعَ وَالدَّمَ ءَايَتٍ مُّفَصَّلَتٍ فَاسْتَكْبَرُوا وَكَانُوا قَوْمًا مُّجْرِمِينَ
{\tiny\colorbox{cl_aya}{134}} وَلَمَّا وَقَعَ عَلَيْهِمُ الرِّجْزُ قَالُوا يَمُوسَى ادْعُ لَنَا رَبَّكَ بِمَا عَهِدَ عِندَكَ لَئِن كَشَفْتَ عَنَّا الرِّجْزَ لَنُؤْمِنَنَّ لَكَ وَلَنُرْسِلَنَّ مَعَكَ بَنِى إِسْرَءِيلَ
{\tiny\colorbox{cl_aya}{135}} فَلَمَّا كَشَفْنَا عَنْهُمُ الرِّجْزَ إِلَى أَجَلٍ هُم بَلِغُوهُ إِذَا هُمْ يَنكُثُونَ
{\tiny\colorbox{cl_aya}{136}} فَانتَقَمْنَا مِنْهُمْ فَأَغْرَقْنَهُمْ فِى الْيَمِّ بِأَنَّهُمْ كَذَّبُوا بَِٔايَتِنَا وَكَانُوا عَنْهَا غَفِلِينَ
{\tiny\colorbox{cl_aya}{137}} وَأَوْرَثْنَا الْقَوْمَ الَّذِينَ كَانُوا يُسْتَضْعَفُونَ مَشَرِقَ الْأَرْضِ وَمَغَرِبَهَا الَّتِى بَرَكْنَا فِيهَا وَتَمَّتْ كَلِمَتُ رَبِّكَ الْحُسْنَى عَلَى بَنِى إِسْرَءِيلَ بِمَا صَبَرُوا وَدَمَّرْنَا مَا كَانَ يَصْنَعُ فِرْعَوْنُ وَقَوْمُهُ وَمَا كَانُوا يَعْرِشُونَ
{\tiny\colorbox{cl_aya}{138}} وَجَوَزْنَا بِبَنِى إِسْرَءِيلَ الْبَحْرَ فَأَتَوْا عَلَى قَوْمٍ يَعْكُفُونَ عَلَى أَصْنَامٍ لَّهُمْ قَالُوا يَمُوسَى اجْعَل لَّنَا إِلَهًا كَمَا لَهُمْ ءَالِهَةٌ قَالَ إِنَّكُمْ قَوْمٌ تَجْهَلُونَ
{\tiny\colorbox{cl_aya}{139}} إِنَّ هَؤُلَاءِ مُتَبَّرٌ مَّا هُمْ فِيهِ وَبَطِلٌ مَّا كَانُوا يَعْمَلُونَ
{\tiny\colorbox{cl_aya}{140}} قَالَ أَغَيْرَ اللَّهِ أَبْغِيكُمْ إِلَهًا وَهُوَ فَضَّلَكُمْ عَلَى الْعَلَمِينَ
{\tiny\colorbox{cl_aya}{141}} وَإِذْ أَنجَيْنَكُم مِّنْ ءَالِ فِرْعَوْنَ يَسُومُونَكُمْ سُوءَ الْعَذَابِ يُقَتِّلُونَ أَبْنَاءَكُمْ وَيَسْتَحْيُونَ نِسَاءَكُمْ وَفِى ذَلِكُم بَلَاءٌ مِّن رَّبِّكُمْ عَظِيمٌ
{\tiny\colorbox{cl_aya}{142}} وَوَعَدْنَا مُوسَى ثَلَثِينَ لَيْلَةً وَأَتْمَمْنَهَا بِعَشْرٍ فَتَمَّ مِيقَتُ رَبِّهِ أَرْبَعِينَ لَيْلَةً وَقَالَ مُوسَى لِأَخِيهِ هَرُونَ اخْلُفْنِى فِى قَوْمِى وَأَصْلِحْ وَلَا تَتَّبِعْ سَبِيلَ الْمُفْسِدِينَ
{\tiny\colorbox{cl_aya}{143}} وَلَمَّا جَاءَ مُوسَى لِمِيقَتِنَا وَكَلَّمَهُ رَبُّهُ قَالَ رَبِّ أَرِنِى أَنظُرْ إِلَيْكَ قَالَ لَن تَرَىنِى وَلَكِنِ انظُرْ إِلَى الْجَبَلِ فَإِنِ اسْتَقَرَّ مَكَانَهُ فَسَوْفَ تَرَىنِى فَلَمَّا تَجَلَّى رَبُّهُ لِلْجَبَلِ جَعَلَهُ دَكًّا وَخَرَّ مُوسَى صَعِقًا فَلَمَّا أَفَاقَ قَالَ سُبْحَنَكَ تُبْتُ إِلَيْكَ وَأَنَا أَوَّلُ الْمُؤْمِنِينَ
{\tiny\colorbox{cl_aya}{144}} قَالَ يَمُوسَى إِنِّى اصْطَفَيْتُكَ عَلَى النَّاسِ بِرِسَلَتِى وَبِكَلَمِى فَخُذْ مَا ءَاتَيْتُكَ وَكُن مِّنَ الشَّكِرِينَ
{\tiny\colorbox{cl_aya}{145}} وَكَتَبْنَا لَهُ فِى الْأَلْوَاحِ مِن كُلِّ شَىْءٍ مَّوْعِظَةً وَتَفْصِيلًا لِّكُلِّ شَىْءٍ فَخُذْهَا بِقُوَّةٍ وَأْمُرْ قَوْمَكَ يَأْخُذُوا بِأَحْسَنِهَا سَأُورِيكُمْ دَارَ الْفَسِقِينَ
{\tiny\colorbox{cl_aya}{146}} سَأَصْرِفُ عَنْ ءَايَتِىَ الَّذِينَ يَتَكَبَّرُونَ فِى الْأَرْضِ بِغَيْرِ الْحَقِّ وَإِن يَرَوْا كُلَّ ءَايَةٍ لَّا يُؤْمِنُوا بِهَا وَإِن يَرَوْا سَبِيلَ الرُّشْدِ لَا يَتَّخِذُوهُ سَبِيلًا وَإِن يَرَوْا سَبِيلَ الْغَىِّ يَتَّخِذُوهُ سَبِيلًا ذَلِكَ بِأَنَّهُمْ كَذَّبُوا بَِٔايَتِنَا وَكَانُوا عَنْهَا غَفِلِينَ
{\tiny\colorbox{cl_aya}{147}} وَالَّذِينَ كَذَّبُوا بَِٔايَتِنَا وَلِقَاءِ الْءَاخِرَةِ حَبِطَتْ أَعْمَلُهُمْ هَلْ يُجْزَوْنَ إِلَّا مَا كَانُوا يَعْمَلُونَ
{\tiny\colorbox{cl_aya}{148}} وَاتَّخَذَ قَوْمُ مُوسَى مِن بَعْدِهِ مِنْ حُلِيِّهِمْ عِجْلًا جَسَدًا لَّهُ خُوَارٌ أَلَمْ يَرَوْا أَنَّهُ لَا يُكَلِّمُهُمْ وَلَا يَهْدِيهِمْ سَبِيلًا اتَّخَذُوهُ وَكَانُوا ظَلِمِينَ
{\tiny\colorbox{cl_aya}{149}} وَلَمَّا سُقِطَ فِى أَيْدِيهِمْ وَرَأَوْا أَنَّهُمْ قَدْ ضَلُّوا قَالُوا لَئِن لَّمْ يَرْحَمْنَا رَبُّنَا وَيَغْفِرْ لَنَا لَنَكُونَنَّ مِنَ الْخَسِرِينَ
{\tiny\colorbox{cl_aya}{150}} وَلَمَّا رَجَعَ مُوسَى إِلَى قَوْمِهِ غَضْبَنَ أَسِفًا قَالَ بِئْسَمَا خَلَفْتُمُونِى مِن بَعْدِى أَعَجِلْتُمْ أَمْرَ رَبِّكُمْ وَأَلْقَى الْأَلْوَاحَ وَأَخَذَ بِرَأْسِ أَخِيهِ يَجُرُّهُ إِلَيْهِ قَالَ ابْنَ أُمَّ إِنَّ الْقَوْمَ اسْتَضْعَفُونِى وَكَادُوا يَقْتُلُونَنِى فَلَا تُشْمِتْ بِىَ الْأَعْدَاءَ وَلَا تَجْعَلْنِى مَعَ الْقَوْمِ الظَّلِمِينَ
{\tiny\colorbox{cl_aya}{151}} قَالَ رَبِّ اغْفِرْ لِى وَلِأَخِى وَأَدْخِلْنَا فِى رَحْمَتِكَ وَأَنتَ أَرْحَمُ الرَّحِمِينَ
{\tiny\colorbox{cl_aya}{152}} إِنَّ الَّذِينَ اتَّخَذُوا الْعِجْلَ سَيَنَالُهُمْ غَضَبٌ مِّن رَّبِّهِمْ وَذِلَّةٌ فِى الْحَيَوةِ الدُّنْيَا وَكَذَلِكَ نَجْزِى الْمُفْتَرِينَ
{\tiny\colorbox{cl_aya}{153}} وَالَّذِينَ عَمِلُوا السَّئَِّاتِ ثُمَّ تَابُوا مِن بَعْدِهَا وَءَامَنُوا إِنَّ رَبَّكَ مِن بَعْدِهَا لَغَفُورٌ رَّحِيمٌ
{\tiny\colorbox{cl_aya}{154}} وَلَمَّا سَكَتَ عَن مُّوسَى الْغَضَبُ أَخَذَ الْأَلْوَاحَ وَفِى نُسْخَتِهَا هُدًى وَرَحْمَةٌ لِّلَّذِينَ هُمْ لِرَبِّهِمْ يَرْهَبُونَ
{\tiny\colorbox{cl_aya}{155}} وَاخْتَارَ مُوسَى قَوْمَهُ سَبْعِينَ رَجُلًا لِّمِيقَتِنَا فَلَمَّا أَخَذَتْهُمُ الرَّجْفَةُ قَالَ رَبِّ لَوْ شِئْتَ أَهْلَكْتَهُم مِّن قَبْلُ وَإِيَّىَ أَتُهْلِكُنَا بِمَا فَعَلَ السُّفَهَاءُ مِنَّا إِنْ هِىَ إِلَّا فِتْنَتُكَ تُضِلُّ بِهَا مَن تَشَاءُ وَتَهْدِى مَن تَشَاءُ أَنتَ وَلِيُّنَا فَاغْفِرْ لَنَا وَارْحَمْنَا وَأَنتَ خَيْرُ الْغَفِرِينَ
{\tiny\colorbox{cl_aya}{156}} وَاكْتُبْ لَنَا فِى هَذِهِ الدُّنْيَا حَسَنَةً وَفِى الْءَاخِرَةِ إِنَّا هُدْنَا إِلَيْكَ قَالَ عَذَابِى أُصِيبُ بِهِ مَنْ أَشَاءُ وَرَحْمَتِى وَسِعَتْ كُلَّ شَىْءٍ فَسَأَكْتُبُهَا لِلَّذِينَ يَتَّقُونَ وَيُؤْتُونَ الزَّكَوةَ وَالَّذِينَ هُم بَِٔايَتِنَا يُؤْمِنُونَ
{\tiny\colorbox{cl_aya}{157}} الَّذِينَ يَتَّبِعُونَ الرَّسُولَ النَّبِىَّ الْأُمِّىَّ الَّذِى يَجِدُونَهُ مَكْتُوبًا عِندَهُمْ فِى التَّوْرَىةِ وَالْإِنجِيلِ يَأْمُرُهُم بِالْمَعْرُوفِ وَيَنْهَىهُمْ عَنِ الْمُنكَرِ وَيُحِلُّ لَهُمُ الطَّيِّبَتِ وَيُحَرِّمُ عَلَيْهِمُ الْخَبَئِثَ وَيَضَعُ عَنْهُمْ إِصْرَهُمْ وَالْأَغْلَلَ الَّتِى كَانَتْ عَلَيْهِمْ فَالَّذِينَ ءَامَنُوا بِهِ وَعَزَّرُوهُ وَنَصَرُوهُ وَاتَّبَعُوا النُّورَ الَّذِى أُنزِلَ مَعَهُ أُولَئِكَ هُمُ الْمُفْلِحُونَ
{\tiny\colorbox{cl_aya}{158}} قُلْ يَأَيُّهَا النَّاسُ إِنِّى رَسُولُ اللَّهِ إِلَيْكُمْ جَمِيعًا الَّذِى لَهُ مُلْكُ السَّمَوَتِ وَالْأَرْضِ لَا إِلَهَ إِلَّا هُوَ يُحْىِ وَيُمِيتُ فََٔامِنُوا بِاللَّهِ وَرَسُولِهِ النَّبِىِّ الْأُمِّىِّ الَّذِى يُؤْمِنُ بِاللَّهِ وَكَلِمَتِهِ وَاتَّبِعُوهُ لَعَلَّكُمْ تَهْتَدُونَ
{\tiny\colorbox{cl_aya}{159}} وَمِن قَوْمِ مُوسَى أُمَّةٌ يَهْدُونَ بِالْحَقِّ وَبِهِ يَعْدِلُونَ
{\tiny\colorbox{cl_aya}{160}} وَقَطَّعْنَهُمُ اثْنَتَىْ عَشْرَةَ أَسْبَاطًا أُمَمًا وَأَوْحَيْنَا إِلَى مُوسَى إِذِ اسْتَسْقَىهُ قَوْمُهُ أَنِ اضْرِب بِّعَصَاكَ الْحَجَرَ فَانبَجَسَتْ مِنْهُ اثْنَتَا عَشْرَةَ عَيْنًا قَدْ عَلِمَ كُلُّ أُنَاسٍ مَّشْرَبَهُمْ وَظَلَّلْنَا عَلَيْهِمُ الْغَمَمَ وَأَنزَلْنَا عَلَيْهِمُ الْمَنَّ وَالسَّلْوَى كُلُوا مِن طَيِّبَتِ مَا رَزَقْنَكُمْ وَمَا ظَلَمُونَا وَلَكِن كَانُوا أَنفُسَهُمْ يَظْلِمُونَ
{\tiny\colorbox{cl_aya}{161}} وَإِذْ قِيلَ لَهُمُ اسْكُنُوا هَذِهِ الْقَرْيَةَ وَكُلُوا مِنْهَا حَيْثُ شِئْتُمْ وَقُولُوا حِطَّةٌ وَادْخُلُوا الْبَابَ سُجَّدًا نَّغْفِرْ لَكُمْ خَطِئَتِكُمْ سَنَزِيدُ الْمُحْسِنِينَ
{\tiny\colorbox{cl_aya}{162}} فَبَدَّلَ الَّذِينَ ظَلَمُوا مِنْهُمْ قَوْلًا غَيْرَ الَّذِى قِيلَ لَهُمْ فَأَرْسَلْنَا عَلَيْهِمْ رِجْزًا مِّنَ السَّمَاءِ بِمَا كَانُوا يَظْلِمُونَ
{\tiny\colorbox{cl_aya}{163}} وَسَْٔلْهُمْ عَنِ الْقَرْيَةِ الَّتِى كَانَتْ حَاضِرَةَ الْبَحْرِ إِذْ يَعْدُونَ فِى السَّبْتِ إِذْ تَأْتِيهِمْ حِيتَانُهُمْ يَوْمَ سَبْتِهِمْ شُرَّعًا وَيَوْمَ لَا يَسْبِتُونَ لَا تَأْتِيهِمْ كَذَلِكَ نَبْلُوهُم بِمَا كَانُوا يَفْسُقُونَ
{\tiny\colorbox{cl_aya}{164}} وَإِذْ قَالَتْ أُمَّةٌ مِّنْهُمْ لِمَ تَعِظُونَ قَوْمًا اللَّهُ مُهْلِكُهُمْ أَوْ مُعَذِّبُهُمْ عَذَابًا شَدِيدًا قَالُوا مَعْذِرَةً إِلَى رَبِّكُمْ وَلَعَلَّهُمْ يَتَّقُونَ
{\tiny\colorbox{cl_aya}{165}} فَلَمَّا نَسُوا مَا ذُكِّرُوا بِهِ أَنجَيْنَا الَّذِينَ يَنْهَوْنَ عَنِ السُّوءِ وَأَخَذْنَا الَّذِينَ ظَلَمُوا بِعَذَابٍ بَِٔيسٍ بِمَا كَانُوا يَفْسُقُونَ
{\tiny\colorbox{cl_aya}{166}} فَلَمَّا عَتَوْا عَن مَّا نُهُوا عَنْهُ قُلْنَا لَهُمْ كُونُوا قِرَدَةً خَسِِٔينَ
{\tiny\colorbox{cl_aya}{167}} وَإِذْ تَأَذَّنَ رَبُّكَ لَيَبْعَثَنَّ عَلَيْهِمْ إِلَى يَوْمِ الْقِيَمَةِ مَن يَسُومُهُمْ سُوءَ الْعَذَابِ إِنَّ رَبَّكَ لَسَرِيعُ الْعِقَابِ وَإِنَّهُ لَغَفُورٌ رَّحِيمٌ
{\tiny\colorbox{cl_aya}{168}} وَقَطَّعْنَهُمْ فِى الْأَرْضِ أُمَمًا مِّنْهُمُ الصَّلِحُونَ وَمِنْهُمْ دُونَ ذَلِكَ وَبَلَوْنَهُم بِالْحَسَنَتِ وَالسَّئَِّاتِ لَعَلَّهُمْ يَرْجِعُونَ
{\tiny\colorbox{cl_aya}{169}} فَخَلَفَ مِن بَعْدِهِمْ خَلْفٌ وَرِثُوا الْكِتَبَ يَأْخُذُونَ عَرَضَ هَذَا الْأَدْنَى وَيَقُولُونَ سَيُغْفَرُ لَنَا وَإِن يَأْتِهِمْ عَرَضٌ مِّثْلُهُ يَأْخُذُوهُ أَلَمْ يُؤْخَذْ عَلَيْهِم مِّيثَقُ الْكِتَبِ أَن لَّا يَقُولُوا عَلَى اللَّهِ إِلَّا الْحَقَّ وَدَرَسُوا مَا فِيهِ وَالدَّارُ الْءَاخِرَةُ خَيْرٌ لِّلَّذِينَ يَتَّقُونَ أَفَلَا تَعْقِلُونَ
{\tiny\colorbox{cl_aya}{170}} وَالَّذِينَ يُمَسِّكُونَ بِالْكِتَبِ وَأَقَامُوا الصَّلَوةَ إِنَّا لَا نُضِيعُ أَجْرَ الْمُصْلِحِينَ
{\tiny\colorbox{cl_aya}{171}} وَإِذْ نَتَقْنَا الْجَبَلَ فَوْقَهُمْ كَأَنَّهُ ظُلَّةٌ وَظَنُّوا أَنَّهُ وَاقِعٌ بِهِمْ خُذُوا مَا ءَاتَيْنَكُم بِقُوَّةٍ وَاذْكُرُوا مَا فِيهِ لَعَلَّكُمْ تَتَّقُونَ
{\tiny\colorbox{cl_aya}{172}} وَإِذْ أَخَذَ رَبُّكَ مِن بَنِى ءَادَمَ مِن ظُهُورِهِمْ ذُرِّيَّتَهُمْ وَأَشْهَدَهُمْ عَلَى أَنفُسِهِمْ أَلَسْتُ بِرَبِّكُمْ قَالُوا بَلَى شَهِدْنَا أَن تَقُولُوا يَوْمَ الْقِيَمَةِ إِنَّا كُنَّا عَنْ هَذَا غَفِلِينَ
{\tiny\colorbox{cl_aya}{173}} أَوْ تَقُولُوا إِنَّمَا أَشْرَكَ ءَابَاؤُنَا مِن قَبْلُ وَكُنَّا ذُرِّيَّةً مِّن بَعْدِهِمْ أَفَتُهْلِكُنَا بِمَا فَعَلَ الْمُبْطِلُونَ
{\tiny\colorbox{cl_aya}{174}} وَكَذَلِكَ نُفَصِّلُ الْءَايَتِ وَلَعَلَّهُمْ يَرْجِعُونَ
{\tiny\colorbox{cl_aya}{175}} وَاتْلُ عَلَيْهِمْ نَبَأَ الَّذِى ءَاتَيْنَهُ ءَايَتِنَا فَانسَلَخَ مِنْهَا فَأَتْبَعَهُ الشَّيْطَنُ فَكَانَ مِنَ الْغَاوِينَ
{\tiny\colorbox{cl_aya}{176}} وَلَوْ شِئْنَا لَرَفَعْنَهُ بِهَا وَلَكِنَّهُ أَخْلَدَ إِلَى الْأَرْضِ وَاتَّبَعَ هَوَىهُ فَمَثَلُهُ كَمَثَلِ الْكَلْبِ إِن تَحْمِلْ عَلَيْهِ يَلْهَثْ أَوْ تَتْرُكْهُ يَلْهَث ذَّلِكَ مَثَلُ الْقَوْمِ الَّذِينَ كَذَّبُوا بَِٔايَتِنَا فَاقْصُصِ الْقَصَصَ لَعَلَّهُمْ يَتَفَكَّرُونَ
{\tiny\colorbox{cl_aya}{177}} سَاءَ مَثَلًا الْقَوْمُ الَّذِينَ كَذَّبُوا بَِٔايَتِنَا وَأَنفُسَهُمْ كَانُوا يَظْلِمُونَ
{\tiny\colorbox{cl_aya}{178}} مَن يَهْدِ اللَّهُ فَهُوَ الْمُهْتَدِى وَمَن يُضْلِلْ فَأُولَئِكَ هُمُ الْخَسِرُونَ
{\tiny\colorbox{cl_aya}{179}} وَلَقَدْ ذَرَأْنَا لِجَهَنَّمَ كَثِيرًا مِّنَ الْجِنِّ وَالْإِنسِ لَهُمْ قُلُوبٌ لَّا يَفْقَهُونَ بِهَا وَلَهُمْ أَعْيُنٌ لَّا يُبْصِرُونَ بِهَا وَلَهُمْ ءَاذَانٌ لَّا يَسْمَعُونَ بِهَا أُولَئِكَ كَالْأَنْعَمِ بَلْ هُمْ أَضَلُّ أُولَئِكَ هُمُ الْغَفِلُونَ
{\tiny\colorbox{cl_aya}{180}} وَلِلَّهِ الْأَسْمَاءُ الْحُسْنَى فَادْعُوهُ بِهَا وَذَرُوا الَّذِينَ يُلْحِدُونَ فِى أَسْمَئِهِ سَيُجْزَوْنَ مَا كَانُوا يَعْمَلُونَ
{\tiny\colorbox{cl_aya}{181}} وَمِمَّنْ خَلَقْنَا أُمَّةٌ يَهْدُونَ بِالْحَقِّ وَبِهِ يَعْدِلُونَ
{\tiny\colorbox{cl_aya}{182}} وَالَّذِينَ كَذَّبُوا بَِٔايَتِنَا سَنَسْتَدْرِجُهُم مِّنْ حَيْثُ لَا يَعْلَمُونَ
{\tiny\colorbox{cl_aya}{183}} وَأُمْلِى لَهُمْ إِنَّ كَيْدِى مَتِينٌ
{\tiny\colorbox{cl_aya}{184}} أَوَلَمْ يَتَفَكَّرُوا مَا بِصَاحِبِهِم مِّن جِنَّةٍ إِنْ هُوَ إِلَّا نَذِيرٌ مُّبِينٌ
{\tiny\colorbox{cl_aya}{185}} أَوَلَمْ يَنظُرُوا فِى مَلَكُوتِ السَّمَوَتِ وَالْأَرْضِ وَمَا خَلَقَ اللَّهُ مِن شَىْءٍ وَأَنْ عَسَى أَن يَكُونَ قَدِ اقْتَرَبَ أَجَلُهُمْ فَبِأَىِّ حَدِيثٍ بَعْدَهُ يُؤْمِنُونَ
{\tiny\colorbox{cl_aya}{186}} مَن يُضْلِلِ اللَّهُ فَلَا هَادِىَ لَهُ وَيَذَرُهُمْ فِى طُغْيَنِهِمْ يَعْمَهُونَ
{\tiny\colorbox{cl_aya}{187}} يَسَْٔلُونَكَ عَنِ السَّاعَةِ أَيَّانَ مُرْسَىهَا قُلْ إِنَّمَا عِلْمُهَا عِندَ رَبِّى لَا يُجَلِّيهَا لِوَقْتِهَا إِلَّا هُوَ ثَقُلَتْ فِى السَّمَوَتِ وَالْأَرْضِ لَا تَأْتِيكُمْ إِلَّا بَغْتَةً يَسَْٔلُونَكَ كَأَنَّكَ حَفِىٌّ عَنْهَا قُلْ إِنَّمَا عِلْمُهَا عِندَ اللَّهِ وَلَكِنَّ أَكْثَرَ النَّاسِ لَا يَعْلَمُونَ
{\tiny\colorbox{cl_aya}{188}} قُل لَّا أَمْلِكُ لِنَفْسِى نَفْعًا وَلَا ضَرًّا إِلَّا مَا شَاءَ اللَّهُ وَلَوْ كُنتُ أَعْلَمُ الْغَيْبَ لَاسْتَكْثَرْتُ مِنَ الْخَيْرِ وَمَا مَسَّنِىَ السُّوءُ إِنْ أَنَا إِلَّا نَذِيرٌ وَبَشِيرٌ لِّقَوْمٍ يُؤْمِنُونَ
{\tiny\colorbox{cl_aya}{189}} هُوَ الَّذِى خَلَقَكُم مِّن نَّفْسٍ وَحِدَةٍ وَجَعَلَ مِنْهَا زَوْجَهَا لِيَسْكُنَ إِلَيْهَا فَلَمَّا تَغَشَّىهَا حَمَلَتْ حَمْلًا خَفِيفًا فَمَرَّتْ بِهِ فَلَمَّا أَثْقَلَت دَّعَوَا اللَّهَ رَبَّهُمَا لَئِنْ ءَاتَيْتَنَا صَلِحًا لَّنَكُونَنَّ مِنَ الشَّكِرِينَ
{\tiny\colorbox{cl_aya}{190}} فَلَمَّا ءَاتَىهُمَا صَلِحًا جَعَلَا لَهُ شُرَكَاءَ فِيمَا ءَاتَىهُمَا فَتَعَلَى اللَّهُ عَمَّا يُشْرِكُونَ
{\tiny\colorbox{cl_aya}{191}} أَيُشْرِكُونَ مَا لَا يَخْلُقُ شَئًْا وَهُمْ يُخْلَقُونَ
{\tiny\colorbox{cl_aya}{192}} وَلَا يَسْتَطِيعُونَ لَهُمْ نَصْرًا وَلَا أَنفُسَهُمْ يَنصُرُونَ
{\tiny\colorbox{cl_aya}{193}} وَإِن تَدْعُوهُمْ إِلَى الْهُدَى لَا يَتَّبِعُوكُمْ سَوَاءٌ عَلَيْكُمْ أَدَعَوْتُمُوهُمْ أَمْ أَنتُمْ صَمِتُونَ
{\tiny\colorbox{cl_aya}{194}} إِنَّ الَّذِينَ تَدْعُونَ مِن دُونِ اللَّهِ عِبَادٌ أَمْثَالُكُمْ فَادْعُوهُمْ فَلْيَسْتَجِيبُوا لَكُمْ إِن كُنتُمْ صَدِقِينَ
{\tiny\colorbox{cl_aya}{195}} أَلَهُمْ أَرْجُلٌ يَمْشُونَ بِهَا أَمْ لَهُمْ أَيْدٍ يَبْطِشُونَ بِهَا أَمْ لَهُمْ أَعْيُنٌ يُبْصِرُونَ بِهَا أَمْ لَهُمْ ءَاذَانٌ يَسْمَعُونَ بِهَا قُلِ ادْعُوا شُرَكَاءَكُمْ ثُمَّ كِيدُونِ فَلَا تُنظِرُونِ
{\tiny\colorbox{cl_aya}{196}} إِنَّ وَلِِّىَ اللَّهُ الَّذِى نَزَّلَ الْكِتَبَ وَهُوَ يَتَوَلَّى الصَّلِحِينَ
{\tiny\colorbox{cl_aya}{197}} وَالَّذِينَ تَدْعُونَ مِن دُونِهِ لَا يَسْتَطِيعُونَ نَصْرَكُمْ وَلَا أَنفُسَهُمْ يَنصُرُونَ
{\tiny\colorbox{cl_aya}{198}} وَإِن تَدْعُوهُمْ إِلَى الْهُدَى لَا يَسْمَعُوا وَتَرَىهُمْ يَنظُرُونَ إِلَيْكَ وَهُمْ لَا يُبْصِرُونَ
{\tiny\colorbox{cl_aya}{199}} خُذِ الْعَفْوَ وَأْمُرْ بِالْعُرْفِ وَأَعْرِضْ عَنِ الْجَهِلِينَ
{\tiny\colorbox{cl_aya}{200}} وَإِمَّا يَنزَغَنَّكَ مِنَ الشَّيْطَنِ نَزْغٌ فَاسْتَعِذْ بِاللَّهِ إِنَّهُ سَمِيعٌ عَلِيمٌ
{\tiny\colorbox{cl_aya}{201}} إِنَّ الَّذِينَ اتَّقَوْا إِذَا مَسَّهُمْ طَئِفٌ مِّنَ الشَّيْطَنِ تَذَكَّرُوا فَإِذَا هُم مُّبْصِرُونَ
{\tiny\colorbox{cl_aya}{202}} وَإِخْوَنُهُمْ يَمُدُّونَهُمْ فِى الْغَىِّ ثُمَّ لَا يُقْصِرُونَ
{\tiny\colorbox{cl_aya}{203}} وَإِذَا لَمْ تَأْتِهِم بَِٔايَةٍ قَالُوا لَوْلَا اجْتَبَيْتَهَا قُلْ إِنَّمَا أَتَّبِعُ مَا يُوحَى إِلَىَّ مِن رَّبِّى هَذَا بَصَائِرُ مِن رَّبِّكُمْ وَهُدًى وَرَحْمَةٌ لِّقَوْمٍ يُؤْمِنُونَ
{\tiny\colorbox{cl_aya}{204}} وَإِذَا قُرِئَ الْقُرْءَانُ فَاسْتَمِعُوا لَهُ وَأَنصِتُوا لَعَلَّكُمْ تُرْحَمُونَ
{\tiny\colorbox{cl_aya}{205}} وَاذْكُر رَّبَّكَ فِى نَفْسِكَ تَضَرُّعًا وَخِيفَةً وَدُونَ الْجَهْرِ مِنَ الْقَوْلِ بِالْغُدُوِّ وَالْءَاصَالِ وَلَا تَكُن مِّنَ الْغَفِلِينَ
{\tiny\colorbox{cl_aya}{206}} إِنَّ الَّذِينَ عِندَ رَبِّكَ لَا يَسْتَكْبِرُونَ عَنْ عِبَادَتِهِ وَيُسَبِّحُونَهُ وَلَهُ يَسْجُدُونَ
\end{document}