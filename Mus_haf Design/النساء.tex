%\documentclass[12pt,a4paper]{article}
\documentclass[20pt,a4paper]{article}
\usepackage[margin=0.5in]{geometry}
\usepackage{polyglossia}
\usepackage[dvipsnames]{xcolor}
\pagenumbering{gobble}
% This beautiful one line disable the initial spacing at the beginning of a line
\usepackage[parfill]{parskip} 
\usepackage{setspace}
\setstretch{2}

\setdefaultlanguage[numerals=maghrib]{arabic}
\newfontfamily\arabicfont[Script=Arabic]{Amiri}

\title{}
\author{}
\date{}
\definecolor{cl_page}{gray}{0.98}
\definecolor{cl_aya}{HTML}{DEEEFF}

\begin{document}
\pagecolor{cl_page}

% Start %


{\tiny\colorbox{cl_aya}{1}} يَأَيُّهَا النَّاسُ اتَّقُوا رَبَّكُمُ الَّذِى خَلَقَكُم مِّن نَّفْسٍ وَحِدَةٍ وَخَلَقَ مِنْهَا زَوْجَهَا وَبَثَّ مِنْهُمَا رِجَالًا كَثِيرًا وَنِسَاءً وَاتَّقُوا اللَّهَ الَّذِى تَسَاءَلُونَ بِهِ وَالْأَرْحَامَ إِنَّ اللَّهَ كَانَ عَلَيْكُمْ رَقِيبًا
{\tiny\colorbox{cl_aya}{2}} وَءَاتُوا الْيَتَمَى أَمْوَلَهُمْ وَلَا تَتَبَدَّلُوا الْخَبِيثَ بِالطَّيِّبِ وَلَا تَأْكُلُوا أَمْوَلَهُمْ إِلَى أَمْوَلِكُمْ إِنَّهُ كَانَ حُوبًا كَبِيرًا
{\tiny\colorbox{cl_aya}{3}} وَإِنْ خِفْتُمْ أَلَّا تُقْسِطُوا فِى الْيَتَمَى فَانكِحُوا مَا طَابَ لَكُم مِّنَ النِّسَاءِ مَثْنَى وَثُلَثَ وَرُبَعَ فَإِنْ خِفْتُمْ أَلَّا تَعْدِلُوا فَوَحِدَةً أَوْ مَا مَلَكَتْ أَيْمَنُكُمْ ذَلِكَ أَدْنَى أَلَّا تَعُولُوا
{\tiny\colorbox{cl_aya}{4}} وَءَاتُوا النِّسَاءَ صَدُقَتِهِنَّ نِحْلَةً فَإِن طِبْنَ لَكُمْ عَن شَىْءٍ مِّنْهُ نَفْسًا فَكُلُوهُ هَنِئًا مَّرِئًا
{\tiny\colorbox{cl_aya}{5}} وَلَا تُؤْتُوا السُّفَهَاءَ أَمْوَلَكُمُ الَّتِى جَعَلَ اللَّهُ لَكُمْ قِيَمًا وَارْزُقُوهُمْ فِيهَا وَاكْسُوهُمْ وَقُولُوا لَهُمْ قَوْلًا مَّعْرُوفًا
{\tiny\colorbox{cl_aya}{6}} وَابْتَلُوا الْيَتَمَى حَتَّى إِذَا بَلَغُوا النِّكَاحَ فَإِنْ ءَانَسْتُم مِّنْهُمْ رُشْدًا فَادْفَعُوا إِلَيْهِمْ أَمْوَلَهُمْ وَلَا تَأْكُلُوهَا إِسْرَافًا وَبِدَارًا أَن يَكْبَرُوا وَمَن كَانَ غَنِيًّا فَلْيَسْتَعْفِفْ وَمَن كَانَ فَقِيرًا فَلْيَأْكُلْ بِالْمَعْرُوفِ فَإِذَا دَفَعْتُمْ إِلَيْهِمْ أَمْوَلَهُمْ فَأَشْهِدُوا عَلَيْهِمْ وَكَفَى بِاللَّهِ حَسِيبًا
{\tiny\colorbox{cl_aya}{7}} لِّلرِّجَالِ نَصِيبٌ مِّمَّا تَرَكَ الْوَلِدَانِ وَالْأَقْرَبُونَ وَلِلنِّسَاءِ نَصِيبٌ مِّمَّا تَرَكَ الْوَلِدَانِ وَالْأَقْرَبُونَ مِمَّا قَلَّ مِنْهُ أَوْ كَثُرَ نَصِيبًا مَّفْرُوضًا
{\tiny\colorbox{cl_aya}{8}} وَإِذَا حَضَرَ الْقِسْمَةَ أُولُوا الْقُرْبَى وَالْيَتَمَى وَالْمَسَكِينُ فَارْزُقُوهُم مِّنْهُ وَقُولُوا لَهُمْ قَوْلًا مَّعْرُوفًا
{\tiny\colorbox{cl_aya}{9}} وَلْيَخْشَ الَّذِينَ لَوْ تَرَكُوا مِنْ خَلْفِهِمْ ذُرِّيَّةً ضِعَفًا خَافُوا عَلَيْهِمْ فَلْيَتَّقُوا اللَّهَ وَلْيَقُولُوا قَوْلًا سَدِيدًا
{\tiny\colorbox{cl_aya}{10}} إِنَّ الَّذِينَ يَأْكُلُونَ أَمْوَلَ الْيَتَمَى ظُلْمًا إِنَّمَا يَأْكُلُونَ فِى بُطُونِهِمْ نَارًا وَسَيَصْلَوْنَ سَعِيرًا
{\tiny\colorbox{cl_aya}{11}} يُوصِيكُمُ اللَّهُ فِى أَوْلَدِكُمْ لِلذَّكَرِ مِثْلُ حَظِّ الْأُنثَيَيْنِ فَإِن كُنَّ نِسَاءً فَوْقَ اثْنَتَيْنِ فَلَهُنَّ ثُلُثَا مَا تَرَكَ وَإِن كَانَتْ وَحِدَةً فَلَهَا النِّصْفُ وَلِأَبَوَيْهِ لِكُلِّ وَحِدٍ مِّنْهُمَا السُّدُسُ مِمَّا تَرَكَ إِن كَانَ لَهُ وَلَدٌ فَإِن لَّمْ يَكُن لَّهُ وَلَدٌ وَوَرِثَهُ أَبَوَاهُ فَلِأُمِّهِ الثُّلُثُ فَإِن كَانَ لَهُ إِخْوَةٌ فَلِأُمِّهِ السُّدُسُ مِن بَعْدِ وَصِيَّةٍ يُوصِى بِهَا أَوْ دَيْنٍ ءَابَاؤُكُمْ وَأَبْنَاؤُكُمْ لَا تَدْرُونَ أَيُّهُمْ أَقْرَبُ لَكُمْ نَفْعًا فَرِيضَةً مِّنَ اللَّهِ إِنَّ اللَّهَ كَانَ عَلِيمًا حَكِيمًا
{\tiny\colorbox{cl_aya}{12}} وَلَكُمْ نِصْفُ مَا تَرَكَ أَزْوَجُكُمْ إِن لَّمْ يَكُن لَّهُنَّ وَلَدٌ فَإِن كَانَ لَهُنَّ وَلَدٌ فَلَكُمُ الرُّبُعُ مِمَّا تَرَكْنَ مِن بَعْدِ وَصِيَّةٍ يُوصِينَ بِهَا أَوْ دَيْنٍ وَلَهُنَّ الرُّبُعُ مِمَّا تَرَكْتُمْ إِن لَّمْ يَكُن لَّكُمْ وَلَدٌ فَإِن كَانَ لَكُمْ وَلَدٌ فَلَهُنَّ الثُّمُنُ مِمَّا تَرَكْتُم مِّن بَعْدِ وَصِيَّةٍ تُوصُونَ بِهَا أَوْ دَيْنٍ وَإِن كَانَ رَجُلٌ يُورَثُ كَلَلَةً أَوِ امْرَأَةٌ وَلَهُ أَخٌ أَوْ أُخْتٌ فَلِكُلِّ وَحِدٍ مِّنْهُمَا السُّدُسُ فَإِن كَانُوا أَكْثَرَ مِن ذَلِكَ فَهُمْ شُرَكَاءُ فِى الثُّلُثِ مِن بَعْدِ وَصِيَّةٍ يُوصَى بِهَا أَوْ دَيْنٍ غَيْرَ مُضَارٍّ وَصِيَّةً مِّنَ اللَّهِ وَاللَّهُ عَلِيمٌ حَلِيمٌ
{\tiny\colorbox{cl_aya}{13}} تِلْكَ حُدُودُ اللَّهِ وَمَن يُطِعِ اللَّهَ وَرَسُولَهُ يُدْخِلْهُ جَنَّتٍ تَجْرِى مِن تَحْتِهَا الْأَنْهَرُ خَلِدِينَ فِيهَا وَذَلِكَ الْفَوْزُ الْعَظِيمُ
{\tiny\colorbox{cl_aya}{14}} وَمَن يَعْصِ اللَّهَ وَرَسُولَهُ وَيَتَعَدَّ حُدُودَهُ يُدْخِلْهُ نَارًا خَلِدًا فِيهَا وَلَهُ عَذَابٌ مُّهِينٌ
{\tiny\colorbox{cl_aya}{15}} وَالَّتِى يَأْتِينَ الْفَحِشَةَ مِن نِّسَائِكُمْ فَاسْتَشْهِدُوا عَلَيْهِنَّ أَرْبَعَةً مِّنكُمْ فَإِن شَهِدُوا فَأَمْسِكُوهُنَّ فِى الْبُيُوتِ حَتَّى يَتَوَفَّىهُنَّ الْمَوْتُ أَوْ يَجْعَلَ اللَّهُ لَهُنَّ سَبِيلًا
{\tiny\colorbox{cl_aya}{16}} وَالَّذَانِ يَأْتِيَنِهَا مِنكُمْ فََٔاذُوهُمَا فَإِن تَابَا وَأَصْلَحَا فَأَعْرِضُوا عَنْهُمَا إِنَّ اللَّهَ كَانَ تَوَّابًا رَّحِيمًا
{\tiny\colorbox{cl_aya}{17}} إِنَّمَا التَّوْبَةُ عَلَى اللَّهِ لِلَّذِينَ يَعْمَلُونَ السُّوءَ بِجَهَلَةٍ ثُمَّ يَتُوبُونَ مِن قَرِيبٍ فَأُولَئِكَ يَتُوبُ اللَّهُ عَلَيْهِمْ وَكَانَ اللَّهُ عَلِيمًا حَكِيمًا
{\tiny\colorbox{cl_aya}{18}} وَلَيْسَتِ التَّوْبَةُ لِلَّذِينَ يَعْمَلُونَ السَّئَِّاتِ حَتَّى إِذَا حَضَرَ أَحَدَهُمُ الْمَوْتُ قَالَ إِنِّى تُبْتُ الَْٔنَ وَلَا الَّذِينَ يَمُوتُونَ وَهُمْ كُفَّارٌ أُولَئِكَ أَعْتَدْنَا لَهُمْ عَذَابًا أَلِيمًا
{\tiny\colorbox{cl_aya}{19}} يَأَيُّهَا الَّذِينَ ءَامَنُوا لَا يَحِلُّ لَكُمْ أَن تَرِثُوا النِّسَاءَ كَرْهًا وَلَا تَعْضُلُوهُنَّ لِتَذْهَبُوا بِبَعْضِ مَا ءَاتَيْتُمُوهُنَّ إِلَّا أَن يَأْتِينَ بِفَحِشَةٍ مُّبَيِّنَةٍ وَعَاشِرُوهُنَّ بِالْمَعْرُوفِ فَإِن كَرِهْتُمُوهُنَّ فَعَسَى أَن تَكْرَهُوا شَئًْا وَيَجْعَلَ اللَّهُ فِيهِ خَيْرًا كَثِيرًا
{\tiny\colorbox{cl_aya}{20}} وَإِنْ أَرَدتُّمُ اسْتِبْدَالَ زَوْجٍ مَّكَانَ زَوْجٍ وَءَاتَيْتُمْ إِحْدَىهُنَّ قِنطَارًا فَلَا تَأْخُذُوا مِنْهُ شَئًْا أَتَأْخُذُونَهُ بُهْتَنًا وَإِثْمًا مُّبِينًا
{\tiny\colorbox{cl_aya}{21}} وَكَيْفَ تَأْخُذُونَهُ وَقَدْ أَفْضَى بَعْضُكُمْ إِلَى بَعْضٍ وَأَخَذْنَ مِنكُم مِّيثَقًا غَلِيظًا
{\tiny\colorbox{cl_aya}{22}} وَلَا تَنكِحُوا مَا نَكَحَ ءَابَاؤُكُم مِّنَ النِّسَاءِ إِلَّا مَا قَدْ سَلَفَ إِنَّهُ كَانَ فَحِشَةً وَمَقْتًا وَسَاءَ سَبِيلًا
{\tiny\colorbox{cl_aya}{23}} حُرِّمَتْ عَلَيْكُمْ أُمَّهَتُكُمْ وَبَنَاتُكُمْ وَأَخَوَتُكُمْ وَعَمَّتُكُمْ وَخَلَتُكُمْ وَبَنَاتُ الْأَخِ وَبَنَاتُ الْأُخْتِ وَأُمَّهَتُكُمُ الَّتِى أَرْضَعْنَكُمْ وَأَخَوَتُكُم مِّنَ الرَّضَعَةِ وَأُمَّهَتُ نِسَائِكُمْ وَرَبَئِبُكُمُ الَّتِى فِى حُجُورِكُم مِّن نِّسَائِكُمُ الَّتِى دَخَلْتُم بِهِنَّ فَإِن لَّمْ تَكُونُوا دَخَلْتُم بِهِنَّ فَلَا جُنَاحَ عَلَيْكُمْ وَحَلَئِلُ أَبْنَائِكُمُ الَّذِينَ مِنْ أَصْلَبِكُمْ وَأَن تَجْمَعُوا بَيْنَ الْأُخْتَيْنِ إِلَّا مَا قَدْ سَلَفَ إِنَّ اللَّهَ كَانَ غَفُورًا رَّحِيمًا
{\tiny\colorbox{cl_aya}{24}} وَالْمُحْصَنَتُ مِنَ النِّسَاءِ إِلَّا مَا مَلَكَتْ أَيْمَنُكُمْ كِتَبَ اللَّهِ عَلَيْكُمْ وَأُحِلَّ لَكُم مَّا وَرَاءَ ذَلِكُمْ أَن تَبْتَغُوا بِأَمْوَلِكُم مُّحْصِنِينَ غَيْرَ مُسَفِحِينَ فَمَا اسْتَمْتَعْتُم بِهِ مِنْهُنَّ فََٔاتُوهُنَّ أُجُورَهُنَّ فَرِيضَةً وَلَا جُنَاحَ عَلَيْكُمْ فِيمَا تَرَضَيْتُم بِهِ مِن بَعْدِ الْفَرِيضَةِ إِنَّ اللَّهَ كَانَ عَلِيمًا حَكِيمًا
{\tiny\colorbox{cl_aya}{25}} وَمَن لَّمْ يَسْتَطِعْ مِنكُمْ طَوْلًا أَن يَنكِحَ الْمُحْصَنَتِ الْمُؤْمِنَتِ فَمِن مَّا مَلَكَتْ أَيْمَنُكُم مِّن فَتَيَتِكُمُ الْمُؤْمِنَتِ وَاللَّهُ أَعْلَمُ بِإِيمَنِكُم بَعْضُكُم مِّن بَعْضٍ فَانكِحُوهُنَّ بِإِذْنِ أَهْلِهِنَّ وَءَاتُوهُنَّ أُجُورَهُنَّ بِالْمَعْرُوفِ مُحْصَنَتٍ غَيْرَ مُسَفِحَتٍ وَلَا مُتَّخِذَتِ أَخْدَانٍ فَإِذَا أُحْصِنَّ فَإِنْ أَتَيْنَ بِفَحِشَةٍ فَعَلَيْهِنَّ نِصْفُ مَا عَلَى الْمُحْصَنَتِ مِنَ الْعَذَابِ ذَلِكَ لِمَنْ خَشِىَ الْعَنَتَ مِنكُمْ وَأَن تَصْبِرُوا خَيْرٌ لَّكُمْ وَاللَّهُ غَفُورٌ رَّحِيمٌ
{\tiny\colorbox{cl_aya}{26}} يُرِيدُ اللَّهُ لِيُبَيِّنَ لَكُمْ وَيَهْدِيَكُمْ سُنَنَ الَّذِينَ مِن قَبْلِكُمْ وَيَتُوبَ عَلَيْكُمْ وَاللَّهُ عَلِيمٌ حَكِيمٌ
{\tiny\colorbox{cl_aya}{27}} وَاللَّهُ يُرِيدُ أَن يَتُوبَ عَلَيْكُمْ وَيُرِيدُ الَّذِينَ يَتَّبِعُونَ الشَّهَوَتِ أَن تَمِيلُوا مَيْلًا عَظِيمًا
{\tiny\colorbox{cl_aya}{28}} يُرِيدُ اللَّهُ أَن يُخَفِّفَ عَنكُمْ وَخُلِقَ الْإِنسَنُ ضَعِيفًا
{\tiny\colorbox{cl_aya}{29}} يَأَيُّهَا الَّذِينَ ءَامَنُوا لَا تَأْكُلُوا أَمْوَلَكُم بَيْنَكُم بِالْبَطِلِ إِلَّا أَن تَكُونَ تِجَرَةً عَن تَرَاضٍ مِّنكُمْ وَلَا تَقْتُلُوا أَنفُسَكُمْ إِنَّ اللَّهَ كَانَ بِكُمْ رَحِيمًا
{\tiny\colorbox{cl_aya}{30}} وَمَن يَفْعَلْ ذَلِكَ عُدْوَنًا وَظُلْمًا فَسَوْفَ نُصْلِيهِ نَارًا وَكَانَ ذَلِكَ عَلَى اللَّهِ يَسِيرًا
{\tiny\colorbox{cl_aya}{31}} إِن تَجْتَنِبُوا كَبَائِرَ مَا تُنْهَوْنَ عَنْهُ نُكَفِّرْ عَنكُمْ سَئَِّاتِكُمْ وَنُدْخِلْكُم مُّدْخَلًا كَرِيمًا
{\tiny\colorbox{cl_aya}{32}} وَلَا تَتَمَنَّوْا مَا فَضَّلَ اللَّهُ بِهِ بَعْضَكُمْ عَلَى بَعْضٍ لِّلرِّجَالِ نَصِيبٌ مِّمَّا اكْتَسَبُوا وَلِلنِّسَاءِ نَصِيبٌ مِّمَّا اكْتَسَبْنَ وَسَْٔلُوا اللَّهَ مِن فَضْلِهِ إِنَّ اللَّهَ كَانَ بِكُلِّ شَىْءٍ عَلِيمًا
{\tiny\colorbox{cl_aya}{33}} وَلِكُلٍّ جَعَلْنَا مَوَلِىَ مِمَّا تَرَكَ الْوَلِدَانِ وَالْأَقْرَبُونَ وَالَّذِينَ عَقَدَتْ أَيْمَنُكُمْ فََٔاتُوهُمْ نَصِيبَهُمْ إِنَّ اللَّهَ كَانَ عَلَى كُلِّ شَىْءٍ شَهِيدًا
{\tiny\colorbox{cl_aya}{34}} الرِّجَالُ قَوَّمُونَ عَلَى النِّسَاءِ بِمَا فَضَّلَ اللَّهُ بَعْضَهُمْ عَلَى بَعْضٍ وَبِمَا أَنفَقُوا مِنْ أَمْوَلِهِمْ فَالصَّلِحَتُ قَنِتَتٌ حَفِظَتٌ لِّلْغَيْبِ بِمَا حَفِظَ اللَّهُ وَالَّتِى تَخَافُونَ نُشُوزَهُنَّ فَعِظُوهُنَّ وَاهْجُرُوهُنَّ فِى الْمَضَاجِعِ وَاضْرِبُوهُنَّ فَإِنْ أَطَعْنَكُمْ فَلَا تَبْغُوا عَلَيْهِنَّ سَبِيلًا إِنَّ اللَّهَ كَانَ عَلِيًّا كَبِيرًا
{\tiny\colorbox{cl_aya}{35}} وَإِنْ خِفْتُمْ شِقَاقَ بَيْنِهِمَا فَابْعَثُوا حَكَمًا مِّنْ أَهْلِهِ وَحَكَمًا مِّنْ أَهْلِهَا إِن يُرِيدَا إِصْلَحًا يُوَفِّقِ اللَّهُ بَيْنَهُمَا إِنَّ اللَّهَ كَانَ عَلِيمًا خَبِيرًا
{\tiny\colorbox{cl_aya}{36}} وَاعْبُدُوا اللَّهَ وَلَا تُشْرِكُوا بِهِ شَئًْا وَبِالْوَلِدَيْنِ إِحْسَنًا وَبِذِى الْقُرْبَى وَالْيَتَمَى وَالْمَسَكِينِ وَالْجَارِ ذِى الْقُرْبَى وَالْجَارِ الْجُنُبِ وَالصَّاحِبِ بِالْجَنبِ وَابْنِ السَّبِيلِ وَمَا مَلَكَتْ أَيْمَنُكُمْ إِنَّ اللَّهَ لَا يُحِبُّ مَن كَانَ مُخْتَالًا فَخُورًا
{\tiny\colorbox{cl_aya}{37}} الَّذِينَ يَبْخَلُونَ وَيَأْمُرُونَ النَّاسَ بِالْبُخْلِ وَيَكْتُمُونَ مَا ءَاتَىهُمُ اللَّهُ مِن فَضْلِهِ وَأَعْتَدْنَا لِلْكَفِرِينَ عَذَابًا مُّهِينًا
{\tiny\colorbox{cl_aya}{38}} وَالَّذِينَ يُنفِقُونَ أَمْوَلَهُمْ رِئَاءَ النَّاسِ وَلَا يُؤْمِنُونَ بِاللَّهِ وَلَا بِالْيَوْمِ الْءَاخِرِ وَمَن يَكُنِ الشَّيْطَنُ لَهُ قَرِينًا فَسَاءَ قَرِينًا
{\tiny\colorbox{cl_aya}{39}} وَمَاذَا عَلَيْهِمْ لَوْ ءَامَنُوا بِاللَّهِ وَالْيَوْمِ الْءَاخِرِ وَأَنفَقُوا مِمَّا رَزَقَهُمُ اللَّهُ وَكَانَ اللَّهُ بِهِمْ عَلِيمًا
{\tiny\colorbox{cl_aya}{40}} إِنَّ اللَّهَ لَا يَظْلِمُ مِثْقَالَ ذَرَّةٍ وَإِن تَكُ حَسَنَةً يُضَعِفْهَا وَيُؤْتِ مِن لَّدُنْهُ أَجْرًا عَظِيمًا
{\tiny\colorbox{cl_aya}{41}} فَكَيْفَ إِذَا جِئْنَا مِن كُلِّ أُمَّةٍ بِشَهِيدٍ وَجِئْنَا بِكَ عَلَى هَؤُلَاءِ شَهِيدًا
{\tiny\colorbox{cl_aya}{42}} يَوْمَئِذٍ يَوَدُّ الَّذِينَ كَفَرُوا وَعَصَوُا الرَّسُولَ لَوْ تُسَوَّى بِهِمُ الْأَرْضُ وَلَا يَكْتُمُونَ اللَّهَ حَدِيثًا
{\tiny\colorbox{cl_aya}{43}} يَأَيُّهَا الَّذِينَ ءَامَنُوا لَا تَقْرَبُوا الصَّلَوةَ وَأَنتُمْ سُكَرَى حَتَّى تَعْلَمُوا مَا تَقُولُونَ وَلَا جُنُبًا إِلَّا عَابِرِى سَبِيلٍ حَتَّى تَغْتَسِلُوا وَإِن كُنتُم مَّرْضَى أَوْ عَلَى سَفَرٍ أَوْ جَاءَ أَحَدٌ مِّنكُم مِّنَ الْغَائِطِ أَوْ لَمَسْتُمُ النِّسَاءَ فَلَمْ تَجِدُوا مَاءً فَتَيَمَّمُوا صَعِيدًا طَيِّبًا فَامْسَحُوا بِوُجُوهِكُمْ وَأَيْدِيكُمْ إِنَّ اللَّهَ كَانَ عَفُوًّا غَفُورًا
{\tiny\colorbox{cl_aya}{44}} أَلَمْ تَرَ إِلَى الَّذِينَ أُوتُوا نَصِيبًا مِّنَ الْكِتَبِ يَشْتَرُونَ الضَّلَلَةَ وَيُرِيدُونَ أَن تَضِلُّوا السَّبِيلَ
{\tiny\colorbox{cl_aya}{45}} وَاللَّهُ أَعْلَمُ بِأَعْدَائِكُمْ وَكَفَى بِاللَّهِ وَلِيًّا وَكَفَى بِاللَّهِ نَصِيرًا
{\tiny\colorbox{cl_aya}{46}} مِّنَ الَّذِينَ هَادُوا يُحَرِّفُونَ الْكَلِمَ عَن مَّوَاضِعِهِ وَيَقُولُونَ سَمِعْنَا وَعَصَيْنَا وَاسْمَعْ غَيْرَ مُسْمَعٍ وَرَعِنَا لَيًّا بِأَلْسِنَتِهِمْ وَطَعْنًا فِى الدِّينِ وَلَوْ أَنَّهُمْ قَالُوا سَمِعْنَا وَأَطَعْنَا وَاسْمَعْ وَانظُرْنَا لَكَانَ خَيْرًا لَّهُمْ وَأَقْوَمَ وَلَكِن لَّعَنَهُمُ اللَّهُ بِكُفْرِهِمْ فَلَا يُؤْمِنُونَ إِلَّا قَلِيلًا
{\tiny\colorbox{cl_aya}{47}} يَأَيُّهَا الَّذِينَ أُوتُوا الْكِتَبَ ءَامِنُوا بِمَا نَزَّلْنَا مُصَدِّقًا لِّمَا مَعَكُم مِّن قَبْلِ أَن نَّطْمِسَ وُجُوهًا فَنَرُدَّهَا عَلَى أَدْبَارِهَا أَوْ نَلْعَنَهُمْ كَمَا لَعَنَّا أَصْحَبَ السَّبْتِ وَكَانَ أَمْرُ اللَّهِ مَفْعُولًا
{\tiny\colorbox{cl_aya}{48}} إِنَّ اللَّهَ لَا يَغْفِرُ أَن يُشْرَكَ بِهِ وَيَغْفِرُ مَا دُونَ ذَلِكَ لِمَن يَشَاءُ وَمَن يُشْرِكْ بِاللَّهِ فَقَدِ افْتَرَى إِثْمًا عَظِيمًا
{\tiny\colorbox{cl_aya}{49}} أَلَمْ تَرَ إِلَى الَّذِينَ يُزَكُّونَ أَنفُسَهُم بَلِ اللَّهُ يُزَكِّى مَن يَشَاءُ وَلَا يُظْلَمُونَ فَتِيلًا
{\tiny\colorbox{cl_aya}{50}} انظُرْ كَيْفَ يَفْتَرُونَ عَلَى اللَّهِ الْكَذِبَ وَكَفَى بِهِ إِثْمًا مُّبِينًا
{\tiny\colorbox{cl_aya}{51}} أَلَمْ تَرَ إِلَى الَّذِينَ أُوتُوا نَصِيبًا مِّنَ الْكِتَبِ يُؤْمِنُونَ بِالْجِبْتِ وَالطَّغُوتِ وَيَقُولُونَ لِلَّذِينَ كَفَرُوا هَؤُلَاءِ أَهْدَى مِنَ الَّذِينَ ءَامَنُوا سَبِيلًا
{\tiny\colorbox{cl_aya}{52}} أُولَئِكَ الَّذِينَ لَعَنَهُمُ اللَّهُ وَمَن يَلْعَنِ اللَّهُ فَلَن تَجِدَ لَهُ نَصِيرًا
{\tiny\colorbox{cl_aya}{53}} أَمْ لَهُمْ نَصِيبٌ مِّنَ الْمُلْكِ فَإِذًا لَّا يُؤْتُونَ النَّاسَ نَقِيرًا
{\tiny\colorbox{cl_aya}{54}} أَمْ يَحْسُدُونَ النَّاسَ عَلَى مَا ءَاتَىهُمُ اللَّهُ مِن فَضْلِهِ فَقَدْ ءَاتَيْنَا ءَالَ إِبْرَهِيمَ الْكِتَبَ وَالْحِكْمَةَ وَءَاتَيْنَهُم مُّلْكًا عَظِيمًا
{\tiny\colorbox{cl_aya}{55}} فَمِنْهُم مَّنْ ءَامَنَ بِهِ وَمِنْهُم مَّن صَدَّ عَنْهُ وَكَفَى بِجَهَنَّمَ سَعِيرًا
{\tiny\colorbox{cl_aya}{56}} إِنَّ الَّذِينَ كَفَرُوا بَِٔايَتِنَا سَوْفَ نُصْلِيهِمْ نَارًا كُلَّمَا نَضِجَتْ جُلُودُهُم بَدَّلْنَهُمْ جُلُودًا غَيْرَهَا لِيَذُوقُوا الْعَذَابَ إِنَّ اللَّهَ كَانَ عَزِيزًا حَكِيمًا
{\tiny\colorbox{cl_aya}{57}} وَالَّذِينَ ءَامَنُوا وَعَمِلُوا الصَّلِحَتِ سَنُدْخِلُهُمْ جَنَّتٍ تَجْرِى مِن تَحْتِهَا الْأَنْهَرُ خَلِدِينَ فِيهَا أَبَدًا لَّهُمْ فِيهَا أَزْوَجٌ مُّطَهَّرَةٌ وَنُدْخِلُهُمْ ظِلًّا ظَلِيلًا
{\tiny\colorbox{cl_aya}{58}} إِنَّ اللَّهَ يَأْمُرُكُمْ أَن تُؤَدُّوا الْأَمَنَتِ إِلَى أَهْلِهَا وَإِذَا حَكَمْتُم بَيْنَ النَّاسِ أَن تَحْكُمُوا بِالْعَدْلِ إِنَّ اللَّهَ نِعِمَّا يَعِظُكُم بِهِ إِنَّ اللَّهَ كَانَ سَمِيعًا بَصِيرًا
{\tiny\colorbox{cl_aya}{59}} يَأَيُّهَا الَّذِينَ ءَامَنُوا أَطِيعُوا اللَّهَ وَأَطِيعُوا الرَّسُولَ وَأُولِى الْأَمْرِ مِنكُمْ فَإِن تَنَزَعْتُمْ فِى شَىْءٍ فَرُدُّوهُ إِلَى اللَّهِ وَالرَّسُولِ إِن كُنتُمْ تُؤْمِنُونَ بِاللَّهِ وَالْيَوْمِ الْءَاخِرِ ذَلِكَ خَيْرٌ وَأَحْسَنُ تَأْوِيلًا
{\tiny\colorbox{cl_aya}{60}} أَلَمْ تَرَ إِلَى الَّذِينَ يَزْعُمُونَ أَنَّهُمْ ءَامَنُوا بِمَا أُنزِلَ إِلَيْكَ وَمَا أُنزِلَ مِن قَبْلِكَ يُرِيدُونَ أَن يَتَحَاكَمُوا إِلَى الطَّغُوتِ وَقَدْ أُمِرُوا أَن يَكْفُرُوا بِهِ وَيُرِيدُ الشَّيْطَنُ أَن يُضِلَّهُمْ ضَلَلًا بَعِيدًا
{\tiny\colorbox{cl_aya}{61}} وَإِذَا قِيلَ لَهُمْ تَعَالَوْا إِلَى مَا أَنزَلَ اللَّهُ وَإِلَى الرَّسُولِ رَأَيْتَ الْمُنَفِقِينَ يَصُدُّونَ عَنكَ صُدُودًا
{\tiny\colorbox{cl_aya}{62}} فَكَيْفَ إِذَا أَصَبَتْهُم مُّصِيبَةٌ بِمَا قَدَّمَتْ أَيْدِيهِمْ ثُمَّ جَاءُوكَ يَحْلِفُونَ بِاللَّهِ إِنْ أَرَدْنَا إِلَّا إِحْسَنًا وَتَوْفِيقًا
{\tiny\colorbox{cl_aya}{63}} أُولَئِكَ الَّذِينَ يَعْلَمُ اللَّهُ مَا فِى قُلُوبِهِمْ فَأَعْرِضْ عَنْهُمْ وَعِظْهُمْ وَقُل لَّهُمْ فِى أَنفُسِهِمْ قَوْلًا بَلِيغًا
{\tiny\colorbox{cl_aya}{64}} وَمَا أَرْسَلْنَا مِن رَّسُولٍ إِلَّا لِيُطَاعَ بِإِذْنِ اللَّهِ وَلَوْ أَنَّهُمْ إِذ ظَّلَمُوا أَنفُسَهُمْ جَاءُوكَ فَاسْتَغْفَرُوا اللَّهَ وَاسْتَغْفَرَ لَهُمُ الرَّسُولُ لَوَجَدُوا اللَّهَ تَوَّابًا رَّحِيمًا
{\tiny\colorbox{cl_aya}{65}} فَلَا وَرَبِّكَ لَا يُؤْمِنُونَ حَتَّى يُحَكِّمُوكَ فِيمَا شَجَرَ بَيْنَهُمْ ثُمَّ لَا يَجِدُوا فِى أَنفُسِهِمْ حَرَجًا مِّمَّا قَضَيْتَ وَيُسَلِّمُوا تَسْلِيمًا
{\tiny\colorbox{cl_aya}{66}} وَلَوْ أَنَّا كَتَبْنَا عَلَيْهِمْ أَنِ اقْتُلُوا أَنفُسَكُمْ أَوِ اخْرُجُوا مِن دِيَرِكُم مَّا فَعَلُوهُ إِلَّا قَلِيلٌ مِّنْهُمْ وَلَوْ أَنَّهُمْ فَعَلُوا مَا يُوعَظُونَ بِهِ لَكَانَ خَيْرًا لَّهُمْ وَأَشَدَّ تَثْبِيتًا
{\tiny\colorbox{cl_aya}{67}} وَإِذًا لَّءَاتَيْنَهُم مِّن لَّدُنَّا أَجْرًا عَظِيمًا
{\tiny\colorbox{cl_aya}{68}} وَلَهَدَيْنَهُمْ صِرَطًا مُّسْتَقِيمًا
{\tiny\colorbox{cl_aya}{69}} وَمَن يُطِعِ اللَّهَ وَالرَّسُولَ فَأُولَئِكَ مَعَ الَّذِينَ أَنْعَمَ اللَّهُ عَلَيْهِم مِّنَ النَّبِيِّنَ وَالصِّدِّيقِينَ وَالشُّهَدَاءِ وَالصَّلِحِينَ وَحَسُنَ أُولَئِكَ رَفِيقًا
{\tiny\colorbox{cl_aya}{70}} ذَلِكَ الْفَضْلُ مِنَ اللَّهِ وَكَفَى بِاللَّهِ عَلِيمًا
{\tiny\colorbox{cl_aya}{71}} يَأَيُّهَا الَّذِينَ ءَامَنُوا خُذُوا حِذْرَكُمْ فَانفِرُوا ثُبَاتٍ أَوِ انفِرُوا جَمِيعًا
{\tiny\colorbox{cl_aya}{72}} وَإِنَّ مِنكُمْ لَمَن لَّيُبَطِّئَنَّ فَإِنْ أَصَبَتْكُم مُّصِيبَةٌ قَالَ قَدْ أَنْعَمَ اللَّهُ عَلَىَّ إِذْ لَمْ أَكُن مَّعَهُمْ شَهِيدًا
{\tiny\colorbox{cl_aya}{73}} وَلَئِنْ أَصَبَكُمْ فَضْلٌ مِّنَ اللَّهِ لَيَقُولَنَّ كَأَن لَّمْ تَكُن بَيْنَكُمْ وَبَيْنَهُ مَوَدَّةٌ يَلَيْتَنِى كُنتُ مَعَهُمْ فَأَفُوزَ فَوْزًا عَظِيمًا
{\tiny\colorbox{cl_aya}{74}} فَلْيُقَتِلْ فِى سَبِيلِ اللَّهِ الَّذِينَ يَشْرُونَ الْحَيَوةَ الدُّنْيَا بِالْءَاخِرَةِ وَمَن يُقَتِلْ فِى سَبِيلِ اللَّهِ فَيُقْتَلْ أَوْ يَغْلِبْ فَسَوْفَ نُؤْتِيهِ أَجْرًا عَظِيمًا
{\tiny\colorbox{cl_aya}{75}} وَمَا لَكُمْ لَا تُقَتِلُونَ فِى سَبِيلِ اللَّهِ وَالْمُسْتَضْعَفِينَ مِنَ الرِّجَالِ وَالنِّسَاءِ وَالْوِلْدَنِ الَّذِينَ يَقُولُونَ رَبَّنَا أَخْرِجْنَا مِنْ هَذِهِ الْقَرْيَةِ الظَّالِمِ أَهْلُهَا وَاجْعَل لَّنَا مِن لَّدُنكَ وَلِيًّا وَاجْعَل لَّنَا مِن لَّدُنكَ نَصِيرًا
{\tiny\colorbox{cl_aya}{76}} الَّذِينَ ءَامَنُوا يُقَتِلُونَ فِى سَبِيلِ اللَّهِ وَالَّذِينَ كَفَرُوا يُقَتِلُونَ فِى سَبِيلِ الطَّغُوتِ فَقَتِلُوا أَوْلِيَاءَ الشَّيْطَنِ إِنَّ كَيْدَ الشَّيْطَنِ كَانَ ضَعِيفًا
{\tiny\colorbox{cl_aya}{77}} أَلَمْ تَرَ إِلَى الَّذِينَ قِيلَ لَهُمْ كُفُّوا أَيْدِيَكُمْ وَأَقِيمُوا الصَّلَوةَ وَءَاتُوا الزَّكَوةَ فَلَمَّا كُتِبَ عَلَيْهِمُ الْقِتَالُ إِذَا فَرِيقٌ مِّنْهُمْ يَخْشَوْنَ النَّاسَ كَخَشْيَةِ اللَّهِ أَوْ أَشَدَّ خَشْيَةً وَقَالُوا رَبَّنَا لِمَ كَتَبْتَ عَلَيْنَا الْقِتَالَ لَوْلَا أَخَّرْتَنَا إِلَى أَجَلٍ قَرِيبٍ قُلْ مَتَعُ الدُّنْيَا قَلِيلٌ وَالْءَاخِرَةُ خَيْرٌ لِّمَنِ اتَّقَى وَلَا تُظْلَمُونَ فَتِيلًا
{\tiny\colorbox{cl_aya}{78}} أَيْنَمَا تَكُونُوا يُدْرِككُّمُ الْمَوْتُ وَلَوْ كُنتُمْ فِى بُرُوجٍ مُّشَيَّدَةٍ وَإِن تُصِبْهُمْ حَسَنَةٌ يَقُولُوا هَذِهِ مِنْ عِندِ اللَّهِ وَإِن تُصِبْهُمْ سَيِّئَةٌ يَقُولُوا هَذِهِ مِنْ عِندِكَ قُلْ كُلٌّ مِّنْ عِندِ اللَّهِ فَمَالِ هَؤُلَاءِ الْقَوْمِ لَا يَكَادُونَ يَفْقَهُونَ حَدِيثًا
{\tiny\colorbox{cl_aya}{79}} مَّا أَصَابَكَ مِنْ حَسَنَةٍ فَمِنَ اللَّهِ وَمَا أَصَابَكَ مِن سَيِّئَةٍ فَمِن نَّفْسِكَ وَأَرْسَلْنَكَ لِلنَّاسِ رَسُولًا وَكَفَى بِاللَّهِ شَهِيدًا
{\tiny\colorbox{cl_aya}{80}} مَّن يُطِعِ الرَّسُولَ فَقَدْ أَطَاعَ اللَّهَ وَمَن تَوَلَّى فَمَا أَرْسَلْنَكَ عَلَيْهِمْ حَفِيظًا
{\tiny\colorbox{cl_aya}{81}} وَيَقُولُونَ طَاعَةٌ فَإِذَا بَرَزُوا مِنْ عِندِكَ بَيَّتَ طَائِفَةٌ مِّنْهُمْ غَيْرَ الَّذِى تَقُولُ وَاللَّهُ يَكْتُبُ مَا يُبَيِّتُونَ فَأَعْرِضْ عَنْهُمْ وَتَوَكَّلْ عَلَى اللَّهِ وَكَفَى بِاللَّهِ وَكِيلًا
{\tiny\colorbox{cl_aya}{82}} أَفَلَا يَتَدَبَّرُونَ الْقُرْءَانَ وَلَوْ كَانَ مِنْ عِندِ غَيْرِ اللَّهِ لَوَجَدُوا فِيهِ اخْتِلَفًا كَثِيرًا
{\tiny\colorbox{cl_aya}{83}} وَإِذَا جَاءَهُمْ أَمْرٌ مِّنَ الْأَمْنِ أَوِ الْخَوْفِ أَذَاعُوا بِهِ وَلَوْ رَدُّوهُ إِلَى الرَّسُولِ وَإِلَى أُولِى الْأَمْرِ مِنْهُمْ لَعَلِمَهُ الَّذِينَ يَسْتَنبِطُونَهُ مِنْهُمْ وَلَوْلَا فَضْلُ اللَّهِ عَلَيْكُمْ وَرَحْمَتُهُ لَاتَّبَعْتُمُ الشَّيْطَنَ إِلَّا قَلِيلًا
{\tiny\colorbox{cl_aya}{84}} فَقَتِلْ فِى سَبِيلِ اللَّهِ لَا تُكَلَّفُ إِلَّا نَفْسَكَ وَحَرِّضِ الْمُؤْمِنِينَ عَسَى اللَّهُ أَن يَكُفَّ بَأْسَ الَّذِينَ كَفَرُوا وَاللَّهُ أَشَدُّ بَأْسًا وَأَشَدُّ تَنكِيلًا
{\tiny\colorbox{cl_aya}{85}} مَّن يَشْفَعْ شَفَعَةً حَسَنَةً يَكُن لَّهُ نَصِيبٌ مِّنْهَا وَمَن يَشْفَعْ شَفَعَةً سَيِّئَةً يَكُن لَّهُ كِفْلٌ مِّنْهَا وَكَانَ اللَّهُ عَلَى كُلِّ شَىْءٍ مُّقِيتًا
{\tiny\colorbox{cl_aya}{86}} وَإِذَا حُيِّيتُم بِتَحِيَّةٍ فَحَيُّوا بِأَحْسَنَ مِنْهَا أَوْ رُدُّوهَا إِنَّ اللَّهَ كَانَ عَلَى كُلِّ شَىْءٍ حَسِيبًا
{\tiny\colorbox{cl_aya}{87}} اللَّهُ لَا إِلَهَ إِلَّا هُوَ لَيَجْمَعَنَّكُمْ إِلَى يَوْمِ الْقِيَمَةِ لَا رَيْبَ فِيهِ وَمَنْ أَصْدَقُ مِنَ اللَّهِ حَدِيثًا
{\tiny\colorbox{cl_aya}{88}} فَمَا لَكُمْ فِى الْمُنَفِقِينَ فِئَتَيْنِ وَاللَّهُ أَرْكَسَهُم بِمَا كَسَبُوا أَتُرِيدُونَ أَن تَهْدُوا مَنْ أَضَلَّ اللَّهُ وَمَن يُضْلِلِ اللَّهُ فَلَن تَجِدَ لَهُ سَبِيلًا
{\tiny\colorbox{cl_aya}{89}} وَدُّوا لَوْ تَكْفُرُونَ كَمَا كَفَرُوا فَتَكُونُونَ سَوَاءً فَلَا تَتَّخِذُوا مِنْهُمْ أَوْلِيَاءَ حَتَّى يُهَاجِرُوا فِى سَبِيلِ اللَّهِ فَإِن تَوَلَّوْا فَخُذُوهُمْ وَاقْتُلُوهُمْ حَيْثُ وَجَدتُّمُوهُمْ وَلَا تَتَّخِذُوا مِنْهُمْ وَلِيًّا وَلَا نَصِيرًا
{\tiny\colorbox{cl_aya}{90}} إِلَّا الَّذِينَ يَصِلُونَ إِلَى قَوْمٍ بَيْنَكُمْ وَبَيْنَهُم مِّيثَقٌ أَوْ جَاءُوكُمْ حَصِرَتْ صُدُورُهُمْ أَن يُقَتِلُوكُمْ أَوْ يُقَتِلُوا قَوْمَهُمْ وَلَوْ شَاءَ اللَّهُ لَسَلَّطَهُمْ عَلَيْكُمْ فَلَقَتَلُوكُمْ فَإِنِ اعْتَزَلُوكُمْ فَلَمْ يُقَتِلُوكُمْ وَأَلْقَوْا إِلَيْكُمُ السَّلَمَ فَمَا جَعَلَ اللَّهُ لَكُمْ عَلَيْهِمْ سَبِيلًا
{\tiny\colorbox{cl_aya}{91}} سَتَجِدُونَ ءَاخَرِينَ يُرِيدُونَ أَن يَأْمَنُوكُمْ وَيَأْمَنُوا قَوْمَهُمْ كُلَّ مَا رُدُّوا إِلَى الْفِتْنَةِ أُرْكِسُوا فِيهَا فَإِن لَّمْ يَعْتَزِلُوكُمْ وَيُلْقُوا إِلَيْكُمُ السَّلَمَ وَيَكُفُّوا أَيْدِيَهُمْ فَخُذُوهُمْ وَاقْتُلُوهُمْ حَيْثُ ثَقِفْتُمُوهُمْ وَأُولَئِكُمْ جَعَلْنَا لَكُمْ عَلَيْهِمْ سُلْطَنًا مُّبِينًا
{\tiny\colorbox{cl_aya}{92}} وَمَا كَانَ لِمُؤْمِنٍ أَن يَقْتُلَ مُؤْمِنًا إِلَّا خَطًَٔا وَمَن قَتَلَ مُؤْمِنًا خَطًَٔا فَتَحْرِيرُ رَقَبَةٍ مُّؤْمِنَةٍ وَدِيَةٌ مُّسَلَّمَةٌ إِلَى أَهْلِهِ إِلَّا أَن يَصَّدَّقُوا فَإِن كَانَ مِن قَوْمٍ عَدُوٍّ لَّكُمْ وَهُوَ مُؤْمِنٌ فَتَحْرِيرُ رَقَبَةٍ مُّؤْمِنَةٍ وَإِن كَانَ مِن قَوْمٍ بَيْنَكُمْ وَبَيْنَهُم مِّيثَقٌ فَدِيَةٌ مُّسَلَّمَةٌ إِلَى أَهْلِهِ وَتَحْرِيرُ رَقَبَةٍ مُّؤْمِنَةٍ فَمَن لَّمْ يَجِدْ فَصِيَامُ شَهْرَيْنِ مُتَتَابِعَيْنِ تَوْبَةً مِّنَ اللَّهِ وَكَانَ اللَّهُ عَلِيمًا حَكِيمًا
{\tiny\colorbox{cl_aya}{93}} وَمَن يَقْتُلْ مُؤْمِنًا مُّتَعَمِّدًا فَجَزَاؤُهُ جَهَنَّمُ خَلِدًا فِيهَا وَغَضِبَ اللَّهُ عَلَيْهِ وَلَعَنَهُ وَأَعَدَّ لَهُ عَذَابًا عَظِيمًا
{\tiny\colorbox{cl_aya}{94}} يَأَيُّهَا الَّذِينَ ءَامَنُوا إِذَا ضَرَبْتُمْ فِى سَبِيلِ اللَّهِ فَتَبَيَّنُوا وَلَا تَقُولُوا لِمَنْ أَلْقَى إِلَيْكُمُ السَّلَمَ لَسْتَ مُؤْمِنًا تَبْتَغُونَ عَرَضَ الْحَيَوةِ الدُّنْيَا فَعِندَ اللَّهِ مَغَانِمُ كَثِيرَةٌ كَذَلِكَ كُنتُم مِّن قَبْلُ فَمَنَّ اللَّهُ عَلَيْكُمْ فَتَبَيَّنُوا إِنَّ اللَّهَ كَانَ بِمَا تَعْمَلُونَ خَبِيرًا
{\tiny\colorbox{cl_aya}{95}} لَّا يَسْتَوِى الْقَعِدُونَ مِنَ الْمُؤْمِنِينَ غَيْرُ أُولِى الضَّرَرِ وَالْمُجَهِدُونَ فِى سَبِيلِ اللَّهِ بِأَمْوَلِهِمْ وَأَنفُسِهِمْ فَضَّلَ اللَّهُ الْمُجَهِدِينَ بِأَمْوَلِهِمْ وَأَنفُسِهِمْ عَلَى الْقَعِدِينَ دَرَجَةً وَكُلًّا وَعَدَ اللَّهُ الْحُسْنَى وَفَضَّلَ اللَّهُ الْمُجَهِدِينَ عَلَى الْقَعِدِينَ أَجْرًا عَظِيمًا
{\tiny\colorbox{cl_aya}{96}} دَرَجَتٍ مِّنْهُ وَمَغْفِرَةً وَرَحْمَةً وَكَانَ اللَّهُ غَفُورًا رَّحِيمًا
{\tiny\colorbox{cl_aya}{97}} إِنَّ الَّذِينَ تَوَفَّىهُمُ الْمَلَئِكَةُ ظَالِمِى أَنفُسِهِمْ قَالُوا فِيمَ كُنتُمْ قَالُوا كُنَّا مُسْتَضْعَفِينَ فِى الْأَرْضِ قَالُوا أَلَمْ تَكُنْ أَرْضُ اللَّهِ وَسِعَةً فَتُهَاجِرُوا فِيهَا فَأُولَئِكَ مَأْوَىهُمْ جَهَنَّمُ وَسَاءَتْ مَصِيرًا
{\tiny\colorbox{cl_aya}{98}} إِلَّا الْمُسْتَضْعَفِينَ مِنَ الرِّجَالِ وَالنِّسَاءِ وَالْوِلْدَنِ لَا يَسْتَطِيعُونَ حِيلَةً وَلَا يَهْتَدُونَ سَبِيلًا
{\tiny\colorbox{cl_aya}{99}} فَأُولَئِكَ عَسَى اللَّهُ أَن يَعْفُوَ عَنْهُمْ وَكَانَ اللَّهُ عَفُوًّا غَفُورًا
{\tiny\colorbox{cl_aya}{100}} وَمَن يُهَاجِرْ فِى سَبِيلِ اللَّهِ يَجِدْ فِى الْأَرْضِ مُرَغَمًا كَثِيرًا وَسَعَةً وَمَن يَخْرُجْ مِن بَيْتِهِ مُهَاجِرًا إِلَى اللَّهِ وَرَسُولِهِ ثُمَّ يُدْرِكْهُ الْمَوْتُ فَقَدْ وَقَعَ أَجْرُهُ عَلَى اللَّهِ وَكَانَ اللَّهُ غَفُورًا رَّحِيمًا
{\tiny\colorbox{cl_aya}{101}} وَإِذَا ضَرَبْتُمْ فِى الْأَرْضِ فَلَيْسَ عَلَيْكُمْ جُنَاحٌ أَن تَقْصُرُوا مِنَ الصَّلَوةِ إِنْ خِفْتُمْ أَن يَفْتِنَكُمُ الَّذِينَ كَفَرُوا إِنَّ الْكَفِرِينَ كَانُوا لَكُمْ عَدُوًّا مُّبِينًا
{\tiny\colorbox{cl_aya}{102}} وَإِذَا كُنتَ فِيهِمْ فَأَقَمْتَ لَهُمُ الصَّلَوةَ فَلْتَقُمْ طَائِفَةٌ مِّنْهُم مَّعَكَ وَلْيَأْخُذُوا أَسْلِحَتَهُمْ فَإِذَا سَجَدُوا فَلْيَكُونُوا مِن وَرَائِكُمْ وَلْتَأْتِ طَائِفَةٌ أُخْرَى لَمْ يُصَلُّوا فَلْيُصَلُّوا مَعَكَ وَلْيَأْخُذُوا حِذْرَهُمْ وَأَسْلِحَتَهُمْ وَدَّ الَّذِينَ كَفَرُوا لَوْ تَغْفُلُونَ عَنْ أَسْلِحَتِكُمْ وَأَمْتِعَتِكُمْ فَيَمِيلُونَ عَلَيْكُم مَّيْلَةً وَحِدَةً وَلَا جُنَاحَ عَلَيْكُمْ إِن كَانَ بِكُمْ أَذًى مِّن مَّطَرٍ أَوْ كُنتُم مَّرْضَى أَن تَضَعُوا أَسْلِحَتَكُمْ وَخُذُوا حِذْرَكُمْ إِنَّ اللَّهَ أَعَدَّ لِلْكَفِرِينَ عَذَابًا مُّهِينًا
{\tiny\colorbox{cl_aya}{103}} فَإِذَا قَضَيْتُمُ الصَّلَوةَ فَاذْكُرُوا اللَّهَ قِيَمًا وَقُعُودًا وَعَلَى جُنُوبِكُمْ فَإِذَا اطْمَأْنَنتُمْ فَأَقِيمُوا الصَّلَوةَ إِنَّ الصَّلَوةَ كَانَتْ عَلَى الْمُؤْمِنِينَ كِتَبًا مَّوْقُوتًا
{\tiny\colorbox{cl_aya}{104}} وَلَا تَهِنُوا فِى ابْتِغَاءِ الْقَوْمِ إِن تَكُونُوا تَأْلَمُونَ فَإِنَّهُمْ يَأْلَمُونَ كَمَا تَأْلَمُونَ وَتَرْجُونَ مِنَ اللَّهِ مَا لَا يَرْجُونَ وَكَانَ اللَّهُ عَلِيمًا حَكِيمًا
{\tiny\colorbox{cl_aya}{105}} إِنَّا أَنزَلْنَا إِلَيْكَ الْكِتَبَ بِالْحَقِّ لِتَحْكُمَ بَيْنَ النَّاسِ بِمَا أَرَىكَ اللَّهُ وَلَا تَكُن لِّلْخَائِنِينَ خَصِيمًا
{\tiny\colorbox{cl_aya}{106}} وَاسْتَغْفِرِ اللَّهَ إِنَّ اللَّهَ كَانَ غَفُورًا رَّحِيمًا
{\tiny\colorbox{cl_aya}{107}} وَلَا تُجَدِلْ عَنِ الَّذِينَ يَخْتَانُونَ أَنفُسَهُمْ إِنَّ اللَّهَ لَا يُحِبُّ مَن كَانَ خَوَّانًا أَثِيمًا
{\tiny\colorbox{cl_aya}{108}} يَسْتَخْفُونَ مِنَ النَّاسِ وَلَا يَسْتَخْفُونَ مِنَ اللَّهِ وَهُوَ مَعَهُمْ إِذْ يُبَيِّتُونَ مَا لَا يَرْضَى مِنَ الْقَوْلِ وَكَانَ اللَّهُ بِمَا يَعْمَلُونَ مُحِيطًا
{\tiny\colorbox{cl_aya}{109}} هَأَنتُمْ هَؤُلَاءِ جَدَلْتُمْ عَنْهُمْ فِى الْحَيَوةِ الدُّنْيَا فَمَن يُجَدِلُ اللَّهَ عَنْهُمْ يَوْمَ الْقِيَمَةِ أَم مَّن يَكُونُ عَلَيْهِمْ وَكِيلًا
{\tiny\colorbox{cl_aya}{110}} وَمَن يَعْمَلْ سُوءًا أَوْ يَظْلِمْ نَفْسَهُ ثُمَّ يَسْتَغْفِرِ اللَّهَ يَجِدِ اللَّهَ غَفُورًا رَّحِيمًا
{\tiny\colorbox{cl_aya}{111}} وَمَن يَكْسِبْ إِثْمًا فَإِنَّمَا يَكْسِبُهُ عَلَى نَفْسِهِ وَكَانَ اللَّهُ عَلِيمًا حَكِيمًا
{\tiny\colorbox{cl_aya}{112}} وَمَن يَكْسِبْ خَطِئَةً أَوْ إِثْمًا ثُمَّ يَرْمِ بِهِ بَرِئًا فَقَدِ احْتَمَلَ بُهْتَنًا وَإِثْمًا مُّبِينًا
{\tiny\colorbox{cl_aya}{113}} وَلَوْلَا فَضْلُ اللَّهِ عَلَيْكَ وَرَحْمَتُهُ لَهَمَّت طَّائِفَةٌ مِّنْهُمْ أَن يُضِلُّوكَ وَمَا يُضِلُّونَ إِلَّا أَنفُسَهُمْ وَمَا يَضُرُّونَكَ مِن شَىْءٍ وَأَنزَلَ اللَّهُ عَلَيْكَ الْكِتَبَ وَالْحِكْمَةَ وَعَلَّمَكَ مَا لَمْ تَكُن تَعْلَمُ وَكَانَ فَضْلُ اللَّهِ عَلَيْكَ عَظِيمًا
{\tiny\colorbox{cl_aya}{114}} لَّا خَيْرَ فِى كَثِيرٍ مِّن نَّجْوَىهُمْ إِلَّا مَنْ أَمَرَ بِصَدَقَةٍ أَوْ مَعْرُوفٍ أَوْ إِصْلَحٍ بَيْنَ النَّاسِ وَمَن يَفْعَلْ ذَلِكَ ابْتِغَاءَ مَرْضَاتِ اللَّهِ فَسَوْفَ نُؤْتِيهِ أَجْرًا عَظِيمًا
{\tiny\colorbox{cl_aya}{115}} وَمَن يُشَاقِقِ الرَّسُولَ مِن بَعْدِ مَا تَبَيَّنَ لَهُ الْهُدَى وَيَتَّبِعْ غَيْرَ سَبِيلِ الْمُؤْمِنِينَ نُوَلِّهِ مَا تَوَلَّى وَنُصْلِهِ جَهَنَّمَ وَسَاءَتْ مَصِيرًا
{\tiny\colorbox{cl_aya}{116}} إِنَّ اللَّهَ لَا يَغْفِرُ أَن يُشْرَكَ بِهِ وَيَغْفِرُ مَا دُونَ ذَلِكَ لِمَن يَشَاءُ وَمَن يُشْرِكْ بِاللَّهِ فَقَدْ ضَلَّ ضَلَلًا بَعِيدًا
{\tiny\colorbox{cl_aya}{117}} إِن يَدْعُونَ مِن دُونِهِ إِلَّا إِنَثًا وَإِن يَدْعُونَ إِلَّا شَيْطَنًا مَّرِيدًا
{\tiny\colorbox{cl_aya}{118}} لَّعَنَهُ اللَّهُ وَقَالَ لَأَتَّخِذَنَّ مِنْ عِبَادِكَ نَصِيبًا مَّفْرُوضًا
{\tiny\colorbox{cl_aya}{119}} وَلَأُضِلَّنَّهُمْ وَلَأُمَنِّيَنَّهُمْ وَلَءَامُرَنَّهُمْ فَلَيُبَتِّكُنَّ ءَاذَانَ الْأَنْعَمِ وَلَءَامُرَنَّهُمْ فَلَيُغَيِّرُنَّ خَلْقَ اللَّهِ وَمَن يَتَّخِذِ الشَّيْطَنَ وَلِيًّا مِّن دُونِ اللَّهِ فَقَدْ خَسِرَ خُسْرَانًا مُّبِينًا
{\tiny\colorbox{cl_aya}{120}} يَعِدُهُمْ وَيُمَنِّيهِمْ وَمَا يَعِدُهُمُ الشَّيْطَنُ إِلَّا غُرُورًا
{\tiny\colorbox{cl_aya}{121}} أُولَئِكَ مَأْوَىهُمْ جَهَنَّمُ وَلَا يَجِدُونَ عَنْهَا مَحِيصًا
{\tiny\colorbox{cl_aya}{122}} وَالَّذِينَ ءَامَنُوا وَعَمِلُوا الصَّلِحَتِ سَنُدْخِلُهُمْ جَنَّتٍ تَجْرِى مِن تَحْتِهَا الْأَنْهَرُ خَلِدِينَ فِيهَا أَبَدًا وَعْدَ اللَّهِ حَقًّا وَمَنْ أَصْدَقُ مِنَ اللَّهِ قِيلًا
{\tiny\colorbox{cl_aya}{123}} لَّيْسَ بِأَمَانِيِّكُمْ وَلَا أَمَانِىِّ أَهْلِ الْكِتَبِ مَن يَعْمَلْ سُوءًا يُجْزَ بِهِ وَلَا يَجِدْ لَهُ مِن دُونِ اللَّهِ وَلِيًّا وَلَا نَصِيرًا
{\tiny\colorbox{cl_aya}{124}} وَمَن يَعْمَلْ مِنَ الصَّلِحَتِ مِن ذَكَرٍ أَوْ أُنثَى وَهُوَ مُؤْمِنٌ فَأُولَئِكَ يَدْخُلُونَ الْجَنَّةَ وَلَا يُظْلَمُونَ نَقِيرًا
{\tiny\colorbox{cl_aya}{125}} وَمَنْ أَحْسَنُ دِينًا مِّمَّنْ أَسْلَمَ وَجْهَهُ لِلَّهِ وَهُوَ مُحْسِنٌ وَاتَّبَعَ مِلَّةَ إِبْرَهِيمَ حَنِيفًا وَاتَّخَذَ اللَّهُ إِبْرَهِيمَ خَلِيلًا
{\tiny\colorbox{cl_aya}{126}} وَلِلَّهِ مَا فِى السَّمَوَتِ وَمَا فِى الْأَرْضِ وَكَانَ اللَّهُ بِكُلِّ شَىْءٍ مُّحِيطًا
{\tiny\colorbox{cl_aya}{127}} وَيَسْتَفْتُونَكَ فِى النِّسَاءِ قُلِ اللَّهُ يُفْتِيكُمْ فِيهِنَّ وَمَا يُتْلَى عَلَيْكُمْ فِى الْكِتَبِ فِى يَتَمَى النِّسَاءِ الَّتِى لَا تُؤْتُونَهُنَّ مَا كُتِبَ لَهُنَّ وَتَرْغَبُونَ أَن تَنكِحُوهُنَّ وَالْمُسْتَضْعَفِينَ مِنَ الْوِلْدَنِ وَأَن تَقُومُوا لِلْيَتَمَى بِالْقِسْطِ وَمَا تَفْعَلُوا مِنْ خَيْرٍ فَإِنَّ اللَّهَ كَانَ بِهِ عَلِيمًا
{\tiny\colorbox{cl_aya}{128}} وَإِنِ امْرَأَةٌ خَافَتْ مِن بَعْلِهَا نُشُوزًا أَوْ إِعْرَاضًا فَلَا جُنَاحَ عَلَيْهِمَا أَن يُصْلِحَا بَيْنَهُمَا صُلْحًا وَالصُّلْحُ خَيْرٌ وَأُحْضِرَتِ الْأَنفُسُ الشُّحَّ وَإِن تُحْسِنُوا وَتَتَّقُوا فَإِنَّ اللَّهَ كَانَ بِمَا تَعْمَلُونَ خَبِيرًا
{\tiny\colorbox{cl_aya}{129}} وَلَن تَسْتَطِيعُوا أَن تَعْدِلُوا بَيْنَ النِّسَاءِ وَلَوْ حَرَصْتُمْ فَلَا تَمِيلُوا كُلَّ الْمَيْلِ فَتَذَرُوهَا كَالْمُعَلَّقَةِ وَإِن تُصْلِحُوا وَتَتَّقُوا فَإِنَّ اللَّهَ كَانَ غَفُورًا رَّحِيمًا
{\tiny\colorbox{cl_aya}{130}} وَإِن يَتَفَرَّقَا يُغْنِ اللَّهُ كُلًّا مِّن سَعَتِهِ وَكَانَ اللَّهُ وَسِعًا حَكِيمًا
{\tiny\colorbox{cl_aya}{131}} وَلِلَّهِ مَا فِى السَّمَوَتِ وَمَا فِى الْأَرْضِ وَلَقَدْ وَصَّيْنَا الَّذِينَ أُوتُوا الْكِتَبَ مِن قَبْلِكُمْ وَإِيَّاكُمْ أَنِ اتَّقُوا اللَّهَ وَإِن تَكْفُرُوا فَإِنَّ لِلَّهِ مَا فِى السَّمَوَتِ وَمَا فِى الْأَرْضِ وَكَانَ اللَّهُ غَنِيًّا حَمِيدًا
{\tiny\colorbox{cl_aya}{132}} وَلِلَّهِ مَا فِى السَّمَوَتِ وَمَا فِى الْأَرْضِ وَكَفَى بِاللَّهِ وَكِيلًا
{\tiny\colorbox{cl_aya}{133}} إِن يَشَأْ يُذْهِبْكُمْ أَيُّهَا النَّاسُ وَيَأْتِ بَِٔاخَرِينَ وَكَانَ اللَّهُ عَلَى ذَلِكَ قَدِيرًا
{\tiny\colorbox{cl_aya}{134}} مَّن كَانَ يُرِيدُ ثَوَابَ الدُّنْيَا فَعِندَ اللَّهِ ثَوَابُ الدُّنْيَا وَالْءَاخِرَةِ وَكَانَ اللَّهُ سَمِيعًا بَصِيرًا
{\tiny\colorbox{cl_aya}{135}} يَأَيُّهَا الَّذِينَ ءَامَنُوا كُونُوا قَوَّمِينَ بِالْقِسْطِ شُهَدَاءَ لِلَّهِ وَلَوْ عَلَى أَنفُسِكُمْ أَوِ الْوَلِدَيْنِ وَالْأَقْرَبِينَ إِن يَكُنْ غَنِيًّا أَوْ فَقِيرًا فَاللَّهُ أَوْلَى بِهِمَا فَلَا تَتَّبِعُوا الْهَوَى أَن تَعْدِلُوا وَإِن تَلْوُا أَوْ تُعْرِضُوا فَإِنَّ اللَّهَ كَانَ بِمَا تَعْمَلُونَ خَبِيرًا
{\tiny\colorbox{cl_aya}{136}} يَأَيُّهَا الَّذِينَ ءَامَنُوا ءَامِنُوا بِاللَّهِ وَرَسُولِهِ وَالْكِتَبِ الَّذِى نَزَّلَ عَلَى رَسُولِهِ وَالْكِتَبِ الَّذِى أَنزَلَ مِن قَبْلُ وَمَن يَكْفُرْ بِاللَّهِ وَمَلَئِكَتِهِ وَكُتُبِهِ وَرُسُلِهِ وَالْيَوْمِ الْءَاخِرِ فَقَدْ ضَلَّ ضَلَلًا بَعِيدًا
{\tiny\colorbox{cl_aya}{137}} إِنَّ الَّذِينَ ءَامَنُوا ثُمَّ كَفَرُوا ثُمَّ ءَامَنُوا ثُمَّ كَفَرُوا ثُمَّ ازْدَادُوا كُفْرًا لَّمْ يَكُنِ اللَّهُ لِيَغْفِرَ لَهُمْ وَلَا لِيَهْدِيَهُمْ سَبِيلًا
{\tiny\colorbox{cl_aya}{138}} بَشِّرِ الْمُنَفِقِينَ بِأَنَّ لَهُمْ عَذَابًا أَلِيمًا
{\tiny\colorbox{cl_aya}{139}} الَّذِينَ يَتَّخِذُونَ الْكَفِرِينَ أَوْلِيَاءَ مِن دُونِ الْمُؤْمِنِينَ أَيَبْتَغُونَ عِندَهُمُ الْعِزَّةَ فَإِنَّ الْعِزَّةَ لِلَّهِ جَمِيعًا
{\tiny\colorbox{cl_aya}{140}} وَقَدْ نَزَّلَ عَلَيْكُمْ فِى الْكِتَبِ أَنْ إِذَا سَمِعْتُمْ ءَايَتِ اللَّهِ يُكْفَرُ بِهَا وَيُسْتَهْزَأُ بِهَا فَلَا تَقْعُدُوا مَعَهُمْ حَتَّى يَخُوضُوا فِى حَدِيثٍ غَيْرِهِ إِنَّكُمْ إِذًا مِّثْلُهُمْ إِنَّ اللَّهَ جَامِعُ الْمُنَفِقِينَ وَالْكَفِرِينَ فِى جَهَنَّمَ جَمِيعًا
{\tiny\colorbox{cl_aya}{141}} الَّذِينَ يَتَرَبَّصُونَ بِكُمْ فَإِن كَانَ لَكُمْ فَتْحٌ مِّنَ اللَّهِ قَالُوا أَلَمْ نَكُن مَّعَكُمْ وَإِن كَانَ لِلْكَفِرِينَ نَصِيبٌ قَالُوا أَلَمْ نَسْتَحْوِذْ عَلَيْكُمْ وَنَمْنَعْكُم مِّنَ الْمُؤْمِنِينَ فَاللَّهُ يَحْكُمُ بَيْنَكُمْ يَوْمَ الْقِيَمَةِ وَلَن يَجْعَلَ اللَّهُ لِلْكَفِرِينَ عَلَى الْمُؤْمِنِينَ سَبِيلًا
{\tiny\colorbox{cl_aya}{142}} إِنَّ الْمُنَفِقِينَ يُخَدِعُونَ اللَّهَ وَهُوَ خَدِعُهُمْ وَإِذَا قَامُوا إِلَى الصَّلَوةِ قَامُوا كُسَالَى يُرَاءُونَ النَّاسَ وَلَا يَذْكُرُونَ اللَّهَ إِلَّا قَلِيلًا
{\tiny\colorbox{cl_aya}{143}} مُّذَبْذَبِينَ بَيْنَ ذَلِكَ لَا إِلَى هَؤُلَاءِ وَلَا إِلَى هَؤُلَاءِ وَمَن يُضْلِلِ اللَّهُ فَلَن تَجِدَ لَهُ سَبِيلًا
{\tiny\colorbox{cl_aya}{144}} يَأَيُّهَا الَّذِينَ ءَامَنُوا لَا تَتَّخِذُوا الْكَفِرِينَ أَوْلِيَاءَ مِن دُونِ الْمُؤْمِنِينَ أَتُرِيدُونَ أَن تَجْعَلُوا لِلَّهِ عَلَيْكُمْ سُلْطَنًا مُّبِينًا
{\tiny\colorbox{cl_aya}{145}} إِنَّ الْمُنَفِقِينَ فِى الدَّرْكِ الْأَسْفَلِ مِنَ النَّارِ وَلَن تَجِدَ لَهُمْ نَصِيرًا
{\tiny\colorbox{cl_aya}{146}} إِلَّا الَّذِينَ تَابُوا وَأَصْلَحُوا وَاعْتَصَمُوا بِاللَّهِ وَأَخْلَصُوا دِينَهُمْ لِلَّهِ فَأُولَئِكَ مَعَ الْمُؤْمِنِينَ وَسَوْفَ يُؤْتِ اللَّهُ الْمُؤْمِنِينَ أَجْرًا عَظِيمًا
{\tiny\colorbox{cl_aya}{147}} مَّا يَفْعَلُ اللَّهُ بِعَذَابِكُمْ إِن شَكَرْتُمْ وَءَامَنتُمْ وَكَانَ اللَّهُ شَاكِرًا عَلِيمًا
{\tiny\colorbox{cl_aya}{148}} لَّا يُحِبُّ اللَّهُ الْجَهْرَ بِالسُّوءِ مِنَ الْقَوْلِ إِلَّا مَن ظُلِمَ وَكَانَ اللَّهُ سَمِيعًا عَلِيمًا
{\tiny\colorbox{cl_aya}{149}} إِن تُبْدُوا خَيْرًا أَوْ تُخْفُوهُ أَوْ تَعْفُوا عَن سُوءٍ فَإِنَّ اللَّهَ كَانَ عَفُوًّا قَدِيرًا
{\tiny\colorbox{cl_aya}{150}} إِنَّ الَّذِينَ يَكْفُرُونَ بِاللَّهِ وَرُسُلِهِ وَيُرِيدُونَ أَن يُفَرِّقُوا بَيْنَ اللَّهِ وَرُسُلِهِ وَيَقُولُونَ نُؤْمِنُ بِبَعْضٍ وَنَكْفُرُ بِبَعْضٍ وَيُرِيدُونَ أَن يَتَّخِذُوا بَيْنَ ذَلِكَ سَبِيلًا
{\tiny\colorbox{cl_aya}{151}} أُولَئِكَ هُمُ الْكَفِرُونَ حَقًّا وَأَعْتَدْنَا لِلْكَفِرِينَ عَذَابًا مُّهِينًا
{\tiny\colorbox{cl_aya}{152}} وَالَّذِينَ ءَامَنُوا بِاللَّهِ وَرُسُلِهِ وَلَمْ يُفَرِّقُوا بَيْنَ أَحَدٍ مِّنْهُمْ أُولَئِكَ سَوْفَ يُؤْتِيهِمْ أُجُورَهُمْ وَكَانَ اللَّهُ غَفُورًا رَّحِيمًا
{\tiny\colorbox{cl_aya}{153}} يَسَْٔلُكَ أَهْلُ الْكِتَبِ أَن تُنَزِّلَ عَلَيْهِمْ كِتَبًا مِّنَ السَّمَاءِ فَقَدْ سَأَلُوا مُوسَى أَكْبَرَ مِن ذَلِكَ فَقَالُوا أَرِنَا اللَّهَ جَهْرَةً فَأَخَذَتْهُمُ الصَّعِقَةُ بِظُلْمِهِمْ ثُمَّ اتَّخَذُوا الْعِجْلَ مِن بَعْدِ مَا جَاءَتْهُمُ الْبَيِّنَتُ فَعَفَوْنَا عَن ذَلِكَ وَءَاتَيْنَا مُوسَى سُلْطَنًا مُّبِينًا
{\tiny\colorbox{cl_aya}{154}} وَرَفَعْنَا فَوْقَهُمُ الطُّورَ بِمِيثَقِهِمْ وَقُلْنَا لَهُمُ ادْخُلُوا الْبَابَ سُجَّدًا وَقُلْنَا لَهُمْ لَا تَعْدُوا فِى السَّبْتِ وَأَخَذْنَا مِنْهُم مِّيثَقًا غَلِيظًا
{\tiny\colorbox{cl_aya}{155}} فَبِمَا نَقْضِهِم مِّيثَقَهُمْ وَكُفْرِهِم بَِٔايَتِ اللَّهِ وَقَتْلِهِمُ الْأَنبِيَاءَ بِغَيْرِ حَقٍّ وَقَوْلِهِمْ قُلُوبُنَا غُلْفٌ بَلْ طَبَعَ اللَّهُ عَلَيْهَا بِكُفْرِهِمْ فَلَا يُؤْمِنُونَ إِلَّا قَلِيلًا
{\tiny\colorbox{cl_aya}{156}} وَبِكُفْرِهِمْ وَقَوْلِهِمْ عَلَى مَرْيَمَ بُهْتَنًا عَظِيمًا
{\tiny\colorbox{cl_aya}{157}} وَقَوْلِهِمْ إِنَّا قَتَلْنَا الْمَسِيحَ عِيسَى ابْنَ مَرْيَمَ رَسُولَ اللَّهِ وَمَا قَتَلُوهُ وَمَا صَلَبُوهُ وَلَكِن شُبِّهَ لَهُمْ وَإِنَّ الَّذِينَ اخْتَلَفُوا فِيهِ لَفِى شَكٍّ مِّنْهُ مَا لَهُم بِهِ مِنْ عِلْمٍ إِلَّا اتِّبَاعَ الظَّنِّ وَمَا قَتَلُوهُ يَقِينًا
{\tiny\colorbox{cl_aya}{158}} بَل رَّفَعَهُ اللَّهُ إِلَيْهِ وَكَانَ اللَّهُ عَزِيزًا حَكِيمًا
{\tiny\colorbox{cl_aya}{159}} وَإِن مِّنْ أَهْلِ الْكِتَبِ إِلَّا لَيُؤْمِنَنَّ بِهِ قَبْلَ مَوْتِهِ وَيَوْمَ الْقِيَمَةِ يَكُونُ عَلَيْهِمْ شَهِيدًا
{\tiny\colorbox{cl_aya}{160}} فَبِظُلْمٍ مِّنَ الَّذِينَ هَادُوا حَرَّمْنَا عَلَيْهِمْ طَيِّبَتٍ أُحِلَّتْ لَهُمْ وَبِصَدِّهِمْ عَن سَبِيلِ اللَّهِ كَثِيرًا
{\tiny\colorbox{cl_aya}{161}} وَأَخْذِهِمُ الرِّبَوا وَقَدْ نُهُوا عَنْهُ وَأَكْلِهِمْ أَمْوَلَ النَّاسِ بِالْبَطِلِ وَأَعْتَدْنَا لِلْكَفِرِينَ مِنْهُمْ عَذَابًا أَلِيمًا
{\tiny\colorbox{cl_aya}{162}} لَّكِنِ الرَّسِخُونَ فِى الْعِلْمِ مِنْهُمْ وَالْمُؤْمِنُونَ يُؤْمِنُونَ بِمَا أُنزِلَ إِلَيْكَ وَمَا أُنزِلَ مِن قَبْلِكَ وَالْمُقِيمِينَ الصَّلَوةَ وَالْمُؤْتُونَ الزَّكَوةَ وَالْمُؤْمِنُونَ بِاللَّهِ وَالْيَوْمِ الْءَاخِرِ أُولَئِكَ سَنُؤْتِيهِمْ أَجْرًا عَظِيمًا
{\tiny\colorbox{cl_aya}{163}} إِنَّا أَوْحَيْنَا إِلَيْكَ كَمَا أَوْحَيْنَا إِلَى نُوحٍ وَالنَّبِيِّنَ مِن بَعْدِهِ وَأَوْحَيْنَا إِلَى إِبْرَهِيمَ وَإِسْمَعِيلَ وَإِسْحَقَ وَيَعْقُوبَ وَالْأَسْبَاطِ وَعِيسَى وَأَيُّوبَ وَيُونُسَ وَهَرُونَ وَسُلَيْمَنَ وَءَاتَيْنَا دَاوُدَ زَبُورًا
{\tiny\colorbox{cl_aya}{164}} وَرُسُلًا قَدْ قَصَصْنَهُمْ عَلَيْكَ مِن قَبْلُ وَرُسُلًا لَّمْ نَقْصُصْهُمْ عَلَيْكَ وَكَلَّمَ اللَّهُ مُوسَى تَكْلِيمًا
{\tiny\colorbox{cl_aya}{165}} رُّسُلًا مُّبَشِّرِينَ وَمُنذِرِينَ لِئَلَّا يَكُونَ لِلنَّاسِ عَلَى اللَّهِ حُجَّةٌ بَعْدَ الرُّسُلِ وَكَانَ اللَّهُ عَزِيزًا حَكِيمًا
{\tiny\colorbox{cl_aya}{166}} لَّكِنِ اللَّهُ يَشْهَدُ بِمَا أَنزَلَ إِلَيْكَ أَنزَلَهُ بِعِلْمِهِ وَالْمَلَئِكَةُ يَشْهَدُونَ وَكَفَى بِاللَّهِ شَهِيدًا
{\tiny\colorbox{cl_aya}{167}} إِنَّ الَّذِينَ كَفَرُوا وَصَدُّوا عَن سَبِيلِ اللَّهِ قَدْ ضَلُّوا ضَلَلًا بَعِيدًا
{\tiny\colorbox{cl_aya}{168}} إِنَّ الَّذِينَ كَفَرُوا وَظَلَمُوا لَمْ يَكُنِ اللَّهُ لِيَغْفِرَ لَهُمْ وَلَا لِيَهْدِيَهُمْ طَرِيقًا
{\tiny\colorbox{cl_aya}{169}} إِلَّا طَرِيقَ جَهَنَّمَ خَلِدِينَ فِيهَا أَبَدًا وَكَانَ ذَلِكَ عَلَى اللَّهِ يَسِيرًا
{\tiny\colorbox{cl_aya}{170}} يَأَيُّهَا النَّاسُ قَدْ جَاءَكُمُ الرَّسُولُ بِالْحَقِّ مِن رَّبِّكُمْ فََٔامِنُوا خَيْرًا لَّكُمْ وَإِن تَكْفُرُوا فَإِنَّ لِلَّهِ مَا فِى السَّمَوَتِ وَالْأَرْضِ وَكَانَ اللَّهُ عَلِيمًا حَكِيمًا
{\tiny\colorbox{cl_aya}{171}} يَأَهْلَ الْكِتَبِ لَا تَغْلُوا فِى دِينِكُمْ وَلَا تَقُولُوا عَلَى اللَّهِ إِلَّا الْحَقَّ إِنَّمَا الْمَسِيحُ عِيسَى ابْنُ مَرْيَمَ رَسُولُ اللَّهِ وَكَلِمَتُهُ أَلْقَىهَا إِلَى مَرْيَمَ وَرُوحٌ مِّنْهُ فََٔامِنُوا بِاللَّهِ وَرُسُلِهِ وَلَا تَقُولُوا ثَلَثَةٌ انتَهُوا خَيْرًا لَّكُمْ إِنَّمَا اللَّهُ إِلَهٌ وَحِدٌ سُبْحَنَهُ أَن يَكُونَ لَهُ وَلَدٌ لَّهُ مَا فِى السَّمَوَتِ وَمَا فِى الْأَرْضِ وَكَفَى بِاللَّهِ وَكِيلًا
{\tiny\colorbox{cl_aya}{172}} لَّن يَسْتَنكِفَ الْمَسِيحُ أَن يَكُونَ عَبْدًا لِّلَّهِ وَلَا الْمَلَئِكَةُ الْمُقَرَّبُونَ وَمَن يَسْتَنكِفْ عَنْ عِبَادَتِهِ وَيَسْتَكْبِرْ فَسَيَحْشُرُهُمْ إِلَيْهِ جَمِيعًا
{\tiny\colorbox{cl_aya}{173}} فَأَمَّا الَّذِينَ ءَامَنُوا وَعَمِلُوا الصَّلِحَتِ فَيُوَفِّيهِمْ أُجُورَهُمْ وَيَزِيدُهُم مِّن فَضْلِهِ وَأَمَّا الَّذِينَ اسْتَنكَفُوا وَاسْتَكْبَرُوا فَيُعَذِّبُهُمْ عَذَابًا أَلِيمًا وَلَا يَجِدُونَ لَهُم مِّن دُونِ اللَّهِ وَلِيًّا وَلَا نَصِيرًا
{\tiny\colorbox{cl_aya}{174}} يَأَيُّهَا النَّاسُ قَدْ جَاءَكُم بُرْهَنٌ مِّن رَّبِّكُمْ وَأَنزَلْنَا إِلَيْكُمْ نُورًا مُّبِينًا
{\tiny\colorbox{cl_aya}{175}} فَأَمَّا الَّذِينَ ءَامَنُوا بِاللَّهِ وَاعْتَصَمُوا بِهِ فَسَيُدْخِلُهُمْ فِى رَحْمَةٍ مِّنْهُ وَفَضْلٍ وَيَهْدِيهِمْ إِلَيْهِ صِرَطًا مُّسْتَقِيمًا
{\tiny\colorbox{cl_aya}{176}} يَسْتَفْتُونَكَ قُلِ اللَّهُ يُفْتِيكُمْ فِى الْكَلَلَةِ إِنِ امْرُؤٌا هَلَكَ لَيْسَ لَهُ وَلَدٌ وَلَهُ أُخْتٌ فَلَهَا نِصْفُ مَا تَرَكَ وَهُوَ يَرِثُهَا إِن لَّمْ يَكُن لَّهَا وَلَدٌ فَإِن كَانَتَا اثْنَتَيْنِ فَلَهُمَا الثُّلُثَانِ مِمَّا تَرَكَ وَإِن كَانُوا إِخْوَةً رِّجَالًا وَنِسَاءً فَلِلذَّكَرِ مِثْلُ حَظِّ الْأُنثَيَيْنِ يُبَيِّنُ اللَّهُ لَكُمْ أَن تَضِلُّوا وَاللَّهُ بِكُلِّ شَىْءٍ عَلِيمٌ
\end{document}