%\documentclass[12pt,a4paper]{article}
\documentclass[20pt,a4paper]{article}
\usepackage[margin=0.5in]{geometry}
\usepackage{polyglossia}
\usepackage[dvipsnames]{xcolor}
\pagenumbering{gobble}
% This beautiful one line disable the initial spacing at the beginning of a line
\usepackage[parfill]{parskip} 
\usepackage{setspace}
\setstretch{2}

\setdefaultlanguage[numerals=maghrib]{arabic}
\newfontfamily\arabicfont[Script=Arabic]{Amiri}

\title{}
\author{}
\date{}
\definecolor{cl_page}{gray}{0.98}
\definecolor{cl_aya}{HTML}{DEEEFF}

\begin{document}
\pagecolor{cl_page}

% Start %


{\tiny\colorbox{cl_aya}{1}} الْحَمْدُ لِلَّهِ الَّذِى خَلَقَ السَّمَوَتِ وَالْأَرْضَ وَجَعَلَ الظُّلُمَتِ وَالنُّورَ ثُمَّ الَّذِينَ كَفَرُوا بِرَبِّهِمْ يَعْدِلُونَ
{\tiny\colorbox{cl_aya}{2}} هُوَ الَّذِى خَلَقَكُم مِّن طِينٍ ثُمَّ قَضَى أَجَلًا وَأَجَلٌ مُّسَمًّى عِندَهُ ثُمَّ أَنتُمْ تَمْتَرُونَ
{\tiny\colorbox{cl_aya}{3}} وَهُوَ اللَّهُ فِى السَّمَوَتِ وَفِى الْأَرْضِ يَعْلَمُ سِرَّكُمْ وَجَهْرَكُمْ وَيَعْلَمُ مَا تَكْسِبُونَ
{\tiny\colorbox{cl_aya}{4}} وَمَا تَأْتِيهِم مِّنْ ءَايَةٍ مِّنْ ءَايَتِ رَبِّهِمْ إِلَّا كَانُوا عَنْهَا مُعْرِضِينَ
{\tiny\colorbox{cl_aya}{5}} فَقَدْ كَذَّبُوا بِالْحَقِّ لَمَّا جَاءَهُمْ فَسَوْفَ يَأْتِيهِمْ أَنبَؤُا مَا كَانُوا بِهِ يَسْتَهْزِءُونَ
{\tiny\colorbox{cl_aya}{6}} أَلَمْ يَرَوْا كَمْ أَهْلَكْنَا مِن قَبْلِهِم مِّن قَرْنٍ مَّكَّنَّهُمْ فِى الْأَرْضِ مَا لَمْ نُمَكِّن لَّكُمْ وَأَرْسَلْنَا السَّمَاءَ عَلَيْهِم مِّدْرَارًا وَجَعَلْنَا الْأَنْهَرَ تَجْرِى مِن تَحْتِهِمْ فَأَهْلَكْنَهُم بِذُنُوبِهِمْ وَأَنشَأْنَا مِن بَعْدِهِمْ قَرْنًا ءَاخَرِينَ
{\tiny\colorbox{cl_aya}{7}} وَلَوْ نَزَّلْنَا عَلَيْكَ كِتَبًا فِى قِرْطَاسٍ فَلَمَسُوهُ بِأَيْدِيهِمْ لَقَالَ الَّذِينَ كَفَرُوا إِنْ هَذَا إِلَّا سِحْرٌ مُّبِينٌ
{\tiny\colorbox{cl_aya}{8}} وَقَالُوا لَوْلَا أُنزِلَ عَلَيْهِ مَلَكٌ وَلَوْ أَنزَلْنَا مَلَكًا لَّقُضِىَ الْأَمْرُ ثُمَّ لَا يُنظَرُونَ
{\tiny\colorbox{cl_aya}{9}} وَلَوْ جَعَلْنَهُ مَلَكًا لَّجَعَلْنَهُ رَجُلًا وَلَلَبَسْنَا عَلَيْهِم مَّا يَلْبِسُونَ
{\tiny\colorbox{cl_aya}{10}} وَلَقَدِ اسْتُهْزِئَ بِرُسُلٍ مِّن قَبْلِكَ فَحَاقَ بِالَّذِينَ سَخِرُوا مِنْهُم مَّا كَانُوا بِهِ يَسْتَهْزِءُونَ
{\tiny\colorbox{cl_aya}{11}} قُلْ سِيرُوا فِى الْأَرْضِ ثُمَّ انظُرُوا كَيْفَ كَانَ عَقِبَةُ الْمُكَذِّبِينَ
{\tiny\colorbox{cl_aya}{12}} قُل لِّمَن مَّا فِى السَّمَوَتِ وَالْأَرْضِ قُل لِّلَّهِ كَتَبَ عَلَى نَفْسِهِ الرَّحْمَةَ لَيَجْمَعَنَّكُمْ إِلَى يَوْمِ الْقِيَمَةِ لَا رَيْبَ فِيهِ الَّذِينَ خَسِرُوا أَنفُسَهُمْ فَهُمْ لَا يُؤْمِنُونَ
{\tiny\colorbox{cl_aya}{13}} وَلَهُ مَا سَكَنَ فِى الَّيْلِ وَالنَّهَارِ وَهُوَ السَّمِيعُ الْعَلِيمُ
{\tiny\colorbox{cl_aya}{14}} قُلْ أَغَيْرَ اللَّهِ أَتَّخِذُ وَلِيًّا فَاطِرِ السَّمَوَتِ وَالْأَرْضِ وَهُوَ يُطْعِمُ وَلَا يُطْعَمُ قُلْ إِنِّى أُمِرْتُ أَنْ أَكُونَ أَوَّلَ مَنْ أَسْلَمَ وَلَا تَكُونَنَّ مِنَ الْمُشْرِكِينَ
{\tiny\colorbox{cl_aya}{15}} قُلْ إِنِّى أَخَافُ إِنْ عَصَيْتُ رَبِّى عَذَابَ يَوْمٍ عَظِيمٍ
{\tiny\colorbox{cl_aya}{16}} مَّن يُصْرَفْ عَنْهُ يَوْمَئِذٍ فَقَدْ رَحِمَهُ وَذَلِكَ الْفَوْزُ الْمُبِينُ
{\tiny\colorbox{cl_aya}{17}} وَإِن يَمْسَسْكَ اللَّهُ بِضُرٍّ فَلَا كَاشِفَ لَهُ إِلَّا هُوَ وَإِن يَمْسَسْكَ بِخَيْرٍ فَهُوَ عَلَى كُلِّ شَىْءٍ قَدِيرٌ
{\tiny\colorbox{cl_aya}{18}} وَهُوَ الْقَاهِرُ فَوْقَ عِبَادِهِ وَهُوَ الْحَكِيمُ الْخَبِيرُ
{\tiny\colorbox{cl_aya}{19}} قُلْ أَىُّ شَىْءٍ أَكْبَرُ شَهَدَةً قُلِ اللَّهُ شَهِيدٌ بَيْنِى وَبَيْنَكُمْ وَأُوحِىَ إِلَىَّ هَذَا الْقُرْءَانُ لِأُنذِرَكُم بِهِ وَمَن بَلَغَ أَئِنَّكُمْ لَتَشْهَدُونَ أَنَّ مَعَ اللَّهِ ءَالِهَةً أُخْرَى قُل لَّا أَشْهَدُ قُلْ إِنَّمَا هُوَ إِلَهٌ وَحِدٌ وَإِنَّنِى بَرِىءٌ مِّمَّا تُشْرِكُونَ
{\tiny\colorbox{cl_aya}{20}} الَّذِينَ ءَاتَيْنَهُمُ الْكِتَبَ يَعْرِفُونَهُ كَمَا يَعْرِفُونَ أَبْنَاءَهُمُ الَّذِينَ خَسِرُوا أَنفُسَهُمْ فَهُمْ لَا يُؤْمِنُونَ
{\tiny\colorbox{cl_aya}{21}} وَمَنْ أَظْلَمُ مِمَّنِ افْتَرَى عَلَى اللَّهِ كَذِبًا أَوْ كَذَّبَ بَِٔايَتِهِ إِنَّهُ لَا يُفْلِحُ الظَّلِمُونَ
{\tiny\colorbox{cl_aya}{22}} وَيَوْمَ نَحْشُرُهُمْ جَمِيعًا ثُمَّ نَقُولُ لِلَّذِينَ أَشْرَكُوا أَيْنَ شُرَكَاؤُكُمُ الَّذِينَ كُنتُمْ تَزْعُمُونَ
{\tiny\colorbox{cl_aya}{23}} ثُمَّ لَمْ تَكُن فِتْنَتُهُمْ إِلَّا أَن قَالُوا وَاللَّهِ رَبِّنَا مَا كُنَّا مُشْرِكِينَ
{\tiny\colorbox{cl_aya}{24}} انظُرْ كَيْفَ كَذَبُوا عَلَى أَنفُسِهِمْ وَضَلَّ عَنْهُم مَّا كَانُوا يَفْتَرُونَ
{\tiny\colorbox{cl_aya}{25}} وَمِنْهُم مَّن يَسْتَمِعُ إِلَيْكَ وَجَعَلْنَا عَلَى قُلُوبِهِمْ أَكِنَّةً أَن يَفْقَهُوهُ وَفِى ءَاذَانِهِمْ وَقْرًا وَإِن يَرَوْا كُلَّ ءَايَةٍ لَّا يُؤْمِنُوا بِهَا حَتَّى إِذَا جَاءُوكَ يُجَدِلُونَكَ يَقُولُ الَّذِينَ كَفَرُوا إِنْ هَذَا إِلَّا أَسَطِيرُ الْأَوَّلِينَ
{\tiny\colorbox{cl_aya}{26}} وَهُمْ يَنْهَوْنَ عَنْهُ وَيَنَْٔوْنَ عَنْهُ وَإِن يُهْلِكُونَ إِلَّا أَنفُسَهُمْ وَمَا يَشْعُرُونَ
{\tiny\colorbox{cl_aya}{27}} وَلَوْ تَرَى إِذْ وُقِفُوا عَلَى النَّارِ فَقَالُوا يَلَيْتَنَا نُرَدُّ وَلَا نُكَذِّبَ بَِٔايَتِ رَبِّنَا وَنَكُونَ مِنَ الْمُؤْمِنِينَ
{\tiny\colorbox{cl_aya}{28}} بَلْ بَدَا لَهُم مَّا كَانُوا يُخْفُونَ مِن قَبْلُ وَلَوْ رُدُّوا لَعَادُوا لِمَا نُهُوا عَنْهُ وَإِنَّهُمْ لَكَذِبُونَ
{\tiny\colorbox{cl_aya}{29}} وَقَالُوا إِنْ هِىَ إِلَّا حَيَاتُنَا الدُّنْيَا وَمَا نَحْنُ بِمَبْعُوثِينَ
{\tiny\colorbox{cl_aya}{30}} وَلَوْ تَرَى إِذْ وُقِفُوا عَلَى رَبِّهِمْ قَالَ أَلَيْسَ هَذَا بِالْحَقِّ قَالُوا بَلَى وَرَبِّنَا قَالَ فَذُوقُوا الْعَذَابَ بِمَا كُنتُمْ تَكْفُرُونَ
{\tiny\colorbox{cl_aya}{31}} قَدْ خَسِرَ الَّذِينَ كَذَّبُوا بِلِقَاءِ اللَّهِ حَتَّى إِذَا جَاءَتْهُمُ السَّاعَةُ بَغْتَةً قَالُوا يَحَسْرَتَنَا عَلَى مَا فَرَّطْنَا فِيهَا وَهُمْ يَحْمِلُونَ أَوْزَارَهُمْ عَلَى ظُهُورِهِمْ أَلَا سَاءَ مَا يَزِرُونَ
{\tiny\colorbox{cl_aya}{32}} وَمَا الْحَيَوةُ الدُّنْيَا إِلَّا لَعِبٌ وَلَهْوٌ وَلَلدَّارُ الْءَاخِرَةُ خَيْرٌ لِّلَّذِينَ يَتَّقُونَ أَفَلَا تَعْقِلُونَ
{\tiny\colorbox{cl_aya}{33}} قَدْ نَعْلَمُ إِنَّهُ لَيَحْزُنُكَ الَّذِى يَقُولُونَ فَإِنَّهُمْ لَا يُكَذِّبُونَكَ وَلَكِنَّ الظَّلِمِينَ بَِٔايَتِ اللَّهِ يَجْحَدُونَ
{\tiny\colorbox{cl_aya}{34}} وَلَقَدْ كُذِّبَتْ رُسُلٌ مِّن قَبْلِكَ فَصَبَرُوا عَلَى مَا كُذِّبُوا وَأُوذُوا حَتَّى أَتَىهُمْ نَصْرُنَا وَلَا مُبَدِّلَ لِكَلِمَتِ اللَّهِ وَلَقَدْ جَاءَكَ مِن نَّبَإِى الْمُرْسَلِينَ
{\tiny\colorbox{cl_aya}{35}} وَإِن كَانَ كَبُرَ عَلَيْكَ إِعْرَاضُهُمْ فَإِنِ اسْتَطَعْتَ أَن تَبْتَغِىَ نَفَقًا فِى الْأَرْضِ أَوْ سُلَّمًا فِى السَّمَاءِ فَتَأْتِيَهُم بَِٔايَةٍ وَلَوْ شَاءَ اللَّهُ لَجَمَعَهُمْ عَلَى الْهُدَى فَلَا تَكُونَنَّ مِنَ الْجَهِلِينَ
{\tiny\colorbox{cl_aya}{36}} إِنَّمَا يَسْتَجِيبُ الَّذِينَ يَسْمَعُونَ وَالْمَوْتَى يَبْعَثُهُمُ اللَّهُ ثُمَّ إِلَيْهِ يُرْجَعُونَ
{\tiny\colorbox{cl_aya}{37}} وَقَالُوا لَوْلَا نُزِّلَ عَلَيْهِ ءَايَةٌ مِّن رَّبِّهِ قُلْ إِنَّ اللَّهَ قَادِرٌ عَلَى أَن يُنَزِّلَ ءَايَةً وَلَكِنَّ أَكْثَرَهُمْ لَا يَعْلَمُونَ
{\tiny\colorbox{cl_aya}{38}} وَمَا مِن دَابَّةٍ فِى الْأَرْضِ وَلَا طَئِرٍ يَطِيرُ بِجَنَاحَيْهِ إِلَّا أُمَمٌ أَمْثَالُكُم مَّا فَرَّطْنَا فِى الْكِتَبِ مِن شَىْءٍ ثُمَّ إِلَى رَبِّهِمْ يُحْشَرُونَ
{\tiny\colorbox{cl_aya}{39}} وَالَّذِينَ كَذَّبُوا بَِٔايَتِنَا صُمٌّ وَبُكْمٌ فِى الظُّلُمَتِ مَن يَشَإِ اللَّهُ يُضْلِلْهُ وَمَن يَشَأْ يَجْعَلْهُ عَلَى صِرَطٍ مُّسْتَقِيمٍ
{\tiny\colorbox{cl_aya}{40}} قُلْ أَرَءَيْتَكُمْ إِنْ أَتَىكُمْ عَذَابُ اللَّهِ أَوْ أَتَتْكُمُ السَّاعَةُ أَغَيْرَ اللَّهِ تَدْعُونَ إِن كُنتُمْ صَدِقِينَ
{\tiny\colorbox{cl_aya}{41}} بَلْ إِيَّاهُ تَدْعُونَ فَيَكْشِفُ مَا تَدْعُونَ إِلَيْهِ إِن شَاءَ وَتَنسَوْنَ مَا تُشْرِكُونَ
{\tiny\colorbox{cl_aya}{42}} وَلَقَدْ أَرْسَلْنَا إِلَى أُمَمٍ مِّن قَبْلِكَ فَأَخَذْنَهُم بِالْبَأْسَاءِ وَالضَّرَّاءِ لَعَلَّهُمْ يَتَضَرَّعُونَ
{\tiny\colorbox{cl_aya}{43}} فَلَوْلَا إِذْ جَاءَهُم بَأْسُنَا تَضَرَّعُوا وَلَكِن قَسَتْ قُلُوبُهُمْ وَزَيَّنَ لَهُمُ الشَّيْطَنُ مَا كَانُوا يَعْمَلُونَ
{\tiny\colorbox{cl_aya}{44}} فَلَمَّا نَسُوا مَا ذُكِّرُوا بِهِ فَتَحْنَا عَلَيْهِمْ أَبْوَبَ كُلِّ شَىْءٍ حَتَّى إِذَا فَرِحُوا بِمَا أُوتُوا أَخَذْنَهُم بَغْتَةً فَإِذَا هُم مُّبْلِسُونَ
{\tiny\colorbox{cl_aya}{45}} فَقُطِعَ دَابِرُ الْقَوْمِ الَّذِينَ ظَلَمُوا وَالْحَمْدُ لِلَّهِ رَبِّ الْعَلَمِينَ
{\tiny\colorbox{cl_aya}{46}} قُلْ أَرَءَيْتُمْ إِنْ أَخَذَ اللَّهُ سَمْعَكُمْ وَأَبْصَرَكُمْ وَخَتَمَ عَلَى قُلُوبِكُم مَّنْ إِلَهٌ غَيْرُ اللَّهِ يَأْتِيكُم بِهِ انظُرْ كَيْفَ نُصَرِّفُ الْءَايَتِ ثُمَّ هُمْ يَصْدِفُونَ
{\tiny\colorbox{cl_aya}{47}} قُلْ أَرَءَيْتَكُمْ إِنْ أَتَىكُمْ عَذَابُ اللَّهِ بَغْتَةً أَوْ جَهْرَةً هَلْ يُهْلَكُ إِلَّا الْقَوْمُ الظَّلِمُونَ
{\tiny\colorbox{cl_aya}{48}} وَمَا نُرْسِلُ الْمُرْسَلِينَ إِلَّا مُبَشِّرِينَ وَمُنذِرِينَ فَمَنْ ءَامَنَ وَأَصْلَحَ فَلَا خَوْفٌ عَلَيْهِمْ وَلَا هُمْ يَحْزَنُونَ
{\tiny\colorbox{cl_aya}{49}} وَالَّذِينَ كَذَّبُوا بَِٔايَتِنَا يَمَسُّهُمُ الْعَذَابُ بِمَا كَانُوا يَفْسُقُونَ
{\tiny\colorbox{cl_aya}{50}} قُل لَّا أَقُولُ لَكُمْ عِندِى خَزَائِنُ اللَّهِ وَلَا أَعْلَمُ الْغَيْبَ وَلَا أَقُولُ لَكُمْ إِنِّى مَلَكٌ إِنْ أَتَّبِعُ إِلَّا مَا يُوحَى إِلَىَّ قُلْ هَلْ يَسْتَوِى الْأَعْمَى وَالْبَصِيرُ أَفَلَا تَتَفَكَّرُونَ
{\tiny\colorbox{cl_aya}{51}} وَأَنذِرْ بِهِ الَّذِينَ يَخَافُونَ أَن يُحْشَرُوا إِلَى رَبِّهِمْ لَيْسَ لَهُم مِّن دُونِهِ وَلِىٌّ وَلَا شَفِيعٌ لَّعَلَّهُمْ يَتَّقُونَ
{\tiny\colorbox{cl_aya}{52}} وَلَا تَطْرُدِ الَّذِينَ يَدْعُونَ رَبَّهُم بِالْغَدَوةِ وَالْعَشِىِّ يُرِيدُونَ وَجْهَهُ مَا عَلَيْكَ مِنْ حِسَابِهِم مِّن شَىْءٍ وَمَا مِنْ حِسَابِكَ عَلَيْهِم مِّن شَىْءٍ فَتَطْرُدَهُمْ فَتَكُونَ مِنَ الظَّلِمِينَ
{\tiny\colorbox{cl_aya}{53}} وَكَذَلِكَ فَتَنَّا بَعْضَهُم بِبَعْضٍ لِّيَقُولُوا أَهَؤُلَاءِ مَنَّ اللَّهُ عَلَيْهِم مِّن بَيْنِنَا أَلَيْسَ اللَّهُ بِأَعْلَمَ بِالشَّكِرِينَ
{\tiny\colorbox{cl_aya}{54}} وَإِذَا جَاءَكَ الَّذِينَ يُؤْمِنُونَ بَِٔايَتِنَا فَقُلْ سَلَمٌ عَلَيْكُمْ كَتَبَ رَبُّكُمْ عَلَى نَفْسِهِ الرَّحْمَةَ أَنَّهُ مَنْ عَمِلَ مِنكُمْ سُوءًا بِجَهَلَةٍ ثُمَّ تَابَ مِن بَعْدِهِ وَأَصْلَحَ فَأَنَّهُ غَفُورٌ رَّحِيمٌ
{\tiny\colorbox{cl_aya}{55}} وَكَذَلِكَ نُفَصِّلُ الْءَايَتِ وَلِتَسْتَبِينَ سَبِيلُ الْمُجْرِمِينَ
{\tiny\colorbox{cl_aya}{56}} قُلْ إِنِّى نُهِيتُ أَنْ أَعْبُدَ الَّذِينَ تَدْعُونَ مِن دُونِ اللَّهِ قُل لَّا أَتَّبِعُ أَهْوَاءَكُمْ قَدْ ضَلَلْتُ إِذًا وَمَا أَنَا مِنَ الْمُهْتَدِينَ
{\tiny\colorbox{cl_aya}{57}} قُلْ إِنِّى عَلَى بَيِّنَةٍ مِّن رَّبِّى وَكَذَّبْتُم بِهِ مَا عِندِى مَا تَسْتَعْجِلُونَ بِهِ إِنِ الْحُكْمُ إِلَّا لِلَّهِ يَقُصُّ الْحَقَّ وَهُوَ خَيْرُ الْفَصِلِينَ
{\tiny\colorbox{cl_aya}{58}} قُل لَّوْ أَنَّ عِندِى مَا تَسْتَعْجِلُونَ بِهِ لَقُضِىَ الْأَمْرُ بَيْنِى وَبَيْنَكُمْ وَاللَّهُ أَعْلَمُ بِالظَّلِمِينَ
{\tiny\colorbox{cl_aya}{59}} وَعِندَهُ مَفَاتِحُ الْغَيْبِ لَا يَعْلَمُهَا إِلَّا هُوَ وَيَعْلَمُ مَا فِى الْبَرِّ وَالْبَحْرِ وَمَا تَسْقُطُ مِن وَرَقَةٍ إِلَّا يَعْلَمُهَا وَلَا حَبَّةٍ فِى ظُلُمَتِ الْأَرْضِ وَلَا رَطْبٍ وَلَا يَابِسٍ إِلَّا فِى كِتَبٍ مُّبِينٍ
{\tiny\colorbox{cl_aya}{60}} وَهُوَ الَّذِى يَتَوَفَّىكُم بِالَّيْلِ وَيَعْلَمُ مَا جَرَحْتُم بِالنَّهَارِ ثُمَّ يَبْعَثُكُمْ فِيهِ لِيُقْضَى أَجَلٌ مُّسَمًّى ثُمَّ إِلَيْهِ مَرْجِعُكُمْ ثُمَّ يُنَبِّئُكُم بِمَا كُنتُمْ تَعْمَلُونَ
{\tiny\colorbox{cl_aya}{61}} وَهُوَ الْقَاهِرُ فَوْقَ عِبَادِهِ وَيُرْسِلُ عَلَيْكُمْ حَفَظَةً حَتَّى إِذَا جَاءَ أَحَدَكُمُ الْمَوْتُ تَوَفَّتْهُ رُسُلُنَا وَهُمْ لَا يُفَرِّطُونَ
{\tiny\colorbox{cl_aya}{62}} ثُمَّ رُدُّوا إِلَى اللَّهِ مَوْلَىهُمُ الْحَقِّ أَلَا لَهُ الْحُكْمُ وَهُوَ أَسْرَعُ الْحَسِبِينَ
{\tiny\colorbox{cl_aya}{63}} قُلْ مَن يُنَجِّيكُم مِّن ظُلُمَتِ الْبَرِّ وَالْبَحْرِ تَدْعُونَهُ تَضَرُّعًا وَخُفْيَةً لَّئِنْ أَنجَىنَا مِنْ هَذِهِ لَنَكُونَنَّ مِنَ الشَّكِرِينَ
{\tiny\colorbox{cl_aya}{64}} قُلِ اللَّهُ يُنَجِّيكُم مِّنْهَا وَمِن كُلِّ كَرْبٍ ثُمَّ أَنتُمْ تُشْرِكُونَ
{\tiny\colorbox{cl_aya}{65}} قُلْ هُوَ الْقَادِرُ عَلَى أَن يَبْعَثَ عَلَيْكُمْ عَذَابًا مِّن فَوْقِكُمْ أَوْ مِن تَحْتِ أَرْجُلِكُمْ أَوْ يَلْبِسَكُمْ شِيَعًا وَيُذِيقَ بَعْضَكُم بَأْسَ بَعْضٍ انظُرْ كَيْفَ نُصَرِّفُ الْءَايَتِ لَعَلَّهُمْ يَفْقَهُونَ
{\tiny\colorbox{cl_aya}{66}} وَكَذَّبَ بِهِ قَوْمُكَ وَهُوَ الْحَقُّ قُل لَّسْتُ عَلَيْكُم بِوَكِيلٍ
{\tiny\colorbox{cl_aya}{67}} لِّكُلِّ نَبَإٍ مُّسْتَقَرٌّ وَسَوْفَ تَعْلَمُونَ
{\tiny\colorbox{cl_aya}{68}} وَإِذَا رَأَيْتَ الَّذِينَ يَخُوضُونَ فِى ءَايَتِنَا فَأَعْرِضْ عَنْهُمْ حَتَّى يَخُوضُوا فِى حَدِيثٍ غَيْرِهِ وَإِمَّا يُنسِيَنَّكَ الشَّيْطَنُ فَلَا تَقْعُدْ بَعْدَ الذِّكْرَى مَعَ الْقَوْمِ الظَّلِمِينَ
{\tiny\colorbox{cl_aya}{69}} وَمَا عَلَى الَّذِينَ يَتَّقُونَ مِنْ حِسَابِهِم مِّن شَىْءٍ وَلَكِن ذِكْرَى لَعَلَّهُمْ يَتَّقُونَ
{\tiny\colorbox{cl_aya}{70}} وَذَرِ الَّذِينَ اتَّخَذُوا دِينَهُمْ لَعِبًا وَلَهْوًا وَغَرَّتْهُمُ الْحَيَوةُ الدُّنْيَا وَذَكِّرْ بِهِ أَن تُبْسَلَ نَفْسٌ بِمَا كَسَبَتْ لَيْسَ لَهَا مِن دُونِ اللَّهِ وَلِىٌّ وَلَا شَفِيعٌ وَإِن تَعْدِلْ كُلَّ عَدْلٍ لَّا يُؤْخَذْ مِنْهَا أُولَئِكَ الَّذِينَ أُبْسِلُوا بِمَا كَسَبُوا لَهُمْ شَرَابٌ مِّنْ حَمِيمٍ وَعَذَابٌ أَلِيمٌ بِمَا كَانُوا يَكْفُرُونَ
{\tiny\colorbox{cl_aya}{71}} قُلْ أَنَدْعُوا مِن دُونِ اللَّهِ مَا لَا يَنفَعُنَا وَلَا يَضُرُّنَا وَنُرَدُّ عَلَى أَعْقَابِنَا بَعْدَ إِذْ هَدَىنَا اللَّهُ كَالَّذِى اسْتَهْوَتْهُ الشَّيَطِينُ فِى الْأَرْضِ حَيْرَانَ لَهُ أَصْحَبٌ يَدْعُونَهُ إِلَى الْهُدَى ائْتِنَا قُلْ إِنَّ هُدَى اللَّهِ هُوَ الْهُدَى وَأُمِرْنَا لِنُسْلِمَ لِرَبِّ الْعَلَمِينَ
{\tiny\colorbox{cl_aya}{72}} وَأَنْ أَقِيمُوا الصَّلَوةَ وَاتَّقُوهُ وَهُوَ الَّذِى إِلَيْهِ تُحْشَرُونَ
{\tiny\colorbox{cl_aya}{73}} وَهُوَ الَّذِى خَلَقَ السَّمَوَتِ وَالْأَرْضَ بِالْحَقِّ وَيَوْمَ يَقُولُ كُن فَيَكُونُ قَوْلُهُ الْحَقُّ وَلَهُ الْمُلْكُ يَوْمَ يُنفَخُ فِى الصُّورِ عَلِمُ الْغَيْبِ وَالشَّهَدَةِ وَهُوَ الْحَكِيمُ الْخَبِيرُ
{\tiny\colorbox{cl_aya}{74}} وَإِذْ قَالَ إِبْرَهِيمُ لِأَبِيهِ ءَازَرَ أَتَتَّخِذُ أَصْنَامًا ءَالِهَةً إِنِّى أَرَىكَ وَقَوْمَكَ فِى ضَلَلٍ مُّبِينٍ
{\tiny\colorbox{cl_aya}{75}} وَكَذَلِكَ نُرِى إِبْرَهِيمَ مَلَكُوتَ السَّمَوَتِ وَالْأَرْضِ وَلِيَكُونَ مِنَ الْمُوقِنِينَ
{\tiny\colorbox{cl_aya}{76}} فَلَمَّا جَنَّ عَلَيْهِ الَّيْلُ رَءَا كَوْكَبًا قَالَ هَذَا رَبِّى فَلَمَّا أَفَلَ قَالَ لَا أُحِبُّ الْءَافِلِينَ
{\tiny\colorbox{cl_aya}{77}} فَلَمَّا رَءَا الْقَمَرَ بَازِغًا قَالَ هَذَا رَبِّى فَلَمَّا أَفَلَ قَالَ لَئِن لَّمْ يَهْدِنِى رَبِّى لَأَكُونَنَّ مِنَ الْقَوْمِ الضَّالِّينَ
{\tiny\colorbox{cl_aya}{78}} فَلَمَّا رَءَا الشَّمْسَ بَازِغَةً قَالَ هَذَا رَبِّى هَذَا أَكْبَرُ فَلَمَّا أَفَلَتْ قَالَ يَقَوْمِ إِنِّى بَرِىءٌ مِّمَّا تُشْرِكُونَ
{\tiny\colorbox{cl_aya}{79}} إِنِّى وَجَّهْتُ وَجْهِىَ لِلَّذِى فَطَرَ السَّمَوَتِ وَالْأَرْضَ حَنِيفًا وَمَا أَنَا مِنَ الْمُشْرِكِينَ
{\tiny\colorbox{cl_aya}{80}} وَحَاجَّهُ قَوْمُهُ قَالَ أَتُحَجُّونِّى فِى اللَّهِ وَقَدْ هَدَىنِ وَلَا أَخَافُ مَا تُشْرِكُونَ بِهِ إِلَّا أَن يَشَاءَ رَبِّى شَئًْا وَسِعَ رَبِّى كُلَّ شَىْءٍ عِلْمًا أَفَلَا تَتَذَكَّرُونَ
{\tiny\colorbox{cl_aya}{81}} وَكَيْفَ أَخَافُ مَا أَشْرَكْتُمْ وَلَا تَخَافُونَ أَنَّكُمْ أَشْرَكْتُم بِاللَّهِ مَا لَمْ يُنَزِّلْ بِهِ عَلَيْكُمْ سُلْطَنًا فَأَىُّ الْفَرِيقَيْنِ أَحَقُّ بِالْأَمْنِ إِن كُنتُمْ تَعْلَمُونَ
{\tiny\colorbox{cl_aya}{82}} الَّذِينَ ءَامَنُوا وَلَمْ يَلْبِسُوا إِيمَنَهُم بِظُلْمٍ أُولَئِكَ لَهُمُ الْأَمْنُ وَهُم مُّهْتَدُونَ
{\tiny\colorbox{cl_aya}{83}} وَتِلْكَ حُجَّتُنَا ءَاتَيْنَهَا إِبْرَهِيمَ عَلَى قَوْمِهِ نَرْفَعُ دَرَجَتٍ مَّن نَّشَاءُ إِنَّ رَبَّكَ حَكِيمٌ عَلِيمٌ
{\tiny\colorbox{cl_aya}{84}} وَوَهَبْنَا لَهُ إِسْحَقَ وَيَعْقُوبَ كُلًّا هَدَيْنَا وَنُوحًا هَدَيْنَا مِن قَبْلُ وَمِن ذُرِّيَّتِهِ دَاوُدَ وَسُلَيْمَنَ وَأَيُّوبَ وَيُوسُفَ وَمُوسَى وَهَرُونَ وَكَذَلِكَ نَجْزِى الْمُحْسِنِينَ
{\tiny\colorbox{cl_aya}{85}} وَزَكَرِيَّا وَيَحْيَى وَعِيسَى وَإِلْيَاسَ كُلٌّ مِّنَ الصَّلِحِينَ
{\tiny\colorbox{cl_aya}{86}} وَإِسْمَعِيلَ وَالْيَسَعَ وَيُونُسَ وَلُوطًا وَكُلًّا فَضَّلْنَا عَلَى الْعَلَمِينَ
{\tiny\colorbox{cl_aya}{87}} وَمِنْ ءَابَائِهِمْ وَذُرِّيَّتِهِمْ وَإِخْوَنِهِمْ وَاجْتَبَيْنَهُمْ وَهَدَيْنَهُمْ إِلَى صِرَطٍ مُّسْتَقِيمٍ
{\tiny\colorbox{cl_aya}{88}} ذَلِكَ هُدَى اللَّهِ يَهْدِى بِهِ مَن يَشَاءُ مِنْ عِبَادِهِ وَلَوْ أَشْرَكُوا لَحَبِطَ عَنْهُم مَّا كَانُوا يَعْمَلُونَ
{\tiny\colorbox{cl_aya}{89}} أُولَئِكَ الَّذِينَ ءَاتَيْنَهُمُ الْكِتَبَ وَالْحُكْمَ وَالنُّبُوَّةَ فَإِن يَكْفُرْ بِهَا هَؤُلَاءِ فَقَدْ وَكَّلْنَا بِهَا قَوْمًا لَّيْسُوا بِهَا بِكَفِرِينَ
{\tiny\colorbox{cl_aya}{90}} أُولَئِكَ الَّذِينَ هَدَى اللَّهُ فَبِهُدَىهُمُ اقْتَدِهْ قُل لَّا أَسَْٔلُكُمْ عَلَيْهِ أَجْرًا إِنْ هُوَ إِلَّا ذِكْرَى لِلْعَلَمِينَ
{\tiny\colorbox{cl_aya}{91}} وَمَا قَدَرُوا اللَّهَ حَقَّ قَدْرِهِ إِذْ قَالُوا مَا أَنزَلَ اللَّهُ عَلَى بَشَرٍ مِّن شَىْءٍ قُلْ مَنْ أَنزَلَ الْكِتَبَ الَّذِى جَاءَ بِهِ مُوسَى نُورًا وَهُدًى لِّلنَّاسِ تَجْعَلُونَهُ قَرَاطِيسَ تُبْدُونَهَا وَتُخْفُونَ كَثِيرًا وَعُلِّمْتُم مَّا لَمْ تَعْلَمُوا أَنتُمْ وَلَا ءَابَاؤُكُمْ قُلِ اللَّهُ ثُمَّ ذَرْهُمْ فِى خَوْضِهِمْ يَلْعَبُونَ
{\tiny\colorbox{cl_aya}{92}} وَهَذَا كِتَبٌ أَنزَلْنَهُ مُبَارَكٌ مُّصَدِّقُ الَّذِى بَيْنَ يَدَيْهِ وَلِتُنذِرَ أُمَّ الْقُرَى وَمَنْ حَوْلَهَا وَالَّذِينَ يُؤْمِنُونَ بِالْءَاخِرَةِ يُؤْمِنُونَ بِهِ وَهُمْ عَلَى صَلَاتِهِمْ يُحَافِظُونَ
{\tiny\colorbox{cl_aya}{93}} وَمَنْ أَظْلَمُ مِمَّنِ افْتَرَى عَلَى اللَّهِ كَذِبًا أَوْ قَالَ أُوحِىَ إِلَىَّ وَلَمْ يُوحَ إِلَيْهِ شَىْءٌ وَمَن قَالَ سَأُنزِلُ مِثْلَ مَا أَنزَلَ اللَّهُ وَلَوْ تَرَى إِذِ الظَّلِمُونَ فِى غَمَرَتِ الْمَوْتِ وَالْمَلَئِكَةُ بَاسِطُوا أَيْدِيهِمْ أَخْرِجُوا أَنفُسَكُمُ الْيَوْمَ تُجْزَوْنَ عَذَابَ الْهُونِ بِمَا كُنتُمْ تَقُولُونَ عَلَى اللَّهِ غَيْرَ الْحَقِّ وَكُنتُمْ عَنْ ءَايَتِهِ تَسْتَكْبِرُونَ
{\tiny\colorbox{cl_aya}{94}} وَلَقَدْ جِئْتُمُونَا فُرَدَى كَمَا خَلَقْنَكُمْ أَوَّلَ مَرَّةٍ وَتَرَكْتُم مَّا خَوَّلْنَكُمْ وَرَاءَ ظُهُورِكُمْ وَمَا نَرَى مَعَكُمْ شُفَعَاءَكُمُ الَّذِينَ زَعَمْتُمْ أَنَّهُمْ فِيكُمْ شُرَكَؤُا لَقَد تَّقَطَّعَ بَيْنَكُمْ وَضَلَّ عَنكُم مَّا كُنتُمْ تَزْعُمُونَ
{\tiny\colorbox{cl_aya}{95}} إِنَّ اللَّهَ فَالِقُ الْحَبِّ وَالنَّوَى يُخْرِجُ الْحَىَّ مِنَ الْمَيِّتِ وَمُخْرِجُ الْمَيِّتِ مِنَ الْحَىِّ ذَلِكُمُ اللَّهُ فَأَنَّى تُؤْفَكُونَ
{\tiny\colorbox{cl_aya}{96}} فَالِقُ الْإِصْبَاحِ وَجَعَلَ الَّيْلَ سَكَنًا وَالشَّمْسَ وَالْقَمَرَ حُسْبَانًا ذَلِكَ تَقْدِيرُ الْعَزِيزِ الْعَلِيمِ
{\tiny\colorbox{cl_aya}{97}} وَهُوَ الَّذِى جَعَلَ لَكُمُ النُّجُومَ لِتَهْتَدُوا بِهَا فِى ظُلُمَتِ الْبَرِّ وَالْبَحْرِ قَدْ فَصَّلْنَا الْءَايَتِ لِقَوْمٍ يَعْلَمُونَ
{\tiny\colorbox{cl_aya}{98}} وَهُوَ الَّذِى أَنشَأَكُم مِّن نَّفْسٍ وَحِدَةٍ فَمُسْتَقَرٌّ وَمُسْتَوْدَعٌ قَدْ فَصَّلْنَا الْءَايَتِ لِقَوْمٍ يَفْقَهُونَ
{\tiny\colorbox{cl_aya}{99}} وَهُوَ الَّذِى أَنزَلَ مِنَ السَّمَاءِ مَاءً فَأَخْرَجْنَا بِهِ نَبَاتَ كُلِّ شَىْءٍ فَأَخْرَجْنَا مِنْهُ خَضِرًا نُّخْرِجُ مِنْهُ حَبًّا مُّتَرَاكِبًا وَمِنَ النَّخْلِ مِن طَلْعِهَا قِنْوَانٌ دَانِيَةٌ وَجَنَّتٍ مِّنْ أَعْنَابٍ وَالزَّيْتُونَ وَالرُّمَّانَ مُشْتَبِهًا وَغَيْرَ مُتَشَبِهٍ انظُرُوا إِلَى ثَمَرِهِ إِذَا أَثْمَرَ وَيَنْعِهِ إِنَّ فِى ذَلِكُمْ لَءَايَتٍ لِّقَوْمٍ يُؤْمِنُونَ
{\tiny\colorbox{cl_aya}{100}} وَجَعَلُوا لِلَّهِ شُرَكَاءَ الْجِنَّ وَخَلَقَهُمْ وَخَرَقُوا لَهُ بَنِينَ وَبَنَتٍ بِغَيْرِ عِلْمٍ سُبْحَنَهُ وَتَعَلَى عَمَّا يَصِفُونَ
{\tiny\colorbox{cl_aya}{101}} بَدِيعُ السَّمَوَتِ وَالْأَرْضِ أَنَّى يَكُونُ لَهُ وَلَدٌ وَلَمْ تَكُن لَّهُ صَحِبَةٌ وَخَلَقَ كُلَّ شَىْءٍ وَهُوَ بِكُلِّ شَىْءٍ عَلِيمٌ
{\tiny\colorbox{cl_aya}{102}} ذَلِكُمُ اللَّهُ رَبُّكُمْ لَا إِلَهَ إِلَّا هُوَ خَلِقُ كُلِّ شَىْءٍ فَاعْبُدُوهُ وَهُوَ عَلَى كُلِّ شَىْءٍ وَكِيلٌ
{\tiny\colorbox{cl_aya}{103}} لَّا تُدْرِكُهُ الْأَبْصَرُ وَهُوَ يُدْرِكُ الْأَبْصَرَ وَهُوَ اللَّطِيفُ الْخَبِيرُ
{\tiny\colorbox{cl_aya}{104}} قَدْ جَاءَكُم بَصَائِرُ مِن رَّبِّكُمْ فَمَنْ أَبْصَرَ فَلِنَفْسِهِ وَمَنْ عَمِىَ فَعَلَيْهَا وَمَا أَنَا عَلَيْكُم بِحَفِيظٍ
{\tiny\colorbox{cl_aya}{105}} وَكَذَلِكَ نُصَرِّفُ الْءَايَتِ وَلِيَقُولُوا دَرَسْتَ وَلِنُبَيِّنَهُ لِقَوْمٍ يَعْلَمُونَ
{\tiny\colorbox{cl_aya}{106}} اتَّبِعْ مَا أُوحِىَ إِلَيْكَ مِن رَّبِّكَ لَا إِلَهَ إِلَّا هُوَ وَأَعْرِضْ عَنِ الْمُشْرِكِينَ
{\tiny\colorbox{cl_aya}{107}} وَلَوْ شَاءَ اللَّهُ مَا أَشْرَكُوا وَمَا جَعَلْنَكَ عَلَيْهِمْ حَفِيظًا وَمَا أَنتَ عَلَيْهِم بِوَكِيلٍ
{\tiny\colorbox{cl_aya}{108}} وَلَا تَسُبُّوا الَّذِينَ يَدْعُونَ مِن دُونِ اللَّهِ فَيَسُبُّوا اللَّهَ عَدْوًا بِغَيْرِ عِلْمٍ كَذَلِكَ زَيَّنَّا لِكُلِّ أُمَّةٍ عَمَلَهُمْ ثُمَّ إِلَى رَبِّهِم مَّرْجِعُهُمْ فَيُنَبِّئُهُم بِمَا كَانُوا يَعْمَلُونَ
{\tiny\colorbox{cl_aya}{109}} وَأَقْسَمُوا بِاللَّهِ جَهْدَ أَيْمَنِهِمْ لَئِن جَاءَتْهُمْ ءَايَةٌ لَّيُؤْمِنُنَّ بِهَا قُلْ إِنَّمَا الْءَايَتُ عِندَ اللَّهِ وَمَا يُشْعِرُكُمْ أَنَّهَا إِذَا جَاءَتْ لَا يُؤْمِنُونَ
{\tiny\colorbox{cl_aya}{110}} وَنُقَلِّبُ أَفِْٔدَتَهُمْ وَأَبْصَرَهُمْ كَمَا لَمْ يُؤْمِنُوا بِهِ أَوَّلَ مَرَّةٍ وَنَذَرُهُمْ فِى طُغْيَنِهِمْ يَعْمَهُونَ
{\tiny\colorbox{cl_aya}{111}} وَلَوْ أَنَّنَا نَزَّلْنَا إِلَيْهِمُ الْمَلَئِكَةَ وَكَلَّمَهُمُ الْمَوْتَى وَحَشَرْنَا عَلَيْهِمْ كُلَّ شَىْءٍ قُبُلًا مَّا كَانُوا لِيُؤْمِنُوا إِلَّا أَن يَشَاءَ اللَّهُ وَلَكِنَّ أَكْثَرَهُمْ يَجْهَلُونَ
{\tiny\colorbox{cl_aya}{112}} وَكَذَلِكَ جَعَلْنَا لِكُلِّ نَبِىٍّ عَدُوًّا شَيَطِينَ الْإِنسِ وَالْجِنِّ يُوحِى بَعْضُهُمْ إِلَى بَعْضٍ زُخْرُفَ الْقَوْلِ غُرُورًا وَلَوْ شَاءَ رَبُّكَ مَا فَعَلُوهُ فَذَرْهُمْ وَمَا يَفْتَرُونَ
{\tiny\colorbox{cl_aya}{113}} وَلِتَصْغَى إِلَيْهِ أَفِْٔدَةُ الَّذِينَ لَا يُؤْمِنُونَ بِالْءَاخِرَةِ وَلِيَرْضَوْهُ وَلِيَقْتَرِفُوا مَا هُم مُّقْتَرِفُونَ
{\tiny\colorbox{cl_aya}{114}} أَفَغَيْرَ اللَّهِ أَبْتَغِى حَكَمًا وَهُوَ الَّذِى أَنزَلَ إِلَيْكُمُ الْكِتَبَ مُفَصَّلًا وَالَّذِينَ ءَاتَيْنَهُمُ الْكِتَبَ يَعْلَمُونَ أَنَّهُ مُنَزَّلٌ مِّن رَّبِّكَ بِالْحَقِّ فَلَا تَكُونَنَّ مِنَ الْمُمْتَرِينَ
{\tiny\colorbox{cl_aya}{115}} وَتَمَّتْ كَلِمَتُ رَبِّكَ صِدْقًا وَعَدْلًا لَّا مُبَدِّلَ لِكَلِمَتِهِ وَهُوَ السَّمِيعُ الْعَلِيمُ
{\tiny\colorbox{cl_aya}{116}} وَإِن تُطِعْ أَكْثَرَ مَن فِى الْأَرْضِ يُضِلُّوكَ عَن سَبِيلِ اللَّهِ إِن يَتَّبِعُونَ إِلَّا الظَّنَّ وَإِنْ هُمْ إِلَّا يَخْرُصُونَ
{\tiny\colorbox{cl_aya}{117}} إِنَّ رَبَّكَ هُوَ أَعْلَمُ مَن يَضِلُّ عَن سَبِيلِهِ وَهُوَ أَعْلَمُ بِالْمُهْتَدِينَ
{\tiny\colorbox{cl_aya}{118}} فَكُلُوا مِمَّا ذُكِرَ اسْمُ اللَّهِ عَلَيْهِ إِن كُنتُم بَِٔايَتِهِ مُؤْمِنِينَ
{\tiny\colorbox{cl_aya}{119}} وَمَا لَكُمْ أَلَّا تَأْكُلُوا مِمَّا ذُكِرَ اسْمُ اللَّهِ عَلَيْهِ وَقَدْ فَصَّلَ لَكُم مَّا حَرَّمَ عَلَيْكُمْ إِلَّا مَا اضْطُرِرْتُمْ إِلَيْهِ وَإِنَّ كَثِيرًا لَّيُضِلُّونَ بِأَهْوَائِهِم بِغَيْرِ عِلْمٍ إِنَّ رَبَّكَ هُوَ أَعْلَمُ بِالْمُعْتَدِينَ
{\tiny\colorbox{cl_aya}{120}} وَذَرُوا ظَهِرَ الْإِثْمِ وَبَاطِنَهُ إِنَّ الَّذِينَ يَكْسِبُونَ الْإِثْمَ سَيُجْزَوْنَ بِمَا كَانُوا يَقْتَرِفُونَ
{\tiny\colorbox{cl_aya}{121}} وَلَا تَأْكُلُوا مِمَّا لَمْ يُذْكَرِ اسْمُ اللَّهِ عَلَيْهِ وَإِنَّهُ لَفِسْقٌ وَإِنَّ الشَّيَطِينَ لَيُوحُونَ إِلَى أَوْلِيَائِهِمْ لِيُجَدِلُوكُمْ وَإِنْ أَطَعْتُمُوهُمْ إِنَّكُمْ لَمُشْرِكُونَ
{\tiny\colorbox{cl_aya}{122}} أَوَمَن كَانَ مَيْتًا فَأَحْيَيْنَهُ وَجَعَلْنَا لَهُ نُورًا يَمْشِى بِهِ فِى النَّاسِ كَمَن مَّثَلُهُ فِى الظُّلُمَتِ لَيْسَ بِخَارِجٍ مِّنْهَا كَذَلِكَ زُيِّنَ لِلْكَفِرِينَ مَا كَانُوا يَعْمَلُونَ
{\tiny\colorbox{cl_aya}{123}} وَكَذَلِكَ جَعَلْنَا فِى كُلِّ قَرْيَةٍ أَكَبِرَ مُجْرِمِيهَا لِيَمْكُرُوا فِيهَا وَمَا يَمْكُرُونَ إِلَّا بِأَنفُسِهِمْ وَمَا يَشْعُرُونَ
{\tiny\colorbox{cl_aya}{124}} وَإِذَا جَاءَتْهُمْ ءَايَةٌ قَالُوا لَن نُّؤْمِنَ حَتَّى نُؤْتَى مِثْلَ مَا أُوتِىَ رُسُلُ اللَّهِ اللَّهُ أَعْلَمُ حَيْثُ يَجْعَلُ رِسَالَتَهُ سَيُصِيبُ الَّذِينَ أَجْرَمُوا صَغَارٌ عِندَ اللَّهِ وَعَذَابٌ شَدِيدٌ بِمَا كَانُوا يَمْكُرُونَ
{\tiny\colorbox{cl_aya}{125}} فَمَن يُرِدِ اللَّهُ أَن يَهْدِيَهُ يَشْرَحْ صَدْرَهُ لِلْإِسْلَمِ وَمَن يُرِدْ أَن يُضِلَّهُ يَجْعَلْ صَدْرَهُ ضَيِّقًا حَرَجًا كَأَنَّمَا يَصَّعَّدُ فِى السَّمَاءِ كَذَلِكَ يَجْعَلُ اللَّهُ الرِّجْسَ عَلَى الَّذِينَ لَا يُؤْمِنُونَ
{\tiny\colorbox{cl_aya}{126}} وَهَذَا صِرَطُ رَبِّكَ مُسْتَقِيمًا قَدْ فَصَّلْنَا الْءَايَتِ لِقَوْمٍ يَذَّكَّرُونَ
{\tiny\colorbox{cl_aya}{127}} لَهُمْ دَارُ السَّلَمِ عِندَ رَبِّهِمْ وَهُوَ وَلِيُّهُم بِمَا كَانُوا يَعْمَلُونَ
{\tiny\colorbox{cl_aya}{128}} وَيَوْمَ يَحْشُرُهُمْ جَمِيعًا يَمَعْشَرَ الْجِنِّ قَدِ اسْتَكْثَرْتُم مِّنَ الْإِنسِ وَقَالَ أَوْلِيَاؤُهُم مِّنَ الْإِنسِ رَبَّنَا اسْتَمْتَعَ بَعْضُنَا بِبَعْضٍ وَبَلَغْنَا أَجَلَنَا الَّذِى أَجَّلْتَ لَنَا قَالَ النَّارُ مَثْوَىكُمْ خَلِدِينَ فِيهَا إِلَّا مَا شَاءَ اللَّهُ إِنَّ رَبَّكَ حَكِيمٌ عَلِيمٌ
{\tiny\colorbox{cl_aya}{129}} وَكَذَلِكَ نُوَلِّى بَعْضَ الظَّلِمِينَ بَعْضًا بِمَا كَانُوا يَكْسِبُونَ
{\tiny\colorbox{cl_aya}{130}} يَمَعْشَرَ الْجِنِّ وَالْإِنسِ أَلَمْ يَأْتِكُمْ رُسُلٌ مِّنكُمْ يَقُصُّونَ عَلَيْكُمْ ءَايَتِى وَيُنذِرُونَكُمْ لِقَاءَ يَوْمِكُمْ هَذَا قَالُوا شَهِدْنَا عَلَى أَنفُسِنَا وَغَرَّتْهُمُ الْحَيَوةُ الدُّنْيَا وَشَهِدُوا عَلَى أَنفُسِهِمْ أَنَّهُمْ كَانُوا كَفِرِينَ
{\tiny\colorbox{cl_aya}{131}} ذَلِكَ أَن لَّمْ يَكُن رَّبُّكَ مُهْلِكَ الْقُرَى بِظُلْمٍ وَأَهْلُهَا غَفِلُونَ
{\tiny\colorbox{cl_aya}{132}} وَلِكُلٍّ دَرَجَتٌ مِّمَّا عَمِلُوا وَمَا رَبُّكَ بِغَفِلٍ عَمَّا يَعْمَلُونَ
{\tiny\colorbox{cl_aya}{133}} وَرَبُّكَ الْغَنِىُّ ذُو الرَّحْمَةِ إِن يَشَأْ يُذْهِبْكُمْ وَيَسْتَخْلِفْ مِن بَعْدِكُم مَّا يَشَاءُ كَمَا أَنشَأَكُم مِّن ذُرِّيَّةِ قَوْمٍ ءَاخَرِينَ
{\tiny\colorbox{cl_aya}{134}} إِنَّ مَا تُوعَدُونَ لَءَاتٍ وَمَا أَنتُم بِمُعْجِزِينَ
{\tiny\colorbox{cl_aya}{135}} قُلْ يَقَوْمِ اعْمَلُوا عَلَى مَكَانَتِكُمْ إِنِّى عَامِلٌ فَسَوْفَ تَعْلَمُونَ مَن تَكُونُ لَهُ عَقِبَةُ الدَّارِ إِنَّهُ لَا يُفْلِحُ الظَّلِمُونَ
{\tiny\colorbox{cl_aya}{136}} وَجَعَلُوا لِلَّهِ مِمَّا ذَرَأَ مِنَ الْحَرْثِ وَالْأَنْعَمِ نَصِيبًا فَقَالُوا هَذَا لِلَّهِ بِزَعْمِهِمْ وَهَذَا لِشُرَكَائِنَا فَمَا كَانَ لِشُرَكَائِهِمْ فَلَا يَصِلُ إِلَى اللَّهِ وَمَا كَانَ لِلَّهِ فَهُوَ يَصِلُ إِلَى شُرَكَائِهِمْ سَاءَ مَا يَحْكُمُونَ
{\tiny\colorbox{cl_aya}{137}} وَكَذَلِكَ زَيَّنَ لِكَثِيرٍ مِّنَ الْمُشْرِكِينَ قَتْلَ أَوْلَدِهِمْ شُرَكَاؤُهُمْ لِيُرْدُوهُمْ وَلِيَلْبِسُوا عَلَيْهِمْ دِينَهُمْ وَلَوْ شَاءَ اللَّهُ مَا فَعَلُوهُ فَذَرْهُمْ وَمَا يَفْتَرُونَ
{\tiny\colorbox{cl_aya}{138}} وَقَالُوا هَذِهِ أَنْعَمٌ وَحَرْثٌ حِجْرٌ لَّا يَطْعَمُهَا إِلَّا مَن نَّشَاءُ بِزَعْمِهِمْ وَأَنْعَمٌ حُرِّمَتْ ظُهُورُهَا وَأَنْعَمٌ لَّا يَذْكُرُونَ اسْمَ اللَّهِ عَلَيْهَا افْتِرَاءً عَلَيْهِ سَيَجْزِيهِم بِمَا كَانُوا يَفْتَرُونَ
{\tiny\colorbox{cl_aya}{139}} وَقَالُوا مَا فِى بُطُونِ هَذِهِ الْأَنْعَمِ خَالِصَةٌ لِّذُكُورِنَا وَمُحَرَّمٌ عَلَى أَزْوَجِنَا وَإِن يَكُن مَّيْتَةً فَهُمْ فِيهِ شُرَكَاءُ سَيَجْزِيهِمْ وَصْفَهُمْ إِنَّهُ حَكِيمٌ عَلِيمٌ
{\tiny\colorbox{cl_aya}{140}} قَدْ خَسِرَ الَّذِينَ قَتَلُوا أَوْلَدَهُمْ سَفَهًا بِغَيْرِ عِلْمٍ وَحَرَّمُوا مَا رَزَقَهُمُ اللَّهُ افْتِرَاءً عَلَى اللَّهِ قَدْ ضَلُّوا وَمَا كَانُوا مُهْتَدِينَ
{\tiny\colorbox{cl_aya}{141}} وَهُوَ الَّذِى أَنشَأَ جَنَّتٍ مَّعْرُوشَتٍ وَغَيْرَ مَعْرُوشَتٍ وَالنَّخْلَ وَالزَّرْعَ مُخْتَلِفًا أُكُلُهُ وَالزَّيْتُونَ وَالرُّمَّانَ مُتَشَبِهًا وَغَيْرَ مُتَشَبِهٍ كُلُوا مِن ثَمَرِهِ إِذَا أَثْمَرَ وَءَاتُوا حَقَّهُ يَوْمَ حَصَادِهِ وَلَا تُسْرِفُوا إِنَّهُ لَا يُحِبُّ الْمُسْرِفِينَ
{\tiny\colorbox{cl_aya}{142}} وَمِنَ الْأَنْعَمِ حَمُولَةً وَفَرْشًا كُلُوا مِمَّا رَزَقَكُمُ اللَّهُ وَلَا تَتَّبِعُوا خُطُوَتِ الشَّيْطَنِ إِنَّهُ لَكُمْ عَدُوٌّ مُّبِينٌ
{\tiny\colorbox{cl_aya}{143}} ثَمَنِيَةَ أَزْوَجٍ مِّنَ الضَّأْنِ اثْنَيْنِ وَمِنَ الْمَعْزِ اثْنَيْنِ قُلْ ءَالذَّكَرَيْنِ حَرَّمَ أَمِ الْأُنثَيَيْنِ أَمَّا اشْتَمَلَتْ عَلَيْهِ أَرْحَامُ الْأُنثَيَيْنِ نَبُِّٔونِى بِعِلْمٍ إِن كُنتُمْ صَدِقِينَ
{\tiny\colorbox{cl_aya}{144}} وَمِنَ الْإِبِلِ اثْنَيْنِ وَمِنَ الْبَقَرِ اثْنَيْنِ قُلْ ءَالذَّكَرَيْنِ حَرَّمَ أَمِ الْأُنثَيَيْنِ أَمَّا اشْتَمَلَتْ عَلَيْهِ أَرْحَامُ الْأُنثَيَيْنِ أَمْ كُنتُمْ شُهَدَاءَ إِذْ وَصَّىكُمُ اللَّهُ بِهَذَا فَمَنْ أَظْلَمُ مِمَّنِ افْتَرَى عَلَى اللَّهِ كَذِبًا لِّيُضِلَّ النَّاسَ بِغَيْرِ عِلْمٍ إِنَّ اللَّهَ لَا يَهْدِى الْقَوْمَ الظَّلِمِينَ
{\tiny\colorbox{cl_aya}{145}} قُل لَّا أَجِدُ فِى مَا أُوحِىَ إِلَىَّ مُحَرَّمًا عَلَى طَاعِمٍ يَطْعَمُهُ إِلَّا أَن يَكُونَ مَيْتَةً أَوْ دَمًا مَّسْفُوحًا أَوْ لَحْمَ خِنزِيرٍ فَإِنَّهُ رِجْسٌ أَوْ فِسْقًا أُهِلَّ لِغَيْرِ اللَّهِ بِهِ فَمَنِ اضْطُرَّ غَيْرَ بَاغٍ وَلَا عَادٍ فَإِنَّ رَبَّكَ غَفُورٌ رَّحِيمٌ
{\tiny\colorbox{cl_aya}{146}} وَعَلَى الَّذِينَ هَادُوا حَرَّمْنَا كُلَّ ذِى ظُفُرٍ وَمِنَ الْبَقَرِ وَالْغَنَمِ حَرَّمْنَا عَلَيْهِمْ شُحُومَهُمَا إِلَّا مَا حَمَلَتْ ظُهُورُهُمَا أَوِ الْحَوَايَا أَوْ مَا اخْتَلَطَ بِعَظْمٍ ذَلِكَ جَزَيْنَهُم بِبَغْيِهِمْ وَإِنَّا لَصَدِقُونَ
{\tiny\colorbox{cl_aya}{147}} فَإِن كَذَّبُوكَ فَقُل رَّبُّكُمْ ذُو رَحْمَةٍ وَسِعَةٍ وَلَا يُرَدُّ بَأْسُهُ عَنِ الْقَوْمِ الْمُجْرِمِينَ
{\tiny\colorbox{cl_aya}{148}} سَيَقُولُ الَّذِينَ أَشْرَكُوا لَوْ شَاءَ اللَّهُ مَا أَشْرَكْنَا وَلَا ءَابَاؤُنَا وَلَا حَرَّمْنَا مِن شَىْءٍ كَذَلِكَ كَذَّبَ الَّذِينَ مِن قَبْلِهِمْ حَتَّى ذَاقُوا بَأْسَنَا قُلْ هَلْ عِندَكُم مِّنْ عِلْمٍ فَتُخْرِجُوهُ لَنَا إِن تَتَّبِعُونَ إِلَّا الظَّنَّ وَإِنْ أَنتُمْ إِلَّا تَخْرُصُونَ
{\tiny\colorbox{cl_aya}{149}} قُلْ فَلِلَّهِ الْحُجَّةُ الْبَلِغَةُ فَلَوْ شَاءَ لَهَدَىكُمْ أَجْمَعِينَ
{\tiny\colorbox{cl_aya}{150}} قُلْ هَلُمَّ شُهَدَاءَكُمُ الَّذِينَ يَشْهَدُونَ أَنَّ اللَّهَ حَرَّمَ هَذَا فَإِن شَهِدُوا فَلَا تَشْهَدْ مَعَهُمْ وَلَا تَتَّبِعْ أَهْوَاءَ الَّذِينَ كَذَّبُوا بَِٔايَتِنَا وَالَّذِينَ لَا يُؤْمِنُونَ بِالْءَاخِرَةِ وَهُم بِرَبِّهِمْ يَعْدِلُونَ
{\tiny\colorbox{cl_aya}{151}} قُلْ تَعَالَوْا أَتْلُ مَا حَرَّمَ رَبُّكُمْ عَلَيْكُمْ أَلَّا تُشْرِكُوا بِهِ شَئًْا وَبِالْوَلِدَيْنِ إِحْسَنًا وَلَا تَقْتُلُوا أَوْلَدَكُم مِّنْ إِمْلَقٍ نَّحْنُ نَرْزُقُكُمْ وَإِيَّاهُمْ وَلَا تَقْرَبُوا الْفَوَحِشَ مَا ظَهَرَ مِنْهَا وَمَا بَطَنَ وَلَا تَقْتُلُوا النَّفْسَ الَّتِى حَرَّمَ اللَّهُ إِلَّا بِالْحَقِّ ذَلِكُمْ وَصَّىكُم بِهِ لَعَلَّكُمْ تَعْقِلُونَ
{\tiny\colorbox{cl_aya}{152}} وَلَا تَقْرَبُوا مَالَ الْيَتِيمِ إِلَّا بِالَّتِى هِىَ أَحْسَنُ حَتَّى يَبْلُغَ أَشُدَّهُ وَأَوْفُوا الْكَيْلَ وَالْمِيزَانَ بِالْقِسْطِ لَا نُكَلِّفُ نَفْسًا إِلَّا وُسْعَهَا وَإِذَا قُلْتُمْ فَاعْدِلُوا وَلَوْ كَانَ ذَا قُرْبَى وَبِعَهْدِ اللَّهِ أَوْفُوا ذَلِكُمْ وَصَّىكُم بِهِ لَعَلَّكُمْ تَذَكَّرُونَ
{\tiny\colorbox{cl_aya}{153}} وَأَنَّ هَذَا صِرَطِى مُسْتَقِيمًا فَاتَّبِعُوهُ وَلَا تَتَّبِعُوا السُّبُلَ فَتَفَرَّقَ بِكُمْ عَن سَبِيلِهِ ذَلِكُمْ وَصَّىكُم بِهِ لَعَلَّكُمْ تَتَّقُونَ
{\tiny\colorbox{cl_aya}{154}} ثُمَّ ءَاتَيْنَا مُوسَى الْكِتَبَ تَمَامًا عَلَى الَّذِى أَحْسَنَ وَتَفْصِيلًا لِّكُلِّ شَىْءٍ وَهُدًى وَرَحْمَةً لَّعَلَّهُم بِلِقَاءِ رَبِّهِمْ يُؤْمِنُونَ
{\tiny\colorbox{cl_aya}{155}} وَهَذَا كِتَبٌ أَنزَلْنَهُ مُبَارَكٌ فَاتَّبِعُوهُ وَاتَّقُوا لَعَلَّكُمْ تُرْحَمُونَ
{\tiny\colorbox{cl_aya}{156}} أَن تَقُولُوا إِنَّمَا أُنزِلَ الْكِتَبُ عَلَى طَائِفَتَيْنِ مِن قَبْلِنَا وَإِن كُنَّا عَن دِرَاسَتِهِمْ لَغَفِلِينَ
{\tiny\colorbox{cl_aya}{157}} أَوْ تَقُولُوا لَوْ أَنَّا أُنزِلَ عَلَيْنَا الْكِتَبُ لَكُنَّا أَهْدَى مِنْهُمْ فَقَدْ جَاءَكُم بَيِّنَةٌ مِّن رَّبِّكُمْ وَهُدًى وَرَحْمَةٌ فَمَنْ أَظْلَمُ مِمَّن كَذَّبَ بَِٔايَتِ اللَّهِ وَصَدَفَ عَنْهَا سَنَجْزِى الَّذِينَ يَصْدِفُونَ عَنْ ءَايَتِنَا سُوءَ الْعَذَابِ بِمَا كَانُوا يَصْدِفُونَ
{\tiny\colorbox{cl_aya}{158}} هَلْ يَنظُرُونَ إِلَّا أَن تَأْتِيَهُمُ الْمَلَئِكَةُ أَوْ يَأْتِىَ رَبُّكَ أَوْ يَأْتِىَ بَعْضُ ءَايَتِ رَبِّكَ يَوْمَ يَأْتِى بَعْضُ ءَايَتِ رَبِّكَ لَا يَنفَعُ نَفْسًا إِيمَنُهَا لَمْ تَكُنْ ءَامَنَتْ مِن قَبْلُ أَوْ كَسَبَتْ فِى إِيمَنِهَا خَيْرًا قُلِ انتَظِرُوا إِنَّا مُنتَظِرُونَ
{\tiny\colorbox{cl_aya}{159}} إِنَّ الَّذِينَ فَرَّقُوا دِينَهُمْ وَكَانُوا شِيَعًا لَّسْتَ مِنْهُمْ فِى شَىْءٍ إِنَّمَا أَمْرُهُمْ إِلَى اللَّهِ ثُمَّ يُنَبِّئُهُم بِمَا كَانُوا يَفْعَلُونَ
{\tiny\colorbox{cl_aya}{160}} مَن جَاءَ بِالْحَسَنَةِ فَلَهُ عَشْرُ أَمْثَالِهَا وَمَن جَاءَ بِالسَّيِّئَةِ فَلَا يُجْزَى إِلَّا مِثْلَهَا وَهُمْ لَا يُظْلَمُونَ
{\tiny\colorbox{cl_aya}{161}} قُلْ إِنَّنِى هَدَىنِى رَبِّى إِلَى صِرَطٍ مُّسْتَقِيمٍ دِينًا قِيَمًا مِّلَّةَ إِبْرَهِيمَ حَنِيفًا وَمَا كَانَ مِنَ الْمُشْرِكِينَ
{\tiny\colorbox{cl_aya}{162}} قُلْ إِنَّ صَلَاتِى وَنُسُكِى وَمَحْيَاىَ وَمَمَاتِى لِلَّهِ رَبِّ الْعَلَمِينَ
{\tiny\colorbox{cl_aya}{163}} لَا شَرِيكَ لَهُ وَبِذَلِكَ أُمِرْتُ وَأَنَا أَوَّلُ الْمُسْلِمِينَ
{\tiny\colorbox{cl_aya}{164}} قُلْ أَغَيْرَ اللَّهِ أَبْغِى رَبًّا وَهُوَ رَبُّ كُلِّ شَىْءٍ وَلَا تَكْسِبُ كُلُّ نَفْسٍ إِلَّا عَلَيْهَا وَلَا تَزِرُ وَازِرَةٌ وِزْرَ أُخْرَى ثُمَّ إِلَى رَبِّكُم مَّرْجِعُكُمْ فَيُنَبِّئُكُم بِمَا كُنتُمْ فِيهِ تَخْتَلِفُونَ
{\tiny\colorbox{cl_aya}{165}} وَهُوَ الَّذِى جَعَلَكُمْ خَلَئِفَ الْأَرْضِ وَرَفَعَ بَعْضَكُمْ فَوْقَ بَعْضٍ دَرَجَتٍ لِّيَبْلُوَكُمْ فِى مَا ءَاتَىكُمْ إِنَّ رَبَّكَ سَرِيعُ الْعِقَابِ وَإِنَّهُ لَغَفُورٌ رَّحِيمٌ
\end{document}