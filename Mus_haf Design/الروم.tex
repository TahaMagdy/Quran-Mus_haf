%\documentclass[12pt,a4paper]{article}
\documentclass[20pt,a4paper]{article}
\usepackage[margin=0.5in]{geometry}
\usepackage{polyglossia}
\usepackage[dvipsnames]{xcolor}
\pagenumbering{gobble}
% This beautiful one line disable the initial spacing at the beginning of a line
\usepackage[parfill]{parskip} 
\usepackage{setspace}
\setstretch{2}

\setdefaultlanguage[numerals=maghrib]{arabic}
\newfontfamily\arabicfont[Script=Arabic]{Amiri}

\title{}
\author{}
\date{}
\definecolor{cl_page}{gray}{0.98}
\definecolor{cl_aya}{HTML}{DEEEFF}

\begin{document}
\pagecolor{cl_page}

% Start %


{\tiny\colorbox{cl_aya}{1}} الم
{\tiny\colorbox{cl_aya}{2}} غُلِبَتِ الرُّومُ
{\tiny\colorbox{cl_aya}{3}} فِى أَدْنَى الْأَرْضِ وَهُم مِّن بَعْدِ غَلَبِهِمْ سَيَغْلِبُونَ
{\tiny\colorbox{cl_aya}{4}} فِى بِضْعِ سِنِينَ لِلَّهِ الْأَمْرُ مِن قَبْلُ وَمِن بَعْدُ وَيَوْمَئِذٍ يَفْرَحُ الْمُؤْمِنُونَ
{\tiny\colorbox{cl_aya}{5}} بِنَصْرِ اللَّهِ يَنصُرُ مَن يَشَاءُ وَهُوَ الْعَزِيزُ الرَّحِيمُ
{\tiny\colorbox{cl_aya}{6}} وَعْدَ اللَّهِ لَا يُخْلِفُ اللَّهُ وَعْدَهُ وَلَكِنَّ أَكْثَرَ النَّاسِ لَا يَعْلَمُونَ
{\tiny\colorbox{cl_aya}{7}} يَعْلَمُونَ ظَهِرًا مِّنَ الْحَيَوةِ الدُّنْيَا وَهُمْ عَنِ الْءَاخِرَةِ هُمْ غَفِلُونَ
{\tiny\colorbox{cl_aya}{8}} أَوَلَمْ يَتَفَكَّرُوا فِى أَنفُسِهِم مَّا خَلَقَ اللَّهُ السَّمَوَتِ وَالْأَرْضَ وَمَا بَيْنَهُمَا إِلَّا بِالْحَقِّ وَأَجَلٍ مُّسَمًّى وَإِنَّ كَثِيرًا مِّنَ النَّاسِ بِلِقَائِ رَبِّهِمْ لَكَفِرُونَ
{\tiny\colorbox{cl_aya}{9}} أَوَلَمْ يَسِيرُوا فِى الْأَرْضِ فَيَنظُرُوا كَيْفَ كَانَ عَقِبَةُ الَّذِينَ مِن قَبْلِهِمْ كَانُوا أَشَدَّ مِنْهُمْ قُوَّةً وَأَثَارُوا الْأَرْضَ وَعَمَرُوهَا أَكْثَرَ مِمَّا عَمَرُوهَا وَجَاءَتْهُمْ رُسُلُهُم بِالْبَيِّنَتِ فَمَا كَانَ اللَّهُ لِيَظْلِمَهُمْ وَلَكِن كَانُوا أَنفُسَهُمْ يَظْلِمُونَ
{\tiny\colorbox{cl_aya}{10}} ثُمَّ كَانَ عَقِبَةَ الَّذِينَ أَسَُٔوا السُّوأَى أَن كَذَّبُوا بَِٔايَتِ اللَّهِ وَكَانُوا بِهَا يَسْتَهْزِءُونَ
{\tiny\colorbox{cl_aya}{11}} اللَّهُ يَبْدَؤُا الْخَلْقَ ثُمَّ يُعِيدُهُ ثُمَّ إِلَيْهِ تُرْجَعُونَ
{\tiny\colorbox{cl_aya}{12}} وَيَوْمَ تَقُومُ السَّاعَةُ يُبْلِسُ الْمُجْرِمُونَ
{\tiny\colorbox{cl_aya}{13}} وَلَمْ يَكُن لَّهُم مِّن شُرَكَائِهِمْ شُفَعَؤُا وَكَانُوا بِشُرَكَائِهِمْ كَفِرِينَ
{\tiny\colorbox{cl_aya}{14}} وَيَوْمَ تَقُومُ السَّاعَةُ يَوْمَئِذٍ يَتَفَرَّقُونَ
{\tiny\colorbox{cl_aya}{15}} فَأَمَّا الَّذِينَ ءَامَنُوا وَعَمِلُوا الصَّلِحَتِ فَهُمْ فِى رَوْضَةٍ يُحْبَرُونَ
{\tiny\colorbox{cl_aya}{16}} وَأَمَّا الَّذِينَ كَفَرُوا وَكَذَّبُوا بَِٔايَتِنَا وَلِقَائِ الْءَاخِرَةِ فَأُولَئِكَ فِى الْعَذَابِ مُحْضَرُونَ
{\tiny\colorbox{cl_aya}{17}} فَسُبْحَنَ اللَّهِ حِينَ تُمْسُونَ وَحِينَ تُصْبِحُونَ
{\tiny\colorbox{cl_aya}{18}} وَلَهُ الْحَمْدُ فِى السَّمَوَتِ وَالْأَرْضِ وَعَشِيًّا وَحِينَ تُظْهِرُونَ
{\tiny\colorbox{cl_aya}{19}} يُخْرِجُ الْحَىَّ مِنَ الْمَيِّتِ وَيُخْرِجُ الْمَيِّتَ مِنَ الْحَىِّ وَيُحْىِ الْأَرْضَ بَعْدَ مَوْتِهَا وَكَذَلِكَ تُخْرَجُونَ
{\tiny\colorbox{cl_aya}{20}} وَمِنْ ءَايَتِهِ أَنْ خَلَقَكُم مِّن تُرَابٍ ثُمَّ إِذَا أَنتُم بَشَرٌ تَنتَشِرُونَ
{\tiny\colorbox{cl_aya}{21}} وَمِنْ ءَايَتِهِ أَنْ خَلَقَ لَكُم مِّنْ أَنفُسِكُمْ أَزْوَجًا لِّتَسْكُنُوا إِلَيْهَا وَجَعَلَ بَيْنَكُم مَّوَدَّةً وَرَحْمَةً إِنَّ فِى ذَلِكَ لَءَايَتٍ لِّقَوْمٍ يَتَفَكَّرُونَ
{\tiny\colorbox{cl_aya}{22}} وَمِنْ ءَايَتِهِ خَلْقُ السَّمَوَتِ وَالْأَرْضِ وَاخْتِلَفُ أَلْسِنَتِكُمْ وَأَلْوَنِكُمْ إِنَّ فِى ذَلِكَ لَءَايَتٍ لِّلْعَلِمِينَ
{\tiny\colorbox{cl_aya}{23}} وَمِنْ ءَايَتِهِ مَنَامُكُم بِالَّيْلِ وَالنَّهَارِ وَابْتِغَاؤُكُم مِّن فَضْلِهِ إِنَّ فِى ذَلِكَ لَءَايَتٍ لِّقَوْمٍ يَسْمَعُونَ
{\tiny\colorbox{cl_aya}{24}} وَمِنْ ءَايَتِهِ يُرِيكُمُ الْبَرْقَ خَوْفًا وَطَمَعًا وَيُنَزِّلُ مِنَ السَّمَاءِ مَاءً فَيُحْىِ بِهِ الْأَرْضَ بَعْدَ مَوْتِهَا إِنَّ فِى ذَلِكَ لَءَايَتٍ لِّقَوْمٍ يَعْقِلُونَ
{\tiny\colorbox{cl_aya}{25}} وَمِنْ ءَايَتِهِ أَن تَقُومَ السَّمَاءُ وَالْأَرْضُ بِأَمْرِهِ ثُمَّ إِذَا دَعَاكُمْ دَعْوَةً مِّنَ الْأَرْضِ إِذَا أَنتُمْ تَخْرُجُونَ
{\tiny\colorbox{cl_aya}{26}} وَلَهُ مَن فِى السَّمَوَتِ وَالْأَرْضِ كُلٌّ لَّهُ قَنِتُونَ
{\tiny\colorbox{cl_aya}{27}} وَهُوَ الَّذِى يَبْدَؤُا الْخَلْقَ ثُمَّ يُعِيدُهُ وَهُوَ أَهْوَنُ عَلَيْهِ وَلَهُ الْمَثَلُ الْأَعْلَى فِى السَّمَوَتِ وَالْأَرْضِ وَهُوَ الْعَزِيزُ الْحَكِيمُ
{\tiny\colorbox{cl_aya}{28}} ضَرَبَ لَكُم مَّثَلًا مِّنْ أَنفُسِكُمْ هَل لَّكُم مِّن مَّا مَلَكَتْ أَيْمَنُكُم مِّن شُرَكَاءَ فِى مَا رَزَقْنَكُمْ فَأَنتُمْ فِيهِ سَوَاءٌ تَخَافُونَهُمْ كَخِيفَتِكُمْ أَنفُسَكُمْ كَذَلِكَ نُفَصِّلُ الْءَايَتِ لِقَوْمٍ يَعْقِلُونَ
{\tiny\colorbox{cl_aya}{29}} بَلِ اتَّبَعَ الَّذِينَ ظَلَمُوا أَهْوَاءَهُم بِغَيْرِ عِلْمٍ فَمَن يَهْدِى مَنْ أَضَلَّ اللَّهُ وَمَا لَهُم مِّن نَّصِرِينَ
{\tiny\colorbox{cl_aya}{30}} فَأَقِمْ وَجْهَكَ لِلدِّينِ حَنِيفًا فِطْرَتَ اللَّهِ الَّتِى فَطَرَ النَّاسَ عَلَيْهَا لَا تَبْدِيلَ لِخَلْقِ اللَّهِ ذَلِكَ الدِّينُ الْقَيِّمُ وَلَكِنَّ أَكْثَرَ النَّاسِ لَا يَعْلَمُونَ
{\tiny\colorbox{cl_aya}{31}} مُنِيبِينَ إِلَيْهِ وَاتَّقُوهُ وَأَقِيمُوا الصَّلَوةَ وَلَا تَكُونُوا مِنَ الْمُشْرِكِينَ
{\tiny\colorbox{cl_aya}{32}} مِنَ الَّذِينَ فَرَّقُوا دِينَهُمْ وَكَانُوا شِيَعًا كُلُّ حِزْبٍ بِمَا لَدَيْهِمْ فَرِحُونَ
{\tiny\colorbox{cl_aya}{33}} وَإِذَا مَسَّ النَّاسَ ضُرٌّ دَعَوْا رَبَّهُم مُّنِيبِينَ إِلَيْهِ ثُمَّ إِذَا أَذَاقَهُم مِّنْهُ رَحْمَةً إِذَا فَرِيقٌ مِّنْهُم بِرَبِّهِمْ يُشْرِكُونَ
{\tiny\colorbox{cl_aya}{34}} لِيَكْفُرُوا بِمَا ءَاتَيْنَهُمْ فَتَمَتَّعُوا فَسَوْفَ تَعْلَمُونَ
{\tiny\colorbox{cl_aya}{35}} أَمْ أَنزَلْنَا عَلَيْهِمْ سُلْطَنًا فَهُوَ يَتَكَلَّمُ بِمَا كَانُوا بِهِ يُشْرِكُونَ
{\tiny\colorbox{cl_aya}{36}} وَإِذَا أَذَقْنَا النَّاسَ رَحْمَةً فَرِحُوا بِهَا وَإِن تُصِبْهُمْ سَيِّئَةٌ بِمَا قَدَّمَتْ أَيْدِيهِمْ إِذَا هُمْ يَقْنَطُونَ
{\tiny\colorbox{cl_aya}{37}} أَوَلَمْ يَرَوْا أَنَّ اللَّهَ يَبْسُطُ الرِّزْقَ لِمَن يَشَاءُ وَيَقْدِرُ إِنَّ فِى ذَلِكَ لَءَايَتٍ لِّقَوْمٍ يُؤْمِنُونَ
{\tiny\colorbox{cl_aya}{38}} فََٔاتِ ذَا الْقُرْبَى حَقَّهُ وَالْمِسْكِينَ وَابْنَ السَّبِيلِ ذَلِكَ خَيْرٌ لِّلَّذِينَ يُرِيدُونَ وَجْهَ اللَّهِ وَأُولَئِكَ هُمُ الْمُفْلِحُونَ
{\tiny\colorbox{cl_aya}{39}} وَمَا ءَاتَيْتُم مِّن رِّبًا لِّيَرْبُوَا فِى أَمْوَلِ النَّاسِ فَلَا يَرْبُوا عِندَ اللَّهِ وَمَا ءَاتَيْتُم مِّن زَكَوةٍ تُرِيدُونَ وَجْهَ اللَّهِ فَأُولَئِكَ هُمُ الْمُضْعِفُونَ
{\tiny\colorbox{cl_aya}{40}} اللَّهُ الَّذِى خَلَقَكُمْ ثُمَّ رَزَقَكُمْ ثُمَّ يُمِيتُكُمْ ثُمَّ يُحْيِيكُمْ هَلْ مِن شُرَكَائِكُم مَّن يَفْعَلُ مِن ذَلِكُم مِّن شَىْءٍ سُبْحَنَهُ وَتَعَلَى عَمَّا يُشْرِكُونَ
{\tiny\colorbox{cl_aya}{41}} ظَهَرَ الْفَسَادُ فِى الْبَرِّ وَالْبَحْرِ بِمَا كَسَبَتْ أَيْدِى النَّاسِ لِيُذِيقَهُم بَعْضَ الَّذِى عَمِلُوا لَعَلَّهُمْ يَرْجِعُونَ
{\tiny\colorbox{cl_aya}{42}} قُلْ سِيرُوا فِى الْأَرْضِ فَانظُرُوا كَيْفَ كَانَ عَقِبَةُ الَّذِينَ مِن قَبْلُ كَانَ أَكْثَرُهُم مُّشْرِكِينَ
{\tiny\colorbox{cl_aya}{43}} فَأَقِمْ وَجْهَكَ لِلدِّينِ الْقَيِّمِ مِن قَبْلِ أَن يَأْتِىَ يَوْمٌ لَّا مَرَدَّ لَهُ مِنَ اللَّهِ يَوْمَئِذٍ يَصَّدَّعُونَ
{\tiny\colorbox{cl_aya}{44}} مَن كَفَرَ فَعَلَيْهِ كُفْرُهُ وَمَنْ عَمِلَ صَلِحًا فَلِأَنفُسِهِمْ يَمْهَدُونَ
{\tiny\colorbox{cl_aya}{45}} لِيَجْزِىَ الَّذِينَ ءَامَنُوا وَعَمِلُوا الصَّلِحَتِ مِن فَضْلِهِ إِنَّهُ لَا يُحِبُّ الْكَفِرِينَ
{\tiny\colorbox{cl_aya}{46}} وَمِنْ ءَايَتِهِ أَن يُرْسِلَ الرِّيَاحَ مُبَشِّرَتٍ وَلِيُذِيقَكُم مِّن رَّحْمَتِهِ وَلِتَجْرِىَ الْفُلْكُ بِأَمْرِهِ وَلِتَبْتَغُوا مِن فَضْلِهِ وَلَعَلَّكُمْ تَشْكُرُونَ
{\tiny\colorbox{cl_aya}{47}} وَلَقَدْ أَرْسَلْنَا مِن قَبْلِكَ رُسُلًا إِلَى قَوْمِهِمْ فَجَاءُوهُم بِالْبَيِّنَتِ فَانتَقَمْنَا مِنَ الَّذِينَ أَجْرَمُوا وَكَانَ حَقًّا عَلَيْنَا نَصْرُ الْمُؤْمِنِينَ
{\tiny\colorbox{cl_aya}{48}} اللَّهُ الَّذِى يُرْسِلُ الرِّيَحَ فَتُثِيرُ سَحَابًا فَيَبْسُطُهُ فِى السَّمَاءِ كَيْفَ يَشَاءُ وَيَجْعَلُهُ كِسَفًا فَتَرَى الْوَدْقَ يَخْرُجُ مِنْ خِلَلِهِ فَإِذَا أَصَابَ بِهِ مَن يَشَاءُ مِنْ عِبَادِهِ إِذَا هُمْ يَسْتَبْشِرُونَ
{\tiny\colorbox{cl_aya}{49}} وَإِن كَانُوا مِن قَبْلِ أَن يُنَزَّلَ عَلَيْهِم مِّن قَبْلِهِ لَمُبْلِسِينَ
{\tiny\colorbox{cl_aya}{50}} فَانظُرْ إِلَى ءَاثَرِ رَحْمَتِ اللَّهِ كَيْفَ يُحْىِ الْأَرْضَ بَعْدَ مَوْتِهَا إِنَّ ذَلِكَ لَمُحْىِ الْمَوْتَى وَهُوَ عَلَى كُلِّ شَىْءٍ قَدِيرٌ
{\tiny\colorbox{cl_aya}{51}} وَلَئِنْ أَرْسَلْنَا رِيحًا فَرَأَوْهُ مُصْفَرًّا لَّظَلُّوا مِن بَعْدِهِ يَكْفُرُونَ
{\tiny\colorbox{cl_aya}{52}} فَإِنَّكَ لَا تُسْمِعُ الْمَوْتَى وَلَا تُسْمِعُ الصُّمَّ الدُّعَاءَ إِذَا وَلَّوْا مُدْبِرِينَ
{\tiny\colorbox{cl_aya}{53}} وَمَا أَنتَ بِهَدِ الْعُمْىِ عَن ضَلَلَتِهِمْ إِن تُسْمِعُ إِلَّا مَن يُؤْمِنُ بَِٔايَتِنَا فَهُم مُّسْلِمُونَ
{\tiny\colorbox{cl_aya}{54}} اللَّهُ الَّذِى خَلَقَكُم مِّن ضَعْفٍ ثُمَّ جَعَلَ مِن بَعْدِ ضَعْفٍ قُوَّةً ثُمَّ جَعَلَ مِن بَعْدِ قُوَّةٍ ضَعْفًا وَشَيْبَةً يَخْلُقُ مَا يَشَاءُ وَهُوَ الْعَلِيمُ الْقَدِيرُ
{\tiny\colorbox{cl_aya}{55}} وَيَوْمَ تَقُومُ السَّاعَةُ يُقْسِمُ الْمُجْرِمُونَ مَا لَبِثُوا غَيْرَ سَاعَةٍ كَذَلِكَ كَانُوا يُؤْفَكُونَ
{\tiny\colorbox{cl_aya}{56}} وَقَالَ الَّذِينَ أُوتُوا الْعِلْمَ وَالْإِيمَنَ لَقَدْ لَبِثْتُمْ فِى كِتَبِ اللَّهِ إِلَى يَوْمِ الْبَعْثِ فَهَذَا يَوْمُ الْبَعْثِ وَلَكِنَّكُمْ كُنتُمْ لَا تَعْلَمُونَ
{\tiny\colorbox{cl_aya}{57}} فَيَوْمَئِذٍ لَّا يَنفَعُ الَّذِينَ ظَلَمُوا مَعْذِرَتُهُمْ وَلَا هُمْ يُسْتَعْتَبُونَ
{\tiny\colorbox{cl_aya}{58}} وَلَقَدْ ضَرَبْنَا لِلنَّاسِ فِى هَذَا الْقُرْءَانِ مِن كُلِّ مَثَلٍ وَلَئِن جِئْتَهُم بَِٔايَةٍ لَّيَقُولَنَّ الَّذِينَ كَفَرُوا إِنْ أَنتُمْ إِلَّا مُبْطِلُونَ
{\tiny\colorbox{cl_aya}{59}} كَذَلِكَ يَطْبَعُ اللَّهُ عَلَى قُلُوبِ الَّذِينَ لَا يَعْلَمُونَ
{\tiny\colorbox{cl_aya}{60}} فَاصْبِرْ إِنَّ وَعْدَ اللَّهِ حَقٌّ وَلَا يَسْتَخِفَّنَّكَ الَّذِينَ لَا يُوقِنُونَ
\end{document}