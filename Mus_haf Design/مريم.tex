%\documentclass[12pt,a4paper]{article}
\documentclass[20pt,a4paper]{article}
\usepackage[margin=0.5in]{geometry}
\usepackage{polyglossia}
\usepackage[dvipsnames]{xcolor}
\pagenumbering{gobble}
% This beautiful one line disable the initial spacing at the beginning of a line
\usepackage[parfill]{parskip} 
\usepackage{setspace}
\setstretch{2}

\setdefaultlanguage[numerals=maghrib]{arabic}
\newfontfamily\arabicfont[Script=Arabic]{Amiri}

\title{}
\author{}
\date{}
\definecolor{cl_page}{gray}{0.98}
\definecolor{cl_aya}{HTML}{DEEEFF}

\begin{document}
\pagecolor{cl_page}

% Start %


{\tiny\colorbox{cl_aya}{1}} كهيعص
{\tiny\colorbox{cl_aya}{2}} ذِكْرُ رَحْمَتِ رَبِّكَ عَبْدَهُ زَكَرِيَّا
{\tiny\colorbox{cl_aya}{3}} إِذْ نَادَى رَبَّهُ نِدَاءً خَفِيًّا
{\tiny\colorbox{cl_aya}{4}} قَالَ رَبِّ إِنِّى وَهَنَ الْعَظْمُ مِنِّى وَاشْتَعَلَ الرَّأْسُ شَيْبًا وَلَمْ أَكُن بِدُعَائِكَ رَبِّ شَقِيًّا
{\tiny\colorbox{cl_aya}{5}} وَإِنِّى خِفْتُ الْمَوَلِىَ مِن وَرَاءِى وَكَانَتِ امْرَأَتِى عَاقِرًا فَهَبْ لِى مِن لَّدُنكَ وَلِيًّا
{\tiny\colorbox{cl_aya}{6}} يَرِثُنِى وَيَرِثُ مِنْ ءَالِ يَعْقُوبَ وَاجْعَلْهُ رَبِّ رَضِيًّا
{\tiny\colorbox{cl_aya}{7}} يَزَكَرِيَّا إِنَّا نُبَشِّرُكَ بِغُلَمٍ اسْمُهُ يَحْيَى لَمْ نَجْعَل لَّهُ مِن قَبْلُ سَمِيًّا
{\tiny\colorbox{cl_aya}{8}} قَالَ رَبِّ أَنَّى يَكُونُ لِى غُلَمٌ وَكَانَتِ امْرَأَتِى عَاقِرًا وَقَدْ بَلَغْتُ مِنَ الْكِبَرِ عِتِيًّا
{\tiny\colorbox{cl_aya}{9}} قَالَ كَذَلِكَ قَالَ رَبُّكَ هُوَ عَلَىَّ هَيِّنٌ وَقَدْ خَلَقْتُكَ مِن قَبْلُ وَلَمْ تَكُ شَئًْا
{\tiny\colorbox{cl_aya}{10}} قَالَ رَبِّ اجْعَل لِّى ءَايَةً قَالَ ءَايَتُكَ أَلَّا تُكَلِّمَ النَّاسَ ثَلَثَ لَيَالٍ سَوِيًّا
{\tiny\colorbox{cl_aya}{11}} فَخَرَجَ عَلَى قَوْمِهِ مِنَ الْمِحْرَابِ فَأَوْحَى إِلَيْهِمْ أَن سَبِّحُوا بُكْرَةً وَعَشِيًّا
{\tiny\colorbox{cl_aya}{12}} يَيَحْيَى خُذِ الْكِتَبَ بِقُوَّةٍ وَءَاتَيْنَهُ الْحُكْمَ صَبِيًّا
{\tiny\colorbox{cl_aya}{13}} وَحَنَانًا مِّن لَّدُنَّا وَزَكَوةً وَكَانَ تَقِيًّا
{\tiny\colorbox{cl_aya}{14}} وَبَرًّا بِوَلِدَيْهِ وَلَمْ يَكُن جَبَّارًا عَصِيًّا
{\tiny\colorbox{cl_aya}{15}} وَسَلَمٌ عَلَيْهِ يَوْمَ وُلِدَ وَيَوْمَ يَمُوتُ وَيَوْمَ يُبْعَثُ حَيًّا
{\tiny\colorbox{cl_aya}{16}} وَاذْكُرْ فِى الْكِتَبِ مَرْيَمَ إِذِ انتَبَذَتْ مِنْ أَهْلِهَا مَكَانًا شَرْقِيًّا
{\tiny\colorbox{cl_aya}{17}} فَاتَّخَذَتْ مِن دُونِهِمْ حِجَابًا فَأَرْسَلْنَا إِلَيْهَا رُوحَنَا فَتَمَثَّلَ لَهَا بَشَرًا سَوِيًّا
{\tiny\colorbox{cl_aya}{18}} قَالَتْ إِنِّى أَعُوذُ بِالرَّحْمَنِ مِنكَ إِن كُنتَ تَقِيًّا
{\tiny\colorbox{cl_aya}{19}} قَالَ إِنَّمَا أَنَا رَسُولُ رَبِّكِ لِأَهَبَ لَكِ غُلَمًا زَكِيًّا
{\tiny\colorbox{cl_aya}{20}} قَالَتْ أَنَّى يَكُونُ لِى غُلَمٌ وَلَمْ يَمْسَسْنِى بَشَرٌ وَلَمْ أَكُ بَغِيًّا
{\tiny\colorbox{cl_aya}{21}} قَالَ كَذَلِكِ قَالَ رَبُّكِ هُوَ عَلَىَّ هَيِّنٌ وَلِنَجْعَلَهُ ءَايَةً لِّلنَّاسِ وَرَحْمَةً مِّنَّا وَكَانَ أَمْرًا مَّقْضِيًّا
{\tiny\colorbox{cl_aya}{22}} فَحَمَلَتْهُ فَانتَبَذَتْ بِهِ مَكَانًا قَصِيًّا
{\tiny\colorbox{cl_aya}{23}} فَأَجَاءَهَا الْمَخَاضُ إِلَى جِذْعِ النَّخْلَةِ قَالَتْ يَلَيْتَنِى مِتُّ قَبْلَ هَذَا وَكُنتُ نَسْيًا مَّنسِيًّا
{\tiny\colorbox{cl_aya}{24}} فَنَادَىهَا مِن تَحْتِهَا أَلَّا تَحْزَنِى قَدْ جَعَلَ رَبُّكِ تَحْتَكِ سَرِيًّا
{\tiny\colorbox{cl_aya}{25}} وَهُزِّى إِلَيْكِ بِجِذْعِ النَّخْلَةِ تُسَقِطْ عَلَيْكِ رُطَبًا جَنِيًّا
{\tiny\colorbox{cl_aya}{26}} فَكُلِى وَاشْرَبِى وَقَرِّى عَيْنًا فَإِمَّا تَرَيِنَّ مِنَ الْبَشَرِ أَحَدًا فَقُولِى إِنِّى نَذَرْتُ لِلرَّحْمَنِ صَوْمًا فَلَنْ أُكَلِّمَ الْيَوْمَ إِنسِيًّا
{\tiny\colorbox{cl_aya}{27}} فَأَتَتْ بِهِ قَوْمَهَا تَحْمِلُهُ قَالُوا يَمَرْيَمُ لَقَدْ جِئْتِ شَئًْا فَرِيًّا
{\tiny\colorbox{cl_aya}{28}} يَأُخْتَ هَرُونَ مَا كَانَ أَبُوكِ امْرَأَ سَوْءٍ وَمَا كَانَتْ أُمُّكِ بَغِيًّا
{\tiny\colorbox{cl_aya}{29}} فَأَشَارَتْ إِلَيْهِ قَالُوا كَيْفَ نُكَلِّمُ مَن كَانَ فِى الْمَهْدِ صَبِيًّا
{\tiny\colorbox{cl_aya}{30}} قَالَ إِنِّى عَبْدُ اللَّهِ ءَاتَىنِىَ الْكِتَبَ وَجَعَلَنِى نَبِيًّا
{\tiny\colorbox{cl_aya}{31}} وَجَعَلَنِى مُبَارَكًا أَيْنَ مَا كُنتُ وَأَوْصَنِى بِالصَّلَوةِ وَالزَّكَوةِ مَا دُمْتُ حَيًّا
{\tiny\colorbox{cl_aya}{32}} وَبَرًّا بِوَلِدَتِى وَلَمْ يَجْعَلْنِى جَبَّارًا شَقِيًّا
{\tiny\colorbox{cl_aya}{33}} وَالسَّلَمُ عَلَىَّ يَوْمَ وُلِدتُّ وَيَوْمَ أَمُوتُ وَيَوْمَ أُبْعَثُ حَيًّا
{\tiny\colorbox{cl_aya}{34}} ذَلِكَ عِيسَى ابْنُ مَرْيَمَ قَوْلَ الْحَقِّ الَّذِى فِيهِ يَمْتَرُونَ
{\tiny\colorbox{cl_aya}{35}} مَا كَانَ لِلَّهِ أَن يَتَّخِذَ مِن وَلَدٍ سُبْحَنَهُ إِذَا قَضَى أَمْرًا فَإِنَّمَا يَقُولُ لَهُ كُن فَيَكُونُ
{\tiny\colorbox{cl_aya}{36}} وَإِنَّ اللَّهَ رَبِّى وَرَبُّكُمْ فَاعْبُدُوهُ هَذَا صِرَطٌ مُّسْتَقِيمٌ
{\tiny\colorbox{cl_aya}{37}} فَاخْتَلَفَ الْأَحْزَابُ مِن بَيْنِهِمْ فَوَيْلٌ لِّلَّذِينَ كَفَرُوا مِن مَّشْهَدِ يَوْمٍ عَظِيمٍ
{\tiny\colorbox{cl_aya}{38}} أَسْمِعْ بِهِمْ وَأَبْصِرْ يَوْمَ يَأْتُونَنَا لَكِنِ الظَّلِمُونَ الْيَوْمَ فِى ضَلَلٍ مُّبِينٍ
{\tiny\colorbox{cl_aya}{39}} وَأَنذِرْهُمْ يَوْمَ الْحَسْرَةِ إِذْ قُضِىَ الْأَمْرُ وَهُمْ فِى غَفْلَةٍ وَهُمْ لَا يُؤْمِنُونَ
{\tiny\colorbox{cl_aya}{40}} إِنَّا نَحْنُ نَرِثُ الْأَرْضَ وَمَنْ عَلَيْهَا وَإِلَيْنَا يُرْجَعُونَ
{\tiny\colorbox{cl_aya}{41}} وَاذْكُرْ فِى الْكِتَبِ إِبْرَهِيمَ إِنَّهُ كَانَ صِدِّيقًا نَّبِيًّا
{\tiny\colorbox{cl_aya}{42}} إِذْ قَالَ لِأَبِيهِ يَأَبَتِ لِمَ تَعْبُدُ مَا لَا يَسْمَعُ وَلَا يُبْصِرُ وَلَا يُغْنِى عَنكَ شَئًْا
{\tiny\colorbox{cl_aya}{43}} يَأَبَتِ إِنِّى قَدْ جَاءَنِى مِنَ الْعِلْمِ مَا لَمْ يَأْتِكَ فَاتَّبِعْنِى أَهْدِكَ صِرَطًا سَوِيًّا
{\tiny\colorbox{cl_aya}{44}} يَأَبَتِ لَا تَعْبُدِ الشَّيْطَنَ إِنَّ الشَّيْطَنَ كَانَ لِلرَّحْمَنِ عَصِيًّا
{\tiny\colorbox{cl_aya}{45}} يَأَبَتِ إِنِّى أَخَافُ أَن يَمَسَّكَ عَذَابٌ مِّنَ الرَّحْمَنِ فَتَكُونَ لِلشَّيْطَنِ وَلِيًّا
{\tiny\colorbox{cl_aya}{46}} قَالَ أَرَاغِبٌ أَنتَ عَنْ ءَالِهَتِى يَإِبْرَهِيمُ لَئِن لَّمْ تَنتَهِ لَأَرْجُمَنَّكَ وَاهْجُرْنِى مَلِيًّا
{\tiny\colorbox{cl_aya}{47}} قَالَ سَلَمٌ عَلَيْكَ سَأَسْتَغْفِرُ لَكَ رَبِّى إِنَّهُ كَانَ بِى حَفِيًّا
{\tiny\colorbox{cl_aya}{48}} وَأَعْتَزِلُكُمْ وَمَا تَدْعُونَ مِن دُونِ اللَّهِ وَأَدْعُوا رَبِّى عَسَى أَلَّا أَكُونَ بِدُعَاءِ رَبِّى شَقِيًّا
{\tiny\colorbox{cl_aya}{49}} فَلَمَّا اعْتَزَلَهُمْ وَمَا يَعْبُدُونَ مِن دُونِ اللَّهِ وَهَبْنَا لَهُ إِسْحَقَ وَيَعْقُوبَ وَكُلًّا جَعَلْنَا نَبِيًّا
{\tiny\colorbox{cl_aya}{50}} وَوَهَبْنَا لَهُم مِّن رَّحْمَتِنَا وَجَعَلْنَا لَهُمْ لِسَانَ صِدْقٍ عَلِيًّا
{\tiny\colorbox{cl_aya}{51}} وَاذْكُرْ فِى الْكِتَبِ مُوسَى إِنَّهُ كَانَ مُخْلَصًا وَكَانَ رَسُولًا نَّبِيًّا
{\tiny\colorbox{cl_aya}{52}} وَنَدَيْنَهُ مِن جَانِبِ الطُّورِ الْأَيْمَنِ وَقَرَّبْنَهُ نَجِيًّا
{\tiny\colorbox{cl_aya}{53}} وَوَهَبْنَا لَهُ مِن رَّحْمَتِنَا أَخَاهُ هَرُونَ نَبِيًّا
{\tiny\colorbox{cl_aya}{54}} وَاذْكُرْ فِى الْكِتَبِ إِسْمَعِيلَ إِنَّهُ كَانَ صَادِقَ الْوَعْدِ وَكَانَ رَسُولًا نَّبِيًّا
{\tiny\colorbox{cl_aya}{55}} وَكَانَ يَأْمُرُ أَهْلَهُ بِالصَّلَوةِ وَالزَّكَوةِ وَكَانَ عِندَ رَبِّهِ مَرْضِيًّا
{\tiny\colorbox{cl_aya}{56}} وَاذْكُرْ فِى الْكِتَبِ إِدْرِيسَ إِنَّهُ كَانَ صِدِّيقًا نَّبِيًّا
{\tiny\colorbox{cl_aya}{57}} وَرَفَعْنَهُ مَكَانًا عَلِيًّا
{\tiny\colorbox{cl_aya}{58}} أُولَئِكَ الَّذِينَ أَنْعَمَ اللَّهُ عَلَيْهِم مِّنَ النَّبِيِّنَ مِن ذُرِّيَّةِ ءَادَمَ وَمِمَّنْ حَمَلْنَا مَعَ نُوحٍ وَمِن ذُرِّيَّةِ إِبْرَهِيمَ وَإِسْرَءِيلَ وَمِمَّنْ هَدَيْنَا وَاجْتَبَيْنَا إِذَا تُتْلَى عَلَيْهِمْ ءَايَتُ الرَّحْمَنِ خَرُّوا سُجَّدًا وَبُكِيًّا
{\tiny\colorbox{cl_aya}{59}} فَخَلَفَ مِن بَعْدِهِمْ خَلْفٌ أَضَاعُوا الصَّلَوةَ وَاتَّبَعُوا الشَّهَوَتِ فَسَوْفَ يَلْقَوْنَ غَيًّا
{\tiny\colorbox{cl_aya}{60}} إِلَّا مَن تَابَ وَءَامَنَ وَعَمِلَ صَلِحًا فَأُولَئِكَ يَدْخُلُونَ الْجَنَّةَ وَلَا يُظْلَمُونَ شَئًْا
{\tiny\colorbox{cl_aya}{61}} جَنَّتِ عَدْنٍ الَّتِى وَعَدَ الرَّحْمَنُ عِبَادَهُ بِالْغَيْبِ إِنَّهُ كَانَ وَعْدُهُ مَأْتِيًّا
{\tiny\colorbox{cl_aya}{62}} لَّا يَسْمَعُونَ فِيهَا لَغْوًا إِلَّا سَلَمًا وَلَهُمْ رِزْقُهُمْ فِيهَا بُكْرَةً وَعَشِيًّا
{\tiny\colorbox{cl_aya}{63}} تِلْكَ الْجَنَّةُ الَّتِى نُورِثُ مِنْ عِبَادِنَا مَن كَانَ تَقِيًّا
{\tiny\colorbox{cl_aya}{64}} وَمَا نَتَنَزَّلُ إِلَّا بِأَمْرِ رَبِّكَ لَهُ مَا بَيْنَ أَيْدِينَا وَمَا خَلْفَنَا وَمَا بَيْنَ ذَلِكَ وَمَا كَانَ رَبُّكَ نَسِيًّا
{\tiny\colorbox{cl_aya}{65}} رَّبُّ السَّمَوَتِ وَالْأَرْضِ وَمَا بَيْنَهُمَا فَاعْبُدْهُ وَاصْطَبِرْ لِعِبَدَتِهِ هَلْ تَعْلَمُ لَهُ سَمِيًّا
{\tiny\colorbox{cl_aya}{66}} وَيَقُولُ الْإِنسَنُ أَءِذَا مَا مِتُّ لَسَوْفَ أُخْرَجُ حَيًّا
{\tiny\colorbox{cl_aya}{67}} أَوَلَا يَذْكُرُ الْإِنسَنُ أَنَّا خَلَقْنَهُ مِن قَبْلُ وَلَمْ يَكُ شَئًْا
{\tiny\colorbox{cl_aya}{68}} فَوَرَبِّكَ لَنَحْشُرَنَّهُمْ وَالشَّيَطِينَ ثُمَّ لَنُحْضِرَنَّهُمْ حَوْلَ جَهَنَّمَ جِثِيًّا
{\tiny\colorbox{cl_aya}{69}} ثُمَّ لَنَنزِعَنَّ مِن كُلِّ شِيعَةٍ أَيُّهُمْ أَشَدُّ عَلَى الرَّحْمَنِ عِتِيًّا
{\tiny\colorbox{cl_aya}{70}} ثُمَّ لَنَحْنُ أَعْلَمُ بِالَّذِينَ هُمْ أَوْلَى بِهَا صِلِيًّا
{\tiny\colorbox{cl_aya}{71}} وَإِن مِّنكُمْ إِلَّا وَارِدُهَا كَانَ عَلَى رَبِّكَ حَتْمًا مَّقْضِيًّا
{\tiny\colorbox{cl_aya}{72}} ثُمَّ نُنَجِّى الَّذِينَ اتَّقَوا وَّنَذَرُ الظَّلِمِينَ فِيهَا جِثِيًّا
{\tiny\colorbox{cl_aya}{73}} وَإِذَا تُتْلَى عَلَيْهِمْ ءَايَتُنَا بَيِّنَتٍ قَالَ الَّذِينَ كَفَرُوا لِلَّذِينَ ءَامَنُوا أَىُّ الْفَرِيقَيْنِ خَيْرٌ مَّقَامًا وَأَحْسَنُ نَدِيًّا
{\tiny\colorbox{cl_aya}{74}} وَكَمْ أَهْلَكْنَا قَبْلَهُم مِّن قَرْنٍ هُمْ أَحْسَنُ أَثَثًا وَرِءْيًا
{\tiny\colorbox{cl_aya}{75}} قُلْ مَن كَانَ فِى الضَّلَلَةِ فَلْيَمْدُدْ لَهُ الرَّحْمَنُ مَدًّا حَتَّى إِذَا رَأَوْا مَا يُوعَدُونَ إِمَّا الْعَذَابَ وَإِمَّا السَّاعَةَ فَسَيَعْلَمُونَ مَنْ هُوَ شَرٌّ مَّكَانًا وَأَضْعَفُ جُندًا
{\tiny\colorbox{cl_aya}{76}} وَيَزِيدُ اللَّهُ الَّذِينَ اهْتَدَوْا هُدًى وَالْبَقِيَتُ الصَّلِحَتُ خَيْرٌ عِندَ رَبِّكَ ثَوَابًا وَخَيْرٌ مَّرَدًّا
{\tiny\colorbox{cl_aya}{77}} أَفَرَءَيْتَ الَّذِى كَفَرَ بَِٔايَتِنَا وَقَالَ لَأُوتَيَنَّ مَالًا وَوَلَدًا
{\tiny\colorbox{cl_aya}{78}} أَطَّلَعَ الْغَيْبَ أَمِ اتَّخَذَ عِندَ الرَّحْمَنِ عَهْدًا
{\tiny\colorbox{cl_aya}{79}} كَلَّا سَنَكْتُبُ مَا يَقُولُ وَنَمُدُّ لَهُ مِنَ الْعَذَابِ مَدًّا
{\tiny\colorbox{cl_aya}{80}} وَنَرِثُهُ مَا يَقُولُ وَيَأْتِينَا فَرْدًا
{\tiny\colorbox{cl_aya}{81}} وَاتَّخَذُوا مِن دُونِ اللَّهِ ءَالِهَةً لِّيَكُونُوا لَهُمْ عِزًّا
{\tiny\colorbox{cl_aya}{82}} كَلَّا سَيَكْفُرُونَ بِعِبَادَتِهِمْ وَيَكُونُونَ عَلَيْهِمْ ضِدًّا
{\tiny\colorbox{cl_aya}{83}} أَلَمْ تَرَ أَنَّا أَرْسَلْنَا الشَّيَطِينَ عَلَى الْكَفِرِينَ تَؤُزُّهُمْ أَزًّا
{\tiny\colorbox{cl_aya}{84}} فَلَا تَعْجَلْ عَلَيْهِمْ إِنَّمَا نَعُدُّ لَهُمْ عَدًّا
{\tiny\colorbox{cl_aya}{85}} يَوْمَ نَحْشُرُ الْمُتَّقِينَ إِلَى الرَّحْمَنِ وَفْدًا
{\tiny\colorbox{cl_aya}{86}} وَنَسُوقُ الْمُجْرِمِينَ إِلَى جَهَنَّمَ وِرْدًا
{\tiny\colorbox{cl_aya}{87}} لَّا يَمْلِكُونَ الشَّفَعَةَ إِلَّا مَنِ اتَّخَذَ عِندَ الرَّحْمَنِ عَهْدًا
{\tiny\colorbox{cl_aya}{88}} وَقَالُوا اتَّخَذَ الرَّحْمَنُ وَلَدًا
{\tiny\colorbox{cl_aya}{89}} لَّقَدْ جِئْتُمْ شَئًْا إِدًّا
{\tiny\colorbox{cl_aya}{90}} تَكَادُ السَّمَوَتُ يَتَفَطَّرْنَ مِنْهُ وَتَنشَقُّ الْأَرْضُ وَتَخِرُّ الْجِبَالُ هَدًّا
{\tiny\colorbox{cl_aya}{91}} أَن دَعَوْا لِلرَّحْمَنِ وَلَدًا
{\tiny\colorbox{cl_aya}{92}} وَمَا يَنبَغِى لِلرَّحْمَنِ أَن يَتَّخِذَ وَلَدًا
{\tiny\colorbox{cl_aya}{93}} إِن كُلُّ مَن فِى السَّمَوَتِ وَالْأَرْضِ إِلَّا ءَاتِى الرَّحْمَنِ عَبْدًا
{\tiny\colorbox{cl_aya}{94}} لَّقَدْ أَحْصَىهُمْ وَعَدَّهُمْ عَدًّا
{\tiny\colorbox{cl_aya}{95}} وَكُلُّهُمْ ءَاتِيهِ يَوْمَ الْقِيَمَةِ فَرْدًا
{\tiny\colorbox{cl_aya}{96}} إِنَّ الَّذِينَ ءَامَنُوا وَعَمِلُوا الصَّلِحَتِ سَيَجْعَلُ لَهُمُ الرَّحْمَنُ وُدًّا
{\tiny\colorbox{cl_aya}{97}} فَإِنَّمَا يَسَّرْنَهُ بِلِسَانِكَ لِتُبَشِّرَ بِهِ الْمُتَّقِينَ وَتُنذِرَ بِهِ قَوْمًا لُّدًّا
{\tiny\colorbox{cl_aya}{98}} وَكَمْ أَهْلَكْنَا قَبْلَهُم مِّن قَرْنٍ هَلْ تُحِسُّ مِنْهُم مِّنْ أَحَدٍ أَوْ تَسْمَعُ لَهُمْ رِكْزًا
\end{document}