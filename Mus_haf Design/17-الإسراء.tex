%\documentclass[12pt,a4paper]{article}
\documentclass[20pt,a4paper]{article}
\usepackage[margin=0.5in]{geometry}
\usepackage{polyglossia}
\usepackage[dvipsnames]{xcolor}
\pagenumbering{gobble}
% This beautiful one line disable the initial spacing at the beginning of a line
\usepackage[parfill]{parskip} 
\usepackage{setspace}
\setstretch{2}

\setdefaultlanguage[numerals=maghrib]{arabic}
\newfontfamily\arabicfont[Script=Arabic]{Amiri}

\title{}
\author{}
\date{}
\definecolor{cl_page}{gray}{0.98}
\definecolor{cl_aya}{HTML}{DEEEFF}

\begin{document}
\pagecolor{cl_page}

% Start %


{\tiny\colorbox{cl_aya}{1}} سُبْحَنَ الَّذِى أَسْرَى بِعَبْدِهِ لَيْلًا مِّنَ الْمَسْجِدِ الْحَرَامِ إِلَى الْمَسْجِدِ الْأَقْصَا الَّذِى بَرَكْنَا حَوْلَهُ لِنُرِيَهُ مِنْ ءَايَتِنَا إِنَّهُ هُوَ السَّمِيعُ الْبَصِيرُ
{\tiny\colorbox{cl_aya}{2}} وَءَاتَيْنَا مُوسَى الْكِتَبَ وَجَعَلْنَهُ هُدًى لِّبَنِى إِسْرَءِيلَ أَلَّا تَتَّخِذُوا مِن دُونِى وَكِيلًا
{\tiny\colorbox{cl_aya}{3}} ذُرِّيَّةَ مَنْ حَمَلْنَا مَعَ نُوحٍ إِنَّهُ كَانَ عَبْدًا شَكُورًا
{\tiny\colorbox{cl_aya}{4}} وَقَضَيْنَا إِلَى بَنِى إِسْرَءِيلَ فِى الْكِتَبِ لَتُفْسِدُنَّ فِى الْأَرْضِ مَرَّتَيْنِ وَلَتَعْلُنَّ عُلُوًّا كَبِيرًا
{\tiny\colorbox{cl_aya}{5}} فَإِذَا جَاءَ وَعْدُ أُولَىهُمَا بَعَثْنَا عَلَيْكُمْ عِبَادًا لَّنَا أُولِى بَأْسٍ شَدِيدٍ فَجَاسُوا خِلَلَ الدِّيَارِ وَكَانَ وَعْدًا مَّفْعُولًا
{\tiny\colorbox{cl_aya}{6}} ثُمَّ رَدَدْنَا لَكُمُ الْكَرَّةَ عَلَيْهِمْ وَأَمْدَدْنَكُم بِأَمْوَلٍ وَبَنِينَ وَجَعَلْنَكُمْ أَكْثَرَ نَفِيرًا
{\tiny\colorbox{cl_aya}{7}} إِنْ أَحْسَنتُمْ أَحْسَنتُمْ لِأَنفُسِكُمْ وَإِنْ أَسَأْتُمْ فَلَهَا فَإِذَا جَاءَ وَعْدُ الْءَاخِرَةِ لِيَسُُٔوا وُجُوهَكُمْ وَلِيَدْخُلُوا الْمَسْجِدَ كَمَا دَخَلُوهُ أَوَّلَ مَرَّةٍ وَلِيُتَبِّرُوا مَا عَلَوْا تَتْبِيرًا
{\tiny\colorbox{cl_aya}{8}} عَسَى رَبُّكُمْ أَن يَرْحَمَكُمْ وَإِنْ عُدتُّمْ عُدْنَا وَجَعَلْنَا جَهَنَّمَ لِلْكَفِرِينَ حَصِيرًا
{\tiny\colorbox{cl_aya}{9}} إِنَّ هَذَا الْقُرْءَانَ يَهْدِى لِلَّتِى هِىَ أَقْوَمُ وَيُبَشِّرُ الْمُؤْمِنِينَ الَّذِينَ يَعْمَلُونَ الصَّلِحَتِ أَنَّ لَهُمْ أَجْرًا كَبِيرًا
{\tiny\colorbox{cl_aya}{10}} وَأَنَّ الَّذِينَ لَا يُؤْمِنُونَ بِالْءَاخِرَةِ أَعْتَدْنَا لَهُمْ عَذَابًا أَلِيمًا
{\tiny\colorbox{cl_aya}{11}} وَيَدْعُ الْإِنسَنُ بِالشَّرِّ دُعَاءَهُ بِالْخَيْرِ وَكَانَ الْإِنسَنُ عَجُولًا
{\tiny\colorbox{cl_aya}{12}} وَجَعَلْنَا الَّيْلَ وَالنَّهَارَ ءَايَتَيْنِ فَمَحَوْنَا ءَايَةَ الَّيْلِ وَجَعَلْنَا ءَايَةَ النَّهَارِ مُبْصِرَةً لِّتَبْتَغُوا فَضْلًا مِّن رَّبِّكُمْ وَلِتَعْلَمُوا عَدَدَ السِّنِينَ وَالْحِسَابَ وَكُلَّ شَىْءٍ فَصَّلْنَهُ تَفْصِيلًا
{\tiny\colorbox{cl_aya}{13}} وَكُلَّ إِنسَنٍ أَلْزَمْنَهُ طَئِرَهُ فِى عُنُقِهِ وَنُخْرِجُ لَهُ يَوْمَ الْقِيَمَةِ كِتَبًا يَلْقَىهُ مَنشُورًا
{\tiny\colorbox{cl_aya}{14}} اقْرَأْ كِتَبَكَ كَفَى بِنَفْسِكَ الْيَوْمَ عَلَيْكَ حَسِيبًا
{\tiny\colorbox{cl_aya}{15}} مَّنِ اهْتَدَى فَإِنَّمَا يَهْتَدِى لِنَفْسِهِ وَمَن ضَلَّ فَإِنَّمَا يَضِلُّ عَلَيْهَا وَلَا تَزِرُ وَازِرَةٌ وِزْرَ أُخْرَى وَمَا كُنَّا مُعَذِّبِينَ حَتَّى نَبْعَثَ رَسُولًا
{\tiny\colorbox{cl_aya}{16}} وَإِذَا أَرَدْنَا أَن نُّهْلِكَ قَرْيَةً أَمَرْنَا مُتْرَفِيهَا فَفَسَقُوا فِيهَا فَحَقَّ عَلَيْهَا الْقَوْلُ فَدَمَّرْنَهَا تَدْمِيرًا
{\tiny\colorbox{cl_aya}{17}} وَكَمْ أَهْلَكْنَا مِنَ الْقُرُونِ مِن بَعْدِ نُوحٍ وَكَفَى بِرَبِّكَ بِذُنُوبِ عِبَادِهِ خَبِيرًا بَصِيرًا
{\tiny\colorbox{cl_aya}{18}} مَّن كَانَ يُرِيدُ الْعَاجِلَةَ عَجَّلْنَا لَهُ فِيهَا مَا نَشَاءُ لِمَن نُّرِيدُ ثُمَّ جَعَلْنَا لَهُ جَهَنَّمَ يَصْلَىهَا مَذْمُومًا مَّدْحُورًا
{\tiny\colorbox{cl_aya}{19}} وَمَنْ أَرَادَ الْءَاخِرَةَ وَسَعَى لَهَا سَعْيَهَا وَهُوَ مُؤْمِنٌ فَأُولَئِكَ كَانَ سَعْيُهُم مَّشْكُورًا
{\tiny\colorbox{cl_aya}{20}} كُلًّا نُّمِدُّ هَؤُلَاءِ وَهَؤُلَاءِ مِنْ عَطَاءِ رَبِّكَ وَمَا كَانَ عَطَاءُ رَبِّكَ مَحْظُورًا
{\tiny\colorbox{cl_aya}{21}} انظُرْ كَيْفَ فَضَّلْنَا بَعْضَهُمْ عَلَى بَعْضٍ وَلَلْءَاخِرَةُ أَكْبَرُ دَرَجَتٍ وَأَكْبَرُ تَفْضِيلًا
{\tiny\colorbox{cl_aya}{22}} لَّا تَجْعَلْ مَعَ اللَّهِ إِلَهًا ءَاخَرَ فَتَقْعُدَ مَذْمُومًا مَّخْذُولًا
{\tiny\colorbox{cl_aya}{23}} وَقَضَى رَبُّكَ أَلَّا تَعْبُدُوا إِلَّا إِيَّاهُ وَبِالْوَلِدَيْنِ إِحْسَنًا إِمَّا يَبْلُغَنَّ عِندَكَ الْكِبَرَ أَحَدُهُمَا أَوْ كِلَاهُمَا فَلَا تَقُل لَّهُمَا أُفٍّ وَلَا تَنْهَرْهُمَا وَقُل لَّهُمَا قَوْلًا كَرِيمًا
{\tiny\colorbox{cl_aya}{24}} وَاخْفِضْ لَهُمَا جَنَاحَ الذُّلِّ مِنَ الرَّحْمَةِ وَقُل رَّبِّ ارْحَمْهُمَا كَمَا رَبَّيَانِى صَغِيرًا
{\tiny\colorbox{cl_aya}{25}} رَّبُّكُمْ أَعْلَمُ بِمَا فِى نُفُوسِكُمْ إِن تَكُونُوا صَلِحِينَ فَإِنَّهُ كَانَ لِلْأَوَّبِينَ غَفُورًا
{\tiny\colorbox{cl_aya}{26}} وَءَاتِ ذَا الْقُرْبَى حَقَّهُ وَالْمِسْكِينَ وَابْنَ السَّبِيلِ وَلَا تُبَذِّرْ تَبْذِيرًا
{\tiny\colorbox{cl_aya}{27}} إِنَّ الْمُبَذِّرِينَ كَانُوا إِخْوَنَ الشَّيَطِينِ وَكَانَ الشَّيْطَنُ لِرَبِّهِ كَفُورًا
{\tiny\colorbox{cl_aya}{28}} وَإِمَّا تُعْرِضَنَّ عَنْهُمُ ابْتِغَاءَ رَحْمَةٍ مِّن رَّبِّكَ تَرْجُوهَا فَقُل لَّهُمْ قَوْلًا مَّيْسُورًا
{\tiny\colorbox{cl_aya}{29}} وَلَا تَجْعَلْ يَدَكَ مَغْلُولَةً إِلَى عُنُقِكَ وَلَا تَبْسُطْهَا كُلَّ الْبَسْطِ فَتَقْعُدَ مَلُومًا مَّحْسُورًا
{\tiny\colorbox{cl_aya}{30}} إِنَّ رَبَّكَ يَبْسُطُ الرِّزْقَ لِمَن يَشَاءُ وَيَقْدِرُ إِنَّهُ كَانَ بِعِبَادِهِ خَبِيرًا بَصِيرًا
{\tiny\colorbox{cl_aya}{31}} وَلَا تَقْتُلُوا أَوْلَدَكُمْ خَشْيَةَ إِمْلَقٍ نَّحْنُ نَرْزُقُهُمْ وَإِيَّاكُمْ إِنَّ قَتْلَهُمْ كَانَ خِطًْٔا كَبِيرًا
{\tiny\colorbox{cl_aya}{32}} وَلَا تَقْرَبُوا الزِّنَى إِنَّهُ كَانَ فَحِشَةً وَسَاءَ سَبِيلًا
{\tiny\colorbox{cl_aya}{33}} وَلَا تَقْتُلُوا النَّفْسَ الَّتِى حَرَّمَ اللَّهُ إِلَّا بِالْحَقِّ وَمَن قُتِلَ مَظْلُومًا فَقَدْ جَعَلْنَا لِوَلِيِّهِ سُلْطَنًا فَلَا يُسْرِف فِّى الْقَتْلِ إِنَّهُ كَانَ مَنصُورًا
{\tiny\colorbox{cl_aya}{34}} وَلَا تَقْرَبُوا مَالَ الْيَتِيمِ إِلَّا بِالَّتِى هِىَ أَحْسَنُ حَتَّى يَبْلُغَ أَشُدَّهُ وَأَوْفُوا بِالْعَهْدِ إِنَّ الْعَهْدَ كَانَ مَسُْٔولًا
{\tiny\colorbox{cl_aya}{35}} وَأَوْفُوا الْكَيْلَ إِذَا كِلْتُمْ وَزِنُوا بِالْقِسْطَاسِ الْمُسْتَقِيمِ ذَلِكَ خَيْرٌ وَأَحْسَنُ تَأْوِيلًا
{\tiny\colorbox{cl_aya}{36}} وَلَا تَقْفُ مَا لَيْسَ لَكَ بِهِ عِلْمٌ إِنَّ السَّمْعَ وَالْبَصَرَ وَالْفُؤَادَ كُلُّ أُولَئِكَ كَانَ عَنْهُ مَسُْٔولًا
{\tiny\colorbox{cl_aya}{37}} وَلَا تَمْشِ فِى الْأَرْضِ مَرَحًا إِنَّكَ لَن تَخْرِقَ الْأَرْضَ وَلَن تَبْلُغَ الْجِبَالَ طُولًا
{\tiny\colorbox{cl_aya}{38}} كُلُّ ذَلِكَ كَانَ سَيِّئُهُ عِندَ رَبِّكَ مَكْرُوهًا
{\tiny\colorbox{cl_aya}{39}} ذَلِكَ مِمَّا أَوْحَى إِلَيْكَ رَبُّكَ مِنَ الْحِكْمَةِ وَلَا تَجْعَلْ مَعَ اللَّهِ إِلَهًا ءَاخَرَ فَتُلْقَى فِى جَهَنَّمَ مَلُومًا مَّدْحُورًا
{\tiny\colorbox{cl_aya}{40}} أَفَأَصْفَىكُمْ رَبُّكُم بِالْبَنِينَ وَاتَّخَذَ مِنَ الْمَلَئِكَةِ إِنَثًا إِنَّكُمْ لَتَقُولُونَ قَوْلًا عَظِيمًا
{\tiny\colorbox{cl_aya}{41}} وَلَقَدْ صَرَّفْنَا فِى هَذَا الْقُرْءَانِ لِيَذَّكَّرُوا وَمَا يَزِيدُهُمْ إِلَّا نُفُورًا
{\tiny\colorbox{cl_aya}{42}} قُل لَّوْ كَانَ مَعَهُ ءَالِهَةٌ كَمَا يَقُولُونَ إِذًا لَّابْتَغَوْا إِلَى ذِى الْعَرْشِ سَبِيلًا
{\tiny\colorbox{cl_aya}{43}} سُبْحَنَهُ وَتَعَلَى عَمَّا يَقُولُونَ عُلُوًّا كَبِيرًا
{\tiny\colorbox{cl_aya}{44}} تُسَبِّحُ لَهُ السَّمَوَتُ السَّبْعُ وَالْأَرْضُ وَمَن فِيهِنَّ وَإِن مِّن شَىْءٍ إِلَّا يُسَبِّحُ بِحَمْدِهِ وَلَكِن لَّا تَفْقَهُونَ تَسْبِيحَهُمْ إِنَّهُ كَانَ حَلِيمًا غَفُورًا
{\tiny\colorbox{cl_aya}{45}} وَإِذَا قَرَأْتَ الْقُرْءَانَ جَعَلْنَا بَيْنَكَ وَبَيْنَ الَّذِينَ لَا يُؤْمِنُونَ بِالْءَاخِرَةِ حِجَابًا مَّسْتُورًا
{\tiny\colorbox{cl_aya}{46}} وَجَعَلْنَا عَلَى قُلُوبِهِمْ أَكِنَّةً أَن يَفْقَهُوهُ وَفِى ءَاذَانِهِمْ وَقْرًا وَإِذَا ذَكَرْتَ رَبَّكَ فِى الْقُرْءَانِ وَحْدَهُ وَلَّوْا عَلَى أَدْبَرِهِمْ نُفُورًا
{\tiny\colorbox{cl_aya}{47}} نَّحْنُ أَعْلَمُ بِمَا يَسْتَمِعُونَ بِهِ إِذْ يَسْتَمِعُونَ إِلَيْكَ وَإِذْ هُمْ نَجْوَى إِذْ يَقُولُ الظَّلِمُونَ إِن تَتَّبِعُونَ إِلَّا رَجُلًا مَّسْحُورًا
{\tiny\colorbox{cl_aya}{48}} انظُرْ كَيْفَ ضَرَبُوا لَكَ الْأَمْثَالَ فَضَلُّوا فَلَا يَسْتَطِيعُونَ سَبِيلًا
{\tiny\colorbox{cl_aya}{49}} وَقَالُوا أَءِذَا كُنَّا عِظَمًا وَرُفَتًا أَءِنَّا لَمَبْعُوثُونَ خَلْقًا جَدِيدًا
{\tiny\colorbox{cl_aya}{50}} قُلْ كُونُوا حِجَارَةً أَوْ حَدِيدًا
{\tiny\colorbox{cl_aya}{51}} أَوْ خَلْقًا مِّمَّا يَكْبُرُ فِى صُدُورِكُمْ فَسَيَقُولُونَ مَن يُعِيدُنَا قُلِ الَّذِى فَطَرَكُمْ أَوَّلَ مَرَّةٍ فَسَيُنْغِضُونَ إِلَيْكَ رُءُوسَهُمْ وَيَقُولُونَ مَتَى هُوَ قُلْ عَسَى أَن يَكُونَ قَرِيبًا
{\tiny\colorbox{cl_aya}{52}} يَوْمَ يَدْعُوكُمْ فَتَسْتَجِيبُونَ بِحَمْدِهِ وَتَظُنُّونَ إِن لَّبِثْتُمْ إِلَّا قَلِيلًا
{\tiny\colorbox{cl_aya}{53}} وَقُل لِّعِبَادِى يَقُولُوا الَّتِى هِىَ أَحْسَنُ إِنَّ الشَّيْطَنَ يَنزَغُ بَيْنَهُمْ إِنَّ الشَّيْطَنَ كَانَ لِلْإِنسَنِ عَدُوًّا مُّبِينًا
{\tiny\colorbox{cl_aya}{54}} رَّبُّكُمْ أَعْلَمُ بِكُمْ إِن يَشَأْ يَرْحَمْكُمْ أَوْ إِن يَشَأْ يُعَذِّبْكُمْ وَمَا أَرْسَلْنَكَ عَلَيْهِمْ وَكِيلًا
{\tiny\colorbox{cl_aya}{55}} وَرَبُّكَ أَعْلَمُ بِمَن فِى السَّمَوَتِ وَالْأَرْضِ وَلَقَدْ فَضَّلْنَا بَعْضَ النَّبِيِّنَ عَلَى بَعْضٍ وَءَاتَيْنَا دَاوُدَ زَبُورًا
{\tiny\colorbox{cl_aya}{56}} قُلِ ادْعُوا الَّذِينَ زَعَمْتُم مِّن دُونِهِ فَلَا يَمْلِكُونَ كَشْفَ الضُّرِّ عَنكُمْ وَلَا تَحْوِيلًا
{\tiny\colorbox{cl_aya}{57}} أُولَئِكَ الَّذِينَ يَدْعُونَ يَبْتَغُونَ إِلَى رَبِّهِمُ الْوَسِيلَةَ أَيُّهُمْ أَقْرَبُ وَيَرْجُونَ رَحْمَتَهُ وَيَخَافُونَ عَذَابَهُ إِنَّ عَذَابَ رَبِّكَ كَانَ مَحْذُورًا
{\tiny\colorbox{cl_aya}{58}} وَإِن مِّن قَرْيَةٍ إِلَّا نَحْنُ مُهْلِكُوهَا قَبْلَ يَوْمِ الْقِيَمَةِ أَوْ مُعَذِّبُوهَا عَذَابًا شَدِيدًا كَانَ ذَلِكَ فِى الْكِتَبِ مَسْطُورًا
{\tiny\colorbox{cl_aya}{59}} وَمَا مَنَعَنَا أَن نُّرْسِلَ بِالْءَايَتِ إِلَّا أَن كَذَّبَ بِهَا الْأَوَّلُونَ وَءَاتَيْنَا ثَمُودَ النَّاقَةَ مُبْصِرَةً فَظَلَمُوا بِهَا وَمَا نُرْسِلُ بِالْءَايَتِ إِلَّا تَخْوِيفًا
{\tiny\colorbox{cl_aya}{60}} وَإِذْ قُلْنَا لَكَ إِنَّ رَبَّكَ أَحَاطَ بِالنَّاسِ وَمَا جَعَلْنَا الرُّءْيَا الَّتِى أَرَيْنَكَ إِلَّا فِتْنَةً لِّلنَّاسِ وَالشَّجَرَةَ الْمَلْعُونَةَ فِى الْقُرْءَانِ وَنُخَوِّفُهُمْ فَمَا يَزِيدُهُمْ إِلَّا طُغْيَنًا كَبِيرًا
{\tiny\colorbox{cl_aya}{61}} وَإِذْ قُلْنَا لِلْمَلَئِكَةِ اسْجُدُوا لِءَادَمَ فَسَجَدُوا إِلَّا إِبْلِيسَ قَالَ ءَأَسْجُدُ لِمَنْ خَلَقْتَ طِينًا
{\tiny\colorbox{cl_aya}{62}} قَالَ أَرَءَيْتَكَ هَذَا الَّذِى كَرَّمْتَ عَلَىَّ لَئِنْ أَخَّرْتَنِ إِلَى يَوْمِ الْقِيَمَةِ لَأَحْتَنِكَنَّ ذُرِّيَّتَهُ إِلَّا قَلِيلًا
{\tiny\colorbox{cl_aya}{63}} قَالَ اذْهَبْ فَمَن تَبِعَكَ مِنْهُمْ فَإِنَّ جَهَنَّمَ جَزَاؤُكُمْ جَزَاءً مَّوْفُورًا
{\tiny\colorbox{cl_aya}{64}} وَاسْتَفْزِزْ مَنِ اسْتَطَعْتَ مِنْهُم بِصَوْتِكَ وَأَجْلِبْ عَلَيْهِم بِخَيْلِكَ وَرَجِلِكَ وَشَارِكْهُمْ فِى الْأَمْوَلِ وَالْأَوْلَدِ وَعِدْهُمْ وَمَا يَعِدُهُمُ الشَّيْطَنُ إِلَّا غُرُورًا
{\tiny\colorbox{cl_aya}{65}} إِنَّ عِبَادِى لَيْسَ لَكَ عَلَيْهِمْ سُلْطَنٌ وَكَفَى بِرَبِّكَ وَكِيلًا
{\tiny\colorbox{cl_aya}{66}} رَّبُّكُمُ الَّذِى يُزْجِى لَكُمُ الْفُلْكَ فِى الْبَحْرِ لِتَبْتَغُوا مِن فَضْلِهِ إِنَّهُ كَانَ بِكُمْ رَحِيمًا
{\tiny\colorbox{cl_aya}{67}} وَإِذَا مَسَّكُمُ الضُّرُّ فِى الْبَحْرِ ضَلَّ مَن تَدْعُونَ إِلَّا إِيَّاهُ فَلَمَّا نَجَّىكُمْ إِلَى الْبَرِّ أَعْرَضْتُمْ وَكَانَ الْإِنسَنُ كَفُورًا
{\tiny\colorbox{cl_aya}{68}} أَفَأَمِنتُمْ أَن يَخْسِفَ بِكُمْ جَانِبَ الْبَرِّ أَوْ يُرْسِلَ عَلَيْكُمْ حَاصِبًا ثُمَّ لَا تَجِدُوا لَكُمْ وَكِيلًا
{\tiny\colorbox{cl_aya}{69}} أَمْ أَمِنتُمْ أَن يُعِيدَكُمْ فِيهِ تَارَةً أُخْرَى فَيُرْسِلَ عَلَيْكُمْ قَاصِفًا مِّنَ الرِّيحِ فَيُغْرِقَكُم بِمَا كَفَرْتُمْ ثُمَّ لَا تَجِدُوا لَكُمْ عَلَيْنَا بِهِ تَبِيعًا
{\tiny\colorbox{cl_aya}{70}} وَلَقَدْ كَرَّمْنَا بَنِى ءَادَمَ وَحَمَلْنَهُمْ فِى الْبَرِّ وَالْبَحْرِ وَرَزَقْنَهُم مِّنَ الطَّيِّبَتِ وَفَضَّلْنَهُمْ عَلَى كَثِيرٍ مِّمَّنْ خَلَقْنَا تَفْضِيلًا
{\tiny\colorbox{cl_aya}{71}} يَوْمَ نَدْعُوا كُلَّ أُنَاسٍ بِإِمَمِهِمْ فَمَنْ أُوتِىَ كِتَبَهُ بِيَمِينِهِ فَأُولَئِكَ يَقْرَءُونَ كِتَبَهُمْ وَلَا يُظْلَمُونَ فَتِيلًا
{\tiny\colorbox{cl_aya}{72}} وَمَن كَانَ فِى هَذِهِ أَعْمَى فَهُوَ فِى الْءَاخِرَةِ أَعْمَى وَأَضَلُّ سَبِيلًا
{\tiny\colorbox{cl_aya}{73}} وَإِن كَادُوا لَيَفْتِنُونَكَ عَنِ الَّذِى أَوْحَيْنَا إِلَيْكَ لِتَفْتَرِىَ عَلَيْنَا غَيْرَهُ وَإِذًا لَّاتَّخَذُوكَ خَلِيلًا
{\tiny\colorbox{cl_aya}{74}} وَلَوْلَا أَن ثَبَّتْنَكَ لَقَدْ كِدتَّ تَرْكَنُ إِلَيْهِمْ شَئًْا قَلِيلًا
{\tiny\colorbox{cl_aya}{75}} إِذًا لَّأَذَقْنَكَ ضِعْفَ الْحَيَوةِ وَضِعْفَ الْمَمَاتِ ثُمَّ لَا تَجِدُ لَكَ عَلَيْنَا نَصِيرًا
{\tiny\colorbox{cl_aya}{76}} وَإِن كَادُوا لَيَسْتَفِزُّونَكَ مِنَ الْأَرْضِ لِيُخْرِجُوكَ مِنْهَا وَإِذًا لَّا يَلْبَثُونَ خِلَفَكَ إِلَّا قَلِيلًا
{\tiny\colorbox{cl_aya}{77}} سُنَّةَ مَن قَدْ أَرْسَلْنَا قَبْلَكَ مِن رُّسُلِنَا وَلَا تَجِدُ لِسُنَّتِنَا تَحْوِيلًا
{\tiny\colorbox{cl_aya}{78}} أَقِمِ الصَّلَوةَ لِدُلُوكِ الشَّمْسِ إِلَى غَسَقِ الَّيْلِ وَقُرْءَانَ الْفَجْرِ إِنَّ قُرْءَانَ الْفَجْرِ كَانَ مَشْهُودًا
{\tiny\colorbox{cl_aya}{79}} وَمِنَ الَّيْلِ فَتَهَجَّدْ بِهِ نَافِلَةً لَّكَ عَسَى أَن يَبْعَثَكَ رَبُّكَ مَقَامًا مَّحْمُودًا
{\tiny\colorbox{cl_aya}{80}} وَقُل رَّبِّ أَدْخِلْنِى مُدْخَلَ صِدْقٍ وَأَخْرِجْنِى مُخْرَجَ صِدْقٍ وَاجْعَل لِّى مِن لَّدُنكَ سُلْطَنًا نَّصِيرًا
{\tiny\colorbox{cl_aya}{81}} وَقُلْ جَاءَ الْحَقُّ وَزَهَقَ الْبَطِلُ إِنَّ الْبَطِلَ كَانَ زَهُوقًا
{\tiny\colorbox{cl_aya}{82}} وَنُنَزِّلُ مِنَ الْقُرْءَانِ مَا هُوَ شِفَاءٌ وَرَحْمَةٌ لِّلْمُؤْمِنِينَ وَلَا يَزِيدُ الظَّلِمِينَ إِلَّا خَسَارًا
{\tiny\colorbox{cl_aya}{83}} وَإِذَا أَنْعَمْنَا عَلَى الْإِنسَنِ أَعْرَضَ وَنََٔا بِجَانِبِهِ وَإِذَا مَسَّهُ الشَّرُّ كَانَ ئَُوسًا
{\tiny\colorbox{cl_aya}{84}} قُلْ كُلٌّ يَعْمَلُ عَلَى شَاكِلَتِهِ فَرَبُّكُمْ أَعْلَمُ بِمَنْ هُوَ أَهْدَى سَبِيلًا
{\tiny\colorbox{cl_aya}{85}} وَيَسَْٔلُونَكَ عَنِ الرُّوحِ قُلِ الرُّوحُ مِنْ أَمْرِ رَبِّى وَمَا أُوتِيتُم مِّنَ الْعِلْمِ إِلَّا قَلِيلًا
{\tiny\colorbox{cl_aya}{86}} وَلَئِن شِئْنَا لَنَذْهَبَنَّ بِالَّذِى أَوْحَيْنَا إِلَيْكَ ثُمَّ لَا تَجِدُ لَكَ بِهِ عَلَيْنَا وَكِيلًا
{\tiny\colorbox{cl_aya}{87}} إِلَّا رَحْمَةً مِّن رَّبِّكَ إِنَّ فَضْلَهُ كَانَ عَلَيْكَ كَبِيرًا
{\tiny\colorbox{cl_aya}{88}} قُل لَّئِنِ اجْتَمَعَتِ الْإِنسُ وَالْجِنُّ عَلَى أَن يَأْتُوا بِمِثْلِ هَذَا الْقُرْءَانِ لَا يَأْتُونَ بِمِثْلِهِ وَلَوْ كَانَ بَعْضُهُمْ لِبَعْضٍ ظَهِيرًا
{\tiny\colorbox{cl_aya}{89}} وَلَقَدْ صَرَّفْنَا لِلنَّاسِ فِى هَذَا الْقُرْءَانِ مِن كُلِّ مَثَلٍ فَأَبَى أَكْثَرُ النَّاسِ إِلَّا كُفُورًا
{\tiny\colorbox{cl_aya}{90}} وَقَالُوا لَن نُّؤْمِنَ لَكَ حَتَّى تَفْجُرَ لَنَا مِنَ الْأَرْضِ يَنبُوعًا
{\tiny\colorbox{cl_aya}{91}} أَوْ تَكُونَ لَكَ جَنَّةٌ مِّن نَّخِيلٍ وَعِنَبٍ فَتُفَجِّرَ الْأَنْهَرَ خِلَلَهَا تَفْجِيرًا
{\tiny\colorbox{cl_aya}{92}} أَوْ تُسْقِطَ السَّمَاءَ كَمَا زَعَمْتَ عَلَيْنَا كِسَفًا أَوْ تَأْتِىَ بِاللَّهِ وَالْمَلَئِكَةِ قَبِيلًا
{\tiny\colorbox{cl_aya}{93}} أَوْ يَكُونَ لَكَ بَيْتٌ مِّن زُخْرُفٍ أَوْ تَرْقَى فِى السَّمَاءِ وَلَن نُّؤْمِنَ لِرُقِيِّكَ حَتَّى تُنَزِّلَ عَلَيْنَا كِتَبًا نَّقْرَؤُهُ قُلْ سُبْحَانَ رَبِّى هَلْ كُنتُ إِلَّا بَشَرًا رَّسُولًا
{\tiny\colorbox{cl_aya}{94}} وَمَا مَنَعَ النَّاسَ أَن يُؤْمِنُوا إِذْ جَاءَهُمُ الْهُدَى إِلَّا أَن قَالُوا أَبَعَثَ اللَّهُ بَشَرًا رَّسُولًا
{\tiny\colorbox{cl_aya}{95}} قُل لَّوْ كَانَ فِى الْأَرْضِ مَلَئِكَةٌ يَمْشُونَ مُطْمَئِنِّينَ لَنَزَّلْنَا عَلَيْهِم مِّنَ السَّمَاءِ مَلَكًا رَّسُولًا
{\tiny\colorbox{cl_aya}{96}} قُلْ كَفَى بِاللَّهِ شَهِيدًا بَيْنِى وَبَيْنَكُمْ إِنَّهُ كَانَ بِعِبَادِهِ خَبِيرًا بَصِيرًا
{\tiny\colorbox{cl_aya}{97}} وَمَن يَهْدِ اللَّهُ فَهُوَ الْمُهْتَدِ وَمَن يُضْلِلْ فَلَن تَجِدَ لَهُمْ أَوْلِيَاءَ مِن دُونِهِ وَنَحْشُرُهُمْ يَوْمَ الْقِيَمَةِ عَلَى وُجُوهِهِمْ عُمْيًا وَبُكْمًا وَصُمًّا مَّأْوَىهُمْ جَهَنَّمُ كُلَّمَا خَبَتْ زِدْنَهُمْ سَعِيرًا
{\tiny\colorbox{cl_aya}{98}} ذَلِكَ جَزَاؤُهُم بِأَنَّهُمْ كَفَرُوا بَِٔايَتِنَا وَقَالُوا أَءِذَا كُنَّا عِظَمًا وَرُفَتًا أَءِنَّا لَمَبْعُوثُونَ خَلْقًا جَدِيدًا
{\tiny\colorbox{cl_aya}{99}} أَوَلَمْ يَرَوْا أَنَّ اللَّهَ الَّذِى خَلَقَ السَّمَوَتِ وَالْأَرْضَ قَادِرٌ عَلَى أَن يَخْلُقَ مِثْلَهُمْ وَجَعَلَ لَهُمْ أَجَلًا لَّا رَيْبَ فِيهِ فَأَبَى الظَّلِمُونَ إِلَّا كُفُورًا
{\tiny\colorbox{cl_aya}{100}} قُل لَّوْ أَنتُمْ تَمْلِكُونَ خَزَائِنَ رَحْمَةِ رَبِّى إِذًا لَّأَمْسَكْتُمْ خَشْيَةَ الْإِنفَاقِ وَكَانَ الْإِنسَنُ قَتُورًا
{\tiny\colorbox{cl_aya}{101}} وَلَقَدْ ءَاتَيْنَا مُوسَى تِسْعَ ءَايَتٍ بَيِّنَتٍ فَسَْٔلْ بَنِى إِسْرَءِيلَ إِذْ جَاءَهُمْ فَقَالَ لَهُ فِرْعَوْنُ إِنِّى لَأَظُنُّكَ يَمُوسَى مَسْحُورًا
{\tiny\colorbox{cl_aya}{102}} قَالَ لَقَدْ عَلِمْتَ مَا أَنزَلَ هَؤُلَاءِ إِلَّا رَبُّ السَّمَوَتِ وَالْأَرْضِ بَصَائِرَ وَإِنِّى لَأَظُنُّكَ يَفِرْعَوْنُ مَثْبُورًا
{\tiny\colorbox{cl_aya}{103}} فَأَرَادَ أَن يَسْتَفِزَّهُم مِّنَ الْأَرْضِ فَأَغْرَقْنَهُ وَمَن مَّعَهُ جَمِيعًا
{\tiny\colorbox{cl_aya}{104}} وَقُلْنَا مِن بَعْدِهِ لِبَنِى إِسْرَءِيلَ اسْكُنُوا الْأَرْضَ فَإِذَا جَاءَ وَعْدُ الْءَاخِرَةِ جِئْنَا بِكُمْ لَفِيفًا
{\tiny\colorbox{cl_aya}{105}} وَبِالْحَقِّ أَنزَلْنَهُ وَبِالْحَقِّ نَزَلَ وَمَا أَرْسَلْنَكَ إِلَّا مُبَشِّرًا وَنَذِيرًا
{\tiny\colorbox{cl_aya}{106}} وَقُرْءَانًا فَرَقْنَهُ لِتَقْرَأَهُ عَلَى النَّاسِ عَلَى مُكْثٍ وَنَزَّلْنَهُ تَنزِيلًا
{\tiny\colorbox{cl_aya}{107}} قُلْ ءَامِنُوا بِهِ أَوْ لَا تُؤْمِنُوا إِنَّ الَّذِينَ أُوتُوا الْعِلْمَ مِن قَبْلِهِ إِذَا يُتْلَى عَلَيْهِمْ يَخِرُّونَ لِلْأَذْقَانِ سُجَّدًا
{\tiny\colorbox{cl_aya}{108}} وَيَقُولُونَ سُبْحَنَ رَبِّنَا إِن كَانَ وَعْدُ رَبِّنَا لَمَفْعُولًا
{\tiny\colorbox{cl_aya}{109}} وَيَخِرُّونَ لِلْأَذْقَانِ يَبْكُونَ وَيَزِيدُهُمْ خُشُوعًا
{\tiny\colorbox{cl_aya}{110}} قُلِ ادْعُوا اللَّهَ أَوِ ادْعُوا الرَّحْمَنَ أَيًّا مَّا تَدْعُوا فَلَهُ الْأَسْمَاءُ الْحُسْنَى وَلَا تَجْهَرْ بِصَلَاتِكَ وَلَا تُخَافِتْ بِهَا وَابْتَغِ بَيْنَ ذَلِكَ سَبِيلًا
{\tiny\colorbox{cl_aya}{111}} وَقُلِ الْحَمْدُ لِلَّهِ الَّذِى لَمْ يَتَّخِذْ وَلَدًا وَلَمْ يَكُن لَّهُ شَرِيكٌ فِى الْمُلْكِ وَلَمْ يَكُن لَّهُ وَلِىٌّ مِّنَ الذُّلِّ وَكَبِّرْهُ تَكْبِيرًا
\end{document}