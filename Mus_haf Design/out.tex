{\tiny\colorbox{cl_aya}{1}} يَأَيُّهَا الَّذِينَ ءَامَنُوا أَوْفُوا بِالْعُقُودِ أُحِلَّتْ لَكُم بَهِيمَةُ الْأَنْعَمِ إِلَّا مَا يُتْلَى عَلَيْكُمْ غَيْرَ مُحِلِّى الصَّيْدِ وَأَنتُمْ حُرُمٌ إِنَّ اللَّهَ يَحْكُمُ مَا يُرِيدُ
{\tiny\colorbox{cl_aya}{2}} يَأَيُّهَا الَّذِينَ ءَامَنُوا لَا تُحِلُّوا شَعَئِرَ اللَّهِ وَلَا الشَّهْرَ الْحَرَامَ وَلَا الْهَدْىَ وَلَا الْقَلَئِدَ وَلَا ءَامِّينَ الْبَيْتَ الْحَرَامَ يَبْتَغُونَ فَضْلًا مِّن رَّبِّهِمْ وَرِضْوَنًا وَإِذَا حَلَلْتُمْ فَاصْطَادُوا وَلَا يَجْرِمَنَّكُمْ شَنََٔانُ قَوْمٍ أَن صَدُّوكُمْ عَنِ الْمَسْجِدِ الْحَرَامِ أَن تَعْتَدُوا وَتَعَاوَنُوا عَلَى الْبِرِّ وَالتَّقْوَى وَلَا تَعَاوَنُوا عَلَى الْإِثْمِ وَالْعُدْوَنِ وَاتَّقُوا اللَّهَ إِنَّ اللَّهَ شَدِيدُ الْعِقَابِ
{\tiny\colorbox{cl_aya}{3}} حُرِّمَتْ عَلَيْكُمُ الْمَيْتَةُ وَالدَّمُ وَلَحْمُ الْخِنزِيرِ وَمَا أُهِلَّ لِغَيْرِ اللَّهِ بِهِ وَالْمُنْخَنِقَةُ وَالْمَوْقُوذَةُ وَالْمُتَرَدِّيَةُ وَالنَّطِيحَةُ وَمَا أَكَلَ السَّبُعُ إِلَّا مَا ذَكَّيْتُمْ وَمَا ذُبِحَ عَلَى النُّصُبِ وَأَن تَسْتَقْسِمُوا بِالْأَزْلَمِ ذَلِكُمْ فِسْقٌ الْيَوْمَ يَئِسَ الَّذِينَ كَفَرُوا مِن دِينِكُمْ فَلَا تَخْشَوْهُمْ وَاخْشَوْنِ الْيَوْمَ أَكْمَلْتُ لَكُمْ دِينَكُمْ وَأَتْمَمْتُ عَلَيْكُمْ نِعْمَتِى وَرَضِيتُ لَكُمُ الْإِسْلَمَ دِينًا فَمَنِ اضْطُرَّ فِى مَخْمَصَةٍ غَيْرَ مُتَجَانِفٍ لِّإِثْمٍ فَإِنَّ اللَّهَ غَفُورٌ رَّحِيمٌ
{\tiny\colorbox{cl_aya}{4}} يَسَْٔلُونَكَ مَاذَا أُحِلَّ لَهُمْ قُلْ أُحِلَّ لَكُمُ الطَّيِّبَتُ وَمَا عَلَّمْتُم مِّنَ الْجَوَارِحِ مُكَلِّبِينَ تُعَلِّمُونَهُنَّ مِمَّا عَلَّمَكُمُ اللَّهُ فَكُلُوا مِمَّا أَمْسَكْنَ عَلَيْكُمْ وَاذْكُرُوا اسْمَ اللَّهِ عَلَيْهِ وَاتَّقُوا اللَّهَ إِنَّ اللَّهَ سَرِيعُ الْحِسَابِ
{\tiny\colorbox{cl_aya}{5}} الْيَوْمَ أُحِلَّ لَكُمُ الطَّيِّبَتُ وَطَعَامُ الَّذِينَ أُوتُوا الْكِتَبَ حِلٌّ لَّكُمْ وَطَعَامُكُمْ حِلٌّ لَّهُمْ وَالْمُحْصَنَتُ مِنَ الْمُؤْمِنَتِ وَالْمُحْصَنَتُ مِنَ الَّذِينَ أُوتُوا الْكِتَبَ مِن قَبْلِكُمْ إِذَا ءَاتَيْتُمُوهُنَّ أُجُورَهُنَّ مُحْصِنِينَ غَيْرَ مُسَفِحِينَ وَلَا مُتَّخِذِى أَخْدَانٍ وَمَن يَكْفُرْ بِالْإِيمَنِ فَقَدْ حَبِطَ عَمَلُهُ وَهُوَ فِى الْءَاخِرَةِ مِنَ الْخَسِرِينَ
{\tiny\colorbox{cl_aya}{6}} يَأَيُّهَا الَّذِينَ ءَامَنُوا إِذَا قُمْتُمْ إِلَى الصَّلَوةِ فَاغْسِلُوا وُجُوهَكُمْ وَأَيْدِيَكُمْ إِلَى الْمَرَافِقِ وَامْسَحُوا بِرُءُوسِكُمْ وَأَرْجُلَكُمْ إِلَى الْكَعْبَيْنِ وَإِن كُنتُمْ جُنُبًا فَاطَّهَّرُوا وَإِن كُنتُم مَّرْضَى أَوْ عَلَى سَفَرٍ أَوْ جَاءَ أَحَدٌ مِّنكُم مِّنَ الْغَائِطِ أَوْ لَمَسْتُمُ النِّسَاءَ فَلَمْ تَجِدُوا مَاءً فَتَيَمَّمُوا صَعِيدًا طَيِّبًا فَامْسَحُوا بِوُجُوهِكُمْ وَأَيْدِيكُم مِّنْهُ مَا يُرِيدُ اللَّهُ لِيَجْعَلَ عَلَيْكُم مِّنْ حَرَجٍ وَلَكِن يُرِيدُ لِيُطَهِّرَكُمْ وَلِيُتِمَّ نِعْمَتَهُ عَلَيْكُمْ لَعَلَّكُمْ تَشْكُرُونَ
{\tiny\colorbox{cl_aya}{7}} وَاذْكُرُوا نِعْمَةَ اللَّهِ عَلَيْكُمْ وَمِيثَقَهُ الَّذِى وَاثَقَكُم بِهِ إِذْ قُلْتُمْ سَمِعْنَا وَأَطَعْنَا وَاتَّقُوا اللَّهَ إِنَّ اللَّهَ عَلِيمٌ بِذَاتِ الصُّدُورِ
{\tiny\colorbox{cl_aya}{8}} يَأَيُّهَا الَّذِينَ ءَامَنُوا كُونُوا قَوَّمِينَ لِلَّهِ شُهَدَاءَ بِالْقِسْطِ وَلَا يَجْرِمَنَّكُمْ شَنََٔانُ قَوْمٍ عَلَى أَلَّا تَعْدِلُوا اعْدِلُوا هُوَ أَقْرَبُ لِلتَّقْوَى وَاتَّقُوا اللَّهَ إِنَّ اللَّهَ خَبِيرٌ بِمَا تَعْمَلُونَ
{\tiny\colorbox{cl_aya}{9}} وَعَدَ اللَّهُ الَّذِينَ ءَامَنُوا وَعَمِلُوا الصَّلِحَتِ لَهُم مَّغْفِرَةٌ وَأَجْرٌ عَظِيمٌ
{\tiny\colorbox{cl_aya}{10}} وَالَّذِينَ كَفَرُوا وَكَذَّبُوا بَِٔايَتِنَا أُولَئِكَ أَصْحَبُ الْجَحِيمِ
{\tiny\colorbox{cl_aya}{11}} يَأَيُّهَا الَّذِينَ ءَامَنُوا اذْكُرُوا نِعْمَتَ اللَّهِ عَلَيْكُمْ إِذْ هَمَّ قَوْمٌ أَن يَبْسُطُوا إِلَيْكُمْ أَيْدِيَهُمْ فَكَفَّ أَيْدِيَهُمْ عَنكُمْ وَاتَّقُوا اللَّهَ وَعَلَى اللَّهِ فَلْيَتَوَكَّلِ الْمُؤْمِنُونَ
{\tiny\colorbox{cl_aya}{12}} وَلَقَدْ أَخَذَ اللَّهُ مِيثَقَ بَنِى إِسْرَءِيلَ وَبَعَثْنَا مِنْهُمُ اثْنَىْ عَشَرَ نَقِيبًا وَقَالَ اللَّهُ إِنِّى مَعَكُمْ لَئِنْ أَقَمْتُمُ الصَّلَوةَ وَءَاتَيْتُمُ الزَّكَوةَ وَءَامَنتُم بِرُسُلِى وَعَزَّرْتُمُوهُمْ وَأَقْرَضْتُمُ اللَّهَ قَرْضًا حَسَنًا لَّأُكَفِّرَنَّ عَنكُمْ سَئَِّاتِكُمْ وَلَأُدْخِلَنَّكُمْ جَنَّتٍ تَجْرِى مِن تَحْتِهَا الْأَنْهَرُ فَمَن كَفَرَ بَعْدَ ذَلِكَ مِنكُمْ فَقَدْ ضَلَّ سَوَاءَ السَّبِيلِ
{\tiny\colorbox{cl_aya}{13}} فَبِمَا نَقْضِهِم مِّيثَقَهُمْ لَعَنَّهُمْ وَجَعَلْنَا قُلُوبَهُمْ قَسِيَةً يُحَرِّفُونَ الْكَلِمَ عَن مَّوَاضِعِهِ وَنَسُوا حَظًّا مِّمَّا ذُكِّرُوا بِهِ وَلَا تَزَالُ تَطَّلِعُ عَلَى خَائِنَةٍ مِّنْهُمْ إِلَّا قَلِيلًا مِّنْهُمْ فَاعْفُ عَنْهُمْ وَاصْفَحْ إِنَّ اللَّهَ يُحِبُّ الْمُحْسِنِينَ
{\tiny\colorbox{cl_aya}{14}} وَمِنَ الَّذِينَ قَالُوا إِنَّا نَصَرَى أَخَذْنَا مِيثَقَهُمْ فَنَسُوا حَظًّا مِّمَّا ذُكِّرُوا بِهِ فَأَغْرَيْنَا بَيْنَهُمُ الْعَدَاوَةَ وَالْبَغْضَاءَ إِلَى يَوْمِ الْقِيَمَةِ وَسَوْفَ يُنَبِّئُهُمُ اللَّهُ بِمَا كَانُوا يَصْنَعُونَ
{\tiny\colorbox{cl_aya}{15}} يَأَهْلَ الْكِتَبِ قَدْ جَاءَكُمْ رَسُولُنَا يُبَيِّنُ لَكُمْ كَثِيرًا مِّمَّا كُنتُمْ تُخْفُونَ مِنَ الْكِتَبِ وَيَعْفُوا عَن كَثِيرٍ قَدْ جَاءَكُم مِّنَ اللَّهِ نُورٌ وَكِتَبٌ مُّبِينٌ
{\tiny\colorbox{cl_aya}{16}} يَهْدِى بِهِ اللَّهُ مَنِ اتَّبَعَ رِضْوَنَهُ سُبُلَ السَّلَمِ وَيُخْرِجُهُم مِّنَ الظُّلُمَتِ إِلَى النُّورِ بِإِذْنِهِ وَيَهْدِيهِمْ إِلَى صِرَطٍ مُّسْتَقِيمٍ
{\tiny\colorbox{cl_aya}{17}} لَّقَدْ كَفَرَ الَّذِينَ قَالُوا إِنَّ اللَّهَ هُوَ الْمَسِيحُ ابْنُ مَرْيَمَ قُلْ فَمَن يَمْلِكُ مِنَ اللَّهِ شَئًْا إِنْ أَرَادَ أَن يُهْلِكَ الْمَسِيحَ ابْنَ مَرْيَمَ وَأُمَّهُ وَمَن فِى الْأَرْضِ جَمِيعًا وَلِلَّهِ مُلْكُ السَّمَوَتِ وَالْأَرْضِ وَمَا بَيْنَهُمَا يَخْلُقُ مَا يَشَاءُ وَاللَّهُ عَلَى كُلِّ شَىْءٍ قَدِيرٌ
{\tiny\colorbox{cl_aya}{18}} وَقَالَتِ الْيَهُودُ وَالنَّصَرَى نَحْنُ أَبْنَؤُا اللَّهِ وَأَحِبَّؤُهُ قُلْ فَلِمَ يُعَذِّبُكُم بِذُنُوبِكُم بَلْ أَنتُم بَشَرٌ مِّمَّنْ خَلَقَ يَغْفِرُ لِمَن يَشَاءُ وَيُعَذِّبُ مَن يَشَاءُ وَلِلَّهِ مُلْكُ السَّمَوَتِ وَالْأَرْضِ وَمَا بَيْنَهُمَا وَإِلَيْهِ الْمَصِيرُ
{\tiny\colorbox{cl_aya}{19}} يَأَهْلَ الْكِتَبِ قَدْ جَاءَكُمْ رَسُولُنَا يُبَيِّنُ لَكُمْ عَلَى فَتْرَةٍ مِّنَ الرُّسُلِ أَن تَقُولُوا مَا جَاءَنَا مِن بَشِيرٍ وَلَا نَذِيرٍ فَقَدْ جَاءَكُم بَشِيرٌ وَنَذِيرٌ وَاللَّهُ عَلَى كُلِّ شَىْءٍ قَدِيرٌ
{\tiny\colorbox{cl_aya}{20}} وَإِذْ قَالَ مُوسَى لِقَوْمِهِ يَقَوْمِ اذْكُرُوا نِعْمَةَ اللَّهِ عَلَيْكُمْ إِذْ جَعَلَ فِيكُمْ أَنبِيَاءَ وَجَعَلَكُم مُّلُوكًا وَءَاتَىكُم مَّا لَمْ يُؤْتِ أَحَدًا مِّنَ الْعَلَمِينَ
{\tiny\colorbox{cl_aya}{21}} يَقَوْمِ ادْخُلُوا الْأَرْضَ الْمُقَدَّسَةَ الَّتِى كَتَبَ اللَّهُ لَكُمْ وَلَا تَرْتَدُّوا عَلَى أَدْبَارِكُمْ فَتَنقَلِبُوا خَسِرِينَ
{\tiny\colorbox{cl_aya}{22}} قَالُوا يَمُوسَى إِنَّ فِيهَا قَوْمًا جَبَّارِينَ وَإِنَّا لَن نَّدْخُلَهَا حَتَّى يَخْرُجُوا مِنْهَا فَإِن يَخْرُجُوا مِنْهَا فَإِنَّا دَخِلُونَ
{\tiny\colorbox{cl_aya}{23}} قَالَ رَجُلَانِ مِنَ الَّذِينَ يَخَافُونَ أَنْعَمَ اللَّهُ عَلَيْهِمَا ادْخُلُوا عَلَيْهِمُ الْبَابَ فَإِذَا دَخَلْتُمُوهُ فَإِنَّكُمْ غَلِبُونَ وَعَلَى اللَّهِ فَتَوَكَّلُوا إِن كُنتُم مُّؤْمِنِينَ
{\tiny\colorbox{cl_aya}{24}} قَالُوا يَمُوسَى إِنَّا لَن نَّدْخُلَهَا أَبَدًا مَّا دَامُوا فِيهَا فَاذْهَبْ أَنتَ وَرَبُّكَ فَقَتِلَا إِنَّا هَهُنَا قَعِدُونَ
{\tiny\colorbox{cl_aya}{25}} قَالَ رَبِّ إِنِّى لَا أَمْلِكُ إِلَّا نَفْسِى وَأَخِى فَافْرُقْ بَيْنَنَا وَبَيْنَ الْقَوْمِ الْفَسِقِينَ
{\tiny\colorbox{cl_aya}{26}} قَالَ فَإِنَّهَا مُحَرَّمَةٌ عَلَيْهِمْ أَرْبَعِينَ سَنَةً يَتِيهُونَ فِى الْأَرْضِ فَلَا تَأْسَ عَلَى الْقَوْمِ الْفَسِقِينَ
{\tiny\colorbox{cl_aya}{27}} وَاتْلُ عَلَيْهِمْ نَبَأَ ابْنَىْ ءَادَمَ بِالْحَقِّ إِذْ قَرَّبَا قُرْبَانًا فَتُقُبِّلَ مِنْ أَحَدِهِمَا وَلَمْ يُتَقَبَّلْ مِنَ الْءَاخَرِ قَالَ لَأَقْتُلَنَّكَ قَالَ إِنَّمَا يَتَقَبَّلُ اللَّهُ مِنَ الْمُتَّقِينَ
{\tiny\colorbox{cl_aya}{28}} لَئِن بَسَطتَ إِلَىَّ يَدَكَ لِتَقْتُلَنِى مَا أَنَا بِبَاسِطٍ يَدِىَ إِلَيْكَ لِأَقْتُلَكَ إِنِّى أَخَافُ اللَّهَ رَبَّ الْعَلَمِينَ
{\tiny\colorbox{cl_aya}{29}} إِنِّى أُرِيدُ أَن تَبُوأَ بِإِثْمِى وَإِثْمِكَ فَتَكُونَ مِنْ أَصْحَبِ النَّارِ وَذَلِكَ جَزَؤُا الظَّلِمِينَ
{\tiny\colorbox{cl_aya}{30}} فَطَوَّعَتْ لَهُ نَفْسُهُ قَتْلَ أَخِيهِ فَقَتَلَهُ فَأَصْبَحَ مِنَ الْخَسِرِينَ
{\tiny\colorbox{cl_aya}{31}} فَبَعَثَ اللَّهُ غُرَابًا يَبْحَثُ فِى الْأَرْضِ لِيُرِيَهُ كَيْفَ يُوَرِى سَوْءَةَ أَخِيهِ قَالَ يَوَيْلَتَى أَعَجَزْتُ أَنْ أَكُونَ مِثْلَ هَذَا الْغُرَابِ فَأُوَرِىَ سَوْءَةَ أَخِى فَأَصْبَحَ مِنَ النَّدِمِينَ
{\tiny\colorbox{cl_aya}{32}} مِنْ أَجْلِ ذَلِكَ كَتَبْنَا عَلَى بَنِى إِسْرَءِيلَ أَنَّهُ مَن قَتَلَ نَفْسًا بِغَيْرِ نَفْسٍ أَوْ فَسَادٍ فِى الْأَرْضِ فَكَأَنَّمَا قَتَلَ النَّاسَ جَمِيعًا وَمَنْ أَحْيَاهَا فَكَأَنَّمَا أَحْيَا النَّاسَ جَمِيعًا وَلَقَدْ جَاءَتْهُمْ رُسُلُنَا بِالْبَيِّنَتِ ثُمَّ إِنَّ كَثِيرًا مِّنْهُم بَعْدَ ذَلِكَ فِى الْأَرْضِ لَمُسْرِفُونَ
{\tiny\colorbox{cl_aya}{33}} إِنَّمَا جَزَؤُا الَّذِينَ يُحَارِبُونَ اللَّهَ وَرَسُولَهُ وَيَسْعَوْنَ فِى الْأَرْضِ فَسَادًا أَن يُقَتَّلُوا أَوْ يُصَلَّبُوا أَوْ تُقَطَّعَ أَيْدِيهِمْ وَأَرْجُلُهُم مِّنْ خِلَفٍ أَوْ يُنفَوْا مِنَ الْأَرْضِ ذَلِكَ لَهُمْ خِزْىٌ فِى الدُّنْيَا وَلَهُمْ فِى الْءَاخِرَةِ عَذَابٌ عَظِيمٌ
{\tiny\colorbox{cl_aya}{34}} إِلَّا الَّذِينَ تَابُوا مِن قَبْلِ أَن تَقْدِرُوا عَلَيْهِمْ فَاعْلَمُوا أَنَّ اللَّهَ غَفُورٌ رَّحِيمٌ
{\tiny\colorbox{cl_aya}{35}} يَأَيُّهَا الَّذِينَ ءَامَنُوا اتَّقُوا اللَّهَ وَابْتَغُوا إِلَيْهِ الْوَسِيلَةَ وَجَهِدُوا فِى سَبِيلِهِ لَعَلَّكُمْ تُفْلِحُونَ
{\tiny\colorbox{cl_aya}{36}} إِنَّ الَّذِينَ كَفَرُوا لَوْ أَنَّ لَهُم مَّا فِى الْأَرْضِ جَمِيعًا وَمِثْلَهُ مَعَهُ لِيَفْتَدُوا بِهِ مِنْ عَذَابِ يَوْمِ الْقِيَمَةِ مَا تُقُبِّلَ مِنْهُمْ وَلَهُمْ عَذَابٌ أَلِيمٌ
{\tiny\colorbox{cl_aya}{37}} يُرِيدُونَ أَن يَخْرُجُوا مِنَ النَّارِ وَمَا هُم بِخَرِجِينَ مِنْهَا وَلَهُمْ عَذَابٌ مُّقِيمٌ
{\tiny\colorbox{cl_aya}{38}} وَالسَّارِقُ وَالسَّارِقَةُ فَاقْطَعُوا أَيْدِيَهُمَا جَزَاءً بِمَا كَسَبَا نَكَلًا مِّنَ اللَّهِ وَاللَّهُ عَزِيزٌ حَكِيمٌ
{\tiny\colorbox{cl_aya}{39}} فَمَن تَابَ مِن بَعْدِ ظُلْمِهِ وَأَصْلَحَ فَإِنَّ اللَّهَ يَتُوبُ عَلَيْهِ إِنَّ اللَّهَ غَفُورٌ رَّحِيمٌ
{\tiny\colorbox{cl_aya}{40}} أَلَمْ تَعْلَمْ أَنَّ اللَّهَ لَهُ مُلْكُ السَّمَوَتِ وَالْأَرْضِ يُعَذِّبُ مَن يَشَاءُ وَيَغْفِرُ لِمَن يَشَاءُ وَاللَّهُ عَلَى كُلِّ شَىْءٍ قَدِيرٌ
{\tiny\colorbox{cl_aya}{41}} يَأَيُّهَا الرَّسُولُ لَا يَحْزُنكَ الَّذِينَ يُسَرِعُونَ فِى الْكُفْرِ مِنَ الَّذِينَ قَالُوا ءَامَنَّا بِأَفْوَهِهِمْ وَلَمْ تُؤْمِن قُلُوبُهُمْ وَمِنَ الَّذِينَ هَادُوا سَمَّعُونَ لِلْكَذِبِ سَمَّعُونَ لِقَوْمٍ ءَاخَرِينَ لَمْ يَأْتُوكَ يُحَرِّفُونَ الْكَلِمَ مِن بَعْدِ مَوَاضِعِهِ يَقُولُونَ إِنْ أُوتِيتُمْ هَذَا فَخُذُوهُ وَإِن لَّمْ تُؤْتَوْهُ فَاحْذَرُوا وَمَن يُرِدِ اللَّهُ فِتْنَتَهُ فَلَن تَمْلِكَ لَهُ مِنَ اللَّهِ شَئًْا أُولَئِكَ الَّذِينَ لَمْ يُرِدِ اللَّهُ أَن يُطَهِّرَ قُلُوبَهُمْ لَهُمْ فِى الدُّنْيَا خِزْىٌ وَلَهُمْ فِى الْءَاخِرَةِ عَذَابٌ عَظِيمٌ
{\tiny\colorbox{cl_aya}{42}} سَمَّعُونَ لِلْكَذِبِ أَكَّلُونَ لِلسُّحْتِ فَإِن جَاءُوكَ فَاحْكُم بَيْنَهُمْ أَوْ أَعْرِضْ عَنْهُمْ وَإِن تُعْرِضْ عَنْهُمْ فَلَن يَضُرُّوكَ شَئًْا وَإِنْ حَكَمْتَ فَاحْكُم بَيْنَهُم بِالْقِسْطِ إِنَّ اللَّهَ يُحِبُّ الْمُقْسِطِينَ
{\tiny\colorbox{cl_aya}{43}} وَكَيْفَ يُحَكِّمُونَكَ وَعِندَهُمُ التَّوْرَىةُ فِيهَا حُكْمُ اللَّهِ ثُمَّ يَتَوَلَّوْنَ مِن بَعْدِ ذَلِكَ وَمَا أُولَئِكَ بِالْمُؤْمِنِينَ
{\tiny\colorbox{cl_aya}{44}} إِنَّا أَنزَلْنَا التَّوْرَىةَ فِيهَا هُدًى وَنُورٌ يَحْكُمُ بِهَا النَّبِيُّونَ الَّذِينَ أَسْلَمُوا لِلَّذِينَ هَادُوا وَالرَّبَّنِيُّونَ وَالْأَحْبَارُ بِمَا اسْتُحْفِظُوا مِن كِتَبِ اللَّهِ وَكَانُوا عَلَيْهِ شُهَدَاءَ فَلَا تَخْشَوُا النَّاسَ وَاخْشَوْنِ وَلَا تَشْتَرُوا بَِٔايَتِى ثَمَنًا قَلِيلًا وَمَن لَّمْ يَحْكُم بِمَا أَنزَلَ اللَّهُ فَأُولَئِكَ هُمُ الْكَفِرُونَ
{\tiny\colorbox{cl_aya}{45}} وَكَتَبْنَا عَلَيْهِمْ فِيهَا أَنَّ النَّفْسَ بِالنَّفْسِ وَالْعَيْنَ بِالْعَيْنِ وَالْأَنفَ بِالْأَنفِ وَالْأُذُنَ بِالْأُذُنِ وَالسِّنَّ بِالسِّنِّ وَالْجُرُوحَ قِصَاصٌ فَمَن تَصَدَّقَ بِهِ فَهُوَ كَفَّارَةٌ لَّهُ وَمَن لَّمْ يَحْكُم بِمَا أَنزَلَ اللَّهُ فَأُولَئِكَ هُمُ الظَّلِمُونَ
{\tiny\colorbox{cl_aya}{46}} وَقَفَّيْنَا عَلَى ءَاثَرِهِم بِعِيسَى ابْنِ مَرْيَمَ مُصَدِّقًا لِّمَا بَيْنَ يَدَيْهِ مِنَ التَّوْرَىةِ وَءَاتَيْنَهُ الْإِنجِيلَ فِيهِ هُدًى وَنُورٌ وَمُصَدِّقًا لِّمَا بَيْنَ يَدَيْهِ مِنَ التَّوْرَىةِ وَهُدًى وَمَوْعِظَةً لِّلْمُتَّقِينَ
{\tiny\colorbox{cl_aya}{47}} وَلْيَحْكُمْ أَهْلُ الْإِنجِيلِ بِمَا أَنزَلَ اللَّهُ فِيهِ وَمَن لَّمْ يَحْكُم بِمَا أَنزَلَ اللَّهُ فَأُولَئِكَ هُمُ الْفَسِقُونَ
{\tiny\colorbox{cl_aya}{48}} وَأَنزَلْنَا إِلَيْكَ الْكِتَبَ بِالْحَقِّ مُصَدِّقًا لِّمَا بَيْنَ يَدَيْهِ مِنَ الْكِتَبِ وَمُهَيْمِنًا عَلَيْهِ فَاحْكُم بَيْنَهُم بِمَا أَنزَلَ اللَّهُ وَلَا تَتَّبِعْ أَهْوَاءَهُمْ عَمَّا جَاءَكَ مِنَ الْحَقِّ لِكُلٍّ جَعَلْنَا مِنكُمْ شِرْعَةً وَمِنْهَاجًا وَلَوْ شَاءَ اللَّهُ لَجَعَلَكُمْ أُمَّةً وَحِدَةً وَلَكِن لِّيَبْلُوَكُمْ فِى مَا ءَاتَىكُمْ فَاسْتَبِقُوا الْخَيْرَتِ إِلَى اللَّهِ مَرْجِعُكُمْ جَمِيعًا فَيُنَبِّئُكُم بِمَا كُنتُمْ فِيهِ تَخْتَلِفُونَ
{\tiny\colorbox{cl_aya}{49}} وَأَنِ احْكُم بَيْنَهُم بِمَا أَنزَلَ اللَّهُ وَلَا تَتَّبِعْ أَهْوَاءَهُمْ وَاحْذَرْهُمْ أَن يَفْتِنُوكَ عَن بَعْضِ مَا أَنزَلَ اللَّهُ إِلَيْكَ فَإِن تَوَلَّوْا فَاعْلَمْ أَنَّمَا يُرِيدُ اللَّهُ أَن يُصِيبَهُم بِبَعْضِ ذُنُوبِهِمْ وَإِنَّ كَثِيرًا مِّنَ النَّاسِ لَفَسِقُونَ
{\tiny\colorbox{cl_aya}{50}} أَفَحُكْمَ الْجَهِلِيَّةِ يَبْغُونَ وَمَنْ أَحْسَنُ مِنَ اللَّهِ حُكْمًا لِّقَوْمٍ يُوقِنُونَ
{\tiny\colorbox{cl_aya}{51}} يَأَيُّهَا الَّذِينَ ءَامَنُوا لَا تَتَّخِذُوا الْيَهُودَ وَالنَّصَرَى أَوْلِيَاءَ بَعْضُهُمْ أَوْلِيَاءُ بَعْضٍ وَمَن يَتَوَلَّهُم مِّنكُمْ فَإِنَّهُ مِنْهُمْ إِنَّ اللَّهَ لَا يَهْدِى الْقَوْمَ الظَّلِمِينَ
{\tiny\colorbox{cl_aya}{52}} فَتَرَى الَّذِينَ فِى قُلُوبِهِم مَّرَضٌ يُسَرِعُونَ فِيهِمْ يَقُولُونَ نَخْشَى أَن تُصِيبَنَا دَائِرَةٌ فَعَسَى اللَّهُ أَن يَأْتِىَ بِالْفَتْحِ أَوْ أَمْرٍ مِّنْ عِندِهِ فَيُصْبِحُوا عَلَى مَا أَسَرُّوا فِى أَنفُسِهِمْ نَدِمِينَ
{\tiny\colorbox{cl_aya}{53}} وَيَقُولُ الَّذِينَ ءَامَنُوا أَهَؤُلَاءِ الَّذِينَ أَقْسَمُوا بِاللَّهِ جَهْدَ أَيْمَنِهِمْ إِنَّهُمْ لَمَعَكُمْ حَبِطَتْ أَعْمَلُهُمْ فَأَصْبَحُوا خَسِرِينَ
{\tiny\colorbox{cl_aya}{54}} يَأَيُّهَا الَّذِينَ ءَامَنُوا مَن يَرْتَدَّ مِنكُمْ عَن دِينِهِ فَسَوْفَ يَأْتِى اللَّهُ بِقَوْمٍ يُحِبُّهُمْ وَيُحِبُّونَهُ أَذِلَّةٍ عَلَى الْمُؤْمِنِينَ أَعِزَّةٍ عَلَى الْكَفِرِينَ يُجَهِدُونَ فِى سَبِيلِ اللَّهِ وَلَا يَخَافُونَ لَوْمَةَ لَائِمٍ ذَلِكَ فَضْلُ اللَّهِ يُؤْتِيهِ مَن يَشَاءُ وَاللَّهُ وَسِعٌ عَلِيمٌ
{\tiny\colorbox{cl_aya}{55}} إِنَّمَا وَلِيُّكُمُ اللَّهُ وَرَسُولُهُ وَالَّذِينَ ءَامَنُوا الَّذِينَ يُقِيمُونَ الصَّلَوةَ وَيُؤْتُونَ الزَّكَوةَ وَهُمْ رَكِعُونَ
{\tiny\colorbox{cl_aya}{56}} وَمَن يَتَوَلَّ اللَّهَ وَرَسُولَهُ وَالَّذِينَ ءَامَنُوا فَإِنَّ حِزْبَ اللَّهِ هُمُ الْغَلِبُونَ
{\tiny\colorbox{cl_aya}{57}} يَأَيُّهَا الَّذِينَ ءَامَنُوا لَا تَتَّخِذُوا الَّذِينَ اتَّخَذُوا دِينَكُمْ هُزُوًا وَلَعِبًا مِّنَ الَّذِينَ أُوتُوا الْكِتَبَ مِن قَبْلِكُمْ وَالْكُفَّارَ أَوْلِيَاءَ وَاتَّقُوا اللَّهَ إِن كُنتُم مُّؤْمِنِينَ
{\tiny\colorbox{cl_aya}{58}} وَإِذَا نَادَيْتُمْ إِلَى الصَّلَوةِ اتَّخَذُوهَا هُزُوًا وَلَعِبًا ذَلِكَ بِأَنَّهُمْ قَوْمٌ لَّا يَعْقِلُونَ
{\tiny\colorbox{cl_aya}{59}} قُلْ يَأَهْلَ الْكِتَبِ هَلْ تَنقِمُونَ مِنَّا إِلَّا أَنْ ءَامَنَّا بِاللَّهِ وَمَا أُنزِلَ إِلَيْنَا وَمَا أُنزِلَ مِن قَبْلُ وَأَنَّ أَكْثَرَكُمْ فَسِقُونَ
{\tiny\colorbox{cl_aya}{60}} قُلْ هَلْ أُنَبِّئُكُم بِشَرٍّ مِّن ذَلِكَ مَثُوبَةً عِندَ اللَّهِ مَن لَّعَنَهُ اللَّهُ وَغَضِبَ عَلَيْهِ وَجَعَلَ مِنْهُمُ الْقِرَدَةَ وَالْخَنَازِيرَ وَعَبَدَ الطَّغُوتَ أُولَئِكَ شَرٌّ مَّكَانًا وَأَضَلُّ عَن سَوَاءِ السَّبِيلِ
{\tiny\colorbox{cl_aya}{61}} وَإِذَا جَاءُوكُمْ قَالُوا ءَامَنَّا وَقَد دَّخَلُوا بِالْكُفْرِ وَهُمْ قَدْ خَرَجُوا بِهِ وَاللَّهُ أَعْلَمُ بِمَا كَانُوا يَكْتُمُونَ
{\tiny\colorbox{cl_aya}{62}} وَتَرَى كَثِيرًا مِّنْهُمْ يُسَرِعُونَ فِى الْإِثْمِ وَالْعُدْوَنِ وَأَكْلِهِمُ السُّحْتَ لَبِئْسَ مَا كَانُوا يَعْمَلُونَ
{\tiny\colorbox{cl_aya}{63}} لَوْلَا يَنْهَىهُمُ الرَّبَّنِيُّونَ وَالْأَحْبَارُ عَن قَوْلِهِمُ الْإِثْمَ وَأَكْلِهِمُ السُّحْتَ لَبِئْسَ مَا كَانُوا يَصْنَعُونَ
{\tiny\colorbox{cl_aya}{64}} وَقَالَتِ الْيَهُودُ يَدُ اللَّهِ مَغْلُولَةٌ غُلَّتْ أَيْدِيهِمْ وَلُعِنُوا بِمَا قَالُوا بَلْ يَدَاهُ مَبْسُوطَتَانِ يُنفِقُ كَيْفَ يَشَاءُ وَلَيَزِيدَنَّ كَثِيرًا مِّنْهُم مَّا أُنزِلَ إِلَيْكَ مِن رَّبِّكَ طُغْيَنًا وَكُفْرًا وَأَلْقَيْنَا بَيْنَهُمُ الْعَدَوَةَ وَالْبَغْضَاءَ إِلَى يَوْمِ الْقِيَمَةِ كُلَّمَا أَوْقَدُوا نَارًا لِّلْحَرْبِ أَطْفَأَهَا اللَّهُ وَيَسْعَوْنَ فِى الْأَرْضِ فَسَادًا وَاللَّهُ لَا يُحِبُّ الْمُفْسِدِينَ
{\tiny\colorbox{cl_aya}{65}} وَلَوْ أَنَّ أَهْلَ الْكِتَبِ ءَامَنُوا وَاتَّقَوْا لَكَفَّرْنَا عَنْهُمْ سَئَِّاتِهِمْ وَلَأَدْخَلْنَهُمْ جَنَّتِ النَّعِيمِ
{\tiny\colorbox{cl_aya}{66}} وَلَوْ أَنَّهُمْ أَقَامُوا التَّوْرَىةَ وَالْإِنجِيلَ وَمَا أُنزِلَ إِلَيْهِم مِّن رَّبِّهِمْ لَأَكَلُوا مِن فَوْقِهِمْ وَمِن تَحْتِ أَرْجُلِهِم مِّنْهُمْ أُمَّةٌ مُّقْتَصِدَةٌ وَكَثِيرٌ مِّنْهُمْ سَاءَ مَا يَعْمَلُونَ
{\tiny\colorbox{cl_aya}{67}} يَأَيُّهَا الرَّسُولُ بَلِّغْ مَا أُنزِلَ إِلَيْكَ مِن رَّبِّكَ وَإِن لَّمْ تَفْعَلْ فَمَا بَلَّغْتَ رِسَالَتَهُ وَاللَّهُ يَعْصِمُكَ مِنَ النَّاسِ إِنَّ اللَّهَ لَا يَهْدِى الْقَوْمَ الْكَفِرِينَ
{\tiny\colorbox{cl_aya}{68}} قُلْ يَأَهْلَ الْكِتَبِ لَسْتُمْ عَلَى شَىْءٍ حَتَّى تُقِيمُوا التَّوْرَىةَ وَالْإِنجِيلَ وَمَا أُنزِلَ إِلَيْكُم مِّن رَّبِّكُمْ وَلَيَزِيدَنَّ كَثِيرًا مِّنْهُم مَّا أُنزِلَ إِلَيْكَ مِن رَّبِّكَ طُغْيَنًا وَكُفْرًا فَلَا تَأْسَ عَلَى الْقَوْمِ الْكَفِرِينَ
{\tiny\colorbox{cl_aya}{69}} إِنَّ الَّذِينَ ءَامَنُوا وَالَّذِينَ هَادُوا وَالصَّبُِٔونَ وَالنَّصَرَى مَنْ ءَامَنَ بِاللَّهِ وَالْيَوْمِ الْءَاخِرِ وَعَمِلَ صَلِحًا فَلَا خَوْفٌ عَلَيْهِمْ وَلَا هُمْ يَحْزَنُونَ
{\tiny\colorbox{cl_aya}{70}} لَقَدْ أَخَذْنَا مِيثَقَ بَنِى إِسْرَءِيلَ وَأَرْسَلْنَا إِلَيْهِمْ رُسُلًا كُلَّمَا جَاءَهُمْ رَسُولٌ بِمَا لَا تَهْوَى أَنفُسُهُمْ فَرِيقًا كَذَّبُوا وَفَرِيقًا يَقْتُلُونَ
{\tiny\colorbox{cl_aya}{71}} وَحَسِبُوا أَلَّا تَكُونَ فِتْنَةٌ فَعَمُوا وَصَمُّوا ثُمَّ تَابَ اللَّهُ عَلَيْهِمْ ثُمَّ عَمُوا وَصَمُّوا كَثِيرٌ مِّنْهُمْ وَاللَّهُ بَصِيرٌ بِمَا يَعْمَلُونَ
{\tiny\colorbox{cl_aya}{72}} لَقَدْ كَفَرَ الَّذِينَ قَالُوا إِنَّ اللَّهَ هُوَ الْمَسِيحُ ابْنُ مَرْيَمَ وَقَالَ الْمَسِيحُ يَبَنِى إِسْرَءِيلَ اعْبُدُوا اللَّهَ رَبِّى وَرَبَّكُمْ إِنَّهُ مَن يُشْرِكْ بِاللَّهِ فَقَدْ حَرَّمَ اللَّهُ عَلَيْهِ الْجَنَّةَ وَمَأْوَىهُ النَّارُ وَمَا لِلظَّلِمِينَ مِنْ أَنصَارٍ
{\tiny\colorbox{cl_aya}{73}} لَّقَدْ كَفَرَ الَّذِينَ قَالُوا إِنَّ اللَّهَ ثَالِثُ ثَلَثَةٍ وَمَا مِنْ إِلَهٍ إِلَّا إِلَهٌ وَحِدٌ وَإِن لَّمْ يَنتَهُوا عَمَّا يَقُولُونَ لَيَمَسَّنَّ الَّذِينَ كَفَرُوا مِنْهُمْ عَذَابٌ أَلِيمٌ
{\tiny\colorbox{cl_aya}{74}} أَفَلَا يَتُوبُونَ إِلَى اللَّهِ وَيَسْتَغْفِرُونَهُ وَاللَّهُ غَفُورٌ رَّحِيمٌ
{\tiny\colorbox{cl_aya}{75}} مَّا الْمَسِيحُ ابْنُ مَرْيَمَ إِلَّا رَسُولٌ قَدْ خَلَتْ مِن قَبْلِهِ الرُّسُلُ وَأُمُّهُ صِدِّيقَةٌ كَانَا يَأْكُلَانِ الطَّعَامَ انظُرْ كَيْفَ نُبَيِّنُ لَهُمُ الْءَايَتِ ثُمَّ انظُرْ أَنَّى يُؤْفَكُونَ
{\tiny\colorbox{cl_aya}{76}} قُلْ أَتَعْبُدُونَ مِن دُونِ اللَّهِ مَا لَا يَمْلِكُ لَكُمْ ضَرًّا وَلَا نَفْعًا وَاللَّهُ هُوَ السَّمِيعُ الْعَلِيمُ
{\tiny\colorbox{cl_aya}{77}} قُلْ يَأَهْلَ الْكِتَبِ لَا تَغْلُوا فِى دِينِكُمْ غَيْرَ الْحَقِّ وَلَا تَتَّبِعُوا أَهْوَاءَ قَوْمٍ قَدْ ضَلُّوا مِن قَبْلُ وَأَضَلُّوا كَثِيرًا وَضَلُّوا عَن سَوَاءِ السَّبِيلِ
{\tiny\colorbox{cl_aya}{78}} لُعِنَ الَّذِينَ كَفَرُوا مِن بَنِى إِسْرَءِيلَ عَلَى لِسَانِ دَاوُدَ وَعِيسَى ابْنِ مَرْيَمَ ذَلِكَ بِمَا عَصَوا وَّكَانُوا يَعْتَدُونَ
{\tiny\colorbox{cl_aya}{79}} كَانُوا لَا يَتَنَاهَوْنَ عَن مُّنكَرٍ فَعَلُوهُ لَبِئْسَ مَا كَانُوا يَفْعَلُونَ
{\tiny\colorbox{cl_aya}{80}} تَرَى كَثِيرًا مِّنْهُمْ يَتَوَلَّوْنَ الَّذِينَ كَفَرُوا لَبِئْسَ مَا قَدَّمَتْ لَهُمْ أَنفُسُهُمْ أَن سَخِطَ اللَّهُ عَلَيْهِمْ وَفِى الْعَذَابِ هُمْ خَلِدُونَ
{\tiny\colorbox{cl_aya}{81}} وَلَوْ كَانُوا يُؤْمِنُونَ بِاللَّهِ وَالنَّبِىِّ وَمَا أُنزِلَ إِلَيْهِ مَا اتَّخَذُوهُمْ أَوْلِيَاءَ وَلَكِنَّ كَثِيرًا مِّنْهُمْ فَسِقُونَ
{\tiny\colorbox{cl_aya}{82}} لَتَجِدَنَّ أَشَدَّ النَّاسِ عَدَوَةً لِّلَّذِينَ ءَامَنُوا الْيَهُودَ وَالَّذِينَ أَشْرَكُوا وَلَتَجِدَنَّ أَقْرَبَهُم مَّوَدَّةً لِّلَّذِينَ ءَامَنُوا الَّذِينَ قَالُوا إِنَّا نَصَرَى ذَلِكَ بِأَنَّ مِنْهُمْ قِسِّيسِينَ وَرُهْبَانًا وَأَنَّهُمْ لَا يَسْتَكْبِرُونَ
{\tiny\colorbox{cl_aya}{83}} وَإِذَا سَمِعُوا مَا أُنزِلَ إِلَى الرَّسُولِ تَرَى أَعْيُنَهُمْ تَفِيضُ مِنَ الدَّمْعِ مِمَّا عَرَفُوا مِنَ الْحَقِّ يَقُولُونَ رَبَّنَا ءَامَنَّا فَاكْتُبْنَا مَعَ الشَّهِدِينَ
{\tiny\colorbox{cl_aya}{84}} وَمَا لَنَا لَا نُؤْمِنُ بِاللَّهِ وَمَا جَاءَنَا مِنَ الْحَقِّ وَنَطْمَعُ أَن يُدْخِلَنَا رَبُّنَا مَعَ الْقَوْمِ الصَّلِحِينَ
{\tiny\colorbox{cl_aya}{85}} فَأَثَبَهُمُ اللَّهُ بِمَا قَالُوا جَنَّتٍ تَجْرِى مِن تَحْتِهَا الْأَنْهَرُ خَلِدِينَ فِيهَا وَذَلِكَ جَزَاءُ الْمُحْسِنِينَ
{\tiny\colorbox{cl_aya}{86}} وَالَّذِينَ كَفَرُوا وَكَذَّبُوا بَِٔايَتِنَا أُولَئِكَ أَصْحَبُ الْجَحِيمِ
{\tiny\colorbox{cl_aya}{87}} يَأَيُّهَا الَّذِينَ ءَامَنُوا لَا تُحَرِّمُوا طَيِّبَتِ مَا أَحَلَّ اللَّهُ لَكُمْ وَلَا تَعْتَدُوا إِنَّ اللَّهَ لَا يُحِبُّ الْمُعْتَدِينَ
{\tiny\colorbox{cl_aya}{88}} وَكُلُوا مِمَّا رَزَقَكُمُ اللَّهُ حَلَلًا طَيِّبًا وَاتَّقُوا اللَّهَ الَّذِى أَنتُم بِهِ مُؤْمِنُونَ
{\tiny\colorbox{cl_aya}{89}} لَا يُؤَاخِذُكُمُ اللَّهُ بِاللَّغْوِ فِى أَيْمَنِكُمْ وَلَكِن يُؤَاخِذُكُم بِمَا عَقَّدتُّمُ الْأَيْمَنَ فَكَفَّرَتُهُ إِطْعَامُ عَشَرَةِ مَسَكِينَ مِنْ أَوْسَطِ مَا تُطْعِمُونَ أَهْلِيكُمْ أَوْ كِسْوَتُهُمْ أَوْ تَحْرِيرُ رَقَبَةٍ فَمَن لَّمْ يَجِدْ فَصِيَامُ ثَلَثَةِ أَيَّامٍ ذَلِكَ كَفَّرَةُ أَيْمَنِكُمْ إِذَا حَلَفْتُمْ وَاحْفَظُوا أَيْمَنَكُمْ كَذَلِكَ يُبَيِّنُ اللَّهُ لَكُمْ ءَايَتِهِ لَعَلَّكُمْ تَشْكُرُونَ
{\tiny\colorbox{cl_aya}{90}} يَأَيُّهَا الَّذِينَ ءَامَنُوا إِنَّمَا الْخَمْرُ وَالْمَيْسِرُ وَالْأَنصَابُ وَالْأَزْلَمُ رِجْسٌ مِّنْ عَمَلِ الشَّيْطَنِ فَاجْتَنِبُوهُ لَعَلَّكُمْ تُفْلِحُونَ
{\tiny\colorbox{cl_aya}{91}} إِنَّمَا يُرِيدُ الشَّيْطَنُ أَن يُوقِعَ بَيْنَكُمُ الْعَدَوَةَ وَالْبَغْضَاءَ فِى الْخَمْرِ وَالْمَيْسِرِ وَيَصُدَّكُمْ عَن ذِكْرِ اللَّهِ وَعَنِ الصَّلَوةِ فَهَلْ أَنتُم مُّنتَهُونَ
{\tiny\colorbox{cl_aya}{92}} وَأَطِيعُوا اللَّهَ وَأَطِيعُوا الرَّسُولَ وَاحْذَرُوا فَإِن تَوَلَّيْتُمْ فَاعْلَمُوا أَنَّمَا عَلَى رَسُولِنَا الْبَلَغُ الْمُبِينُ
{\tiny\colorbox{cl_aya}{93}} لَيْسَ عَلَى الَّذِينَ ءَامَنُوا وَعَمِلُوا الصَّلِحَتِ جُنَاحٌ فِيمَا طَعِمُوا إِذَا مَا اتَّقَوا وَّءَامَنُوا وَعَمِلُوا الصَّلِحَتِ ثُمَّ اتَّقَوا وَّءَامَنُوا ثُمَّ اتَّقَوا وَّأَحْسَنُوا وَاللَّهُ يُحِبُّ الْمُحْسِنِينَ
{\tiny\colorbox{cl_aya}{94}} يَأَيُّهَا الَّذِينَ ءَامَنُوا لَيَبْلُوَنَّكُمُ اللَّهُ بِشَىْءٍ مِّنَ الصَّيْدِ تَنَالُهُ أَيْدِيكُمْ وَرِمَاحُكُمْ لِيَعْلَمَ اللَّهُ مَن يَخَافُهُ بِالْغَيْبِ فَمَنِ اعْتَدَى بَعْدَ ذَلِكَ فَلَهُ عَذَابٌ أَلِيمٌ
{\tiny\colorbox{cl_aya}{95}} يَأَيُّهَا الَّذِينَ ءَامَنُوا لَا تَقْتُلُوا الصَّيْدَ وَأَنتُمْ حُرُمٌ وَمَن قَتَلَهُ مِنكُم مُّتَعَمِّدًا فَجَزَاءٌ مِّثْلُ مَا قَتَلَ مِنَ النَّعَمِ يَحْكُمُ بِهِ ذَوَا عَدْلٍ مِّنكُمْ هَدْيًا بَلِغَ الْكَعْبَةِ أَوْ كَفَّرَةٌ طَعَامُ مَسَكِينَ أَوْ عَدْلُ ذَلِكَ صِيَامًا لِّيَذُوقَ وَبَالَ أَمْرِهِ عَفَا اللَّهُ عَمَّا سَلَفَ وَمَنْ عَادَ فَيَنتَقِمُ اللَّهُ مِنْهُ وَاللَّهُ عَزِيزٌ ذُو انتِقَامٍ
{\tiny\colorbox{cl_aya}{96}} أُحِلَّ لَكُمْ صَيْدُ الْبَحْرِ وَطَعَامُهُ مَتَعًا لَّكُمْ وَلِلسَّيَّارَةِ وَحُرِّمَ عَلَيْكُمْ صَيْدُ الْبَرِّ مَا دُمْتُمْ حُرُمًا وَاتَّقُوا اللَّهَ الَّذِى إِلَيْهِ تُحْشَرُونَ
{\tiny\colorbox{cl_aya}{97}} جَعَلَ اللَّهُ الْكَعْبَةَ الْبَيْتَ الْحَرَامَ قِيَمًا لِّلنَّاسِ وَالشَّهْرَ الْحَرَامَ وَالْهَدْىَ وَالْقَلَئِدَ ذَلِكَ لِتَعْلَمُوا أَنَّ اللَّهَ يَعْلَمُ مَا فِى السَّمَوَتِ وَمَا فِى الْأَرْضِ وَأَنَّ اللَّهَ بِكُلِّ شَىْءٍ عَلِيمٌ
{\tiny\colorbox{cl_aya}{98}} اعْلَمُوا أَنَّ اللَّهَ شَدِيدُ الْعِقَابِ وَأَنَّ اللَّهَ غَفُورٌ رَّحِيمٌ
{\tiny\colorbox{cl_aya}{99}} مَّا عَلَى الرَّسُولِ إِلَّا الْبَلَغُ وَاللَّهُ يَعْلَمُ مَا تُبْدُونَ وَمَا تَكْتُمُونَ
{\tiny\colorbox{cl_aya}{100}} قُل لَّا يَسْتَوِى الْخَبِيثُ وَالطَّيِّبُ وَلَوْ أَعْجَبَكَ كَثْرَةُ الْخَبِيثِ فَاتَّقُوا اللَّهَ يَأُولِى الْأَلْبَبِ لَعَلَّكُمْ تُفْلِحُونَ
{\tiny\colorbox{cl_aya}{101}} يَأَيُّهَا الَّذِينَ ءَامَنُوا لَا تَسَْٔلُوا عَنْ أَشْيَاءَ إِن تُبْدَ لَكُمْ تَسُؤْكُمْ وَإِن تَسَْٔلُوا عَنْهَا حِينَ يُنَزَّلُ الْقُرْءَانُ تُبْدَ لَكُمْ عَفَا اللَّهُ عَنْهَا وَاللَّهُ غَفُورٌ حَلِيمٌ
{\tiny\colorbox{cl_aya}{102}} قَدْ سَأَلَهَا قَوْمٌ مِّن قَبْلِكُمْ ثُمَّ أَصْبَحُوا بِهَا كَفِرِينَ
{\tiny\colorbox{cl_aya}{103}} مَا جَعَلَ اللَّهُ مِن بَحِيرَةٍ وَلَا سَائِبَةٍ وَلَا وَصِيلَةٍ وَلَا حَامٍ وَلَكِنَّ الَّذِينَ كَفَرُوا يَفْتَرُونَ عَلَى اللَّهِ الْكَذِبَ وَأَكْثَرُهُمْ لَا يَعْقِلُونَ
{\tiny\colorbox{cl_aya}{104}} وَإِذَا قِيلَ لَهُمْ تَعَالَوْا إِلَى مَا أَنزَلَ اللَّهُ وَإِلَى الرَّسُولِ قَالُوا حَسْبُنَا مَا وَجَدْنَا عَلَيْهِ ءَابَاءَنَا أَوَلَوْ كَانَ ءَابَاؤُهُمْ لَا يَعْلَمُونَ شَئًْا وَلَا يَهْتَدُونَ
{\tiny\colorbox{cl_aya}{105}} يَأَيُّهَا الَّذِينَ ءَامَنُوا عَلَيْكُمْ أَنفُسَكُمْ لَا يَضُرُّكُم مَّن ضَلَّ إِذَا اهْتَدَيْتُمْ إِلَى اللَّهِ مَرْجِعُكُمْ جَمِيعًا فَيُنَبِّئُكُم بِمَا كُنتُمْ تَعْمَلُونَ
{\tiny\colorbox{cl_aya}{106}} يَأَيُّهَا الَّذِينَ ءَامَنُوا شَهَدَةُ بَيْنِكُمْ إِذَا حَضَرَ أَحَدَكُمُ الْمَوْتُ حِينَ الْوَصِيَّةِ اثْنَانِ ذَوَا عَدْلٍ مِّنكُمْ أَوْ ءَاخَرَانِ مِنْ غَيْرِكُمْ إِنْ أَنتُمْ ضَرَبْتُمْ فِى الْأَرْضِ فَأَصَبَتْكُم مُّصِيبَةُ الْمَوْتِ تَحْبِسُونَهُمَا مِن بَعْدِ الصَّلَوةِ فَيُقْسِمَانِ بِاللَّهِ إِنِ ارْتَبْتُمْ لَا نَشْتَرِى بِهِ ثَمَنًا وَلَوْ كَانَ ذَا قُرْبَى وَلَا نَكْتُمُ شَهَدَةَ اللَّهِ إِنَّا إِذًا لَّمِنَ الْءَاثِمِينَ
{\tiny\colorbox{cl_aya}{107}} فَإِنْ عُثِرَ عَلَى أَنَّهُمَا اسْتَحَقَّا إِثْمًا فََٔاخَرَانِ يَقُومَانِ مَقَامَهُمَا مِنَ الَّذِينَ اسْتَحَقَّ عَلَيْهِمُ الْأَوْلَيَنِ فَيُقْسِمَانِ بِاللَّهِ لَشَهَدَتُنَا أَحَقُّ مِن شَهَدَتِهِمَا وَمَا اعْتَدَيْنَا إِنَّا إِذًا لَّمِنَ الظَّلِمِينَ
{\tiny\colorbox{cl_aya}{108}} ذَلِكَ أَدْنَى أَن يَأْتُوا بِالشَّهَدَةِ عَلَى وَجْهِهَا أَوْ يَخَافُوا أَن تُرَدَّ أَيْمَنٌ بَعْدَ أَيْمَنِهِمْ وَاتَّقُوا اللَّهَ وَاسْمَعُوا وَاللَّهُ لَا يَهْدِى الْقَوْمَ الْفَسِقِينَ
{\tiny\colorbox{cl_aya}{109}} يَوْمَ يَجْمَعُ اللَّهُ الرُّسُلَ فَيَقُولُ مَاذَا أُجِبْتُمْ قَالُوا لَا عِلْمَ لَنَا إِنَّكَ أَنتَ عَلَّمُ الْغُيُوبِ
{\tiny\colorbox{cl_aya}{110}} إِذْ قَالَ اللَّهُ يَعِيسَى ابْنَ مَرْيَمَ اذْكُرْ نِعْمَتِى عَلَيْكَ وَعَلَى وَلِدَتِكَ إِذْ أَيَّدتُّكَ بِرُوحِ الْقُدُسِ تُكَلِّمُ النَّاسَ فِى الْمَهْدِ وَكَهْلًا وَإِذْ عَلَّمْتُكَ الْكِتَبَ وَالْحِكْمَةَ وَالتَّوْرَىةَ وَالْإِنجِيلَ وَإِذْ تَخْلُقُ مِنَ الطِّينِ كَهَئَْةِ الطَّيْرِ بِإِذْنِى فَتَنفُخُ فِيهَا فَتَكُونُ طَيْرًا بِإِذْنِى وَتُبْرِئُ الْأَكْمَهَ وَالْأَبْرَصَ بِإِذْنِى وَإِذْ تُخْرِجُ الْمَوْتَى بِإِذْنِى وَإِذْ كَفَفْتُ بَنِى إِسْرَءِيلَ عَنكَ إِذْ جِئْتَهُم بِالْبَيِّنَتِ فَقَالَ الَّذِينَ كَفَرُوا مِنْهُمْ إِنْ هَذَا إِلَّا سِحْرٌ مُّبِينٌ
{\tiny\colorbox{cl_aya}{111}} وَإِذْ أَوْحَيْتُ إِلَى الْحَوَارِيِّنَ أَنْ ءَامِنُوا بِى وَبِرَسُولِى قَالُوا ءَامَنَّا وَاشْهَدْ بِأَنَّنَا مُسْلِمُونَ
{\tiny\colorbox{cl_aya}{112}} إِذْ قَالَ الْحَوَارِيُّونَ يَعِيسَى ابْنَ مَرْيَمَ هَلْ يَسْتَطِيعُ رَبُّكَ أَن يُنَزِّلَ عَلَيْنَا مَائِدَةً مِّنَ السَّمَاءِ قَالَ اتَّقُوا اللَّهَ إِن كُنتُم مُّؤْمِنِينَ
{\tiny\colorbox{cl_aya}{113}} قَالُوا نُرِيدُ أَن نَّأْكُلَ مِنْهَا وَتَطْمَئِنَّ قُلُوبُنَا وَنَعْلَمَ أَن قَدْ صَدَقْتَنَا وَنَكُونَ عَلَيْهَا مِنَ الشَّهِدِينَ
{\tiny\colorbox{cl_aya}{114}} قَالَ عِيسَى ابْنُ مَرْيَمَ اللَّهُمَّ رَبَّنَا أَنزِلْ عَلَيْنَا مَائِدَةً مِّنَ السَّمَاءِ تَكُونُ لَنَا عِيدًا لِّأَوَّلِنَا وَءَاخِرِنَا وَءَايَةً مِّنكَ وَارْزُقْنَا وَأَنتَ خَيْرُ الرَّزِقِينَ
{\tiny\colorbox{cl_aya}{115}} قَالَ اللَّهُ إِنِّى مُنَزِّلُهَا عَلَيْكُمْ فَمَن يَكْفُرْ بَعْدُ مِنكُمْ فَإِنِّى أُعَذِّبُهُ عَذَابًا لَّا أُعَذِّبُهُ أَحَدًا مِّنَ الْعَلَمِينَ
{\tiny\colorbox{cl_aya}{116}} وَإِذْ قَالَ اللَّهُ يَعِيسَى ابْنَ مَرْيَمَ ءَأَنتَ قُلْتَ لِلنَّاسِ اتَّخِذُونِى وَأُمِّىَ إِلَهَيْنِ مِن دُونِ اللَّهِ قَالَ سُبْحَنَكَ مَا يَكُونُ لِى أَنْ أَقُولَ مَا لَيْسَ لِى بِحَقٍّ إِن كُنتُ قُلْتُهُ فَقَدْ عَلِمْتَهُ تَعْلَمُ مَا فِى نَفْسِى وَلَا أَعْلَمُ مَا فِى نَفْسِكَ إِنَّكَ أَنتَ عَلَّمُ الْغُيُوبِ
{\tiny\colorbox{cl_aya}{117}} مَا قُلْتُ لَهُمْ إِلَّا مَا أَمَرْتَنِى بِهِ أَنِ اعْبُدُوا اللَّهَ رَبِّى وَرَبَّكُمْ وَكُنتُ عَلَيْهِمْ شَهِيدًا مَّا دُمْتُ فِيهِمْ فَلَمَّا تَوَفَّيْتَنِى كُنتَ أَنتَ الرَّقِيبَ عَلَيْهِمْ وَأَنتَ عَلَى كُلِّ شَىْءٍ شَهِيدٌ
{\tiny\colorbox{cl_aya}{118}} إِن تُعَذِّبْهُمْ فَإِنَّهُمْ عِبَادُكَ وَإِن تَغْفِرْ لَهُمْ فَإِنَّكَ أَنتَ الْعَزِيزُ الْحَكِيمُ
{\tiny\colorbox{cl_aya}{119}} قَالَ اللَّهُ هَذَا يَوْمُ يَنفَعُ الصَّدِقِينَ صِدْقُهُمْ لَهُمْ جَنَّتٌ تَجْرِى مِن تَحْتِهَا الْأَنْهَرُ خَلِدِينَ فِيهَا أَبَدًا رَّضِىَ اللَّهُ عَنْهُمْ وَرَضُوا عَنْهُ ذَلِكَ الْفَوْزُ الْعَظِيمُ
{\tiny\colorbox{cl_aya}{120}} لِلَّهِ مُلْكُ السَّمَوَتِ وَالْأَرْضِ وَمَا فِيهِنَّ وَهُوَ عَلَى كُلِّ شَىْءٍ قَدِيرٌ
