%\documentclass[12pt,a4paper]{article}
\documentclass[20pt,a4paper]{article}
\usepackage[margin=0.5in]{geometry}
\usepackage{polyglossia}
\usepackage[dvipsnames]{xcolor}
\pagenumbering{gobble}
% This beautiful one line disable the initial spacing at the beginning of a line
\usepackage[parfill]{parskip} 
\usepackage{setspace}
\setstretch{2}

\setdefaultlanguage[numerals=maghrib]{arabic}
\newfontfamily\arabicfont[Script=Arabic]{Amiri}

\title{}
\author{}
\date{}
\definecolor{cl_page}{gray}{0.98}
\definecolor{cl_aya}{HTML}{DEEEFF}

\begin{document}
\pagecolor{cl_page}

% Start %


{\tiny\colorbox{cl_aya}{1}} الْحَمْدُ لِلَّهِ الَّذِى أَنزَلَ عَلَى عَبْدِهِ الْكِتَبَ وَلَمْ يَجْعَل لَّهُ عِوَجَا
{\tiny\colorbox{cl_aya}{2}} قَيِّمًا لِّيُنذِرَ بَأْسًا شَدِيدًا مِّن لَّدُنْهُ وَيُبَشِّرَ الْمُؤْمِنِينَ الَّذِينَ يَعْمَلُونَ الصَّلِحَتِ أَنَّ لَهُمْ أَجْرًا حَسَنًا
{\tiny\colorbox{cl_aya}{3}} مَّكِثِينَ فِيهِ أَبَدًا
{\tiny\colorbox{cl_aya}{4}} وَيُنذِرَ الَّذِينَ قَالُوا اتَّخَذَ اللَّهُ وَلَدًا
{\tiny\colorbox{cl_aya}{5}} مَّا لَهُم بِهِ مِنْ عِلْمٍ وَلَا لِءَابَائِهِمْ كَبُرَتْ كَلِمَةً تَخْرُجُ مِنْ أَفْوَهِهِمْ إِن يَقُولُونَ إِلَّا كَذِبًا
{\tiny\colorbox{cl_aya}{6}} فَلَعَلَّكَ بَخِعٌ نَّفْسَكَ عَلَى ءَاثَرِهِمْ إِن لَّمْ يُؤْمِنُوا بِهَذَا الْحَدِيثِ أَسَفًا
{\tiny\colorbox{cl_aya}{7}} إِنَّا جَعَلْنَا مَا عَلَى الْأَرْضِ زِينَةً لَّهَا لِنَبْلُوَهُمْ أَيُّهُمْ أَحْسَنُ عَمَلًا
{\tiny\colorbox{cl_aya}{8}} وَإِنَّا لَجَعِلُونَ مَا عَلَيْهَا صَعِيدًا جُرُزًا
{\tiny\colorbox{cl_aya}{9}} أَمْ حَسِبْتَ أَنَّ أَصْحَبَ الْكَهْفِ وَالرَّقِيمِ كَانُوا مِنْ ءَايَتِنَا عَجَبًا
{\tiny\colorbox{cl_aya}{10}} إِذْ أَوَى الْفِتْيَةُ إِلَى الْكَهْفِ فَقَالُوا رَبَّنَا ءَاتِنَا مِن لَّدُنكَ رَحْمَةً وَهَيِّئْ لَنَا مِنْ أَمْرِنَا رَشَدًا
{\tiny\colorbox{cl_aya}{11}} فَضَرَبْنَا عَلَى ءَاذَانِهِمْ فِى الْكَهْفِ سِنِينَ عَدَدًا
{\tiny\colorbox{cl_aya}{12}} ثُمَّ بَعَثْنَهُمْ لِنَعْلَمَ أَىُّ الْحِزْبَيْنِ أَحْصَى لِمَا لَبِثُوا أَمَدًا
{\tiny\colorbox{cl_aya}{13}} نَّحْنُ نَقُصُّ عَلَيْكَ نَبَأَهُم بِالْحَقِّ إِنَّهُمْ فِتْيَةٌ ءَامَنُوا بِرَبِّهِمْ وَزِدْنَهُمْ هُدًى
{\tiny\colorbox{cl_aya}{14}} وَرَبَطْنَا عَلَى قُلُوبِهِمْ إِذْ قَامُوا فَقَالُوا رَبُّنَا رَبُّ السَّمَوَتِ وَالْأَرْضِ لَن نَّدْعُوَا مِن دُونِهِ إِلَهًا لَّقَدْ قُلْنَا إِذًا شَطَطًا
{\tiny\colorbox{cl_aya}{15}} هَؤُلَاءِ قَوْمُنَا اتَّخَذُوا مِن دُونِهِ ءَالِهَةً لَّوْلَا يَأْتُونَ عَلَيْهِم بِسُلْطَنٍ بَيِّنٍ فَمَنْ أَظْلَمُ مِمَّنِ افْتَرَى عَلَى اللَّهِ كَذِبًا
{\tiny\colorbox{cl_aya}{16}} وَإِذِ اعْتَزَلْتُمُوهُمْ وَمَا يَعْبُدُونَ إِلَّا اللَّهَ فَأْوُا إِلَى الْكَهْفِ يَنشُرْ لَكُمْ رَبُّكُم مِّن رَّحْمَتِهِ وَيُهَيِّئْ لَكُم مِّنْ أَمْرِكُم مِّرْفَقًا
{\tiny\colorbox{cl_aya}{17}} وَتَرَى الشَّمْسَ إِذَا طَلَعَت تَّزَوَرُ عَن كَهْفِهِمْ ذَاتَ الْيَمِينِ وَإِذَا غَرَبَت تَّقْرِضُهُمْ ذَاتَ الشِّمَالِ وَهُمْ فِى فَجْوَةٍ مِّنْهُ ذَلِكَ مِنْ ءَايَتِ اللَّهِ مَن يَهْدِ اللَّهُ فَهُوَ الْمُهْتَدِ وَمَن يُضْلِلْ فَلَن تَجِدَ لَهُ وَلِيًّا مُّرْشِدًا
{\tiny\colorbox{cl_aya}{18}} وَتَحْسَبُهُمْ أَيْقَاظًا وَهُمْ رُقُودٌ وَنُقَلِّبُهُمْ ذَاتَ الْيَمِينِ وَذَاتَ الشِّمَالِ وَكَلْبُهُم بَسِطٌ ذِرَاعَيْهِ بِالْوَصِيدِ لَوِ اطَّلَعْتَ عَلَيْهِمْ لَوَلَّيْتَ مِنْهُمْ فِرَارًا وَلَمُلِئْتَ مِنْهُمْ رُعْبًا
{\tiny\colorbox{cl_aya}{19}} وَكَذَلِكَ بَعَثْنَهُمْ لِيَتَسَاءَلُوا بَيْنَهُمْ قَالَ قَائِلٌ مِّنْهُمْ كَمْ لَبِثْتُمْ قَالُوا لَبِثْنَا يَوْمًا أَوْ بَعْضَ يَوْمٍ قَالُوا رَبُّكُمْ أَعْلَمُ بِمَا لَبِثْتُمْ فَابْعَثُوا أَحَدَكُم بِوَرِقِكُمْ هَذِهِ إِلَى الْمَدِينَةِ فَلْيَنظُرْ أَيُّهَا أَزْكَى طَعَامًا فَلْيَأْتِكُم بِرِزْقٍ مِّنْهُ وَلْيَتَلَطَّفْ وَلَا يُشْعِرَنَّ بِكُمْ أَحَدًا
{\tiny\colorbox{cl_aya}{20}} إِنَّهُمْ إِن يَظْهَرُوا عَلَيْكُمْ يَرْجُمُوكُمْ أَوْ يُعِيدُوكُمْ فِى مِلَّتِهِمْ وَلَن تُفْلِحُوا إِذًا أَبَدًا
{\tiny\colorbox{cl_aya}{21}} وَكَذَلِكَ أَعْثَرْنَا عَلَيْهِمْ لِيَعْلَمُوا أَنَّ وَعْدَ اللَّهِ حَقٌّ وَأَنَّ السَّاعَةَ لَا رَيْبَ فِيهَا إِذْ يَتَنَزَعُونَ بَيْنَهُمْ أَمْرَهُمْ فَقَالُوا ابْنُوا عَلَيْهِم بُنْيَنًا رَّبُّهُمْ أَعْلَمُ بِهِمْ قَالَ الَّذِينَ غَلَبُوا عَلَى أَمْرِهِمْ لَنَتَّخِذَنَّ عَلَيْهِم مَّسْجِدًا
{\tiny\colorbox{cl_aya}{22}} سَيَقُولُونَ ثَلَثَةٌ رَّابِعُهُمْ كَلْبُهُمْ وَيَقُولُونَ خَمْسَةٌ سَادِسُهُمْ كَلْبُهُمْ رَجْمًا بِالْغَيْبِ وَيَقُولُونَ سَبْعَةٌ وَثَامِنُهُمْ كَلْبُهُمْ قُل رَّبِّى أَعْلَمُ بِعِدَّتِهِم مَّا يَعْلَمُهُمْ إِلَّا قَلِيلٌ فَلَا تُمَارِ فِيهِمْ إِلَّا مِرَاءً ظَهِرًا وَلَا تَسْتَفْتِ فِيهِم مِّنْهُمْ أَحَدًا
{\tiny\colorbox{cl_aya}{23}} وَلَا تَقُولَنَّ لِشَاىْءٍ إِنِّى فَاعِلٌ ذَلِكَ غَدًا
{\tiny\colorbox{cl_aya}{24}} إِلَّا أَن يَشَاءَ اللَّهُ وَاذْكُر رَّبَّكَ إِذَا نَسِيتَ وَقُلْ عَسَى أَن يَهْدِيَنِ رَبِّى لِأَقْرَبَ مِنْ هَذَا رَشَدًا
{\tiny\colorbox{cl_aya}{25}} وَلَبِثُوا فِى كَهْفِهِمْ ثَلَثَ مِائَةٍ سِنِينَ وَازْدَادُوا تِسْعًا
{\tiny\colorbox{cl_aya}{26}} قُلِ اللَّهُ أَعْلَمُ بِمَا لَبِثُوا لَهُ غَيْبُ السَّمَوَتِ وَالْأَرْضِ أَبْصِرْ بِهِ وَأَسْمِعْ مَا لَهُم مِّن دُونِهِ مِن وَلِىٍّ وَلَا يُشْرِكُ فِى حُكْمِهِ أَحَدًا
{\tiny\colorbox{cl_aya}{27}} وَاتْلُ مَا أُوحِىَ إِلَيْكَ مِن كِتَابِ رَبِّكَ لَا مُبَدِّلَ لِكَلِمَتِهِ وَلَن تَجِدَ مِن دُونِهِ مُلْتَحَدًا
{\tiny\colorbox{cl_aya}{28}} وَاصْبِرْ نَفْسَكَ مَعَ الَّذِينَ يَدْعُونَ رَبَّهُم بِالْغَدَوةِ وَالْعَشِىِّ يُرِيدُونَ وَجْهَهُ وَلَا تَعْدُ عَيْنَاكَ عَنْهُمْ تُرِيدُ زِينَةَ الْحَيَوةِ الدُّنْيَا وَلَا تُطِعْ مَنْ أَغْفَلْنَا قَلْبَهُ عَن ذِكْرِنَا وَاتَّبَعَ هَوَىهُ وَكَانَ أَمْرُهُ فُرُطًا
{\tiny\colorbox{cl_aya}{29}} وَقُلِ الْحَقُّ مِن رَّبِّكُمْ فَمَن شَاءَ فَلْيُؤْمِن وَمَن شَاءَ فَلْيَكْفُرْ إِنَّا أَعْتَدْنَا لِلظَّلِمِينَ نَارًا أَحَاطَ بِهِمْ سُرَادِقُهَا وَإِن يَسْتَغِيثُوا يُغَاثُوا بِمَاءٍ كَالْمُهْلِ يَشْوِى الْوُجُوهَ بِئْسَ الشَّرَابُ وَسَاءَتْ مُرْتَفَقًا
{\tiny\colorbox{cl_aya}{30}} إِنَّ الَّذِينَ ءَامَنُوا وَعَمِلُوا الصَّلِحَتِ إِنَّا لَا نُضِيعُ أَجْرَ مَنْ أَحْسَنَ عَمَلًا
{\tiny\colorbox{cl_aya}{31}} أُولَئِكَ لَهُمْ جَنَّتُ عَدْنٍ تَجْرِى مِن تَحْتِهِمُ الْأَنْهَرُ يُحَلَّوْنَ فِيهَا مِنْ أَسَاوِرَ مِن ذَهَبٍ وَيَلْبَسُونَ ثِيَابًا خُضْرًا مِّن سُندُسٍ وَإِسْتَبْرَقٍ مُّتَّكِِٔينَ فِيهَا عَلَى الْأَرَائِكِ نِعْمَ الثَّوَابُ وَحَسُنَتْ مُرْتَفَقًا
{\tiny\colorbox{cl_aya}{32}} وَاضْرِبْ لَهُم مَّثَلًا رَّجُلَيْنِ جَعَلْنَا لِأَحَدِهِمَا جَنَّتَيْنِ مِنْ أَعْنَبٍ وَحَفَفْنَهُمَا بِنَخْلٍ وَجَعَلْنَا بَيْنَهُمَا زَرْعًا
{\tiny\colorbox{cl_aya}{33}} كِلْتَا الْجَنَّتَيْنِ ءَاتَتْ أُكُلَهَا وَلَمْ تَظْلِم مِّنْهُ شَئًْا وَفَجَّرْنَا خِلَلَهُمَا نَهَرًا
{\tiny\colorbox{cl_aya}{34}} وَكَانَ لَهُ ثَمَرٌ فَقَالَ لِصَحِبِهِ وَهُوَ يُحَاوِرُهُ أَنَا أَكْثَرُ مِنكَ مَالًا وَأَعَزُّ نَفَرًا
{\tiny\colorbox{cl_aya}{35}} وَدَخَلَ جَنَّتَهُ وَهُوَ ظَالِمٌ لِّنَفْسِهِ قَالَ مَا أَظُنُّ أَن تَبِيدَ هَذِهِ أَبَدًا
{\tiny\colorbox{cl_aya}{36}} وَمَا أَظُنُّ السَّاعَةَ قَائِمَةً وَلَئِن رُّدِدتُّ إِلَى رَبِّى لَأَجِدَنَّ خَيْرًا مِّنْهَا مُنقَلَبًا
{\tiny\colorbox{cl_aya}{37}} قَالَ لَهُ صَاحِبُهُ وَهُوَ يُحَاوِرُهُ أَكَفَرْتَ بِالَّذِى خَلَقَكَ مِن تُرَابٍ ثُمَّ مِن نُّطْفَةٍ ثُمَّ سَوَّىكَ رَجُلًا
{\tiny\colorbox{cl_aya}{38}} لَّكِنَّا هُوَ اللَّهُ رَبِّى وَلَا أُشْرِكُ بِرَبِّى أَحَدًا
{\tiny\colorbox{cl_aya}{39}} وَلَوْلَا إِذْ دَخَلْتَ جَنَّتَكَ قُلْتَ مَا شَاءَ اللَّهُ لَا قُوَّةَ إِلَّا بِاللَّهِ إِن تَرَنِ أَنَا أَقَلَّ مِنكَ مَالًا وَوَلَدًا
{\tiny\colorbox{cl_aya}{40}} فَعَسَى رَبِّى أَن يُؤْتِيَنِ خَيْرًا مِّن جَنَّتِكَ وَيُرْسِلَ عَلَيْهَا حُسْبَانًا مِّنَ السَّمَاءِ فَتُصْبِحَ صَعِيدًا زَلَقًا
{\tiny\colorbox{cl_aya}{41}} أَوْ يُصْبِحَ مَاؤُهَا غَوْرًا فَلَن تَسْتَطِيعَ لَهُ طَلَبًا
{\tiny\colorbox{cl_aya}{42}} وَأُحِيطَ بِثَمَرِهِ فَأَصْبَحَ يُقَلِّبُ كَفَّيْهِ عَلَى مَا أَنفَقَ فِيهَا وَهِىَ خَاوِيَةٌ عَلَى عُرُوشِهَا وَيَقُولُ يَلَيْتَنِى لَمْ أُشْرِكْ بِرَبِّى أَحَدًا
{\tiny\colorbox{cl_aya}{43}} وَلَمْ تَكُن لَّهُ فِئَةٌ يَنصُرُونَهُ مِن دُونِ اللَّهِ وَمَا كَانَ مُنتَصِرًا
{\tiny\colorbox{cl_aya}{44}} هُنَالِكَ الْوَلَيَةُ لِلَّهِ الْحَقِّ هُوَ خَيْرٌ ثَوَابًا وَخَيْرٌ عُقْبًا
{\tiny\colorbox{cl_aya}{45}} وَاضْرِبْ لَهُم مَّثَلَ الْحَيَوةِ الدُّنْيَا كَمَاءٍ أَنزَلْنَهُ مِنَ السَّمَاءِ فَاخْتَلَطَ بِهِ نَبَاتُ الْأَرْضِ فَأَصْبَحَ هَشِيمًا تَذْرُوهُ الرِّيَحُ وَكَانَ اللَّهُ عَلَى كُلِّ شَىْءٍ مُّقْتَدِرًا
{\tiny\colorbox{cl_aya}{46}} الْمَالُ وَالْبَنُونَ زِينَةُ الْحَيَوةِ الدُّنْيَا وَالْبَقِيَتُ الصَّلِحَتُ خَيْرٌ عِندَ رَبِّكَ ثَوَابًا وَخَيْرٌ أَمَلًا
{\tiny\colorbox{cl_aya}{47}} وَيَوْمَ نُسَيِّرُ الْجِبَالَ وَتَرَى الْأَرْضَ بَارِزَةً وَحَشَرْنَهُمْ فَلَمْ نُغَادِرْ مِنْهُمْ أَحَدًا
{\tiny\colorbox{cl_aya}{48}} وَعُرِضُوا عَلَى رَبِّكَ صَفًّا لَّقَدْ جِئْتُمُونَا كَمَا خَلَقْنَكُمْ أَوَّلَ مَرَّةٍ بَلْ زَعَمْتُمْ أَلَّن نَّجْعَلَ لَكُم مَّوْعِدًا
{\tiny\colorbox{cl_aya}{49}} وَوُضِعَ الْكِتَبُ فَتَرَى الْمُجْرِمِينَ مُشْفِقِينَ مِمَّا فِيهِ وَيَقُولُونَ يَوَيْلَتَنَا مَالِ هَذَا الْكِتَبِ لَا يُغَادِرُ صَغِيرَةً وَلَا كَبِيرَةً إِلَّا أَحْصَىهَا وَوَجَدُوا مَا عَمِلُوا حَاضِرًا وَلَا يَظْلِمُ رَبُّكَ أَحَدًا
{\tiny\colorbox{cl_aya}{50}} وَإِذْ قُلْنَا لِلْمَلَئِكَةِ اسْجُدُوا لِءَادَمَ فَسَجَدُوا إِلَّا إِبْلِيسَ كَانَ مِنَ الْجِنِّ فَفَسَقَ عَنْ أَمْرِ رَبِّهِ أَفَتَتَّخِذُونَهُ وَذُرِّيَّتَهُ أَوْلِيَاءَ مِن دُونِى وَهُمْ لَكُمْ عَدُوٌّ بِئْسَ لِلظَّلِمِينَ بَدَلًا
{\tiny\colorbox{cl_aya}{51}} مَّا أَشْهَدتُّهُمْ خَلْقَ السَّمَوَتِ وَالْأَرْضِ وَلَا خَلْقَ أَنفُسِهِمْ وَمَا كُنتُ مُتَّخِذَ الْمُضِلِّينَ عَضُدًا
{\tiny\colorbox{cl_aya}{52}} وَيَوْمَ يَقُولُ نَادُوا شُرَكَاءِىَ الَّذِينَ زَعَمْتُمْ فَدَعَوْهُمْ فَلَمْ يَسْتَجِيبُوا لَهُمْ وَجَعَلْنَا بَيْنَهُم مَّوْبِقًا
{\tiny\colorbox{cl_aya}{53}} وَرَءَا الْمُجْرِمُونَ النَّارَ فَظَنُّوا أَنَّهُم مُّوَاقِعُوهَا وَلَمْ يَجِدُوا عَنْهَا مَصْرِفًا
{\tiny\colorbox{cl_aya}{54}} وَلَقَدْ صَرَّفْنَا فِى هَذَا الْقُرْءَانِ لِلنَّاسِ مِن كُلِّ مَثَلٍ وَكَانَ الْإِنسَنُ أَكْثَرَ شَىْءٍ جَدَلًا
{\tiny\colorbox{cl_aya}{55}} وَمَا مَنَعَ النَّاسَ أَن يُؤْمِنُوا إِذْ جَاءَهُمُ الْهُدَى وَيَسْتَغْفِرُوا رَبَّهُمْ إِلَّا أَن تَأْتِيَهُمْ سُنَّةُ الْأَوَّلِينَ أَوْ يَأْتِيَهُمُ الْعَذَابُ قُبُلًا
{\tiny\colorbox{cl_aya}{56}} وَمَا نُرْسِلُ الْمُرْسَلِينَ إِلَّا مُبَشِّرِينَ وَمُنذِرِينَ وَيُجَدِلُ الَّذِينَ كَفَرُوا بِالْبَطِلِ لِيُدْحِضُوا بِهِ الْحَقَّ وَاتَّخَذُوا ءَايَتِى وَمَا أُنذِرُوا هُزُوًا
{\tiny\colorbox{cl_aya}{57}} وَمَنْ أَظْلَمُ مِمَّن ذُكِّرَ بَِٔايَتِ رَبِّهِ فَأَعْرَضَ عَنْهَا وَنَسِىَ مَا قَدَّمَتْ يَدَاهُ إِنَّا جَعَلْنَا عَلَى قُلُوبِهِمْ أَكِنَّةً أَن يَفْقَهُوهُ وَفِى ءَاذَانِهِمْ وَقْرًا وَإِن تَدْعُهُمْ إِلَى الْهُدَى فَلَن يَهْتَدُوا إِذًا أَبَدًا
{\tiny\colorbox{cl_aya}{58}} وَرَبُّكَ الْغَفُورُ ذُو الرَّحْمَةِ لَوْ يُؤَاخِذُهُم بِمَا كَسَبُوا لَعَجَّلَ لَهُمُ الْعَذَابَ بَل لَّهُم مَّوْعِدٌ لَّن يَجِدُوا مِن دُونِهِ مَوْئِلًا
{\tiny\colorbox{cl_aya}{59}} وَتِلْكَ الْقُرَى أَهْلَكْنَهُمْ لَمَّا ظَلَمُوا وَجَعَلْنَا لِمَهْلِكِهِم مَّوْعِدًا
{\tiny\colorbox{cl_aya}{60}} وَإِذْ قَالَ مُوسَى لِفَتَىهُ لَا أَبْرَحُ حَتَّى أَبْلُغَ مَجْمَعَ الْبَحْرَيْنِ أَوْ أَمْضِىَ حُقُبًا
{\tiny\colorbox{cl_aya}{61}} فَلَمَّا بَلَغَا مَجْمَعَ بَيْنِهِمَا نَسِيَا حُوتَهُمَا فَاتَّخَذَ سَبِيلَهُ فِى الْبَحْرِ سَرَبًا
{\tiny\colorbox{cl_aya}{62}} فَلَمَّا جَاوَزَا قَالَ لِفَتَىهُ ءَاتِنَا غَدَاءَنَا لَقَدْ لَقِينَا مِن سَفَرِنَا هَذَا نَصَبًا
{\tiny\colorbox{cl_aya}{63}} قَالَ أَرَءَيْتَ إِذْ أَوَيْنَا إِلَى الصَّخْرَةِ فَإِنِّى نَسِيتُ الْحُوتَ وَمَا أَنسَىنِيهُ إِلَّا الشَّيْطَنُ أَنْ أَذْكُرَهُ وَاتَّخَذَ سَبِيلَهُ فِى الْبَحْرِ عَجَبًا
{\tiny\colorbox{cl_aya}{64}} قَالَ ذَلِكَ مَا كُنَّا نَبْغِ فَارْتَدَّا عَلَى ءَاثَارِهِمَا قَصَصًا
{\tiny\colorbox{cl_aya}{65}} فَوَجَدَا عَبْدًا مِّنْ عِبَادِنَا ءَاتَيْنَهُ رَحْمَةً مِّنْ عِندِنَا وَعَلَّمْنَهُ مِن لَّدُنَّا عِلْمًا
{\tiny\colorbox{cl_aya}{66}} قَالَ لَهُ مُوسَى هَلْ أَتَّبِعُكَ عَلَى أَن تُعَلِّمَنِ مِمَّا عُلِّمْتَ رُشْدًا
{\tiny\colorbox{cl_aya}{67}} قَالَ إِنَّكَ لَن تَسْتَطِيعَ مَعِىَ صَبْرًا
{\tiny\colorbox{cl_aya}{68}} وَكَيْفَ تَصْبِرُ عَلَى مَا لَمْ تُحِطْ بِهِ خُبْرًا
{\tiny\colorbox{cl_aya}{69}} قَالَ سَتَجِدُنِى إِن شَاءَ اللَّهُ صَابِرًا وَلَا أَعْصِى لَكَ أَمْرًا
{\tiny\colorbox{cl_aya}{70}} قَالَ فَإِنِ اتَّبَعْتَنِى فَلَا تَسَْٔلْنِى عَن شَىْءٍ حَتَّى أُحْدِثَ لَكَ مِنْهُ ذِكْرًا
{\tiny\colorbox{cl_aya}{71}} فَانطَلَقَا حَتَّى إِذَا رَكِبَا فِى السَّفِينَةِ خَرَقَهَا قَالَ أَخَرَقْتَهَا لِتُغْرِقَ أَهْلَهَا لَقَدْ جِئْتَ شَئًْا إِمْرًا
{\tiny\colorbox{cl_aya}{72}} قَالَ أَلَمْ أَقُلْ إِنَّكَ لَن تَسْتَطِيعَ مَعِىَ صَبْرًا
{\tiny\colorbox{cl_aya}{73}} قَالَ لَا تُؤَاخِذْنِى بِمَا نَسِيتُ وَلَا تُرْهِقْنِى مِنْ أَمْرِى عُسْرًا
{\tiny\colorbox{cl_aya}{74}} فَانطَلَقَا حَتَّى إِذَا لَقِيَا غُلَمًا فَقَتَلَهُ قَالَ أَقَتَلْتَ نَفْسًا زَكِيَّةً بِغَيْرِ نَفْسٍ لَّقَدْ جِئْتَ شَئًْا نُّكْرًا
{\tiny\colorbox{cl_aya}{75}} قَالَ أَلَمْ أَقُل لَّكَ إِنَّكَ لَن تَسْتَطِيعَ مَعِىَ صَبْرًا
{\tiny\colorbox{cl_aya}{76}} قَالَ إِن سَأَلْتُكَ عَن شَىْءٍ بَعْدَهَا فَلَا تُصَحِبْنِى قَدْ بَلَغْتَ مِن لَّدُنِّى عُذْرًا
{\tiny\colorbox{cl_aya}{77}} فَانطَلَقَا حَتَّى إِذَا أَتَيَا أَهْلَ قَرْيَةٍ اسْتَطْعَمَا أَهْلَهَا فَأَبَوْا أَن يُضَيِّفُوهُمَا فَوَجَدَا فِيهَا جِدَارًا يُرِيدُ أَن يَنقَضَّ فَأَقَامَهُ قَالَ لَوْ شِئْتَ لَتَّخَذْتَ عَلَيْهِ أَجْرًا
{\tiny\colorbox{cl_aya}{78}} قَالَ هَذَا فِرَاقُ بَيْنِى وَبَيْنِكَ سَأُنَبِّئُكَ بِتَأْوِيلِ مَا لَمْ تَسْتَطِع عَّلَيْهِ صَبْرًا
{\tiny\colorbox{cl_aya}{79}} أَمَّا السَّفِينَةُ فَكَانَتْ لِمَسَكِينَ يَعْمَلُونَ فِى الْبَحْرِ فَأَرَدتُّ أَنْ أَعِيبَهَا وَكَانَ وَرَاءَهُم مَّلِكٌ يَأْخُذُ كُلَّ سَفِينَةٍ غَصْبًا
{\tiny\colorbox{cl_aya}{80}} وَأَمَّا الْغُلَمُ فَكَانَ أَبَوَاهُ مُؤْمِنَيْنِ فَخَشِينَا أَن يُرْهِقَهُمَا طُغْيَنًا وَكُفْرًا
{\tiny\colorbox{cl_aya}{81}} فَأَرَدْنَا أَن يُبْدِلَهُمَا رَبُّهُمَا خَيْرًا مِّنْهُ زَكَوةً وَأَقْرَبَ رُحْمًا
{\tiny\colorbox{cl_aya}{82}} وَأَمَّا الْجِدَارُ فَكَانَ لِغُلَمَيْنِ يَتِيمَيْنِ فِى الْمَدِينَةِ وَكَانَ تَحْتَهُ كَنزٌ لَّهُمَا وَكَانَ أَبُوهُمَا صَلِحًا فَأَرَادَ رَبُّكَ أَن يَبْلُغَا أَشُدَّهُمَا وَيَسْتَخْرِجَا كَنزَهُمَا رَحْمَةً مِّن رَّبِّكَ وَمَا فَعَلْتُهُ عَنْ أَمْرِى ذَلِكَ تَأْوِيلُ مَا لَمْ تَسْطِع عَّلَيْهِ صَبْرًا
{\tiny\colorbox{cl_aya}{83}} وَيَسَْٔلُونَكَ عَن ذِى الْقَرْنَيْنِ قُلْ سَأَتْلُوا عَلَيْكُم مِّنْهُ ذِكْرًا
{\tiny\colorbox{cl_aya}{84}} إِنَّا مَكَّنَّا لَهُ فِى الْأَرْضِ وَءَاتَيْنَهُ مِن كُلِّ شَىْءٍ سَبَبًا
{\tiny\colorbox{cl_aya}{85}} فَأَتْبَعَ سَبَبًا
{\tiny\colorbox{cl_aya}{86}} حَتَّى إِذَا بَلَغَ مَغْرِبَ الشَّمْسِ وَجَدَهَا تَغْرُبُ فِى عَيْنٍ حَمِئَةٍ وَوَجَدَ عِندَهَا قَوْمًا قُلْنَا يَذَا الْقَرْنَيْنِ إِمَّا أَن تُعَذِّبَ وَإِمَّا أَن تَتَّخِذَ فِيهِمْ حُسْنًا
{\tiny\colorbox{cl_aya}{87}} قَالَ أَمَّا مَن ظَلَمَ فَسَوْفَ نُعَذِّبُهُ ثُمَّ يُرَدُّ إِلَى رَبِّهِ فَيُعَذِّبُهُ عَذَابًا نُّكْرًا
{\tiny\colorbox{cl_aya}{88}} وَأَمَّا مَنْ ءَامَنَ وَعَمِلَ صَلِحًا فَلَهُ جَزَاءً الْحُسْنَى وَسَنَقُولُ لَهُ مِنْ أَمْرِنَا يُسْرًا
{\tiny\colorbox{cl_aya}{89}} ثُمَّ أَتْبَعَ سَبَبًا
{\tiny\colorbox{cl_aya}{90}} حَتَّى إِذَا بَلَغَ مَطْلِعَ الشَّمْسِ وَجَدَهَا تَطْلُعُ عَلَى قَوْمٍ لَّمْ نَجْعَل لَّهُم مِّن دُونِهَا سِتْرًا
{\tiny\colorbox{cl_aya}{91}} كَذَلِكَ وَقَدْ أَحَطْنَا بِمَا لَدَيْهِ خُبْرًا
{\tiny\colorbox{cl_aya}{92}} ثُمَّ أَتْبَعَ سَبَبًا
{\tiny\colorbox{cl_aya}{93}} حَتَّى إِذَا بَلَغَ بَيْنَ السَّدَّيْنِ وَجَدَ مِن دُونِهِمَا قَوْمًا لَّا يَكَادُونَ يَفْقَهُونَ قَوْلًا
{\tiny\colorbox{cl_aya}{94}} قَالُوا يَذَا الْقَرْنَيْنِ إِنَّ يَأْجُوجَ وَمَأْجُوجَ مُفْسِدُونَ فِى الْأَرْضِ فَهَلْ نَجْعَلُ لَكَ خَرْجًا عَلَى أَن تَجْعَلَ بَيْنَنَا وَبَيْنَهُمْ سَدًّا
{\tiny\colorbox{cl_aya}{95}} قَالَ مَا مَكَّنِّى فِيهِ رَبِّى خَيْرٌ فَأَعِينُونِى بِقُوَّةٍ أَجْعَلْ بَيْنَكُمْ وَبَيْنَهُمْ رَدْمًا
{\tiny\colorbox{cl_aya}{96}} ءَاتُونِى زُبَرَ الْحَدِيدِ حَتَّى إِذَا سَاوَى بَيْنَ الصَّدَفَيْنِ قَالَ انفُخُوا حَتَّى إِذَا جَعَلَهُ نَارًا قَالَ ءَاتُونِى أُفْرِغْ عَلَيْهِ قِطْرًا
{\tiny\colorbox{cl_aya}{97}} فَمَا اسْطَعُوا أَن يَظْهَرُوهُ وَمَا اسْتَطَعُوا لَهُ نَقْبًا
{\tiny\colorbox{cl_aya}{98}} قَالَ هَذَا رَحْمَةٌ مِّن رَّبِّى فَإِذَا جَاءَ وَعْدُ رَبِّى جَعَلَهُ دَكَّاءَ وَكَانَ وَعْدُ رَبِّى حَقًّا
{\tiny\colorbox{cl_aya}{99}} وَتَرَكْنَا بَعْضَهُمْ يَوْمَئِذٍ يَمُوجُ فِى بَعْضٍ وَنُفِخَ فِى الصُّورِ فَجَمَعْنَهُمْ جَمْعًا
{\tiny\colorbox{cl_aya}{100}} وَعَرَضْنَا جَهَنَّمَ يَوْمَئِذٍ لِّلْكَفِرِينَ عَرْضًا
{\tiny\colorbox{cl_aya}{101}} الَّذِينَ كَانَتْ أَعْيُنُهُمْ فِى غِطَاءٍ عَن ذِكْرِى وَكَانُوا لَا يَسْتَطِيعُونَ سَمْعًا
{\tiny\colorbox{cl_aya}{102}} أَفَحَسِبَ الَّذِينَ كَفَرُوا أَن يَتَّخِذُوا عِبَادِى مِن دُونِى أَوْلِيَاءَ إِنَّا أَعْتَدْنَا جَهَنَّمَ لِلْكَفِرِينَ نُزُلًا
{\tiny\colorbox{cl_aya}{103}} قُلْ هَلْ نُنَبِّئُكُم بِالْأَخْسَرِينَ أَعْمَلًا
{\tiny\colorbox{cl_aya}{104}} الَّذِينَ ضَلَّ سَعْيُهُمْ فِى الْحَيَوةِ الدُّنْيَا وَهُمْ يَحْسَبُونَ أَنَّهُمْ يُحْسِنُونَ صُنْعًا
{\tiny\colorbox{cl_aya}{105}} أُولَئِكَ الَّذِينَ كَفَرُوا بَِٔايَتِ رَبِّهِمْ وَلِقَائِهِ فَحَبِطَتْ أَعْمَلُهُمْ فَلَا نُقِيمُ لَهُمْ يَوْمَ الْقِيَمَةِ وَزْنًا
{\tiny\colorbox{cl_aya}{106}} ذَلِكَ جَزَاؤُهُمْ جَهَنَّمُ بِمَا كَفَرُوا وَاتَّخَذُوا ءَايَتِى وَرُسُلِى هُزُوًا
{\tiny\colorbox{cl_aya}{107}} إِنَّ الَّذِينَ ءَامَنُوا وَعَمِلُوا الصَّلِحَتِ كَانَتْ لَهُمْ جَنَّتُ الْفِرْدَوْسِ نُزُلًا
{\tiny\colorbox{cl_aya}{108}} خَلِدِينَ فِيهَا لَا يَبْغُونَ عَنْهَا حِوَلًا
{\tiny\colorbox{cl_aya}{109}} قُل لَّوْ كَانَ الْبَحْرُ مِدَادًا لِّكَلِمَتِ رَبِّى لَنَفِدَ الْبَحْرُ قَبْلَ أَن تَنفَدَ كَلِمَتُ رَبِّى وَلَوْ جِئْنَا بِمِثْلِهِ مَدَدًا
{\tiny\colorbox{cl_aya}{110}} قُلْ إِنَّمَا أَنَا بَشَرٌ مِّثْلُكُمْ يُوحَى إِلَىَّ أَنَّمَا إِلَهُكُمْ إِلَهٌ وَحِدٌ فَمَن كَانَ يَرْجُوا لِقَاءَ رَبِّهِ فَلْيَعْمَلْ عَمَلًا صَلِحًا وَلَا يُشْرِكْ بِعِبَادَةِ رَبِّهِ أَحَدًا
\end{document}